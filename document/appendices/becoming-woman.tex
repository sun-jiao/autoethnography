\subsection{From Interpellation to Performativity}\label{subsec:from-interpellation-to-performativity}

During this time, I was in a state of confused self-exploration, mostly treating it as a meaningless fantasy. However, it increasingly affected my life. The most serious issue was that I would cast myself in the role of the female protagonist when reading some adult stories, and sometimes I couldn't distinguish between fantasy and reality. A few months ago, I was reading an adult story about psychological manipulation. I identified with the manipulated character and, for a long time, did not realise it was a story about psychological manipulation, believing the manipulator was genuinely trying to protect ``me.'' However, when I realised he was just manipulating and using ``me,'' I had a severe nightmare. I dreamt that ``I'' was crying and begging him to continue deceiving ``me.'' I woke up from the dream in the middle of the night and sat in my room for a long time, unable to fall back asleep.

After waking up the next day, I realised I had to take this matter seriously. Being suggested by friends, I began to read the works of Foucault, Althusser, and Butler to understand my experiences.

Butler points out that gender is not what we intrinsically ``are,'' but what we constantly ``do.'' It is not a stable essence, but an effect produced by the ``stylised repetition of acts in time.'' There is no pre-existing ``doer'' behind the deed. On the contrary, it is the deed itself, in its constant repetition, that fictitiously constitutes and consolidates the illusion of a stable ``doer.'' I found that many of my actions perfectly illustrate this theory.

In informal settings, I used Latin letters or cursive script for my signature to make my name ambiguous. In my artificial intelligence class, I changed the dataset. These were secret performances about the name. I even used that feminine name myself, and if discovered, I would blame the ``mistake'' on the input method. This series of performances—using an ambiguous signature, altering the dataset, and blaming the input method—is precisely what Judith Butler describes as the ``stylised repetition of acts.'' They are not expressions of an inner essence but produce the effect of identity through constant reiteration in a specific social context.

My feeling happy when praised by friends as ``quiet'' and ``virtuous'' was also a performative citation of the script of traditional femininity. Making a bracelet and enclosing a note with delicate handwriting, the ``King'' in the game said they thought their ``angel'' would be a girl, and I ``secretly felt extremely delighted.'' These are not passive reactions, but a series of active, ``stylised acts.'' By repeating these acts that conform to social imaginings of ``femininity,'' I produced the effect of a female gender identity.

A strong sense of pleasure always accompanied these moments of successful performance. The source of this pleasure is key to understanding Butler's theory. When friends defended my feminine name, when the ``King'' in the game mistook my gender, and especially when the junior female student replied, ``Wow, a pretty sis,'' I felt delighted. This gender euphoria is not the joy of an inner, ``true'' self being seen—that view still falls into the trap of essentialism. Instead, this pleasure is the profound satisfaction and relief that comes after a successful performative utterance. At that moment, the social world reflects and confirms the identity that my actions sought to create. It is the joy of making a fiction feel real, the satisfaction of successfully bringing a social reality into existence.

This discourse also plunged me into a brief confusion: if gender is a performance, then what is my gender identity, and where did it come from? Is it also the result of my performances at a younger age? I began to try to reflect on all gender-related matters in my upbringing, trying to understand how my gender identity was constructed.

My desire for a feminine name may not have been fabricated out of thin air, but like a ghost, had long been lurking in my life, stemming from the experience of my name being frequently misspelt or mistyped by others in my childhood.

A few other childhood memories also flooded my mind: adults used to say my ``personality was like a little girl's,'' probably because I liked playing with stuffed animals and disliked sports and fighting. I was also punished for imitating a little girl on TV covering her mouth to laugh, being told, ``Boys can't laugh like girls.'' Children have no gender bias and spontaneously explore the world, imitating and performing all behaviours within their capabilities as a naive performative.

Whether it's an input method error or the correction of behaviour, it is the continuous interpellation of the ideological state apparatus. Louis Althusser describes interpellation as the process by which ideology ``hails'' individuals and constitutes them as subjects, like a policeman shouting on the street: ``Hey, you there!'' Every misuse of a name, every disciplining of behaviour, is an interpellation from the social machine: ``Hey, the little girl there!'' Initially, my reaction to this call was anger and rejection. I was eager to correct it every time, trying to maintain the integrity of my identity as a ``boy.'' This anger was an instinctive resistance to a ``wrong'' ideological positioning.

This long negotiation around my name reveals the contingent origins of gender identity. The initial seed of my feminised identity did not come from some \textit{a priori} inner essence, but from a systematic ``error''—a product of an input method algorithm, a technological noise. This origin negates the Cartesian, Western Enlightenment narrative of ``finding the true self.'' The construction of identity is sometimes not a heroic journey of self-exploration, but more like an opportunistic, creative use of system loopholes and contingencies. The generation of queer identity may often occur in the cracks, glitches, and noises of the normative system. It is a form of bricolage, not an architect's creation. The trajectory of emotion, from initial anger at the ``wrong'' interpellation to the eventual joyful acceptance of an alternative subject position, reveals the core of ideology's operation. Ideology is not a cold set of ideas, but a power that orchestrates, manages, and even produces emotion. When mainstream ideology fails to provide emotional comfort, an alternative, marginal call can bring liberating pleasure, because it happens to bridge the pre-existing, ineffable gap between emotion and social positioning.

We can see a dynamic feedback loop between Althusser's interpellation and Butler's performativity. Initially, an interpellation from the external world (like the misspelt name) provides an ideological ``script.'' Then, the subject actively ``adopts'' and ``performs'' this script through a series of stylised, repetitive acts (like deliberately blurring a signature, cooking, making a bracelet). These successful performances, in turn, elicit new, more precise interpellations from others (like the address ``pretty sis''), and these new interpellations further strengthen the subject's desire to continue performing. Gender construction is thus a never-ending circular process: we are interpellated into a specific subject position (Althusser), then we consolidate and inhabit these positions through performativity (Butler), and our actions in turn invite more interpellations. Identity is not a destination to be reached, but a continuous, dynamic process of negotiation between action and response.

\subsection{A Pair of Shoes}\label{subsec:a-pair-of-shoes}

If the interpellation of a name was a relatively gentle negotiation at the symbolic level, the ``gymnastics shoes'' incident was a cruel disciplinary practice inscribed upon the body. This memory is an excellent model for understanding Foucauldian power: power does not always oppress in a grand, visible, monarchical way; it more often exists in a micro, diffuse form of discipline aimed at producing ``docile bodies.''

In elementary school, I wore a type of white cloth shoe with red trim, colloquially known as ``gymnastics shoes.'' These shoes were far more than just a colour and style; they were a strictly encoded gender symbol in the school. They were part of the girls' school uniform. In all occasions requiring a uniform, such as ceremonies and performances, girls were required to wear the red version and boys the blue one. Therefore, in this micro-society, ``red shoes are for girls'' was not a debatable opinion, but an unquestionable ``truth.'' By wearing red shoes, I blatantly violated the established norm. My body, because of these shoes, became an ``abnormal,'' transgressive body, exposed to the gaze of disciplinary power.

The power that executed the punishment did not come from the ``monarchical'' authority of the principal or teachers, but was diffused among the peer group. A few boys snatched my shoes and threw them away, pulled down my pants, and pushed me onto the broken branches of a bush in the garden. These acts of bullying were a cruel, decentralised enforcement of gender norms. The classmates at that moment acted as guards in a prison; they were the most terminal and effective nodes of the power network. The punishment was directly applied to my body: the shoes were stripped away, the body was exposed, and the skin was pierced. This precisely confirms Foucault's assertion: the target of disciplinary power is the body. It operates on the body, shaping it into a compliant ``docile body'' through training, marking, and punishment.

This micro-power operates not only by punishing transgressors but also by constantly reproducing gender through the ``truth mechanisms'' within the group. I recall that the boys in my class often established hierarchies through fighting, and I preferred to play with the girls because their interactions seemed more friendly. This preference did not stem from any innate gender essence but was a reaction to two different regimes of discipline. During a museum visit, this difference was dramatically displayed: the boys' group responded to failure with ridicule and expulsion, a punitive practice aimed at eliminating the weak and reinforcing competitive norms; the girls' group, on the other hand, offered encouragement to those who failed, a technology of care aimed at maintaining the collective and emphasising cooperation. My ``envy'' at that moment was a subject's longing for a more attractive subject position within two different micro-power networks. This reveals that the gender divide is not only marked by symbols like clothing, but is also continuously reproduced through distinctly different behavioural norms, emotional patterns, and power technologies.

Faced with this violence inscribed on the body, psychological resistance unfolded peculiarly. I often dreamt of losing my feet in an accident and would feel a strong sense of comfort and relief upon waking. It is a profound psychological metaphor: since the feet wearing ``girls' shoes'' were the source of social punishment and pain, removing these ``girls' feet'' was a fantasy of liberation from the grasp of power/knowledge. It was a complete refusal to become the ``docile body'' that the system demanded. This experience provides a powerful, non-essentialist genealogical explanation for the emergence of body dysphoria. It shows that body dysphoria does not necessarily stem from a pre-social, innate sense of mismatch. Instead, it can be the direct consequence of disciplinary power violently inscribing social meaning (``girlish'') onto the body. Dysphoria arises when this inscription is so painful that the subject internalises it as an alienation from their own flesh. I initially had no aversion to my body. Still, when a part of it (my feet) became the target of social punishment because of its association with a ``transgressive'' gender symbol (red shoes), I developed a psychological desire to get rid of that body part, seeing it as a foreign object. This localised sense of alienation may be the origin of a more widespread body dysphoria. This ``dysphoria'' is the scar left on the psyche by normalised violence.

I read some literature and researched their history, discovering that these shoes likely originated from Japanese indoor shoes (uwabaki), mainly used in schools and kindergartens in Japan. Chinese clothing factories likely took on some Japanese orders to produce these shoes, which were later sold in China. Due to their simple style and low price, they became widely used as uniform shoes for elementary school students. In some Japanese schools, the colour of indoor shoes also depends on gender, but some schools assign colours by grade level. \parencite{Kanzaki2019Shogakko} Such a genealogical analysis reveals the contingency and arbitrariness of the gendered signifier. The ``truth'' that ``red shoes are for girls'' is filled with historical contingency and arbitrariness, a result of the operation of a power/knowledge mechanism, but not a metaphysical truth.

As an adult, I developed a unique, complex attachment to these shoes. I took a photo of myself wearing them, uploaded it to Wikimedia Commons, and added it to the relevant Wikipedia article. This act is a highly self-aware reversal of power. It repositions the disciplined body as the controller, transforming a symbol laden with personal trauma and shame into public knowledge. This is a modern form of ``reverse discourse.'' It utilises one of the most powerful tools of knowledge production and dissemination of our time to rewrite the public meaning of a symbol once used to discipline me. It is an attempt to transform from an object of power/knowledge into a subject that produces power/knowledge. In the digital age, resistance sometimes manifests as an information war, a battle over the public meaning of symbols.

\subsection{The Heterotopia of Childhood}\label{subsec:the-heterotopia-of-childhood}

The operation of power is not only inscribed on the body but also divides and encodes space. My childhood memories related to the girls' bathroom reveal how this space became what Foucault called a ``heterotopia''—a real place that is nevertheless outside of all other places, an ``other space.'' The girls' bathroom was not a neutral, functional place, but a symbolic domain full of contradictory meanings, where the conventional logic of power was suspended, inverted, and reconfigured.

In elementary school, I was playing tag with a few girls, and they repeatedly ran into the girls' bathroom to hide. I stood guard at the door, waiting to catch them when they came out, but a teacher saw me and scolded me. Here, power was not about upholding the rules of the game, but about maintaining the gendered segregation of space. The punishment I received was not for disrupting the game, but because my male body ``lingering'' outside a space coded as female constituted a potential threat to the spatial order.

Another incident was when a maths teacher used ``cleaning the girls' bathroom'' as a punishment for mischievous boys. This act was highly symbolic. It transformed a private space usually associated with femininity and ``impurity'' into a public theatre of punishment. The boys were forced to enter this forbidden zone, and the punishment lay not only in the hard labour but also in the ``humiliation'' of being forced to cross a gender boundary. Here, space was used as a disciplinary technology, reinforcing the norms of masculinity through forced ``effeminacy''—that is, ``real boys'' do not belong here.

Finally, at certain times, it offered care that violated the rules. Once, I got sick and vomited during class. Perhaps because the teacher was female, our classroom was close to the girls' bathroom, and it was during class time, when no one was there, the teacher took me to the girls' bathroom to clean the vomit off my clothes and body. In my most vulnerable moment, all the taboos of this space were temporarily lifted. In a state of bodily crisis, the strict rules of gender segregation gave way to a more pressing ethic of care, creating a ``state of exception'' that instantly transformed this forbidden space into a caring, healing place.

The superposition of these three memories constructs the girls' bathroom as a paradoxical heterotopia. It was simultaneously a safe zone for girls, a place of degradation for boys, and a place of care that could be transgressed in a crisis. It was both a symbol of strict gender order and proof that this order could be broken. This chaotic, contradictory meaning, in a pre-linguistic way, shaped the female space, female symbols, and even femininity itself in my mind into a complex amalgam of fascination, taboo, safety, and care, profoundly influencing my early perception of gender.

\subsection{Becoming-woman}\label{subsec:becoming-woman}

I am significantly different from ``typical'' transgender people in the mainstream discourse, as my dislike and anxiety about my biological body do not seem to be an ontological rejection. One piece of evidence is that I discovered the mechanism of masturbation at a very young age (around kindergarten age) and excitedly shared it with others. I know this behaviour might be considered ``shameless'' or ``perverted'' by adults, but for my kindergarten self, it was just an objective exploration of the body. I believe it is no different in essence from sucking one's thumb or playing with one's hair. A child sees the body as a field of possibilities, not a cage pre-coded by gender norms.

One thing that left a deep impression on me was that I had a crush on a girl in my childhood, but she didn't like me back. I happened to read Stefan Zweig's Letter from an Unknown Woman, in which the female protagonist has a one-night stand with the male protagonist and raises their child on her own. I thought at the time, ``Wow, I also want to have XX's child and raise it secretly.'' It's so great that girls can have babies; it's so enviable. Why can't a boy's body have babies? What a pity. This was an envy of a specific function, stemming from a longing for a romantic relationship.

I find a kind of beauty in the female body that is hard to describe. It's like an aesthetic feeling, which could perhaps be loosely called a sense of elegance. I find the female body very pleasing to look at and feel envious. A DeepNude version of my feminised self has sexually aroused me. This experience has an unsettling phenomenological similarity to Blanchard's  controversial theory of autogynephilia, which is often used as a weapon to pathologise and deny the identity of transgender women. This phenomenological similarity is my undeniable ``lived experience,'' but we do not have to accept his explanation. He explains this phenomenon by positing that sexual orientation is the root cause, from which gender identity is derived, which implies the existence of an innate, ontological sexual orientation.

This experience radically deconstructs the binary opposition between identity and desire. Traditional models of identity, including many mainstream LGBTQ+ popular discourses, usually treat gender identity (who you are) and sexual orientation (who you are attracted to) as two independent axes. However, my experience of being sexually aroused by the gender image I identify with completely dismantles this clear division. Is this an act of identity affirmation or an expression of sexual desire?

This demonstrates that my ``sexual orientation'' (towards women) and my ``gender identity'' may stem from some shared, more fundamental aesthetic preference. The preference itself is neutral. It is only when combined with different sociocultural scripts that it is differentiated and interpreted as different phenomena: when combined with intimate emotions and sexual instincts, it is named ``sexual orientation''; when it interacts with body image and external discipline (like the ``red shoes'' incident), it is constructed as part of ``gender identity.'' Identity, desire and other affairs are not independent axes in a Cartesian coordinate system, but fluid, entangled, paradoxical, and rhizomatic connected. My body is therefore not an error waiting to be ``corrected,'' but a dynamic site where various desires, memories, and aesthetic judgments constantly flow, recombine, and generate meaning.

\subsection{Cyber-Pharmakon}\label{subsec:cyber-pharmakon}

Among all the practices of constructing gender, the experience of using FaceApp to generate a feminised version of myself and then using DeepNude to create a nude version is undoubtedly the most ethically and psychologically complex and contradictory. It embodies Jacques Derrida's concept of the ``pharmakon.'' Pharmakon is a core concept Derrida proposed in his reading of Plato's Phaedrus, pointing to the inherent ``undecidability'' of a word in ancient Greek: pharmakon means both ``remedy'' and ``poison,'' and these two meanings are inseparable; it is always both at the same time. My use of DeepNude is a vivid enactment of the pharmakon concept in contemporary digital technology.

The ``poisonous'' nature of this application is obvious. We must admit that its algorithm, design purpose, and even its very existence are steeped in the violent logic of misogyny and the objectification of the female body. It was initially a tool for sexual exploitation. When I was sexually aroused by the feminised image of myself generated by DeepNude, this experience, though closely linked to my exploration of gender identity, became extremely complex due to its toxic origin. Using a tool designed to objectify others for self-exploration inherently carries the great risk of self-objectification. This is its ``poison'' side.

However, at the same time, this application acted as a powerful ``remedy.'' For the immense mental suffering caused by the mismatch between social roles and inner feelings, it provided a visual, concrete comfort. Those generated female nude images, along with the scanned ID documents I modified with Photoshop, provided a tangible, visible confirmation for my then-vague and unrecognised identity. This is its ``remedy'' side.

The core is that these two aspects are inseparable; we cannot cleanly separate the ``remedy'' from the ``poison.'' The power of this tool to heal my gender anxiety is derived from the same visual, sexualizing logic as its power as a tool of exploitation. It is the ability to generate realistic female nude images that allows it to generate a feminised body image that I find gender-affirming. My female identity was generated under the interpellation and performativity of the same misogyny and male gaze that are at the root of the AI software that creates objectified women. The affirmation of identity and sexual arousal are intertwined here, making it impossible to distinguish which is the cause and which is the effect. They are two sides of the same coin.

This act is a prime example of queer reappropriation, and queer reappropriation itself is a practice with the nature of a pharmakon. The word ``queer'' itself is the most famous example of the queer community reappropriating an oppressive tool and stigma (poison) into a symbol of community pride and identity (remedy). My action was to turn a technological tool designed for misogyny inward, using it for self-construction and self-healing. The political potential of queer practice often lies not in creating ``pure,'' utopian tools free from the contamination of the oppressive system, but in seizing the contaminated, ``toxic'' tools within that system and using them to resist the system's own logic. The power of reappropriation comes precisely from its impurity.

\subsection{Strategic Habitation}\label{subsec:strategic-habitation}

The archaeological excavation of this autoethnography finally arrives at the present. I turn to another question: Is my gender identity really ``female''? This seems natural, but in fact, I sometimes genuinely believe I am ``male,'' especially when arguing with trans-exclusionary radical feminists (TERFs) online.

When they criticise or insult all ``phenotypic males'' in some gender-essentialist way (e.g., ``all men are oppressors''), I feel very angry and use myself as a counterexample to refute them. Of course, this is partly because I know very well that their so-called ``men'' refers to phenotypic sex, not gender identity. It at least shows that although I usually feel uncomfortable when being called or classified as ``male,'' this discomfort is not greater than my hatred for irrationality. I feel comfortable with it if classifying myself as ``male'' can provide a valid counterexample to refute their argument.

I am not sure if this counts as a kind of ``gender identity.'' When I argue with TERFs, the structure of the anger is the same as the anger I feel when arguing with extreme nationalists who proclaim that ``all Japanese people are guilty, there are no innocent souls under the atomic bomb.'' I believe my anger stems from the ``collective responsibility.''

While for me personally, there is a significant difference between these two situations: I am obviously not Japanese, so when I argue with extreme nationalists, I am very clear that this is a purely moral anger against collective punishment. In the other situation, because I had not deeply thought about gender identity at the time, and their definition of ``male,'' along with that of the broader society, did include me, the target and direction of this anger were often confused. I sometimes genuinely felt that I was a (specifically defined) ``male'' and that I was being insulted.

From a particular perspective, this is also ``internalising a gender identity through interaction with a pervasively gendered society''. TERFs use a crude, essentialist method to impose the label ``male'' on me. For the sake of debate, I strategically accept this label and develop complex emotional reactions around it. This ``contextual male identity'' and the ``female identity,'' I feel, in many other scenarios, are formed by the exact same mechanism.

It demonstrated that this phenomenon not only occurs in childhood but also appears continuously throughout a person's life. It's just that for most people, whether cisgender or transgender, after their gender identity is formed, they will consciously resist the invasion from another gender. I happen to not care much about ``gender.'' I don't think my gender identity is essential to me, not my core identity, so I didn't resist it. Ironically, my ignorance of gender has allowed ``gender'' to be able to freely ``invade'' my ``self.'' Some other ``gender fluid'' people may also stem from a similar cognitive mechanism.

This is an unconscious practice of ``strategic essentialism'': I strategically occupy the category of ``male'' imposed on me by my opponents to deconstruct their arguments from within. This demonstrates an understanding of identity as a tool to be used, a position to be occupied, rather than a fixed, inner truth. It ultimately embodies the core idea of post-structuralism: ``the subject does not pre-exist discourse, but is constituted within discourse,'' and can therefore be strategically reconfigured and reused according to context and need.

\subsection{Towards a Practice of Queer Life}\label{subsec:towards-a-practice-of-queer-life}

Through the theoretical analysis of a series of my personal experiences, we have painted a picture of the generation of gender identity: it is interpellated by ideology, disciplined by power, constituted through performativity, and navigated with the help of contradictory pharmakon-like tools. The conclusion is that gender identity is not a stable essence, but a contingent, socially mediated, and ongoing process. The fluidity of gender is not ``indecisiveness,'' but a politically conscious practice of refusing to be fixed by any single category. It ultimately embodies the idea that ``the subject does not pre-exist discourse, but is constituted within discourse,'' and can therefore be strategically reconfigured as needed.

Thus, what this autoethnography describes is not a journey of ``discovering'' a true self, but a journey of learning how to deconstruct and use identity. From the body disciplined by ``red shoes'' in childhood to the strategic deployment of a ``male'' identity in online debates as an adult, this is a process of transforming from an object shaped by power to a subject capable of using discourse and identity for resistance. The conclusion is that gender identity is not a stable essence, but a contingent, socially mediated, and ongoing process of becoming.

Queer theory is therefore not just an academic discipline, but a crucial set of life tools. The most liberating response to the gendered power diffused throughout society may be continuously self-reflecting on how our own identities are shaped by power; refusing the temptation to find a single ``true'' gender identity, and instead embracing the complexity, contradiction, and fluidity of life experience; and learning to performatively and strategically use identity categories to challenge power, rather than letting them define us. The ultimate goal is not, as we usually think, to ``discover'' who we are, but to actively refuse the person that power tries to shape us into. This is a never-ending, generative self-creation.