\subsection[“Assigned Sex/Gender at Birth” Lacks Inclusivity for Intersex People]{``Assigned Sex/Gender at Birth'' Lacks Inclusivity for Intersex People}\label{subsec:assigned-sex-gender-at-birth-lacks-inclusivity-for-intersex-people}

Let's start with a basic thought experiment. There is an intersex child whose external genitalia more closely resemble a male. At birth, in a hospital with a high level of medical expertise, doctors noticed ambiguities in the external genitalia. After performing karyotype analysis and PCR, they ``assigned'' the child as female and advised the parents to consider genital reconstruction surgery when the child was older. The parents, being highly irresponsible and finding it troublesome, abandoned the child. The child was then found by a couple with a lower level of education and was raised as a boy. Theoretically, this child's ASAB/AGAB is female, but this assignment has no bearing on the child's entire life. If this child grows up and identifies as female, according to current mainstream definitions, this child would be a cisgender woman. This is clearly absurd, as their experience is vastly different from that of a cisgender woman.

This illustrates two points:

\begin{enumerate}
    \item The method of sex assignment is artificial. Although visual inspection of external genitalia is the common method, theoretically, karyotype or PCR sequencing could be used.
    \item Later ``sex assignment'' can potentially override the assignment at birth, becoming the main gender used to raise the child.
\end{enumerate}

Although this thought experiment is extreme, it involves highly skilled doctors, irresponsible biological parents, and less educated adoptive parents. There are less extreme but more common cases in reality. In the following examples, ASAB/AGAB had some impact on the child's early development, but subsequent experiences were equally or more important.

Many intersex individuals undergo medical ``reassignment'' at some point during childhood or adolescence.2 Some parents with strong stereotypes may immediately make a 180-degree turn in their raising methods. For example, I read a story by a patient with Persistent Müllerian Duct Syndrome (PMDS). After being reassigned as female, their parents immediately bought them dresses and bras, told them to pay attention to their appearance, deportment, and posture to avoid indecent exposure, and forbade them from playing with ``boys' toys'' (though the good news is that the author self-identifies as female in adulthood). This would not happen in the lives of typical ``phenotypic males'' or ``phenotypic females'' (even if they identify as transgender). For them, their phenotypic sex and assigned sex are completely consistent (it's hard for them not to be, barring extreme medical errors) and will not be ``reassigned'' during their lives.

For intersex people, labels such as ``AGAB to Identify'' (e.g., MtF) cannot fully encapsulate their experiences of interacting with and resisting externally imposed gender roles. This also differs from the experiences of transgender individuals with ``typical phenotypic sex'' who share the same label (e.g., a typical MtF).

Therefore, ASAB/AGAB erases the unique experiences of intersex people. It attempts to categorise the patterns of a child's social gendering into two typical phenotypic sexes (most medical systems only offer male and female options for sex assignment), ignoring the experiences of children whose gender roles and expectations undergo drastic shifts during their development. Consequently, the so-called ASAB/AGAB replicates the power framework it intended to oppose using a new discursive system; it is, in fact, gender binary and ``typical sex hegemonism'' (a term I coined).

Furthermore, when we interact with others daily, no one asks them to show identification to see their legal ASAB/AGAB. We merely judge their sex or gender based on their physical and social appearance. In other words, we are constantly performing such ``reassignment'' on others. Moreover, to some extent, this ``daily assignment'' we perform also influences the gender experiences of others (especially children). Both medical ``reassignment'' and our ``daily assignment'' can lead to unique experiences for individuals that cannot be encapsulated by ASAB/AGAB. Indeed, this ongoing ``daily assignment,'' based on perceived social cues, can be seen as an integral part of the broader phenomenon often described as gender performativity, where gender is constituted through repeated social interactions and interpretations.

\subsection{Philosophical Critique of ``Mechanistic Idealism''}\label{subsec:philosophical-critique-of-``mechanistic-idealism''}

Philosophically, ASAB/AGAB has a rather bizarre and, for me, incomprehensible philosophical basis.

I believe it is primarily idealistic because it denies the role of phenotypic sex, or the phenotypic basis (material level), in the child's social gendering process, reducing it instead to an ``assignment'' -- in other words, a specific social activity (mental level). (Although phenotypic sex is also a socially constructed concept, it is relatively more ``material'' compared to ASAB/AGAB, encompassing many material phenotypic characteristics, even if these characteristics are organised into the concept of ``phenotypic sex'' through human mental activities.)

This view might have some reasonable aspects, as the ways of social gendering are diverse and cannot be fully represented by phenotypic sex (though it remains a very influential factor, as guardians cannot invent a gender role out of thin air and apply it to a child). I can understand what it intends to convey: transgender people are not denying their phenotypic sex but rather the externally imposed gender roles. The concern behind this viewpoint may be valid.

However, it further attempts to use an initial mental act to determine everything, bestowing upon it an absolute, ontological priority, and on this basis, it further negates all subsequent mental activities. It treats the initial mental act (ASAB/AGAB) as a static, unchanging, and ultimately decisive factor, disregarding the impact of all subsequent ``mental activities'' (such as medical and daily ``reassignment'') on individual experiences. It places all subsequent mental activities under the absolute dominion of this ``initial mental act.'' It reduces a complex, dynamic, lifelong process of social gendering and the individual's reactions and interactions with this process to a static, single-point, initial ``label.''

This philosophical view is not comprehensible yet profoundly striking. I name it ``mechanistic idealism.'' To my knowledge, many idealistic philosophies (such as Hegel's absolute idealism, Berkeley's subjective idealism, etc.) incorporate change, development, dialectical movement, or agency. Few idealistic philosophies advocate for mechanism and dogmatism. Therefore, in my view, this is a rather atypical and anomalous philosophy, possessing a certain high degree of ``intellectual originality.''

\subsection{New Definitions: Balancing Accuracy and Operability}\label{subsec:new-definitions:-balancing-accuracy-and-operability}

Under my definition, transgender is essentially about the ego feeling discomfort with and resisting externally imposed ``social gendering'' patterns. The resistance can manifest in myriad ways, from overt political activism to deeply personal, internal acts of self-definition and emotional negotiation.

ASAB/AGAB itself is just a part of the ``social gendering pattern.'' The source of oppression is an externally imposed, artificially constructed social norm, not the word written on a birth certificate. This process is dynamic and diverse, rather than fixed and monolithic.

Under this definition, being transgender is not a passive, objectively existing state, but an individual's non-conformity with and resistance to externally imposed rigid social norms, an active act of self-empowerment. It begins with the internal refusal to accept imposed norms, regardless of whether this leads to external action. This definition mentions the specificity of gender issues under the current social construction and can attract broader support from those resisting oppression beyond gender issues by relying on a narrative of resistance.

However, this definition is too abstract and lacks operability in daily life. Nobody's social gendering pattern can be completely the same with others. What constitutes ``discomfort''? What constitutes ``resistance''? It is fundamentally undefinable and inapplicable.

\subsubsection{``Phenotypic Sex'' with a More Inclusive Definition}

From the perspective of ``All models are wrong, but some are useful,'' I believe ``phenotypic sex'' is a criterion that can better balance operability and accuracy. Phenotypic sex is the primary factor determining how we are socially gendered; the way others perceive us depends on this material existence (at least under current social norms), and based on this, they further actively impose socialization patterns on us, including how elders raise us and how others interact with us, for both ``typical phenotypic sex'' individuals and intersex people.

Phenotypic sex includes not only male and female (typical phenotypic sexes) but also intersex people. Intersex people can even be further subdivided into different groups, all of whom may have experiences different from typical males and females. We should acknowledge XtM, XtF, XtX cases (where the first X represents atypical phenotypic sex), and can even extend this to, for example, KtM (Klinefelter syndrome identifying as male), TtF (Turner syndrome identifying as female), 5tQ (5-alpha-reductase deficiency identifying as queer), etc. (Of course, specific symbols can be negotiated and adjusted). At the same time, phenotypic sex also avoids the problem of ``social gendering patterns'' being too abstract and inoperable, thus being a good standard that combines accuracy, inclusivity, and operability.

\subsubsection{All Formally Assigned Genders (AFAG)}

Another useful ``model'' is an extension of ASAB/AGAB, incorporating all formally assigned genders:

Transgender individuals are those whose gender identity does not align with at least one of their formally assigned genders (AFAG). ``Formal'' refers to designations made in legal or medical documents and the gender role \textit{de facto} used by guardians.

The value of introducing the concept of ``\textit{de facto} use by guardians'' lies in its pointing to an important social reality -- the individual's early social gendering experiences in the absence of formal documentation or when documentation conflicts with upbringing practices, like the one in our initial thought experiment, where the child was not formally assigned male but was raised as male. It can also include stateless persons without documents. Even if it is challenging to quantify perfectly, acknowledging its existence is itself progress.

``At least one'' emphasises that if any formal assignment in an individual's life does not align with their gender identity (meaning a social gendering pattern was once imposed in this way), they can be classified as transgender.

Although this definition omits ``daily assignment,'' which we previously suggested could also create unique social gendering experiences, omitting a factor with relatively less impact is reasonable for the sake of operability.

Under this definition, an individual's gender identity needs only to be inconsistent with any one of their AFAG during their lifetime to meet the definition of transgender.

\subsubsection{Comparison of Definitions}

The expanded definition of phenotypic sex emphasises the material reference point society uses when socially gendering an individual. Its advantage is the high inclusivity, accommodating any phenotypic sex, such as the many different types of intersex people we mentioned. Moreover, it points out that the social gendering process is mainly based on phenotypic characteristics, reminding us to reflect on this practice. The disadvantage is that it might be misinterpreted by some as a degree of biological determinism if they deliberately ignore that ``this is just a model.''

AFAB, All Formally Assigned Genders, emphasises that social gendering patterns are externally imposed and also highlights the continuity of this process throughout an individual's life. However, its disadvantage is that it might be too simplistic for different types of intersex people. If an individual is required to fill out their AFAB on a document, should they list all of them? What if there is not enough space? Furthermore, ``\textit{de facto} used by guardians'' lacks empirical evidence and strict quantification or criteria. For example, does it count if parents dress their child in different gendered clothing on a whim one afternoon? In the future, we might develop more detailed sociological or psychological methods to assess the impact of such ``\textit{de facto} use.''

Which one to choose? Frankly, I do not know. Because both models are simplifications and generalisations of objective reality, not reality itself, they are only ``useful'' or not. The final adoption depends on consultation among all relevant parties, including the transgender community, medical professionals, psychologists, sociologists, policymakers, etc. I am merely conducting purely theoretical deductions and philosophical critiques, aiming to provide new perspectives and possibilities for discussions in related fields. Specific applications and choices still await interdisciplinary cooperation and practical testing, and I will not make any irresponsible predictions about this.