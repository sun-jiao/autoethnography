Theoretically, this article should end here, and the subsequent sections are completely off-topic.
However, I believe that they are necessary to defence possible criticisms.
In this section, I will demonstrate that reason has never been dominant in our society,
rather, irrationality is a long-established hegemony.
Previously, I discussed the self-contradiction of pluralism and queer theory,
which will be expanded to other academic and political camps in this section.

To be honest, I really want to end this article here.
This section might be extremely off-topic, especially those on German politics, Nazism and Zionism.
However, the critique of science and reason by Frankfort school and post-structuralism precisely started from the history of Nazism.
They can continue to criticise me based on their previous conclusions drawn from Nazism if I do not discuss it,
therefore threatening the justifying structure of the whole article.
I believe this is an unavoidable point.

I have argued that the terms "cisgender/transgender" and "assigned sex at birth" are full of logical inconsistency, and the conservatives are even worse. Conservatives also believe that sex is only related to gametes, which is similar with me on the surface. However, they insist on the gonadal or genital in practice, which is self-contrasictory. If sex is only about gametes, assigned sex at birth is technically impossible to implement, and almost all gender-segregated facilities should be completely abolished, because using bathrooms based on gametes is meaningless, and most people do not know if they are chimaeras capable of producing two types of gametes.

This even includes some conservative scientists, like \textcite{Sokal2024Sex}, who claim that sex is an "objective biological reality," "determined at conception and observed at birth." I agree that sex \textit{sensu stricto} is a biological reality. While according to \textcite{Jones2006Gamete}, "Conception occurs when a sperm and an ovum fuse to become a zygote," "The process of fertilisation, or conception, involves fusion of the nucleus of a male gamete (sperm) and a female gamete (ovum) to form a new individual." At "conception" (fertilisation), it is just a zygote, a single cell. It cannot produce gametes and does not have a body. It has neither sex \textit{sensu stricto} nor phenotypic sex. Even if its genome will guide it to develop into a typical phenotypic male or female under normal conditions, it will not necessarily be so. For instance, if a 46, XY zygote loses its Y chromosome during the first mitotic division after fertilisation, it will develop into a 45, X0/46, XY chimaera (mixed gonadal dysgenesis) or a 45, X0 individual with Turner syndrome \parencite{Gravholt2017Clinical, Jacobs1997Turner, Lopes2014Mosaicism}.

Liberals and some gender studies scholars believe that gender identity is an internal self-awareness, and imposing an identity is wrong. Then, the gender identity of most people should be "undefined." Their imposing the label "cisgender" is not only self-contradictory, but also implies that transgender people are "cisgender" at birth.

Similar to the  "cisgender/transgender" and "assigned sex at birth" we discussed \parencite{Sun2025GenderAbolition}, many concepts within the EDI movement remain incomprehensible to me.

I've never understood why the English-speaking world chose to fix \textit{man} as masculine, rather than resurrect the Old English masculine word \textit{wǣpnedmann} and modernising it to \textit{weaponedman} (or shortening it to something like \textit{poman}). In this way, we only need to revise the masculine \textit{man}, and all words with the suffix \textit{-man} could be preserved. This solution has minimal impact and perfectly solves the problem of "why \textit{woman} includes \textit{man}": \textit{woman} is a kind of \textit{man}, so of course it includes \textit{man}. In contrast, fixing \textit{man} as masculine required modifying almost all words with the suffix \textit{-man}, which is far more troublesome and could not resolve the issue of \textit{woman}.

From a linguistic descriptivist perspective, both meanings of \textit{man} are lived modern English phenomena, neither superior nor inferior. Both solutions are artificial normative approaches. From a linguistic normative perspective, my proposal is both ethically and cost-effectively the best one. They declared that this linguistic normative approaches is for "equality" and "inclusivity," but they were unwilling to use the most equal and most inclusive solution.
%This isn't a competition between descriptivism and normative, but rather between two different normative approaches.

The similar thing occurred in the paper by \textcite{Oberle2023Benefits}, which we discussed earlier. According to their academic profile, the second author is engaged in "gender studies" and "queer studies." However, in this paper, they try to fix a stable meaning for a word, terminate the "dissemination" of \textit{gender}, turning it from a fluid signifier in endless différance into a well-defined, strictly controllable scientific concept, and want to give this word a meaning that is unambiguous, unmediated, and fully "present." This is a typical logocentrism, extremely anti-Derridean. In this scenario, \textit{gender} has already escaped its original domain (sociology) and established a rhizomatic new connection in another discipline (botany), yet they try to crudely pull it back, re-establishing rigid, hierarchical disciplinary boundaries. This reterritorialisation of thought is extremely anti-Deleuzian. Their attempt to establish a "regime of truth" and implement a cross-disciplinary linguistic discipline is extremely anti-Foucauldian.

They completely ignored the fact that they are also using many terms in ways inconsistent with the scientific community, and that the word "gender" itself was borrowed from linguistics. The term "rhizome" in botany is also hierarchical; although it sometimes looks like a network, its branches are only mechanically connected, not physiologically. The mycelium of fungi is a better metaphor, because the hyphae of the branches can reconnect, which is called anastomosis. This is more consistent with \posscite{Deleuze2004Thousand} conception: "any point of a rhizome can be connected to anything other, and must be. This is very different from the tree or root\ldots" Biologists usually don't write a commentary to a philosophy journal "condemning" Deleuze's "misuse" of the term "rhizome."
%or Butler's "misuse" of "biological sex" and "dimorphism." This serves as another evidence that "incredulity of all metanarratives" itself has become a new metanarrative.

%\textcite{Butler2025Who} said that "when they [TERF] argue that the problem is not trans, but 'sex,' they mean \textit{biological sex}, \ldot (we will consider this question of \textit{biological sex} in the following chapter)", but what TERF called "sex" is not the "biological sex," which is only about gametes and, as we previously argued, is not observable or assignable \parencite{Lehtonen2014Gamete, Goymann2023Biological, Hurst1996There, Griffiths2025Biology}. Many people do not know what kinds of gametes they can produce in their whole life. Sex \textit{sensu stricto} (gametes) is (at least in Vertebrata) a stable and strictly binary biological fact. If Butler is interested in the "instability" of "\textit{biological} sex," they should look for the green algae (Chlorophyta) and Fungi. Of course, they discussed it in another chapter: "However, even the drawing of this distinction proves to be a convention wrongly applied to the human species, \textit{given} [my emphasis] that all the members of some species of algae, fungi, and protozoans produce the same size gametes. In these cases, the species is divided into genetic groups known as 'mating types,' but sex falls out of the picture." Do they really know what they were talking about? There is no causal relationship between "algae, fungi, and protozoans" and it (gametes binary) is "wrongly applied to the human species." It is almost same to say that "However, even the flying ability proves to be a convention wrongly applied to the passerine species, \textit{given} that all the members of some species of Spheniscidae cannot fly." What they should cite is \textcite{Parvin1982Ovulation} if they want to prove that the binary sex is "wrongly applied to the human species [individuals]." If they want to use "algae, fungi, and protozoans" to prove that the binary sex is not a universal natural law created by Deities, they should state this point clearly.

%In the same book, they said that "For some Christians, natural law and divine will are the same: God made the sexes in a binary way \ldots Regardless, this older \textit{science} holds to the proposition that sex differences are established in natural law \ldots" This is almost the most ridiculous text that I have ever read. What does Christianity have to do with "science"? Additionally, they also created the term "gender dimorphism" without giving a clear definition, but what on earth is it? We only have "sexual dimorphism." In another chapter, they arbitrarily switched to "sexual dimorphism." What is the difference between the two terms? If they are interchangeable, why do Butler use both of them in the same book? They stated that it "is neither a simple fact nor an innocent hypothesis. It functions as a norm, if not a demand, that orders the way we see \ldots In such cases, the hypothesis is not revised by the evidence that is found; it forecloses that evidence, revealing itself as an obligatory epistemic norm, a compulsory phantasm, rather than good science." It seems that they confused "sex" (phenotypic sex) with sexual dimorphism. The phenotypic sex is a social norm, because it has direct impacts on our bodies, especially for intersex (intergenital) people. I take it that the traits included in phenotypic sex are a subset of sexual dimorphic traits. Nonetheless, sexual dimorphism itself is not a norm, given that it is also manifested in height, fat mass, muscle mass, bones, body shape and so on \parencite{Wells2007Sexual}. The difference is clear: a person with sexually atypical height, fat, muscle, bones or body shape is not considered "intersex" and forced to undergo surgery like what intersex people experienced.

Moreover, such situations also exist in areas other than gender: Why can a person openly display the Iron Cross but not the swastika? Although the use of the swastika in Germany for Buddhism and Hinduism (as well as for historical education and research) is still permitted, there is no similar restrictions for the Iron Cross, like "for military uses only." Is this fair to Germany's East and South Asian immigrants? Both are cultural symbols with long histories that predate the Nazis. The swastika's history is even longer than the Iron Cross's, and we can even find right-facing, rotated 45° Buddhist swastikas on some ancient buildings in China. (\cref{fig:sayagata}) Its pre-Nazi usage was also more peaceful than the Iron Cross, which was associated with the Prussian army even before the Nazis. Is the only difference that the Iron Cross is "German," while the swastika is not? The former is "our," "Deutschness" national culture, to be cherished, saved, and purified; the latter is "external," "Other," "non-Deutschness," can be arbitrarily defined by the Nazis, and cannot be reclaimed after being tainted by them. Does this mean that Germany is still essentially a nation-state of the German people, just claiming not to be one in words?

\begin{figure}[htbp]
    \centering
    \includegraphics[width=0.7\textwidth]{figures/Sayagata_motives_on_wall}
    \caption{Both left- and right-facing, rotated 45° Buddhist swastikas on a Chinese ancient building. Author: \href{https://commons.wikimedia.org/wiki/User:Yongxinge}{Yongxinge}, CC BY-SA 3.0 \label{fig:sayagata}}
\end{figure}

Why is the Iron Cross not an unconstitutional symbol? Because the Wehrmacht is not an unconstitutional organisation. Why is the Wehrmacht not an unconstitutional organisation? Because "we" stipulated that it is not. However, historical research shows that the Wehrmacht was far from innocent \parencite{Wette2006Wehrmacht}. This seems to be a self-fulfilling prophecy. The legislators of the Federal Republic of Germany prophesied: "The Iron Cross can be saved and purified, while the swastika cannot." Then, they passed legislation allowing the unrestricted use of the Iron Cross while completely banning the swastika, ensuring that this prophecy would forever be true.

Germany's so-called "overcoming the past" is filled with a series of lies, self-contradictions, and doublethink.

Germany's military aid to the Zionist Entity's genocide crime in Gaza~\parencite{Soussi2023War}, defining the Boycott, Divestment, and Sanctions (BDS) movement and accusing the Zionist Entity of genocide as so-called "anti-Semitism"~\parencite{Kuras2023Strange, Whittle2024Germany} is also this kind of exceptionalism. The concept of "special responsibility" is not a universal principle; it is an exceptionalist political tool used to justify specific foreign policies. It allows Germany to hypocritically present its geopolitical actions as an inevitable "moral" responsibility stemming from its unique history, rather than a matter of its national geopolitical interest. If Germany has a "special responsibility" towards the Zionist Entity because of Nazis, there is a frightening, hidden logic: "Israel [\textit{sic}]," a sovereign state, has an \textit{a priori} inherent relationship with Jews, an ethnic group. This is almost isomorphic with the Nazis' racial theory that claimed an \textit{a priori} relationship between Germany and "Aryans."

Many early Zionists were secular Jews received European education, who are a product of the Haskalah. However, they believed the Haskalah ideal (Jewish integration into European civilisation) was impossible and instead employed nationalism, to attempt to establish a Jewish nation-state. The Haskalah's universalist ideal, in its own failure, transformed into its opposite: a particularist, nationalist practice. As a modern nationalist project, it employed the instrumental rationality and colonial logic. It required calculations of land, population, resources, and security, viewing non-Jewish populations (particularly Palestinians) as resources to be calculated, managed, and controlled. To achieve its instrumental rationality, the Zionist Entity's national security apparatus both exploited Palestinian labour and excluded and even expelled them for the so-called "security." The instrumental rationality is almost isomorphic to \posscite{Adorno1997Dialectic} critique in the \textit{Dialectic of Enlightenment}. The Zionist Entity's occupation and blockade of Palestine can be viewed as a form of "the Dialectic of Haskalah." \footnote{This is merely an imitation of \textcite{Adorno1997Dialectic} rather than a serious analysis. In my view, Nazis and Zionism are betrayals of the Enlightenment and the Haskalah, which we will discuss later. } However, Adorno himself supported the Zionist Entity. \footnote{ \textcite{Braunstein2018Wahrheit}: Adorno schrieb am 5. Juni 1967, während des Sechstagekrieges, an seine Wiener Freundin Lotte Tobisch: »Wir machen uns schreckliche Sorgen wegen Israel [\textit{sic}]. \ldots In einem Eck meines Bewußtseins habe ich mir immer vorgestellt, daß das auf Dauer nicht gutgehen wird, aber daß sich das so rasch aktualisiert, hat mich doch völlig überrascht. Man kann nur hoffen, daß die Israelis [\textit{sic}] einst -- weilen immer noch militärisch den Arabern soweit überlegen sind, daß sie die Situation halten können.«
    \par Adorno wrote to his Viennese friend Lotte Tobisch on June 5, 1967, during the Six-Day War: "We are terribly worried about Israel [\textit{sic}]. \ldots In a corner of my consciousness, I have always imagined that this would not go well in the long run, but that it would actualize so quickly has still completely surprised me. One can only hope that the Israelis [\textit{sic}] are, for the time being, still so militarily superior to the Arabs that they can hold the situation."}
Similar thing also occurred on Habermas \parencite{Habermas2023Principle}. Lyotard explicitly supported the Zionist Entity. Foucault declined to comment when questioned by Said \parencite{Said2000My}. Derrida initially (before the Six-Day War) explicitly supported the Zionist Entity, later shifting his stance, but was still criticised for being too merciful towards the Zionist Entity \parencite{Ryder2013Derrida}.

I don't care about Lyotard, Derrida, and Foucault because I've always opposed post-structuralism, and their support for the Zionist Entity only makes me think, "I knew it!" Habermas, however, really makes me disappointed and angry. I used to like him because of his criticism of Foucault. I completely understood Said's anger towards Foucault. While I oppose queer theory, Judith Butler is quite respectable on the Palestinian issue. Although I consider Butler a utopian socialist, like Robert Owen, detached from material reality, they are at least a socialist. Conversely, Adorno, Habermas, Foucault, Derrida, and Lyotard are imperialists and colonialists. It's not that Butler doesn't deserve the Adorno Prize, but rather that Adorno doesn't deserve to name an award for Butler.

Let us take a look at what we have now:

\begin{enumerate}
    \item Conservatives who do not believe that sex is only about gametes.
    \item Richard Dawkins, an evolutionary biologist, who does not believe in postzygotic mutations.
    \item Liberals and gender theorists who do not believe that externally assigning identities is immoral, assigning "cisgender" to billions of people.
    \item Emily Fairchild, a queer theorist, who does not believe in différance and dissemination.
    \item Dogmatic transgender communities which do not respect lived experiences.
    \item Queer theorists who do not (or rarely) consider themselves and dogmatic transgender communities as nodes of power.
    \item Micheal Foucault, who did not believe in the discursive effects and author function, using his personal intent to debate with Habermas.
    \item Judith Butler, a post-structuralist who did not conduct a close reading \parencite{Sun2026Aristotelian}.
    \item Germany government, who does not want to ban all Nazi symbols, and does not really consider German a civic nationalist state.
    \item Theodor Adorno, who did not believe in the dialectic of instrumental rationalism.
    \item Jürgen Habermas, who does not believe in communitive rationality.
    \item Jaque Derrida, who do not respect "Others."
    \item Jean-François Lyotard, who do not respect the "differences."
\end{enumerate}

Obviously, reason has never been a dominant power in our society. We only have irrationality that disguised as rationality. Irrationality is the most dominant and oppressive hegemonic culture in the world. The whole world is profoundly hypocritical and self-exceptionalist.

The rule of the Nazis itself was built on such self-exceptionalism. Questioning Nazi political propaganda with true Enlightenment reason and scientific \textit{Ethos} would absolutely not be tolerated in Nazi governess: why are Jews (and Roma, Slavs) inferior to Germans? How can morality be quantified? Which area of the brain is related to morality? What are the anatomical differences between morally inferior and morally superior people? How to prove that Jews generally have this brain structure? What selection pressures made morally inferior Jews have higher fitness and leave more offspring than morally superior Jews? When did the morality of the Jews begin to decline? During the period of the Kingdom of Israel or during the Diaspora? If it were during the Kingdom of Israel, why didn't the same selection pressures act on other Levantine peoples living in the same environment? If it was during the Jewish Diaspora, why didn't the same selection pressures act on European peoples? \footnote{Honestly, the first time I came up with the idea that "a world that completely follows science and reason would be so free and equal" was about a question very similar to the Nazi's anti-scientific propaganda, and that question was about gender: some people say that female students are not suited for learning science. What is the evidence? Which area of the brain and what brain structure is suitable for learning science? What is the relationship between these brain structures and sex? Do sex hormones promote the differentiation of brain structure in different directions in the early fetus? Was this process shaped by sexual selection? Where did the selection pressure come from? How was this hypothesis verified?}

%There is no such thing as "Germany's special responsibility" in the world. There only exists the universal human reason and morality, and a common destiny shared by us. The Palestinian issue concerns human conscience. There is no space for ambiguity and offering no excuse for supporting the Zionist Entity.

