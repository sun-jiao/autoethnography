The first thing I needed to consider was whether to view ``gender identity'' as internal or external. I almost immediately chose the latter, as the former contradicts some fundamental principles of evolutionary biology and neurology.

I recognise that from a pure physicalist perspective, ``gender identity'' is just a physical state of one's brain, involving the connection patterns of neurons. The human brains could be initialised in this state as long as there are enough genes about it. However, this claim is too \textit{ad hoc}. As an analogy, infants have the sucking reflex without knowing what a ``breast'' is. Innately encoding a ``gender identity'' is more useless than encoding a ``breast.'' The latter would allow infants to recognise breasts innately, distinguish what should and should not suck, and avoid sucking on harmful things. Therefore, an innate ``gender identity'' is less probable than an innate concept of breasts. Moreover, using so many genes to encode such a metaphysical, abstract concept is biologically inefficient and impractical, without reasonable selection pressure. In contrast, evolving a brain with high neuroplasticity to learn abstract concepts is far more reasonable. Claiming that humans have an innate abstract, socially constructed concept is too anti-Darwinist, almost being a regression to Plato's theory of Forms.

Almost all neuroscience research claiming an ``innate gender identity'' involves circular reasoning. They used the brains of individuals who conform to assigned gender roles (so-called ``cisgender'') as ``male brains''/``female brains.'' However, without human society and culture, this would just be a sexual dimorphic neurological trait, but not ``male'' or ``female.'' If it could be considered ``male'' or ``female,'' tall/short individuals would also be considered ``male height'' and ``female height,'' so would fat/thin individuals, because they are also sexual dimorphic characteristics \parencite{Wells2007Sexual}.

Discomfort with or preference for a specific body morphology may indeed stem from innate body schema. Some neuroscience studies revealed that some transgender men have innate phantom penis sensations \parencite{Ramachandran2008Phantom}. However, according to the sex/gender division, a preference for a specific body morphology should be considered an identity with \textit{sex}, not with \textit{gender} (gender identity). Moreover, the reason they are interpreted within the framework of ``gender'' is still a product of society and culture. The sex \textit{sensu stricto} is strictly confined to the gametes in biology \parencite{Lehtonen2014Gamete, Goymann2023Biological, Hurst1996There, Griffiths2025Biology}, and its determination mechanisms are homologous among the entire vertebrate lineage \parencite{Bellott2017Avian, Graves2010Homologies, Smith2007Bird}. Nonetheless, different groups of organisms have developed different sex chromosomes, sex organs, and sexual dimorphism in the evolutionary history. These characteristics are dynamically shaped by sexual selection.

However, the phenotypic sex (so-called ``biological sex'') is just a set of sexual dimorphic traits that are artificially selected. The boundaries are very blurry, and the standards are inconsistent and unstable. For example, breasts are sometimes considered part of phenotypic sex (e.g., in gender-affirming surgery), but not when determining intersex conditions. Intersex individuals are judged only by their genitals; individuals with gynaecomastia are not considered intersex. For individuals with gynaecomastia who self-identify as male, medical treatment for their breasts is also to ``make their body consistent with their gender identity.'' Why isn't this surgery a gender-affirming surgery? It is shaped by the sociocultural construct that why we understand body schema incongruence about this part as ``gender dysphoria,'' while another part is ``body dysmorphic disorder'' or ``body integrity identity disorder.''

Therefore, ``I feel my body should have a penis'' or ``I feel an alienation from my breasts'' are innate, physical, neurological experiences. ``I am a girl/boy/man/woman/non-binary person'' is social, cultural, and political. For these transgender individuals with innate body schema inconsistency, their ``gender identity'' might be a socially constructed conceptual chimaera -- it connects an innate, pre-linguistic feeling about the body with a purely socially constructed category of identity. Then it claims this connection is ``natural,'' using the innate feeling to legitimise a social construct, claiming that the former directly and necessarily leads to the latter.

%If we were to judge intersex conditions in humans as we do in birds, based on external genitalia alone, then most birds (except for a few groups like ducks, whose males have a penis) have the same external genitalia (a cloaca), so they should not have the concept of ``intersex'' at all. Some ornithologists classify birds as intersex based on mixed plumage colour \parencite{Choudhary2024Intersex}, but feathers' relationship to reproduction is less direct than that of mammalian breasts (used for lactation). Their function is more like our body hair, Adam's apple, beard, and facial features, which are purely for attracting mates and not physiologically related to reproduction.

%If birds' mixed plumage can be used to classify intersex, then individuals with gynaecomastia should also be intersex. The neutral neurological traits related to an individual's personality, behaviour patterns, and body schema, which make them more likely to form a specific gender identity when interacting with current social gender norms, are also a form of sexual dimorphism and have a significant statistical correlation with gametes (sex \textit{sensu stricto}). Thus, theoretically, they should also be part of ``phenotypic sex.''

%If we apply the standard for birds to ourselves, an individual whose certain sexually dimorphic traits differ from the statistical norm for their gametic sex could be considered intersex. Then, transgender people whose identity stems from innate body schema incongruence are all intersexes, and in fact, almost everyone is intersex, such as a ``female'' with a lot of body hair or a short ``male.'' The ``standard'' and ``typical'' ``male'' and ``female'' (endosex) would be rare exceptions.

%Also, ``conversion therapy cannot change it'' is not a valid argument for ``gender identity is innate.'' Let's conduct a thought experiment: try using ``conversion therapy'' to change a person's native language. If we apply this logic honestly, consistently, and without compromise, we will inevitably conclude that ``native language is also innate.'' \footnote{I am not seriously suggesting a ``forced native language conversion''; this is clearly a violent, anti-human act.}

Without the intervention of society and culture, an individual would not choose words like ``male'' and ``female'' to describe themselves. Our social constructs have encoded these two words with meanings beyond their reproductive biological sense. Neurological features are not gender; they are simply personality traits, behavioural patterns, and body schemas. In social and cultural interactions, they may lead an individual to develop a sense of ``identity'' with specific gender constructs and gender roles. Still, they themselves are not ``gender.''
%This process occurs within the research domains of sociology and psychology. Using biological methods to study them is a serious category error, as ineffective as calculating the relativistic velocity of each car to study traffic flow on a road.