From a physicalist, reductionist perspective, ``gender identity'' is just a physical state of one's brain, involving the connection patterns of neurons. The human brains could be initialised in this state as long as there are enough genes about it. However, this claim is too \textit{ad hoc}. As an analogy, infants have the sucking reflex without knowing what a ``breast'' is. Innately encoding a ``gender identity'' is more useless than encoding a ``breast.'' The latter would allow infants to recognise breasts innately, distinguish what should and should not suck, and avoid sucking on harmful things. Therefore, an innate ``gender identity'' is less probable than an innate concept of breasts. Moreover, using so many genes to encode such a metaphysical, abstract concept is biologically inefficient and impractical, without reasonable selection pressure. In contrast, evolving a brain with high neuroplasticity to learn abstract concepts is far more reasonable. Claiming that humans have an innate abstract, socially constructed concept is too anti-Darwinist.

Many neuroscience research claiming an ``innate gender identity'' involves circular reasoning and explanatory gaps. They used the brains of individuals who conform to assigned gender roles (so-called ``cisgender'') as ``male brains''/``female brains.'' However, without human society and culture, this would just be a sexual dimorphic neurological trait, but not ``male'' or ``female.'' If it could be considered ``male'' or ``female,'' tall/short individuals would also be considered ``male height'' and ``female height,'' so would fat/thin individuals, because they are also sexual dimorphic characteristics \parencite{Wells2007Sexual}.

Discomfort with a specific body morphology starts from an early age, reported by some transgender individuals, may indeed stem from internal body representation, which was revealed by some neuroscience studies \parencite{Case2017Altered, Lin2014Neural, Ramachandran2008Phantom}. However, according to the sex/gender division, a preference for a specific body morphology should be considered an identity with \textit{sex} (``sex identity''), not with \textit{gender} (gender identity), which is a category error.

The reason they are interpreted within the framework of ``gender'' is still a product of society and culture. The sex \textit{sensu stricto} is strictly confined to the gametes in biology \parencite{Lehtonen2014Gamete, Goymann2023Biological, Hurst1996There, Griffiths2025Biology}, and its determination mechanisms are homologous among the entire vertebrate lineage \parencite{Bellott2017Avian, Graves2010Homologies, Smith2007Bird}. However, different clades of organisms have developed different sex chromosomes, sex organs, and sexual dimorphism in the evolutionary history. These characteristics are dynamically shaped by sexual selection. And the phenotypic sex (so-called ``biological sex'') is just a set of sexual dimorphic traits that are artificially selected from them. The boundaries are very blurry, and the standards are inconsistent and unstable. For example, breasts are sometimes considered part of phenotypic sex (e.g., in gender-affirming surgery), but not when determining intersex conditions. Intersex individuals are judged only by their genitals; individuals with gynaecomastia are not considered intersex. For individuals with gynaecomastia who self-identify as male, medical treatment for their breasts is also to ``make their body consistent with their gender identity,'' which therefore should also be considered a gender-affirming surgery. It is shaped by the sociocultural construct that why we understand body representation incongruence about this part as ``gender dysphoria,'' while another part is ``body dysmorphic disorder'' or ``body integrity identity disorder.''

A nature-nurture co-operation hypothesis is strongly supported by the aforementioned neurological studies. \textcite{Case2017Altered} suggest an innate multimodal body representation but acknowledge that culture and experience can shape it, making the ``innate'' vs. ``acquired'' line blurry. \textcite{Lin2014Neural} revealed that transgender individual's brains also showed high connectivity between the body representation areas and visual/auditory processing areas. They suggest this means transgender individuals are ``integrating massive visual and auditory cues to shape their body image.'' This strongly implies a learned, social component. The brain uses these learned social cues to interpret a vague, underlying neural discomfort into a specific, socially coherent narrative.

However, these studies, and similar neurological studies, mainly focus on ``typical,'' early-onset transgender individuals. In contrast, studies focus on ``atypical'' transgender or non-binary individuals are relatively rare. \textcite{Bonazzi2025Gender} demonstrate the significant relevance between autism and transgender identity. The neurological heterogeneity between body representation incongruence and autism implies that ``transgender'' is a pure socio-cultural/political concept. Some people's raw feeling is ``this body part isn't mine,'' some people might be ``I dislike the personality/temperament forced on me,'' and others might develop their transgender identity through personal history. The gender construct ties disparate causes together and interpret them as ``This means I am a girl/boy/man/woman/non-binary person.''

For these transgender individuals with innate body representation incongruence, their ``gender identity'' might be a socially constructed conceptual chimaera -- it connects an innate, pre-linguistic feeling about the body with a purely socially constructed category of identity. Then it claims this connection is ``natural,'' using the innate feeling to legitimise a social construct, claiming that the former directly and necessarily leads to the latter.

From a connectionist perspective, the human brain could be considered a biological neural network whose properties emerge from the connections between neurons. Our ``self'' is not a pre-programmed essence, but a product of being shaped by (``trained on'') the data it receives from the external environment (the ``training set'') via neuroplasticity. The social construct of gender, as a crucial part of the training set, is a mess of unrelated items (body parts, colours, objects, social roles) that have been nonsensically grouped together under a single label (applying wrong labels to the data). The gender identity of an individual -- both cisgender and transgender individuals -- is shaped by this cultural ``dataset.'' Our brain (as a neural network) tries to find a coherent pattern in this messy dataset of social labels. It creates a ``self'' that is perfectly adapted to the nonsensical rules of the gender system, but which is not a true or flexible representation of reality. It's a flawed model. Unlike an AI model, the human brain isn't a blank slate. It's ``pre-trained'' for millions of years in the evolutionary history, subsequently shaped by environment and one's whole life history. This is why individuals don't fit neatly into a single class like ``female'' or ``male.'' Because no one can be ``born this way'' and do not being imposed to the opposite gender's ``training set'' in their whole life. The brain is always learning and updating itself based on everything in our daily life. This explains why someone's gender identity can be fluid over a lifetime. \footnote{This is a metaphor about their structural and functional similarity, instead of claiming that there are mathematically literal ``activation function'' or ``backpropagation'' in our brains. }

%Neurological features are not gender; they are simply personality traits, behavioural patterns, and body representations. In social and cultural interactions, they may lead an individual to develop a sense of ``identity'' with specific gender constructs and gender roles. Still, they themselves are not ``gender.''

%If we were to judge intersex conditions in humans as we do in birds, based on external genitalia alone, then most birds (except for a few groups like ducks, whose males have a penis) have the same external genitalia (a cloaca), so they should not have the concept of ``intersex'' at all. Some ornithologists classify birds as intersex based on mixed plumage colour \parencite{Choudhary2024Intersex}, but feathers' relationship to reproduction is less direct than that of mammalian breasts (used for lactation). Their function is more like our body hair, Adam's apple, beard, and facial features, which are purely for attracting mates and not physiologically related to reproduction.

%If birds' mixed plumage can be used to classify intersex, then individuals with gynaecomastia should also be intersex. The neutral neurological traits related to an individual's personality, behaviour patterns, and body representation, which make them more likely to form a specific gender identity when interacting with current social gender norms, are also a form of sexual dimorphism and have a significant statistical correlation with gametes (sex \textit{sensu stricto}). Thus, theoretically, they should also be part of ``phenotypic sex.''

%If we apply the standard for birds to ourselves, an individual whose certain sexually dimorphic traits differ from the statistical norm for their gametic sex could be considered intersex. Then, transgender people whose identity stems from innate body representation incongruence are all intersexes, and in fact, almost everyone is intersex, such as a ``female'' with a lot of body hair or a short ``male.'' The ``standard'' and ``typical'' ``male'' and ``female'' (endosex) would be rare exceptions.

%Also, ``conversion therapy cannot change it'' is not a valid argument for ``gender identity is innate.'' Let's conduct a thought experiment: try using ``conversion therapy'' to change a person's native language. If we apply this logic honestly, consistently, and without compromise, we will inevitably conclude that ``native language is also innate.'' \footnote{I am not seriously suggesting a ``forced native language conversion''; this is clearly a violent, anti-human act.}

%Without the intervention of society and culture, an individual would not choose words like ``male'' and ``female'' to describe themselves. Our social constructs have encoded these two words with meanings beyond their reproductive biological sense. Neurological features are not gender; they are simply personality traits, behavioural patterns, and body schemas. In social and cultural interactions, they may lead an individual to develop a sense of ``identity'' with specific gender constructs and gender roles. Still, they themselves are not ``gender.''
%This process occurs within the research domains of sociology and psychology. Using biological methods to study them is a serious category error, as ineffective as calculating the relativistic velocity of each car to study traffic flow on a road.