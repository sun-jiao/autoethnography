In this section, I will demonstrate that the idea of an "innate," or "inherent" gender identity is groundless and directly contradicts many biological laws, multiple theories, especially ones about "male/female brain" and body representation, cannot explain "gender identity" satisfactorily, finally establishing the predictive coding theory as a biological and philosophical apex of gender identity.

Before the critique of gender identity, I will firstly illustrate that the definition of the so-called "biological sex" involves the fallacy of begging the question and contradicts the Occam's razor as a scientific redundancy. This argument is necessary for further discussion because many scientific models I will discuss repeat this fallacy when explaining gender identity. In this article, I use ``phenotypic sex'' as an alternative to the conventional term ``biological sex,'' because the academic biological definition of sex (sex \textit{sensu stricto}) is strictly confined to the types of gametes \parencite{Lehtonen2014Gamete, Goymann2023Biological, Hurst1996There, Griffiths2025Biology}, and its determination mechanisms are homologous among the entire vertebrate lineage \parencite{Bellott2017Avian, Graves2010Homologies, Smith2007Bird}. Different clades have developed different sex chromosomes, genitals, and sexual dimorphic characteristics in the evolutionary history. These characteristics are dynamically shaped by sexual selection. Genitals can change extremely fast in some groups, e.g., some ducks \parencite{Brennan2007Coevolution, Orbach2018Evolution}, which is not different from other sexual dimorphic traits. The phenotypic sex (so-called ``biological sex'') is merely a arbitrarily organised subset of sexual dimorphic traits. The range of it is unjustified. \textcite{Zieminska2022Toward} proposed the so-called "5 layers of sex," namely sex chromosomes, gonads, internal sex organs, external genitals, and secondary sex characteristics. From an evolutionary reductionist perspective, the sexual dimorphism of heights possesses no essential difference from the one of sex chromosomes, genitals, or secondary sex characteristics. The difference lies solely in the degree of overlap, which is high in height compared to the low degree of overlap in genitalia. This is a quantitative difference, not a fundamental one. They are all adaptive traits that promote the rate of reproduction success. Then why are tall "women" or short "men" not considered "intersex?" In which way do the 5 layers different from other sexual dimorphic traits? This is a typical instance of begging the question. As an instrumental proxy of sex \textit{sensu stricto}, its functions can be replaced by sexual dimorphism or some specific sexual dimorphic traits, therefore framing it as a scientific redundancy. Moreover, using the same term "sex" for gametes and an arbitrary subset of sexual dimorphic traits wrongly implies that they are more closely related than other sexual dimorphic traits. 

% The boundaries are very blurry, and the standards are inconsistent and unstable. For example, breasts are sometimes considered part of phenotypic sex (e.g., in gender-affirming surgery), but not when determining intersex conditions. Intersex individuals are judged only by their genitals; individuals with gynaecomastia are not considered intersex. For individuals with gynaecomastia who self-identify as male, medical treatment for their breasts is also to ``make their body consistent with their gender identity,'' which therefore should also be considered a gender-affirming surgery. It is shaped by the sociocultural construct that why we understand body representation incongruence about some parts as ``gender dysphoria,'' while other parts as ``body dysmorphic disorder'' or ``body integrity identity disorder.''

\textcite{APA2015Guidelines} defines "gender identity" as "a person's deeply felt, inherent sense of being a girl, woman, or female; a boy, a man, or male; a blend of male or female; or an alternative gender." \textcite{NHS2022Gender} defines it as "a way to describe a person's innate sense of their own gender, whether male, female, or non-binary." It is untenable that we have an innate feeling about "gender" because it is a social construct. From a physicalist and reductionist perspective, one's gender identity is the product of a specific physical state of their brain, involving the connection patterns of neurons. Theoretically, the human brains could be initialised in this state as long as there are enough genes about it. However, this claim is too \textit{ad hoc} because similar phenomena are not found elsewhere. Using so many genes to encode such a metaphysical, abstract concept is biologically inefficient and impractical, without reasonable selection pressure. In contrast, evolving a brain with high neuroplasticity to learn abstract concepts is far more reasonable.

% As an analogy, infants have the sucking reflex without knowing what a ``breast'' is. Innately encoding a ``gender identity'' is more useless than encoding a ``breast.'' The latter would allow infants to recognise breasts innately, distinguish what should and should not suck, and avoid sucking on harmful things. Therefore, an innate ``gender identity'' is less probable than an innate concept of breasts. Mo

%By definition, it clearly falls under the concept of "narrative identity" or "narrative self", specifically, "self-concept," which is defined as the "conscious beliefs about the self that are descriptive or evaluative" \parencite{Fanti2024Dual}. In contemporary psychology, cognitive science and philosophy of mind, there is a broad consensus that narrative identity is essentially a posteriori. It is formed through the integration and interpretation of personal life experiences and is deeply influenced by sociocultural frameworks \parencite{Fanti2024Dual}. Therefore, claiming that gender identity is a priori or innate creates a profound philosophical contradiction.

Many neuroscience research claiming an ``innate gender identity'' involves circular reasoning and explanatory gaps. They used the brains of so-called ``cisgender'' individuals as ``male brains''/``female brains.'' However, without human society and culture, this would just be a sexual dimorphic neurological trait, but not ``male'' or ``female.'' As I previously argued, if it could be considered ``male brain'' or ``female brain,'' tall/short individuals would also be considered ``male height'' and ``female height,'' because they are also sexual dimorphic characteristics \parencite{Wells2007Sexual}. Additionally, research shows that almost no one's brain is entirely "male" or entirely "female"; the vast majority of people have a mixture of "typical male" and "typical female" characteristics \parencite{Baxendale2025Brain, Joel2015Sex}.

Primatologist Frans de Waal claimed that ``primates are born with a gender identity'' because their behaviour does not conform to the statistical type for their phenotypic sex, such as the phenotypically female chimpanzee Donna, who occupies territory and fights with others~\parencite{DeWaal2022Different, Morin2022Frans}. However, nobody has ever asked ``Hi, Donna, what is your gender identity?'' De Waal's claim is effectively saying that gender identity is a behavioural characteristic of animals. Animals exhibit sexual dimorphic behaviour, while equating "aggression" with "male identity" is not only anthropomorphism, but also a logical leap. From an evolutionary perspective, "aggression" offers a survival advantage, but an abstract concept like "I am a male chimpanzee" would have been entirely unprofitable and highly energy-intensive for animals without language. This is a sexual dimorphism behaviour characteristic, its relationship with "gender identity" is unjustified. 

Some transgender individuals reports that they have a miscomfort with their body characteristics starts from an early age. This may indeed stem from internal body representation, which was revealed by some neuroscience studies \parencite{Case2017Altered, Lin2014Neural, Ramachandran2008Phantom}. However, according to the sex/gender division, a preference for a specific body morphology should be considered "(phenotypic) sex identity" or "body identity," not "gender identity" itself. This is a category error. The reason they are interpreted within the framework of ``gender'' is a product of the questionable definition of phenotypic sex. Why not a short "woman" who wants to be tall considered a part of her "gender identity?" 

A nature-nurture co-operation hypothesis is strongly supported by the aforementioned neurological studies. \textcite{Case2017Altered} suggest an innate multimodal body representation but acknowledge that culture and experience can shape it, making the ``innate'' vs. ``acquired'' line blurry. \textcite{Lin2014Neural} revealed that transgender individual's brains showed high connectivity between the body representation areas and visual/auditory processing areas. They suggest this means transgender individuals are ``integrating massive visual and auditory cues to shape their body image,'' and transgender people's ``distinct neural network of body representation can be coterminal to genetic constitution, developmental factors and learned experience in their life.'' This means that our brains use learned socio-cultural knowledge to interpret a vague, underlying discomfort into a specific, socially coherent narrative. Evolution may have encoded certain raw feelings about body representation, personality, or other traits. Nonetheless, these studies, as well as similar neurological studies, mainly focus on ``typical,'' early-onset transgender individuals. In contrast, studies focus on ``atypical'' transgender or non-binary individuals are relatively rare. \textcite{Bonazzi2025Gender} demonstrate the significant relevance between autism and transgender identity. The neurological heterogeneity between body representation incongruence and autism implies that ``transgender'' people do not share a biological essence.

According to the predictive coding framework, the human brain functions as a continuous prediction engine, leveraging past knowledge and internal models to efficiently anticipate sensory data and thereby minimise unexpected outcomes \parencite{Clark2013Whatever}. \textcite{Tacikowski2020Fluidity} demonstrated that the perceptual illusion of owning a phenotypic opposite-sex body causes a dynamic, robust, and automatic shift in gender identity, characterized by a more balanced subjective identification with both genders, updated implicit self-associations, and reduced gender-stereotypical beliefs regarding one's own personality. If an individual has a sensory processing variation (e.g., the breast feels ``alien''), this creates a prediction error. The brain can resolve this error by updating its ``self-model.'' If the social environment offers a category (``Transgender Man'') that explains this feeling, the brain adopts this identity to minimise the predictive error. Similarly, as demonstrated by \textcite{Clausen2021Action}, footstep sounds can modulate gender identity and the sense of self-group relation in cisgender participants. Therefore, ``gender identity'' is shaped through updating our self model to resolve the continuous prediction failure in multiple aspects because one or many of our raw feelings are incongruence with gender categories. Body representation incongruent people's raw feeling might be ``this body part isn't mine.'' Autistic people's raw feeling might be ``the arbitrary gender norm is hard to conform and makes me uncomfortable.'' Some people with gender non-conforming personalities and behaviours might feel ``I dislike the personality/temperament externally forced on me.'' Other transgender individuals might develop their gender identity through personal history. The gender construct ties disparate causes together and make us interpret them as ``This means I am a girl/boy/man/woman/non-binary person.'' ``Gender identity'' is therefore not an internal essence, but a post-hoc rationalisation, a ``conceptual chimaera.'' It connects a pre-linguistic raw feeling with a purely socially constructed category of identity. Then it claims this connection is ``natural,'' using the raw feeling to legitimise a social construct.

Using the predictive coding theory to explain the emergence of gender identity provides distinct advantages, especially compared with the "male/female brain" theory and the body representation theory. It successfully bridges "gender" (social construct) and "gender identity" and answers the question "what \textit{gender} means in \textit{gender identity}" by connecting neurological states with external gender construct, and avoids fallacies like begging the question or the category error exhibited in other models.  

% Therefore, we must distinguish between the biological causes and the psychological explanation. The former are continuous, mosaic, and low-level biological traits. The latter is a discrete, high-level semantic label recruited by the brain's predictive machinery to make sense of the former within a specific cultural framework. Confusing them is the fundamental error of gender essentialism.

%If we were to judge intersex conditions in humans as we do in birds, based on external genitalia alone, then most birds (except for a few groups like ducks, whose males have a penis) have the same external genitalia (a cloaca), so they should not have the concept of ``intersex'' at all. Some ornithologists classify birds as intersex based on mixed plumage colour \parencite{Choudhary2024Intersex}, but feathers' relationship to reproduction is less direct than that of mammalian breasts (used for lactation). Their function is more like our body hair, Adam's apple, beard, and facial features, which are purely for attracting mates and not physiologically related to reproduction.

%If birds' mixed plumage can be used to classify intersex, then individuals with gynaecomastia should also be intersex. The neutral neurological traits related to an individual's personality, behaviour patterns, and body representation, which make them more likely to form a specific gender identity when interacting with current social gender norms, are also a form of sexual dimorphism and have a significant statistical correlation with gametes (sex \textit{sensu stricto}). Thus, theoretically, they should also be part of ``phenotypic sex.''

%If we apply the standard for birds to ourselves, an individual whose certain sexually dimorphic traits differ from the statistical norm for their gametic sex could be considered intersex. Then, transgender people whose identity stems from innate body representation incongruence are all intersexes, and in fact, almost everyone is intersex, such as a ``female'' with a lot of body hair or a short ``male.'' The ``standard'' and ``typical'' ``male'' and ``female'' (endosex) would be rare exceptions.

%Also, ``conversion therapy cannot change it'' is not a valid argument for ``gender identity is innate.'' Let's conduct a thought experiment: try using ``conversion therapy'' to change a person's native language. If we apply this logic honestly, consistently, and without compromise, we will inevitably conclude that ``native language is also innate.'' \footnote{I am not seriously suggesting a ``forced native language conversion''; this is clearly a violent, anti-human act.}

%Without the intervention of society and culture, an individual would not choose words like ``male'' and ``female'' to describe themselves. Our social constructs have encoded these two words with meanings beyond their reproductive biological sense. Neurological features are not gender; they are simply personality traits, behavioural patterns, and body schemas. In social and cultural interactions, they may lead an individual to develop a sense of ``identity'' with specific gender constructs and gender roles. Still, they themselves are not ``gender.''
%This process occurs within the research domains of sociology and psychology. Using biological methods to study them is a serious category error, as ineffective as calculating the relativistic velocity of each car to study traffic flow on a road.