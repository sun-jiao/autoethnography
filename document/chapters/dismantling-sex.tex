In this section, I will illustrate that the definition of the so-called "biological sex" involves the fallacy of begging the question and contradicts Occam's razor as a scientific redundancy. This argument is necessary for further discussion because many scientific models I will discuss repeat this fallacy when explaining gender identity. In this article, I use ``phenotypic sex'' as an alternative to the conventional term ``biological sex,'' because the academic biological definition of sex is strictly confined to the types of gametes \parencite{Lehtonen2014Gamete, Goymann2023Biological, Hurst1996There, Griffiths2025Biology}, and its determination mechanisms are homologous among the entire vertebrate lineage \parencite{Bellott2017Avian, Graves2010Homologies, Smith2007Bird}. Different clades have developed different sex chromosomes, genitals, and sexual dimorphic characteristics in the evolutionary history. These characteristics are dynamically shaped by sexual selection. Genitals can change extremely fast in some groups, e.g., some ducks \parencite{Brennan2007Coevolution, Orbach2018Evolution}, which is not different from other sexual dimorphic traits. The phenotypic sex (so-called ``biological sex'') is merely an arbitrarily organised subset of sexual dimorphic traits. Its scope is unjustified. \textcite{Zieminska2022Toward} proposed the so-called "five layers of sex," namely sex chromosomes, gonads, internal sex organs, external genitals, and secondary sex characteristics. From an evolutionary perspective, the sexual dimorphism of height possesses no essential difference from that of sex chromosomes, genitals, or secondary sex characteristics. The difference lies solely in the degree of overlap; height exhibits significantly greater overlap than genitalia. This is a quantitative difference rather than a fundamental one. They are all adaptive traits that promote the rate of reproduction success. Then why are tall "women" or short "men" not considered "intersex?" In what way are the five layers different from other sexual dimorphic traits? This is a typical instance of begging the question. As an instrumental proxy of gametic sex, its functions can be replaced by sexual dimorphism or some specific sexual dimorphic traits, therefore framing it as a scientific redundancy. Moreover, using the same term "sex" for gametes and an arbitrary subset of sexual dimorphic traits wrongly implies that they are more closely related than other sexual dimorphic traits. It should be named "strongly sexual dimorphic traits" rather than "sex."

Secondary sexual characteristics are sometimes considered part of phenotypic sex (e.g., \cite{Zieminska2022Toward}). However, intersex conditions are practically diagnosed only by their gonads and genitals, not the whole range of "sex." Individuals with gynaecomastia are not considered intersex. For individuals with gynaecomastia who self-identify as male, medical treatment for their breasts is also to ``make their body consistent with their gender identity,'' while it is not considered a "gender-affirming surgery." What does the component "sex" mean in this term? \textcite{AMA2021Advancing} define "intersex" as "individuals whose reproductive organs and anatomy (e.g., primary sex characteristics, hormones, chromosomes, etc.) do not align with medically defined and socially expected notions of male and female." Why aren't individuals with polycystic ovaries syndrome (PCOS) "intersex," since their hormones do not align with typical male or female phenotypes. The definition of phenotypic sex and related terms are not only unjustified, but also blurred, nebulous, and unstable, even sometimes self-contradictory. 
