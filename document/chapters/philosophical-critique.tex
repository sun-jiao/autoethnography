%As we said above, if we apply the standards of Aves to Mammalia, we will conclude that almost all humans are intersex. However, in the following argument, let's not jump to such a radical stage.

%Let's first achieve a kind of logical consistency within humanity: individuals with gynaecomastia are intersex. This is the necessary result of applying the standard of breasts as part of ``phenotypic sex'' from other domains to the domain of ``intersex.''

%Then, according to the old diagnostic criteria, like in ICD-10~\parencite{WHO1992ICD10} or DSM-IV-TR~\parencite{APA2000DSM4}, an individual with gynaecomastia who self-identifies as male should be considered a specific kind of transsexual (XtM), because they have ``a sense of discomfort with their anatomic sex,'' and want to make it ``as congruent as possible with one's preferred sex.'' The medical treatment should also be considered ``gender affirming surgery.'' Ironically, both diagnostic criteria explicitly exclude intersex individuals, which is extremely hard to understand.

%However, rather than achieving greater inclusivity by including intersex people, the editors continue to maintain a \textit{de facto} exclusion of intersex people by introducing the concept of  ``assigned sex at birth''~\parencite{APA2013DSM5, WHO2019ICD11}. Not only individuals with gynaecomastia were excluded from transgender, but many traditionally defined

The preceding biological analysis established that ``gender identity'' is not an innate, immutable essence housed within the brain or soul, but rather a product of neuroplasticity formed as the brain ``overfits'' to a noisy and arbitrary social norms. Consequently, because the biological hardware is merely processing the software of culture, I shift the focus from the biological mechanism to the philosophical structure of the social constructs themselves. I will demonstrate that the contemporary theoretical framework surrounding gender -- specifically the dichotomy of ``cisgender'' and ``transgender'' and the concept of ``assigned sex at birth'' (ASAB) -- is full of internal contradictions to maintain a precarious ideological order.

Let us have a thought experiment: an intersex child is assigned female at birth, but gets lost at a very young age, is reassigned male by adoptive parents, and is raised as a boy. After adulthood, their body is closer to a ``phenotypic male,'' but their gender identity is female. According to the definition of \textcite{APA2013DSM5, WHO2019ICD11}, this individual is a ``cisgender female.''

By introducing the concept of  ``assigned sex at birth,'' some intersex people are excluded from transgender. Ironically, if we use ``phenotypic sex'' to define transgender people, we can capture this individual's unique gender socialisation experience (XtF) very well. Since the so-called ``assigned sex'' is almost always binary in actual medical practice, it simplifies the patterns of children's gender socialisation using two typical phenotypic sexes, replicating the power framework it seeks to oppose, and in fact, brings back the ghost of gender binary.

When we examine ASAB more furtherly, we will inevitably ask: Is ``assigned sex/gender at birth'' \footnote{In English, both ``assigned sex'' and ``assigned gender'' are used. } a kind of ``sex'' or ``gender?'' What is the relationship between ASAB and gender identity? If gender is unrelated to phenotypic sex and gender identity is not identifying with sex, then logically, being phenotypic or assigned male/female and identifying as male/female are two or three unrelated things. They just share the words ``male/female,'' like the ecological and sociological ``community.''

Therefore, everyone's phenotypic sex (a series of biological characteristics, although artificially selected) or ASAB (a label on a legal document) is ``inconsistent'' with their gender identity (a psychological state). Or more accurately, it is impossible to discuss whether they are ``consistent'' or ``inconsistent.''. Because they describe three essentially different kinds of things, just as we would not say that the social ``community'' a person belongs to is ``consistent'' with the ecological ``community'' they inhabitants.

Conversely, if those whose ASAB use the same word as their gender identity are called ``cisgender,'' and their phenotypic sex or ASAB is ``consistent'' with their gender identity, this means that phenotypic sex and gender are related in some way. Their view on the relationship between sex and gender is not a rigorous philosophical nominalism, but a vulgarised, selectively applied nominalist fallacy. When they need to separate the body from identity, they use nominalism; when they need to establish the cisgender/transgender binary opposition, they secretly retreat to an unreflective realist position.

The conflict between ``nominalism'' and ``realism'' occurs not only between ASAB and gender identity, but also within ``gender identity'' itself: is the ``woman'' of feminism and the ``woman'' of patriarchy the same concept? The ``woman'' who supports feminism and the traditional ``woman'' both self-identify as ``woman.'' Are their gender identities the same? They claim that all people who identify as ``woman'' share some real, existing ``essence'' called ``female identity'', no matter how different their political stances, lifestyles, and values are. And it uses realist labels like ``cisgender/transgender'' to classify these completely unrelated self-identities as the same kind, simply because their name is or is not consistent with their birth certificate.

Moreover, the cisgender/transgender binary opposition fundamentally violates the ``principle of self-identification,'' because it is equivalent to assigning a gender identity that is ``consistent'' with the ASAB to all people who have not explicitly identified as a different one. Most ``cisgender'' people have never made such a statement, nor have they ever been asked ``What is your gender identity?'' This act is almost completely isomorphic to the ASAB; both are top-down, external assignments without consent. A movement, whose important ethical principle is ``opposing non-consensual identity assignment,'' \footnote{\textcite{Butler2025Who}: \ldots Can we decide what being or having a sex means outside of a framework that establishes and reestablishes sex, that is, a framework that has to be imposed with regularity through time, one where the power to self-assign is exercised by those who have already \textit{been} assigned? Some trans people turn against all assignment, claiming that it invariably works in the service of hierarchy.} relies on the non-consensual identity assignment of most people. It repeats the oppressive structure it criticises, but only packaging itself in a ``critical'' language.

It implicitly introduces an \textit{a priori} assumption that ``everyone has a gender identity,'' because the definition of ``cisgender/transgender'' depends on the existence of it and its ``consistency'' with ASAB. By establishing a classification based on ``gender identity,'' it made ``having a gender identity'' a necessary condition for being a complete-socialised person. If you are not ``transgender,'' then you must be ``cisgender'' -- humans are deprived of the options ``I don't have this thing at all,'' ``null,'' or ``undefined.'' Under this framework, the position of ``I have no gender identity'' becomes invisible or politically incorrect. Individuals are deprived of the right to ``quit the game''; they are either ``cisgender'' or ``transgender.''
%\footnote{This topic reminds me that I once intended to declare my gender identity as null. I was using it in a very sincere, technical sense (value unassigned). Nevertheless, I immediately realised that I cannot express it like how a computer handles null. When someone asks about my gender identity, I cannot directly throw a ``NullPointerException: field \`gender\` not found in object \`user\`'' in their brains. In real life, it is still a textual label. The ``null'' I am writing in this article is four Latin letters, not the technical null.}

Most fatally, when they set ``cisgender'' as a synonym for ``non-transgender'' or ``having not stated their gender identity,'' they are saying that all transgender people are cisgender at birth. This is equivalent to saying that ``transgender people were cisgender and changed for specific reasons,'' which becomes the conservative viewpoint. This flaw is logically and politically devastating.

%``Transgender'' is also this kind of imposed identity if we view it from a specific perspective. Some activists and scholars like \textcite{Arraiza2024After} argue that neurological markers are ``biological determinism'' and strip trans people of their rights. However, if we completely abandon neurological markers, it is effectively saying a person must participate in the social gender construct and have a ``gender identity'' to get medical treatment. People are striped the right of refusing to participate in gender constructs and identity politics, and just going to a doctor and say, ``I have a persistent rejection of this part, it affects my life, and I want to remove it to resolve my suffering.'' The only reason this isn't the primary path is that the ``gender identity'' narrative currently has more followers.
