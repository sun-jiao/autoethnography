Having established this premise, what I should do next was clear: to introspect on everything related to gender in my grown-up and analyse which neurological features they might stem from, how they interacted with society and culture, and how these interactions might have shaped my so-called ``gender identity.''

I began with my current experiences, as described earlier, to identify the experiences that might have shaped them.

My desire for a feminine name might stem from my name being frequently miswritten or mistyped as the feminine \textit{Jiāo} by others in my childhood due to the input method issue mentioned earlier. (\cref{fig:2}) I was furious at the time. I suspect my preference for the feminine \textit{Jiāo} was shaped by repeated misuse.

\begin{figure}[htbp]
    \centering
    \includegraphics[width=0.7\textwidth]{F2}
    \caption{The author's name was spelt as the feminine \textit{Jiāo} by others. Left: The author's middle school name tags, with the middle one written as the feminine \textit{Jiāo}, and the top and bottom ones as the author's legal name; right: The author's border pass in Nyalam County, Tibet, with the name spelt as the feminine \textit{Jiāo} by officers.\label{fig:2}}
\end{figure}

The second thing that came to mind quite clearly was that in primary school, the shoes I wore were a type of white cloth shoe with red trim, locally known as ``gymnastics shoes'' (体操鞋 \textit{tǐ cāo xié}). My classmates considered them ``girls' shoes,'' and I was mocked and bullied for it.

It wasn't just that the colour and style were ``feminine,'' but were part of the girls' school uniform (the boys' version was the same style but blue). For occasions such as ceremonies and performances, girls were required to wear the red version, and boys the blue one. Therefore, in the micro-society of the school, they were absolutely and unquestionably considered girls' shoes. A few bullies snatched my shoes and threw them away, pulled down my pants, and pushed me onto the broken branches of a bush in the school garden, injuring me.

During that time, I often dreamt of losing my feet in an accident and would feel a strong sense of comfort and relief upon waking. I don't think this is body integrity identity disorder. I hypothesise that the dream might have been because in my dream, I believed that wearing girls' shoes meant having girls' feet, and I wanted to separate them from myself. This might indicate that I did not identify as a girl at the time, was very resistant, and wanted to separate that from myself.\footnote{I am not saying that my feeling of ``male'' or ``boy'' was internal and innate; it was necessarily learned as well. It may occur too early to be remembered by me. } Then, this idea of seeing a specific part of my body as a girl's might have gradually extended to my entire body being a girl.

After I grew up, I developed a unique attachment to these shoes. I even took a photo of myself wearing them, uploaded it to Wikimedia Commons,\footnote{\url{https://commons.wikimedia.org/wiki/File:Red_Uwabaki.jpg}} and added it to its Wikipedia article, making a specific private childhood experience a part of humanity's public knowledge repository.

I read some literature and researched their history, discovering that these shoes likely originated from Japanese indoor shoes (上履き, \textit{uwabaki}), mainly used in schools and kindergartens in Japan. Chinese clothing factories might got some Japanese orders to produce these shoes, which were later sold in China. Due to their simple style and low price, they became widely used as uniform shoes for elementary school students. In some Japanese schools, the colour of indoor shoes also depends on gender, but some schools assign colours by grade level. \parencite{Kanzaki2019Shogakko} Their study revealed that the shoe itself has no fixed gender meaning; its gender meaning is artificially assigned in a specific context. (\cref{fig:3})

\begin{figure}[htbp]
    \centering
    \includegraphics[width=0.5\textwidth]{F3}
    \caption{Application of gymnastics shoes/indoor shoes in different contexts: a-b) Photos of the author wearing red gymnastics shoes in childhood; c) An performance at a Chinese kindergarten, where boys wear the blue version and girls wear the red version, from Qilu.com; d) Screenshot from the social platform Xiaohongshu, where a Japanese blogger shares their school life, with both boys and girls wearing the red version of indoor shoes. Chinese users in the comments discuss this phenomenon with the Japanese blogger.\label{fig:3}}
\end{figure}

Following the thread of being bullied in elementary school, I recalled that the boys in my class often fought. I disliked playing with them. The girls were harmonious and friendly, and they were kind to me, so I enjoyed playing with them.\footnote{I am not saying that this gender-specific behaviour pattern is innate. } One thing I remember vividly is when we visited a museum with an interactive exhibit. Our class was split into two groups, boys and girls. When the boys' group played, if someone failed, they were harshly mocked and heckled by the majority, telling them to get down quickly and not waste others' time. When the same thing happened in the girls' group, it was filled with encouragement. I really envied the girls' group at that moment.\footnote{Again: I am not saying that this gender-specific behaviour pattern is innate.}

Then, a few other childhood memories flooded my mind: adults used to say my ``personality was like a little girl's,'' probably because I liked playing with stuffed animal toys and disliked sports and fighting. Another thing was that I was punished for imitating a little girl on a TV series, covering her mouth to laugh, being told, ``Boys can't laugh like girls.'' Children have no gender bias and imitate and learn all behaviours within their capabilities, and the gender meaning is externally imposed. I might have some innate personality traits and temperaments from my nervous system that are like the ``feminine temperament'' in gender stereotypes, making me more willing to imitate specific behaviours. However, these innate traits are fundamentally neutral. It has nothing to do with gender before being interpreted by adults as ``this is girls' behaviour.''

I also recalled a few things related to the girls' bathroom. In elementary school, I was playing tag with a few girls, and they repeatedly ran into the girls' bathroom to hide. I stood guard at the door, waiting to catch them when they came out, but a teacher saw me and punished me. Another incident was when a maths teacher punished some mischievous boys by making them clean the girls' bathroom. Another time, I got sick and vomited during class, and the teacher took me to the girls' bathroom to clean up. This was probably because the teacher was female, our classroom was close to the girls' bathroom, and it was during class time, so no one was in there.

These incidents are difficult to interpret with normal logic. I suspect that, in some way, they shaped the girls' bathroom in my young mind into a symbolic place filled with indescribable, chaotic, and contradictory implies and metaphors: it was a safe zone for girls, where they could hide during a game, while my attempt to use a logical strategy to win the game was inexplicably punished; it was a place of ``degradation'' for boys, where misbehaving boys were forced to enter and clean as a form of humiliation; it was a place of care, where when I was unwell, the usually forbidden rules were broken, and the teacher helped me clean my body and clothes. These events may have somehow shaped my fascination with female spaces, femininity, and female symbols.

Another significant difference between me and ``typical'' transgender people is that my dislike and anxiety about my body do not seem to be ontological. One piece of evidence is that I discovered the mechanism of masturbation at a very young age (around kindergarten age) and excitedly shared it with others. This indicates that my body schema was consistent with my physiological body.\footnote{I know this behaviour might be considered ``shameless'' or ``perverted'' by adults, but for my kindergarten self, it was just an objective exploration of the body, which I believe is no different in essence from sucking one's thumb or playing with one's hair.}

As we've discussed before, gender incongruence about physical characteristics and gender incongruence about social roles do not have the same neurological mechanisms. Otherwise, it would imply that humans can innately feel the traditionally defined ``gender'' (the sex-gender complex), which suggests a ``return to gender essentialism.'' Innate body incongruence may stem from the body schema. In contrast, nurture body incongruence may stem from the meaning that society and culture assign to the body and its impact on body image. Mine seems to be the latter. My dysphoria is primarily about identity and social roles\footnote{It has been explained by childhood experiences like the shoes and the girls' bathroom. }, and my longing for a female body is much milder than what other transgender people describe.

One thing that left a deep impression on me was that I had a crush on a girl in my childhood, but she didn't love me. I happened to read Stefan Zweig's \textit{Letter from an Unknown Woman}, in which the female protagonist has a one-night stand with the male protagonist and raises their child on her own. I thought at the time, ``Wow, I also want to have XX's child and raise them secretly.'' It's so great that girls can have babies; it's so enviable. Why can't a boy's body have babies? What a pity. This was an envy of a specific function, stemming from a longing for a romantic relationship.

Moreover, I find a kind of beauty in the female body that is hard to describe. It's a pre-linguistic aesthetic feeling, which could perhaps be described as ``a sense of elegance.'' I find the female body very pleasing to look at and feel envious. It's like how one might envy a bird for being able to fly, but it's unlikely to cause a body integrity disorder regarding one's own arms.\footnote{Our arms are homology with birds' wings. } I interpreted it as my longing for a female body is aesthetic and functional, not metaphysical.

I suspect that both my so-called ``sexual orientation'' (towards women) and the aesthetic part of my ``gender identity'' are products of this aesthetic experience, combined with different ``other factors'' (to use the term loosely). Combined with intimate emotions and sexual instincts, it becomes sexual orientation; combined with body image and external stimuli, it becomes the aesthetic part of ``gender identity.''

Another piece of evidence is that a DeepNude version of my feminised self has sexually aroused me.\footnote{I have to say, it's quite ironic that an AI application originally intended for sexual abuse was used by me to alleviate my gender dysphoria. } It is revealed that in my case, ``sexual orientation'' and ``gender identity'' share some more fundamental factors. This is phenotypically like \textcite{Blanchard1991Clinical}'s theory of autogynephilia. However, he explained that sexual orientation is the root cause, from which gender identity stems. This implies an innate, ontological sexual orientation, which I disagree with. I believe humans have genetically determined, neurological innate aesthetic and mate preferences, but they are not innately ``gendered.'' They are interpreted by society and culture. Individuals who have internalised social norms then integrated them into a ``gender.''

On the other hand, I have neurodermatitis (a.k.a. lichen simplex chronicus) in my genital area. There are multiple studies supporting that chronic itching and pain can affect one's body image. \parencite{Simsek2020Body, Vamos1993Body} This could be another reason why I dislike my reproductive organs. However, it should be noted that I developed neurodermatitis much later than the childhood events mentioned above, and neurodermatitis itself is a psychosomatic disease heavily influenced by psychological and mental states. \parencite{Lotti2008Prurigo, Tey2013Psychosomatic} Therefore, it could be a result of gender dysphoria rather than a cause.

My conclusion is: my current so-called ``gender identity'' is a synthesis of this aesthetic longing for the female body, envy of the reproductive function, and an internalised reaction to childhood experiences and trauma. Our life experiences and psychological responses to external stimuli are interwoven like a dense net, or rather, like a chain reaction, where one event triggers multiple preceding events, which continue to trigger subsequent events. Then we pick out a few phenotypically similar phenomena, give them a name: ``gender identity.'' This is pure tautology and has no other meaning.

The ``similarity'' between them is also guided and shaped by society and culture; otherwise, they would be independent and unrelated phenomena. Does a name have a gender?\footnote{Chinese is a language without grammatical gender. } Does a specific colour and style of shoe have a gender? Does encouragement and care from friends have a gender? Does playing with stuffed animals, disliking sports, roughhousing, or covering one's mouth to laugh have a gender? Does a specific geographical space, if not labelled ``Girls' Bathroom,'' have a gender? Does a specific body morphology and aesthetic preference have a gender? Does wanting to establish a relationship with a romantic partner through childbirth have a gender?\footnote{Some might argue that the last two do have a ``gender,'' which involves innate body schema differences, a point we have already discussed. Moreover, even from a biological perspective, a person can have a ``phenotypic female'' body or a uterus and simultaneously produce sperm.}

From beginning to end, gender is not inherent in the individual (me), but in the external society. It is the pervasively gendered society and culture that assigns a gender to all things. The individual, in their interaction with these ``gendered'' things, internalises social norms and develops a so-called ``gender identity.'' Therefore, ``gender identity'' is not an internal attribute but a product of humans' internalisation of social norms through interaction with the external world.