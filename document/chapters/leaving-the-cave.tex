Having established this premise, what I should do next was clear: to introspect on everything related to gender in my grown-up and how they might have shaped my ``gender identity.''

I suspect my preference for the feminine \textit{Jiāo} was shaped by repeated misuse. It was frequently miswritten or mistyped as the feminine \textit{Jiāo} in my life (due to the input method issue mentioned earlier). (\cref{fig:2}) I felt ashamed and unconfortable about it when I was in elementary and middle school, but I gradually came to like it.

\begin{figure}[htbp]
    \centering
    \includegraphics[width=0.7\textwidth]{IDs_with_feminine_name}
    \caption{The author's name was spelt as the feminine \textit{Jiāo} by others. Left: The author's middle school name tags, with the middle one written as the feminine \textit{Jiāo}, and the top and bottom ones as the author's legal name; right: The author's border pass in Nyalam County, Xizang (Tibet), with the name spelt as the feminine \textit{Jiāo} by officers.\label{fig:2}}
\end{figure}

Another memory involves the footwear I wore in primary school, which my classmates labelled ``girls' shoes.'' These were locally referred to as ``gymnastics shoes'' (体操鞋 \textit{tǐ cāo xié}). Although the style was uniform, the colours were segregated by gender: red for girls and blue for boys, and were mandatory in school. Because of this, I faced physical bullying: my shoes were thrown away, my trousers were pulled down to ``check whether I was a girl,'' and I was once pushed into the broken branches of a bush, resulting in injury. At that time, I often experienced dreams where I lost my feet in an accident, which brought me a strong sense of relief upon waking. In contrast to Body Integrity Identity Disorder, I hypothesise that I believed wearing ``girls' shoes'' rendered my feet ``girls' feet,'' prompting a desire to sever them from my identity. This likely indicates that I did not identify as a girl at the time and strongly resisted that categorisation. \footnote{I am not saying that my feeling of ``male'' or ``boy'' was internal and innate; it was necessarily learned as well. It may occur too early to be remembered by me. } This perception of a specific body part as ``female'' might have gradually extended to viewing my entire person through that lens.

After I grew up, I developed a unique attachment to these shoes. I took a photo of myself wearing them, uploaded it to Wikimedia Commons, and added it to its Wikipedia article, making a specific private childhood experience a part of humanity's public knowledge repository.

I read some literature and researched their history, discovering that these shoes likely originated from Japanese indoor shoes (上履き, \textit{uwabaki}), mainly used in schools and kindergartens in Japan. Chinese clothing factories might got some Japanese orders to produce these shoes, which were later sold in China. Due to their simple style and low price, they became widely used as uniform shoes for elementary school students. In some Japanese schools, the colour of indoor shoes also depends on gender, but some schools assign colours by grade level \parencite{Kanzaki2019Shogakko}. Their study revealed that the shoe itself has no fixed gender meaning; its gender meaning is artificially assigned in a specific context. (\cref{fig:shoes})

\begin{figure}[htbp]
    \centering
    \includegraphics[width=0.5\textwidth]{shoes}
    \caption{Application of gymnastics shoes/indoor shoes in different contexts: a-b) Photos of the author wearing red gymnastics shoes in childhood; c) An performance at a Chinese kindergarten, where boys wear the blue version and girls wear the red version, from Qilu.com; d) Screenshot from the social platform Xiaohongshu, where a Japanese blogger shares their school life, with both boys and girls wearing the red version of indoor shoes. Chinese users in the comments discuss this phenomenon with the Japanese blogger.\label{fig:shoes}}
\end{figure}


Then, a few other childhood memories flooded my mind: adults used to say my ``personality was like a little girl's,'' probably because I liked playing with stuffed animal toys and disliked sports and fighting. Another thing was that I was punished for imitating a little girl on a TV series, covering her mouth to laugh, being told, ``Boys can't laugh like girls.'' Children have no gender bias and imitate and learn all behaviours within their capabilities, and the gender meaning is externally imposed. I might have some innate personality traits and temperaments from my nervous system that are like the ``feminine temperament'' in gender stereotypes, making me more willing to imitate specific behaviours. However, these innate traits are fundamentally neutral. It has nothing to do with gender before being interpreted by adults as ``this is girls' behaviour.'' Additionally, the boys in my class often fought. I disliked playing with them. The girls were friendly, and they were kind to me, so I enjoyed playing with them. \footnote{I am not saying that this gender-specific behaviour pattern is innate.}

%One thing I remember vividly is when we visited a museum with an interactive exhibit. Our class was split into two groups, boys and girls. When the boys' group played, if someone failed, they were harshly mocked and heckled by the majority, telling them to get down quickly and not waste others' time. When the same thing happened in the girls' group, it was filled with encouragement. I really envied the girls' group at that moment.

I also recalled a few things related to the girls' bathroom. In elementary school, I was playing tag with a few girls, and they repeatedly ran into the girls' bathroom to hide. I stood guard at the door, waiting to catch them when they came out, but a teacher saw me and punished me. Another incident was when a maths teacher punished some mischievous boys by making them clean the girls' bathroom. Another time, I got sick and vomited during class, and the teacher took me to the girls' bathroom to clean up. This was probably because the teacher was female, our classroom was close to the girls' bathroom, and it was during class time, so no one was in there.

These incidents are difficult to interpret with normal logic. I suspect that, in some way, they shaped the girls' bathroom in my young mind into a symbolic place filled with indescribable, chaotic, and contradictory implies and metaphors: it was a safe zone for girls, where they could hide during a game, while my attempt to use a logical strategy to win the game was inexplicably punished; it was a place of ``degradation'' for boys, where misbehaving boys were forced to enter and clean as a form of humiliation; it was a place of care, where when I was unwell, the usually forbidden rules were broken, and the teacher helped me clean my body and clothes. These events may have somehow shaped my fascination with female spaces, femininity, and female symbols.

Another significant difference between me and ``typical'' transgender people is that my dislike and anxiety about my body do not seem to be ontological. One piece of evidence is that I discovered the mechanism of masturbation at a very young age (around kindergarten age) and excitedly shared it with others. This indicates that my body schema was consistent with my physiological body.
%\footnote{I know this behaviour might be considered ``shameless'' or ``perverted'' by adults, but for my kindergarten self, it was just an objective exploration of the body, which I believe is no different in essence from sucking one's thumb or playing with one's hair.}

As we've discussed before, gender incongruence about physical characteristics and gender incongruence about social roles do not have the same neurological mechanisms. Otherwise, it would imply that humans can innately feel the traditionally defined ``gender'' (the sex-gender complex), which suggests a ``return to gender essentialism.'' Innate body incongruence may stem from the body schema. In contrast, nurture body incongruence may stem from the meaning that society and culture assign to the body and its impact on body image. Mine seems to be the latter. My dysphoria is primarily about identity and social roles \footnote{It has been explained by childhood experiences like the shoes and the girls' bathroom. }, and my longing for a female body is much milder than what other transgender people describe.

One thing that left a deep impression on me was that I had a crush on a girl in my childhood, but she didn't love me. I happened to read Stefan Zweig's \textit{Letter from an Unknown Woman}, in which the female protagonist has a one-night stand with the male protagonist and raises their child on her own. I thought at the time, ``Wow, I also want to have XX's child and raise them secretly.'' It's so great that girls can have babies; it's so enviable. Why can't a boy's body have babies? What a pity. This was an envy of a specific function, stemming from a longing for a romantic relationship.

%Moreover, I find a kind of beauty in the female body that is hard to describe. It's a pre-linguistic aesthetic feeling, which could perhaps be described as ``a sense of elegance.'' I find the female body very pleasing to look at and feel envious. It's like how one might envy a bird for being able to fly, but it's unlikely to cause a body integrity disorder regarding one's own arms. \footnote{Our arms are homology with birds' wings. } I interpreted it as my longing for a female body is aesthetic and functional, not metaphysical.

I suspect that both my so-called ``sexual orientation'' (towards women) and the aesthetic part of my ``gender identity'' are products of this aesthetic experience, combined with different ``other factors'' (to use the term loosely). Combined with intimate emotions and sexual instincts, it becomes sexual orientation; combined with body image and external stimuli, it becomes the aesthetic part of ``gender identity.'' A piece of evidence is that a DeepNude version of my feminised self has sexually aroused me. \footnote{I have to say, it's quite ironic that an AI application originally intended for sexual abuse was used by me to alleviate my gender dysphoria. } It is revealed that in my case, ``sexual orientation'' and ``gender identity'' share some more fundamental factors. This is phenotypically like \textcite{Blanchard1991Clinical}'s theory of autogynephilia. However, he explained that sexual orientation is the root cause, from which gender identity stems. This implies an innate, ontological sexual orientation, which I disagree with. I believe humans have genetically determined, neurological innate aesthetic and mate preferences, but they are not innately ``gendered.'' They are interpreted by society and culture. Individuals who have internalised social norms then integrated them into a ``gender.''

On the other hand, I have neurodermatitis (a.k.a. lichen simplex chronicus) in my genital area. There are multiple studies supporting that chronic itching and pain can affect one's body image \parencite{Simsek2020Body, Vamos1993Body}. This could be another reason why I dislike my reproductive organs. However, it should be noted that I developed neurodermatitis much later than the childhood events mentioned above, and neurodermatitis itself is a psychosomatic disease heavily influenced by psychological and mental states \parencite{Lotti2008Prurigo, Tey2013Psychosomatic}. Therefore, it might be a result of gender dysphoria rather than a cause.

My conclusion is: my current so-called ``gender identity'' is a synthesis of this aesthetic longing for the female body, envy of the reproductive function, and an internalised reaction to childhood experiences and trauma. Our life experiences and psychological responses to external stimuli are interwoven like a dense net, or rather, like a chain reaction, where one event triggers multiple preceding events, which continue to trigger subsequent events. Then we pick out a few phenotypically similar phenomena, give them a name: ``gender identity.'' This is pure tautology and has no other meaning.

The ``similarity'' and ``relatedness'' between them is also shaped by society and culture; otherwise, they would be independent and unrelated phenomena. From beginning to end, gender is not inherent in the individual (me), but in the external society. It is the pervasively gendered society and culture that assigns a gender to everything, including our bodies, clothes, behaviours and personalities. The individual, in the interaction with these ``gendered'' things, internalises social norms and develops a so-called ``gender identity.'' Therefore, ``gender identity'' is not an internal attribute but one of many components of our ``narrative identity'', it is the product of the neuroplasticity of our brains. The childhood traumas (e.g., my experience with the shoes) are also an important part that shaped our gender identities, just like they could shape other components in our narrative identities.

% Does a name have a gender? \footnote{Chinese is a language without grammatical gender. } Does a specific colour and style of shoe have a gender? Does encouragement and care from friends have a gender? Does playing with stuffed animals, disliking sports, roughhousing, or covering one's mouth to laugh have a gender? Does a specific geographical space, if not labelled ``Girls' Bathroom,'' have a gender? Does a specific body morphology and aesthetic preference have a gender? Does wanting to establish a relationship with a romantic partner through childbirth have a gender? \footnote{Some might argue that the last two do have a ``gender,'' which involves innate body schema differences. Still, we have already argued that this is a category error. Moreover, even from a biological perspective, a person can have a ``phenotypic female'' body or a uterus and simultaneously produce sperm.}

Subsequently, I turned to another question: Is my ``gender identity'' really ``female''? This still seems to be a Texas Sharpshooter Fallacy. Before I began my analysis, I had presupposed that ``I have a strong sense of identity with female identity and related social symbols.'' However, sometimes I very naturally and automatically think of myself as ``male,'' especially when arguing with trans-exclusionary radical feminists (TERFs) online.

When they criticise or insult all ``phenotypic males'' in some gender-essentialist way (e.g., ``all men are oppressors''), I feel very angry and use myself as a counterexample to refute them. Of course, this is partly because I know very well that their so-called ``men'' refers to phenotypic sex, not gender identity, and I had not thought about the gender issue at that time. It at least shows that although I usually feel uncomfortable when being called or classified as ``male,'' this discomfort is not greater than my hatred for irrationality. If classifying myself as ``male'' can provide a valid counterexample to refute their argument. I am happy to substitute myself into $ \text{Male}(x) $ and logically falsify their universal proposition of $ \forall x (\text{Male}(x) \rightarrow P(x)) $.

I am not sure if this counts as a kind of ``gender identity.'' When I argue with TERFs, the structure of the anger is the same as the anger I feel when arguing with extreme nationalists who proclaim that ``all Japanese people are guilty, there are no innocent souls under the atomic bomb.'' \footnote{Referring to Hiroshima and Nagasaki.} I believe my anger is from the irrationality and collective responsibility. These two scenarios are almost logically isomorphic: an oppressive power (patriarchy/Japanese militarism) commits evil in the name of a specific group, which does not grant the oppressed group the right to indiscriminately attack the former group in return, because this power clearly did not receive authorization from the group it claims to represent.

While for me personally, there is a significant difference between these two situations: I am obviously not Japanese, so when I argue with extreme nationalists, I am very clear that this is a purely rational anger against irrationality. In the other situation, because I had not deeply thought about gender identity at the time, and their definition of ``male,'' along with that of the broader society, did include me, the target and direction of this anger were often confused. I sometimes genuinely felt that I was a (specifically defined) ``male'' and that I was being insulted.

From a particular perspective, this is also ``internalising a gender identity through interaction with a pervasively gendered society.'' TERFs use a crude, essentialist method to impose the label ``male'' on me. For the sake of debate, I strategically accept this label and develop complex emotional reactions around it. This ``contextual male identity'' and the ``female identity,'' I feel, in many other scenarios, are formed by the exact same mechanism.

Thus, this phenomenon not only occurs in childhood but also appears continuously throughout a person's life. It's just that for most people, whether cisgender or transgender, after their gender identity is formed, they will consciously resist the invasion from another gender. I happen to not care much about ``gender.'' I don't think my gender identity is very important to me, not my core identity, so I didn't resist it. Ironically, my ignorance of gender has allowed ``gender'' to be able to freely ``invade'' my ``self.'' Some other ``gender fluid'' people may also stem from a similar cognitive mechanism. \footnote{I did not rule out other mechanisms of gender fluid. }