%I posted my initial viewpoint on three different platforms. Except for insults and banning, I received a more subtle form of violence. They asked me, ``Isn't this just queer?'' or ``You've reinvented queer theory.'' Some others said that my viewpoint was ``solipsism'' or ``Landian Accelerationism,'' which I completely failed to understand. \footnote{Ironically, queer theory is more like ``gender accelerationism'' to me. Both left-wing accelerationism and Land's right-wing accelerationism are theories about ``pushing the inherent logic of capitalism to its limits, leading to its self-destruction and creating a new world.'' The difference lies in the ``new world'' they envision (socialism/anti-humanism). They and queer theory are all about resisting a system by repeating/accelerating its own mechanism. } Moreover, they responded to my analysis of ``Indo-European-centrism in gendered pronouns'' with the personal attack, ``Your mum is also a form of cultural invasion.'' (\cref{fig:comments})

% I told them, ``This is not queer. The difference between queer theory and my viewpoint is like the difference between Giordano Bruno's pantheistic philosophy and the Hubble model of the universe. They are within two completely different disciplines and frameworks. This is also not a reinvention. You can't say that Hubble reinvented Bruno's pantheism.'' \footnote{This is a fact. I started thinking entirely based on the first principles of evolutionary biology. I read \textit{Gender Trouble} after I completed my introspection.} When they interpreted my points as ``reinventing queer theory,'' they did what Derrida spent his whole life opposing: they eliminated the uniqueness of the Other.
%
%\footnote{I am not a linguistic purist. I completely accept reasonable loanwords. However, gendered pronouns is not a reasonable case. Chinese originally lacked gendered third-person pronouns. Before the New Culture Movement, 他 (\textit{tā}, will be mentioned as ``neutral \textit{tā}'' in the following text) referred to anyone, regardless of gender. During the Westernization wave in the 20th century, Liu Bannong created 她 (\textit{tā}, she, will be mentioned as ``female \textit{tā}'' in the following text) in 1917 to mirror the gendered pronouns in European languages, especially English. This sparked controversy; for instance, the magazine \textit{Women's Resonance} (妇女共鸣) argued that replacing the ``human'' radical (亻) with the female radical (女) from the neutral \textit{tā} to create the female \textit{tā} dehumanized women. Nonetheless, the utility of the distinction, especially in translation and new literature, led to its widespread adoption. Ironically, some ``modern Liu Bannongs'' brainwashed by the West propose creating a ``gender-neutral pronoun'' to mirror English's \textit{they}. It is not a natural and equal language contact, but a linguistic pattern that continuously replicates in history: imitating English pronouns, making English almost the ``upstream'' (in the software engineering sense) of Chinese. Additionally, to be honest, I completely cannot understand why do you use gendered pronouns in English. }

%\begin{figure}[htbp]
%    \centering
%    \includegraphics[width=0.95\textwidth]{figures/comments}
%    \caption{The author's experience of being insulted in dogmatic communities on Zhihu, Xiaohongshu, and Reddit. Chinese comments translated using AI.\label{fig:comments}}
%\end{figure}
%
%Later, with the goal of so-called ``community solidarity,'' I wanted to participate in a more moderate way, no longer mentioning that ``gender is an oppressive concept,'' ``we should completely abolish gender.'' I still couldn't understand why, after sharing an article about the lack of inclusivity of ASAB for intersex people in one community, I was immediately banned. Another community accused me of ``gender essentialism'' for thinking that ``phenotypic sex is more accurate and inclusive than ASAB.'' I don't believe ``phenotypic sex'' is a real thing at all (as we discussed earlier). It is a proxy, like ASAB, to represent the gender socialisation patterns. If ``phenotypic sex'' is more accurate, then of course we should use it. A more precise representation means higher inclusivity, recognising a richer biological and sociological diversity. As a man-made scientific model, what does it have to do with ``essentialism''? I don't believe there is a metaphysical essence or a Platonic Form of ``male'' and ``female'' behind this model. I never discussed the metaphysical status of this concept at all. This viewpoint is moderate and morally necessary. I am curious if they really believe that the person in the thought experiment is a ``cisgender female.'' If they had seriously read my post, they could not have reached this conclusion. It completely violates their own values.
%
%It took me a long time to understand that their standard of ``inclusivity'' seems to be whether it sounds comfortable to them, and their standard of ``comfort'' seems to be arbitrary and without any fixed logic. The inclusivity I understand is using more accurate scientific models to represent human biological and sociological diversity. Suppose a model can describe and distinguish more diversity (e.g., different types of intersex people). In that case, it is more precise and more inclusive than a model that flattens or ignores these situations (like ``assigned sex at birth'').

%The prerequisite for any dialogue is that both parties are equal, mutually respectful, willing to listen, engage with the opponent's actual arguments, and willing to modify their own positions based on logic and empirical evidence. They don't care what ``gender essentialism'', ``accelerationism'' or ``solipsism'' really are. They are not academic philosophical classification, but the ultimate political tool like ``blasphemy'' or ``witchcraft,'' which an ideological system uses when facing a challenge it cannot respond.

When I started exploring my gender identity, I kept encountering intellectual barriers, such as: What is gender identity? What is its relationship to ``gender''? The term ``gender'' refers to a sociocultural category, then it is essentially an identity with a social construct. This contradicts the so-called ``innate, profound feeling,'' because social norms are nurtured. It is also politically problematic, as it seems to advocate that people should identify with the oppressive gender roles. If it refers to ``gender identity'' itself, this constitutes a ridiculous tautology, i.e., ``gender identity is an identity with gender identity.'' I felt it very mystical and obscurantist.

I posted my questions on some platforms, including Reddit, RedNote (Xiaohongshu) and Zhihu (a Chinese Q\&A website), hoping for some advice or help. However, almost no one responded to me seriously. I was insulted or degraded on all three platforms, and my account was banned on Reddit. These exclusionary transgender community completely refuses to understand what others say and explain their own views.

This reminds me of \posscite{Habermas1987Philosophical} critique of Michel Foucault. He argued that Foucault's reduction of everything to ``power'' made rational communication impossible and destroyed the foundation for building a truly free society based on communicative rationality. \footnote{\textcite{Habermas1987Philosophical}:
\begin{quotation}
    Thus, the attempt to preserve genealogical historiography from a relativist self-denial by means of its own tools falls short. In becoming aware of its own provenance from this alliance of scholarly and disqualified knowledge, genealogy only confirms that the validity claims of counterdiscourses count no more and no less than those of the discourses in power  --  they, too, are nothing else than the effects of power they unleash. Foucault sees this dilemma, but once again he evades any response.
    Foucault cannot adequately deal with the persistent problems that come up in connection with an interpretative approach to the object domain, a self-referential denial of universal validity claims, and a normative justification for critique.
    If one admits only the model of empowerment, the socialization of succeeding generations can also be presented only in the image of wily confrontation. Then, however, the socialization of subjects capable of speech and action cannot be simultaneously conceived as individuation, \ldots both theories lack a mechanism for social integration such as language, with its interlacing of the performative attitudes of speakers and hearers, which could explain the individuating effects of socialization.
\end{quotation}
} This is perfectly shown in the communities deeply influenced by post-structuralism. Foucault has taught them that every communication is just a power operation. When you tell people that there is no truth, only ``regimes of truth''; no reason, only ``power/knowledge''; no living author, only the ``author function''; no common, universal human destiny, only various discourses fighting each other -- what do you expect them to do? If someone presents a challenging argument, you have no obligation to listen to its reasoning; you only need to analyse its ``discursive effect'' \footnote{In my case, it was not even a real discursive effect, but their own prediction of a possible discursive effect. } and see it as an ``operation of power.''

According to this logic, Foucault himself should be the first to be judged, because ``morally judging and punishing an author based on discursive effects'' is precisely the real discursive effect produced in the discourse network by Foucault as an author function. Although this was not his own intention, its discursive effects have caused exclusion and harm in the real world. Foucault is the only author who cannot claim ``that's not what I meant,'' because he himself (as the author function) has forbidden himself (as a natural person) from doing so.

The greatest strength of Habermas's critique is its immediacy and structurality. Habermas's critique was fully articulated before Foucault's theory was ``vulgarised'' by the followers we see today. It did not need to wait for any subsequent events for verification, because it targeted an inherent structural flaw -- the ``performative contradiction'' of ``using reason to destroy reason.'' If Foucault was still using academic language, making arguments, writing, and expecting to be understood, he was inevitably caught in this contradiction. It is somewhat ironic that Foucault, when interviewed by \textcite{Boesers1977Folter}, appealed to his intent to defend himself; he claimed that the French \textit{raison} is different from the German \textit{Vernunft}, \textit{raison} is instrumental rationality, \textit{Vernunft} also includes value rationality, they are not the same. In this event, the natural person Foucault ``resurrected'' the ``dead'' author-function Foucault to practice Habermas's communicative rationality, trying to use rational argument to make others understand, he was not destroying reason. This demonstrated that it is not just a ``vulgarisation'' that has occurred in a few specific communities. It is the inevitable result of his theory that even Foucault himself cannot avoid. Because his world is inhabitable.

Out of the principle of communicative rationality, I accept Foucault's own defence about his original intent. However, analysing it with his own theory: \footnote{\textcite{Foucault1972Archaeology}:
\begin{quotation}
    In the nineteenth century, psychiatric discourse is characterized not by privileged objects, but by the way in which it forms objects that are in fact highly dispersed. This formation is made possible by a group of relations established between authorities of emergence, delimitation, and specification. One might say, then, that a discursive formation is defined (as far as its objects are concerned, at least) if one can establish such a group; if one can show how any particular object of discourse finds in it its place and law of emergence; if one can show that it may give birth simultaneously or successively to mutually exclusive objects, without having to modify itself.
\end{quotation}
} Foucault's description of the world as an anonymous, impersonal discursive field, claiming that the author's own intention is not important, is not just descriptive but also productive. His works, through discursive effects, ``produced'' this impersonal discourse operation, providing people with a powerful new discourse and new arguments as intellectual weapons, allowing them to ignore the author's intent, refuse good-faithed interpretation, and dismiss demands for intellectual coherence and logical self-consistency as ``power'' or ``discipline,'' equating it with ``instrumental rationality'' or ``governmentality,'' thereby further exacerbating this pre-existing problem. It might be a powerful framework in literature critique, while when becoming a communication strategy, it is a very form of dehumanisation. Foucault should have foreseen this consequence using his own theory, despite his own intention.

Everything the exclusionary transgender community has done, academic post-structuralists have also done, just packaged it in complex philosophical theories. \textcite{Butler1990Gender} used straw man fallacy and Texas sharpshooter fallacy when criticising \posscite{Page1987Sex} study, lacking this kind of respect for the works of others.  (We will discuss it in detail in \hyperref[subsec:concluding-scientific-postscript]{a separate subsection}.) What post-structuralists did to Rebecca Tuvel in the \textit{Hypatia} transracialism affair is also a typical case.

This issue is not only intellectual, but also ethical and political, having caused real exclusion and violence. Like Habermas's critique, post-structuralist-influenced discourse refuses to engage in normal dialogue. Instead, it just applies labels to the questioner. When someone raises a logical query about ``gender identity,'' it is not seen as an intellectual act of truth-seeking. It is interpreted as a political act, a form of discursive violence.

Rational dialogue is valuable because if critics are well-intentioned, dialogue fosters mutual understanding. If they are malicious, dialogue exposes their unreasonableness to onlookers and helps those who have misunderstanding, especially transgender individuals during their self-exploration. When this crude diagnosis becomes an unthinking, conditioned response, those who ask questions in good faith, including some transgender people themselves, are also crudely labelled and insulted as having ``transphobia'' or ``internalised transphobia.'' This strategy assumes that all transgender people think one way and all TERFs or conservatives think another way. Therefore, attacking someone based on their epistemology won't harm a transgender person. This is the real essentialism and epistemological hegemony. What they seem to do is ``finding a marginalised group, assimilating them with their theoretical framework, making them understand their experience with the post-structuralist terms, and ultimately expelling those who refuse post-structuralism.'' Is it ``empowerment'' or a form of intellectual colonisation?

An Asian, transgender, gender-abolitionist, anti-essentialist, working class (my mother was a textile worker, and my father is a machine worker) evolutionary biologist is clearly a multiple-dimensional ``Other,'' whether to conservatives and TERF (obviously), liberals (they like the ``born this way'' narrative), the mainstream academic community (I oppose both some conservative biologists' ``scientific'' gender essentialism and the ``scientific'' practice of searching for a ``neurological proxy'' for ``gender identity''), the feminist philosophy of science (I agree with many of their scientific models, such as \textcite{Arraiza2024After}. However, I strongly oppose their ``standpoint epistemology,'' this neurological model is correct only because it is consistent with an external reality), the mainstream essentialist transgender community (they believe in an internal, essential gender identity), or the mainstream anti-essentialist transgender community (they are often post-structuralist, like to talk about discourse and power, and are wary of people who talk about science)\ldots I have never seen post-structuralists stand with this kind of politically useless ``Other.'' They only support ``Others'' who comply with their philosophical stance.

%Peer review presupposes the existence of public standards. The very act of submitting a paper to peer-review implicitly acknowledges that there exists a public standard that transcends positions and can be commonly understood and judged. Without this public standard, peer review becomes purely a matter of ``taking sides'' -- ``if you are one of us (whether this `us' is divided by economic interests, academic factions, or `lived experience'), I'll pass you; if you are not, I'll reject you.'' Although this situation does exist in reality, at least in theory, this is not the stated purpose of peer review.

%Although the framework I use is evolutionary biology, neuroscience and cognitive science, my personal experience can be described perfectly well using Butler's gender performativity, Althusser's ``interpellation,'' and Foucault's diffuse power and discipline. My views and post-structuralism point to the same conclusion: identity is constructed in interaction with the internal and external environment, not an innate, fixed, \textit{a priori} essence. The more I understand their theory, the more I cannot understand why they attacked me. %(I am actually fully capable of rewriting this article using their frameworks and terms, see \cref{app:b}.)
%
%My views were rejected, most likely because I broke a discourse monopoly. In their eyes, the form is more important than the content. What kind of gender-view I actually described is irrelevant; what matters most is what language I used to say it. In that specific ideological framework, only discourse originating from specific thinkers is considered ``orthodox'' and ``safe.'' Scientific discourse, even if it reaches similar conclusions, is seen as a ``heretical,'' unwelcome external competitor. It threatens the purity and authority of that closed theoretical system.
%% The tools of critique are no longer used for difficult and painful analysis; they have become obscure postmodern jargon, a sophistry to dismiss any criticism as ``you don't understand,'' and a way to show allegiance to the tribe.
%
%Another reason is that they view so-called ``logocentrism,'' ``scientism,'' ``rationality,'' and ``grand narratives'' as a form of oppression even more terrifying than patriarchy and sexism. So much so that when they encounter someone speaking scientific language, who could originally be a companion of them, they immediately turn their guns on them, completely ignoring the fact that we have the same enemy. This is precisely an ideological grand narrative. \textcite{Lyotard1994Postmodern} argued that ``postmodernism is the incredulity of all metanarratives.'' However, ``incredulity of all metanarratives'' itself has become an unquestionable metanarrative. I understand that the self-reference of relativism is a very classical and old-fashioned critique that can be dated to Plato's critique of sophists. While in my case, it has truly become a new ``metanarrative'' that caused real exclusion and violence. This is where all sophisticated arguments failed.
%
%Post-Structuralists, queer theorists, and the communities influenced by it claim to protect the ``Other'' and respect ``lived experience.'' \footnote{Butler being interviewed by \textcite{Williams2014Gender}: I think I needed to pay more attention to what people feel, how the primary experience of the body is registered, ... \par \textcite{Butler2025Who}: ``Gender identity'' is a deeply felt sense of how one fits in the gendered scheme of things, the lived reality of one's own body in the world.} However, what they seem to do is respect only those experiences that are consistent with their own theoretical and ideological framework. A rationalist, materialist, transgender biologist trying to understand their own experiences and feelings in their own way does not seem to qualify as a valid, respectable human experience in their eyes. Furthermore, my critique of cooking and kinship as a child should absolutely be considered a valid personal experience of universalism and rationalism.
%
%They claim to respect human experience, but they arbitrarily set boundaries to maintain the absolute authority of a specific kind of experience, and systematically disrespect, censor, and suppress all other conflicting experiences. What they seem to do is ``finding a marginalised group, assimilating them with their theoretical framework, making them understand their experience with the post-structuralist terms, and ultimately expelling those who refuse post-structuralism.'' Is it ``empowerment'' or a form of intellectual colonisation?
%
%According to Foucault's definition of power, a queer theorist who teaches at a top university, can influence the thinking of a generation of students, can decide the academic future and fate of students, defines what are ``valuable'' questions and ``legitimate'' research methods in a specific field, publishes articles in famous journals, and is honoured by medias as a ``progressive thinker,'' should also be considered an essential node of power. Their self-proclaimed role of ``speaking for the marginalised,'' claiming to be on the margins of ``power'' and the Other, is sociologically untenable. From a Foucauldian perspective, this self-claiming is not an objective description of power, but a power strategy that attempts to produce authority for themselves. Similarly, the mainstream liberal transgender community produces the ``innate gender identity'' through discursive effect and, through a series of social movements, implements its own political claims into written law, setting the threshold for community recognition, medical resources, and legal identity. This is also a form of power. They are not powerless victims, but an actively operating ``regime of truth.''
%
%However, post-structuralists rarely include themselves in the list of ``powers'' when analysing the operation of power. They always use external power to define the ``centre'' and the ``Other.'' When essentialist transgender people (e.g., the mechanical materialist ``gender brain'' or the idealist ``gender soul'') declare ``do not deconstruct our gender identity,'' queer theorists tend to become very gentle and retract to an essentialist position (as in~\cite{Williams2014Gender}), even if they theoretically disagree with these people's essentialist claims (as in~\cite{Butler1990Gender}).
%
%This is not an accident, but an inherent structural problem of their narratives. Because the size of a politically effective group has the lowest limit (although this is not a mathematically precise boundary). If post-structuralists want to maintain the political effectiveness of their actions, they must stop when they touch this lowest limit. Therefore, those who ultimately receive the most attention and protection are definitely not the most special, most vulnerable, and most excluded people, but a self-declared ``minority'' group that is large and major enough to make the loudest sound and form effective political movements.
%
%%If they consistently applied their own philosophy, they would have to stand with the ``Other'' of the dogmatic transgender community, and then, after these Others have formed a power, continue to stand with the Other's Other\ldots until one day, they must stand with the individual. In this way, they would return to the position of individualism and universalism.
%
%A transgender, gender-abolitionist, anti-essentialist evolutionary biologist is clearly a multiple-dimensional ``Other,'' whether to conservatives and TERF (obviously), liberals (they like the ``born this way'' narrative), the mainstream academic community (I oppose both some conservative biologists' ``scientific'' gender essentialism and the ``scientific'' practice of searching for a ``neurological proxy'' for ``gender identity''), the mainstream essentialist transgender community (they believe in an internal, essential gender identity), or the mainstream anti-essentialist transgender community (they are often post-structuralist, like to talk about discourse and power, and are wary of people who talk about science)\ldots I have never seen post-structuralists stand with this kind of politically useless ``Other.''
%
%If they truly respect ``lived experience,'' most people subjectively feel that the earth is flat and still, that spicy is a sense of taste, that we have a complete, unified, innate self-awareness, but almost no one rigorously respects these subjective experiences. Including post-structuralism itself, from Foucault onwards, it is built on the systematic negation of the subjective experience of a ``complete self-subject.'' If they defend themselves by saying ``the subject is a social construct that originated in the 18th century''~\parencite{Foucault1994Order}, then gender identity is exactly the same, a social construct that originated in the 20th century, thus should also be deconstructed as they did with ``subject.''
%
%The respect for ``experience'' by post-structuralism and its branch, queer theory, is highly selective, conditional, and full of unstated political considerations. It is not a consistent principle. They are actually executing a hidden rule: ``We respect those subjective experiences consistent with our philosophy and political agenda.'' As revealed by my experience, what they love is the abstract ``other'' -- the ``conceptualised other'' that exists in their theoretical texts, fits their narrative, and can be used to prove the ``oppression of logocentrism.''
%
%The reason is simple. For questions about the shape of the earth, whether spiciness is a taste or pain, and the formation of self-identity, the scientific conclusions are unequivocal. Doubting these things would immediately associate them with anti-science conspiracy theories like ``flat-earth theory'' and would make post-structuralists lose all credibility. In contrast, gender identity is still a complex and controversial area in neuroscience and cognitive science, far from reaching a definitive conclusion. This is a kind of ``critique in the gaps'' like ``God in the gaps.'' They claim to be ``postmodern,'' but their behaviour in this regard is very pre-modern.
%%
%%This is, in fact, a Cartesian mind-body dualism. They claim to oppose Descartes, but they have secretly resurrected Descartes's ghost, attempting to preserve a domain in human consciousness that is exempt from the tests of natural science, arguing that the general scientific methodology based on empirical evidence is invalid in this category. This is a pre-modern fantasy doomed to fail.

\subsection*{Concluding Scientific Postscript}\label{subsec:concluding-scientific-postscript}

\textcite{Butler1990Gender} criticised \posscite{Page1987Sex} study on sex-determining gene \footnote{This gene is not the famous SRY gene but the Zinc finger Y-chromosomal protein gene, which was once an important candidate for the testis-determining factor.} in the subchapter \textit{Concluding Unscientific Postscript} of their famous work \textit{Gender Trouble}, proposing two main points:

\begin{enumerate}
    \item Why do we want to find a \textit{master gene} [sic]?
    \begin{quotation}
        The framework suggests a refusal from the outset to consider that these individuals implicitly challenge the descriptive force of the available categories of sex; the question he pursues is that of how the “binary switch” gets started, not whether the description of bodies in terms of binary sex is adequate to the task at hand.
    \end{quotation}
    \item Why do we want to find a \textit{mater gene} [sic] determining males?
    \begin{quotation}
        Ovary-determination is never considered in the literature on sex-determination and that femaleness is always conceptualized in terms of the absence of the male-determining factor or of the passive presence of that factor. As absent or passive, it is definitionally disqualified as an object of study.
        \par The concentration on the ``master gene'' suggests that femaleness ought to be understood as the presence or absence of maleness or, at best, the presence of a passivity that, in men, would invariably be active.
        \par Unfortunately for Page, there was one persistent problem that haunted the claims made on behalf of the discovery of the DNA sequence. Exactly the same stretch of DNA said to determine maleness was, in fact, found to be present on the X chromosomes of females. \footnote{The Zinc finger X-chromosomal protein gene.} Page first responded to this curious discovery by claiming that perhaps it was not the presence of the gene sequence in males versus its absence in females that was determining, but that it was active in males and passive in females (Aristotle lives!).
    \end{quotation}
\end{enumerate}

I agree with the first point that the hypothesis about a ``binary switch'' \footnote{\textcite{Page1987Sex}: The mammalian Y chromosome, by its presence or absence, constitutes a binary switch upon which hinge all sexually dimorphic characteristics. \ldots There must exist on the Y chromosome one or more genes whose products, directly or indirectly, determine all aspects of sexual dimorphism.} is over-simplified. Nevertheless, Butler's second critique was based on a misinterpretation of Page. He, actually, proposed four models:

\begin{enumerate}
    \item The X-encoded protein does not function in gonadal sex determination. (This is what Butler criticised.)
    \item The X and Y loci determine sex antagonistically.
    \item The X and Y loci determine sex in concert, while they are not interchangeable.
    \item The X and Y loci are interchangeable. However, in females, one of the two X chromosomes is inactivated. Therefore, a single dose (on the active X) determines female and two doses (on X and Y, respectively) determine male.
\end{enumerate}

The first model perfectly aligns with Butler's critique. Nonetheless, Page himself preferred the fourth model \footnote{\textcite{Page1987Sex}: Models 1, 2, and 3 all fit well with the prevailing notion of a dominantly acting sex-determining factor unique to the Y chromosome. A fourth model does not fit with this prevailing notion, but its simplicity is attractive.} and he used more than half of this section to discuss it. In other words, Page considered the most anti-Aristotelian model simple and attractive. This is a quantitative hypotheses (one dose vs two doses), this gene is not ``inactive'' or ``passive'' in females, on the contrary, the single dose actively determines the gonadal sex. Moreover, he did not use Aristotelian terms ``active/passive'' in the original paper. Yet Butler imposed these terms on their study and satirised him by saying ``Aristotle lives.'' I agree that analysing the cultural and philosophical load behind a scientific study is reasonable, which is helpful for better scientific practice. However, it is irresponsible to attribute theorist's analysis to scientists' own ``claiming.''

Another issue lies in Butler's criticise is that even Page really proposed a ``master gene'' for male determining, what is its relationship with ``Aristotle lives?'' The same structure can, on the contrary, be interpreted as ``female is natural and default (the first sex) and male is derived (the second sex).'' This was, in fact, the mainstream interpretation of the Sry gene before the discovery of the WNT4 gene \parencite{Ainsworth2015Sex}, which was, and is still, used by some feminists and transgender advocators to rebut the androcentrism and sex ideology \footnote{I coined this term to imitate conservative's ``gender ideology.'' }, including \posscite{Trump2025Defending} Executive Order (as in~\cite{Garcia2025McBride, Yeo2025Trump}).

If the same scientific theory can be interpreted as both ``male is active, female is passive'' and ``female is default, male is derived,'' it is untenable to choose a specific interpretation and use it to criticise the original study. This indicates that Butler has fallen into a confirmation bias. They presupposed a template of ``metaphysical binary opposition'' rooted in the ``Western philosophical tradition,'' then misinterpret the scientific text to comfort their preset framework, ultimately using this to ``prove'' their conclusion: I have found the unstated power behind the metaphysical preset of scientists. This is both the straw man fallacy (misinterpret others' viewpoints) and Texas sharpshooter fallacy (choose the interpretation that is easiest to criticise from two or many of them). It devastatingly demonstrates what happens when a post-structuralist, even the most elite of them, places their theoretical framework above the rigorous close reading of text.

Ironically, Page himself published a genomic study in 2023 \parencite{San2023Human}, which revealed that the ``inactive'' X chromosome regulates the active X chromosome in humans. In other words: in a binary opposition (active X/inactive X), the one previously thought to be secondary and supplementary is, in fact, necessary for the privileged one. This is a classic deconstructive, almost Derridean, finding. This shows that scientists like Page don't care if the conclusion is Aristotelian or Derridean. What matters is the evidence.
