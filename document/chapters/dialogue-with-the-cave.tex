I posted my initial viewpoint on three different platforms. On two of them, I received insults or was banned. On the third, I received a more subtle form of violence. They asked me, ``Isn't this just queer?'' or ``You've reinvented queer theory.'' I told them, ``This is not queer. The difference between queer theory and my viewpoint is like the difference between Giordano Bruno's pantheistic philosophy and the Hubble model of the universe. They are within two completely different disciplines and frameworks. This is also not a reinvention.\footnote{This is a fact. I started thinking entirely based on the first principles of evolutionary biology. I read \textit{Gender Trouble} after I completed my introspection.} You can't say that Hubble reinvented Bruno's pantheism.'' When they interpreted my points as ``reinventing queer theory,'' they did what Derrida spent his whole life opposing: they eliminated the uniqueness of the Other. Some others said that my viewpoint was ``solipsism'' or ``Landian Accelerationism'' and I completely fail to understand where these labels came from.\footnote{``Accelerationism'' is the most absurd label that I have ever encountered. Both left-wing accelerationism and Land's right-wing accelerationism are theories about ``pushing the inherent logic of capitalism to its limits, leading to its self-destruction and creating a new world.'' The difference lies in the ``new world'' they envision (socialism/anti-humanism). Therefore, queer theory is more like ``gender accelerationism.'' It is, like other accelerationsims, about exploring a system, finding its inherent, self-destructive features, exploiting and accelerating them until the system collapses within its own mechanism. } And they responded to my analysis of ``Indo-European-centric cultural invasion in gender study and LGBTQ terms'' with the personal attack, ``Your mom is also a form of cultural invasion.'' (\cref{fig:6})

\begin{figure}[htbp]
    \centering
    \includegraphics[width=\textwidth]{F6}
    \caption{The author's experience of being insulted in dogmatic communities on Zhihu, Xiaohongshu, and Reddit. Chinese comments translated using AI.\label{fig:6}}
\end{figure}

Later, with the goal of so-called ``community solidarity,'' I wanted to participate in a more moderate way, no longer mentioning that ``gender is an evil concept,'' ``we should completely abolish gender.'' I still couldn't understand why, after sharing an article about the lack of inclusivity of ASAB for intersex people (a brief version was mentioned earlier; the full version is in \cref{app:a}) in one community, I was immediately banned. Another community accused me of ``gender essentialism'' for thinking that ``phenotypic sex is more accurate and inclusive than ASAB.'' Moreover, I don't believe ``phenotypic sex'' is a real thing at all (as we discussed earlier). It's just a model created by humans, a tool, like ASAB, to represent the gender socialisation patterns. If ``phenotypic sex'' can more accurately represent gender socialisation patterns than ASAB, then of course we should use it. A more precise representation means higher inclusivity, recognising a richer biological and sociological diversity. As a man-made scientific model, what does it have to do with ``essentialism''? I don't believe there is a metaphysical essence or a Platonic Form of ``male'' and ``female'' behind this model. I never discussed the metaphysical status of this concept at all.

I was extremely puzzled. Why did it seem that no one was willing to read others' viewpoints with basic respect and seriousness? This viewpoint is moderate and morally necessary. I am curious if they really believe that the person in the thought experiment is a ``cisgender female.'' If they had seriously considered my point, they could not have reached this conclusion. It is extremely absurd and completely violates their own values.

It took me a long time to understand that their standard of ``inclusivity'' seems to be whether it sounds comfortable to them, and this standard of comfort seems to be arbitrary and without any fixed logic. The inclusivity I understand is using more accurate scientific models to represent the human biological and sociological diversity accurately. Suppose a model can describe and distinguish more diversity (e.g., different types of intersex people). In that case, it is more precise, and therefore more inclusive, than a model that flattens or ignores these situations (like ``assigned sex at birth'').

The prerequisite for any dialogue is that both parties are equal, mutually respectful, willing to listen, engage with the opponent's actual arguments, and willing to modify their own positions based on logic and empirical evidence. A dogmatic community completely refuses to engage in these activities, refusing to understand what others are saying, refusing to explain the specific content of their own views. They don't care what ``gender essentialism'' or ``solipsism'' really are. It's a trumped-up charge like ``blasphemy'' or ``witchcraft,'' the ultimate political tool an ideological system uses when facing a challenge it cannot respond to with logic.

Extending from this, what I find even more incomprehensible is why many people not only lack respect for others' views, but also lack respect for their own. Like the cases we have already discussed:

Conservatives also believe that sex is only related to gametes. Then, assigned sex at birth is technically impossible to implement, and almost all gender-segregated facilities should be completely abolished, because using bathrooms based on gametes is meaningless, and most people will never know if they are chimaeras capable of producing two types of gametes. While they dare not, as I do (I also believe sex should be strictly limited to gametes), push this view to this point.

This even includes some conservative scientists, like \textcite{Sokal2024Sex}, who claim that sex is an ``objective biological reality'', ``determined at conception and observed at birth.'' I agree that sex \textit{sensu stricto} is a biological reality. While according to \textcite{Jones2006Gamete}, ``Conception occurs when a sperm and an ovum fuse to become a zygote,'' ``The process of fertilisation, or conception, involves fusion of the nucleus of a male gamete (sperm) and a female gamete (ovum) to form a new individual.'' At ``conception'' (fertilisation), it is just a zygote. A zygote is just one cell. It cannot produce gametes and does not have a body. It has neither sex \textit{sensu stricto} nor phenotypic sex. Even if its genome will guide it to develop into a typical phenotypic male or female under normal conditions, it will not necessarily be a typical phenotypic male or female. To give a simple example, if a normal 46, XY zygote loses its Y chromosome during the first mitotic division after fertilisation, it will develop into a 45, X0/46, XY chimaera (mixed gonadal dysgenesis) or a 45, X0 individual with Turner syndrome. \parencite{Gravholt2017Clinical, Jacobs1997Turner, Lopes2014Mosaicism} Richard Dawkins, as an evolutionary biologist, cannot be unaware of this.

Liberals and some gender studies scholars believe that gender identity is an internal self-awareness, and imposing an identity is wrong. Then, the gender identity of most people should be undefined. Their imposing the label ``cisgender'' is not only self-contradictory, but also implies that transgender people are ``cisgender'' at birth. Liberals also believe that all gender identities are completely equal. Therefore, these gender identities can only be seen as a metaphor logically. Then we should completely abolish gender-segregated spaces. Yet they advocate for ``the right of transgender women to use women's bathrooms,'' which is a meaningless proposition within their own framework.

Similar to the distinction between ``emotional inclusivity'' and ``epistemological inclusivity'' we mentioned earlier, many concepts within the EDI movement remain incomprehensible to me.

I've never understood why the English-speaking world chose to fix \textit{man} as masculine, rather than reviving the Old English masculine word \textit{wǣpnedmann} and modernising it to \textit{weaponedman} (or shortening it to something like \textit{poman}). In this way, we only need to modify the masculine \textit{man}, preserving all words with the suffix \textit{-man}. This solution has minimal impact and perfectly solves the problem of ``why \textit{woman} includes \textit{man}'': \textit{woman} is a kind of \textit{man}, so of course it includes \textit{man}. In contrast, fixing \textit{man} as masculine required modifying almost all words with the suffix \textit{-man}, which would be a far more troublesome and could not resolve the issue of \textit{woman}.

From a linguistic descriptivist perspective, both meanings of \textit{man} are lived modern English phenomena, neither superior nor inferior. Both solutions are artificial normative approaches. This isn't a competition between descriptivism and normative, but rather between two different normative approaches. From a linguistic normative perspective, my proposal is both ethically and cost-effectively the best one. They declared that this linguistic normative approaches is for ``equality'' and ``inclusivity,'' but they were unwilling to use the most equal and most inclusive solution.

The paper by \textcite{Oberle2023Benefits}, which we discussed earlier, remains the same. The conclusion they reach at the end of the paper is to recommend that botanists not address the reproductive function of plants \textit{gender}. Ironically, the second author, Emily Fairchild, is engaged in so-called ``gender studies'' and ``queer studies.'' Yet, in this paper, they try to fix a stable meaning for a word. They try to terminate the ``dissemination'' of the word \textit{gender}, turning it from a signifier constantly flowing in endless différance into a well-defined, strictly controllable scientific concept, and want to give this word a meaning that is unambiguous, unmediated, and fully ``present.'' This is a typical logocentric obsession with presence, and is extremely anti-Derridean. In this scenario, \textit{gender} has already escaped its original domain (sociology) and established a rhizomatic new connection in another discipline (botany), yet they try to crudely pull it back, re-establishing rigid, hierarchical disciplinary boundaries. This reterritorialisation of thought is extremely anti-Deleuzian. Their attempt to establish a ``regime of truth'' and implement a cross-disciplinary linguistic discipline is extremely anti-Foucauldian.

They completely ignored that they are also using ``sex'' and ``rhizome'' in ways inconsistent with biology, and that the word gender itself was borrowed from linguistics. The ``rhizome'' in botany is also hierarchical; although it sometimes looks like a network, its branches are only mechanically connected, not physiologically. I think the mycelium of fungus is a better metaphor, because the hyphae of the branches can reconnect, which is called anastomosis. This is more consistent with Deleuze's conception: a non-hierarchical, non-central, interconnected network where any point can connect to any other point. Botanists usually don't send a commentary to a philosophy journal ``condemning'' Deleuze's ``misuse'' of the word ``rhizome.''

Similarly, queer theory and the communities influenced by it claim to protect the ``Other'' and respect ``lived experience.'' However, the concept of ``lived experience,'' which originates from Husserl's phenomenology, has precisely been deconstructed by their parent post-structuralism (especially Derrida) as an obsession with the metaphysics of unmediated, pure ``presence.'' What they seem to do is respect only those experiences that are consistent with their own theoretical framework. A rationalist, materialist, transgender biologist trying to understand their own experiences and feelings in their own way does not seem to count as a valid, respectable human experience in their eyes. Or perhaps my experience is a ``dead experience.'' Furthermore, I believe that my practice of the categorical imperative as a child before reading Kant should absolutely be considered a valid personal experience of universalism and rationalism.

They claim to respect human experience, but they arbitrarily set boundaries to maintain the absolute authority of a specific kind of experience, and systematically disrespect, censor, and suppress all other conflicting experiences. What they seem to do is ``finding an Other, assimilating the Other with their theoretical framework, making the Other understand and speak of their experience with the post-structuralist terms, and ultimately expelling those who refuse the post-structuralist paradigm.'' Is it empowerment or a form of intellectual colonisation?

Although the framework I use is evolutionary biology, neuroscience, psychology, and cognitive science, my personal experience can be described perfectly well using Butler's gender performativity, Althusser's ``interpellation,'' and Foucault's diffuse power and discipline. My views and post-structuralism, although one from an internal individual perspective, one from a social perspective, both point to the same conclusion: identity is constructed in interaction with the internal and external environment, not an innate, fixed, \textit{a priori} essence. I am actually fully capable of rewriting this article using their frameworks and terms; I just don't want to. (I did write it, see \cref{app:b}.) As I said in the introduction, a firm commitment to science and Enlightenment rationality is an inseparable part of my ``self.'' At the same time, my analysis of ``using gendered pronouns to disrupt gender is Indo-European-centric'' is also fully consistent with post-colonialism. The more I understand their views, the more I cannot understand why they reject me.

My views were rejected, most likely because they broke a discourse monopoly. In their eyes, the form is more important than the content. What kind of gender worldview I actually described is irrelevant; what matters most is what language I used to say it. The tools of critique are no longer used for difficult and painful analysis; they have become obscure postmodern jargon, a sophistry to dismiss any criticism as ``you don't understand,'' and a way to show allegiance to the tribe. In that specific ideological framework, only discourse originating from specific thinkers is considered ``orthodox'' and ``safe.'' Scientific discourse, even if it reaches similar conclusions, is seen as a ``heretical,'' unwelcome external competitor. It threatens the purity and authority of that closed theoretical system.

Another reason is that they view so-called ``logocentrism,'' ``scientism,'' ``rationality,'' and ``grand narratives'' as a source of oppression even more terrifying than patriarchy and sexism. So much so that when they encounter someone speaking scientific language, who could originally be a companion of them, they immediately turn their guns on them, completely ignoring the fact that they have the same enemy. This is precisely a manifestation of ideological struggle and grand narratives. \textcite{Lyotard1994Postmodern} argued that ``postmodernism is the incredulity of all metanarratives.'' But what will happen when ``incredulity of all metanarratives'' itself becomes an unquestionable metanarrative?

This is what I find most sad and desperate: almost no one—whether conservative, liberal, or post-structuralist—is interested in pure, consistent, self-coherent thought. No one is willing to bear the full logical consequences of their own ideology. They are only interested in winning. They are keen to defend their tribe, traditions, emotions, and political goals, and will adopt any temporarily useful premise. Once these premises become inconvenient or lead to unpleasant conclusions, they will quietly abandon them. They proclaim a series of premises but refuse to live in a world built on those premises as axioms. This is a destructive act full of intellectual bad faith.

This is why Queer theory will not succeed. Most people do not care if a political concept (yes, gender is a political concept) is meaningful or consistent. Otherwise, almost all political camps would have already collapsed. As we have argued earlier, since ``agender'' is also a ``gender'', ``gender'' has already been worthless. In contrast, it did not collapse. Queer theorists overestimate the political importance of logical consistency and underestimate the insistence of power to maintain an empty while useful symbol.

Furthermore, this is itself a meta-level ``intellectual inconsistency.'' In post-structuralist tradition, especially Derridian theory, all texts are inherently unstable. However, Queer strategy actually implies that ``stable text is a normal, while unstable text will collapse.'' This is a pre-Derridian metaphysics illusion about ``presence.''  How will queer theorists respond if others declare that ``the inflation of gender is an aporia, we should enjoy it''? Additionally, given the two promises: ``all texts are inherently unstable'' and ``unstableness will cause collapse,'' the logical conclusion is that all texts must collapse. Obviously, we all know that the real world, especially politics, does not work in this way.

This reminds me of \textcite{Habermas1987Philosophical}'s critique of Michel Foucault. He argued that Foucault's reduction of everything to ``power'' made rational communication impossible and destroyed the foundation for building a truly free society based on communicative rationality. This is perfectly shown in the communities deeply influenced by Foucault. Foucault has taught some social activists that every communication is just a move in a power game. When you tell people that there is no truth, only ``regimes of truth''; no reason, only ``power/knowledge''; no living author, only the ``author function''; no common, universal human destiny, only various discourses fighting each other—what do you expect them to do? Under this influence, if someone presents a challenging argument, you have no obligation to listen to its reasoning; you only need to analyse its ``discursive effect''\footnote{In my case, it was not even a real discursive effect, but their own prediction of a possible discursive effect. } and see it as an ``operation of power.''

According to this logic, Foucault himself should be the first to be judged, because ``morally judging and punishing an author based on discursive effects'' is precisely the real discursive effect produced in the discourse network by Foucault as an author function. Regardless of whether this was his own intention,\footnote{I know that it was not his intention, but completely anti-Foucauldian. } it has caused exclusion and harm in the real world. Foucault is the only author who cannot, like others (such as myself), claim ``that's not what I meant,'' because he himself (as the author function) has forbidden himself (as a natural person) from doing so.

This is not just a ``vulgarisation'' that has occurred in a few specific communities influenced by post-structuralism. As Habermas criticised Foucault, Foucault used reason to destroy reason, which is a performative contradiction, already containing the seeds of self-destruction.

The greatest strength of Habermas's critique is its immediacy and structural nature. Habermas's critique was fully articulated before Foucault's theory was ``vulgarised'' by the followers we see today. It did not need to wait for any subsequent events for verification, because it targeted an inherent structural flaw—the ``performative contradiction'' of ``using reason to destroy reason.'' If Foucault was still using academic language, making arguments, writing, and expecting to be understood, he was inevitably caught in this contradiction. It is somewhat ironic that Foucault~\parencite{Boesers1977Folter} appealed to intent to defend himself; he claimed, ``I was talking about the French \textit{raison}, not the German \textit{Vernunft}; \textit{raison} is instrumental rationality, \textit{Vernunft} also includes value rationality, they are not the same.'' In this event, the natural person Foucault ``resurrected'' the ``dead'' author Foucault to practice Habermas's communicative rationality, trying to use rational argument to make others understand, ``I am not destroying reason.''

Out of the principle of communicative rationality, I will accept Foucault's own defence. We must also acknowledge that this phenomenon likely existed before Foucault; it could not have been created by him, at least the intellectual dishonesty of conservatives cannot be attributed to him. However, analysing it with their own theory: Foucault's description of the world as an anonymous, impersonal discursive field, claiming that the author's own intention is not essential, is not just descriptive but also performative. His works, through discursive effects, ``reproduced'' this impersonal discourse operation, providing people with a powerful new discourse and new arguments as intellectual weapons, allowing them to ignore the author's intent comfortably, refuse interpretation in good faith, and dismiss demands for intellectual coherence and logical self-consistency as ``power'' or ``discipline,'' equating it with ``instrumental rationality'' or ``governmentality,'' thereby further exacerbating this pre-existing problem. Foucault himself should have been the one to foresee this consequence most accurately.

Present-day post-structuralist scholars also live in such a performative contradiction, because academic activities presuppose the validity of rationality. Writing papers, organising arguments, citing literature, and responding to criticism all require the use of the ``reason'' they claim to critique, and require the expectation that others will sincerely understand them. Peer review presupposes the existence of public standards. The very act of submitting a paper to peer review implicitly acknowledges that there exists a public standard that transcends positions and can be commonly understood and judged. Without this public standard, peer review becomes purely a matter of ``taking sides''— ``if you are one of us (whether this `us' is divided by economic interests, academic factions, or `lived experience'), I'll pass you; if you are not, I'll reject you.'' Although this situation does exist in reality, at least in theory, this is not the stated purpose of peer review.

According to Foucault's definition of power, a queer theorist who teaches at a top university, can influence the thinking of a generation of students, can decide the academic future and fate of students, defines what are ``valuable'' questions and ``legitimate'' research methods in a specific field, publishes articles in famous journals, and is honoured by the media as a representative of ``progressive thought,'' should also be considered an essential node of power. Their self-proclaimed role of ``speaking for the marginalised,'' claiming to be on the margins of ``power'' and the Other, is sociologically absurd. From a Foucauldian perspective, this self-claiming is not an objective description of power, but a power strategy that attempts to produce a kind of authority for itself through discursive effects. At the same time, the mainstream liberal transgender community produces the ``innate gender identity'' through discursive effect and, through a series of social movements, implements its own political claims into written law, setting the threshold for community recognition, medical resources, and legal identity. This is also a form of power. They are not powerless victims, but an actively operating ``regime of truth.''

However, post-structuralists never include them in the list of ``powers'' when analysing the operation of power. They always use external power to define the ``centre'' and the ``Other.'' When essentialist transgender people (e.g., the mechanical materialist ``gender brain'' or the idealist ``gender soul'') declare ``do not deconstruct our gender identity,'' queer theorists tend to become very gentle, even if they theoretically disagree with these people's claims (as in~\cite{Butler1990Gender}).

This is not an accident, but an inherent structural problem of their narratives. Because the size of a politically effective group has the lowest limit, although this is not a mathematically precise boundary. If post-structuralists want to maintain the political effectiveness of their actions, they must stop when they touch this lowest limit. Therefore, those who ultimately receive the most attention and protection are definitely not the most special, the most vulnerable, and the most excluded people, but a self-declared ``minority'' group that is large and major enough to make the loudest sound and form effective political actions.

If they consistently applied their own philosophy, they would have to stand with the ``Other'' of the dogmatic transgender community, and then, after these Others have formed a power, continue to stand with the Other's Other\ldots until one day, they must stand with the individual. In this way, they would return to the position of individualism and universalism.

A transgender, gender-abolitionist, anti-essentialist evolutionary biologist is clearly a multiple ``Other,'' whether to conservatives (obviously), liberals (they like the ``born this way'' narrative), the mainstream biological community (I oppose both Richard Dawkins's ``scientific'' gender essentialism and the search for a ``neurological proxy'' for ``gender identity''), the mainstream essentialist transgender community (they believe in an internal, essential gender identity), or the mainstream anti-essentialist transgender community (they are often post-structuralist, like to talk about discourse and power, and are wary of people who talk about science)\ldots I have never seen post-structuralists stand with this kind of politically valueless ``Other.''

If they truly respect ``lived experience,'' most people subjectively feel that the earth is flat and still, that spicy is a sense of taste, that we have a complete, unified, innate self-awareness, but almost no one rigorously respects these subjective experiences. Including post-structuralism itself, from Foucault onwards, it is built on the systematic negation of the subjective experience of a ``complete self-subject.'' If they defend themselves by saying ``the subject is a social construct that originated in the 18th century''~\parencite{Foucault1994Order}, then gender identity is exactly the same, a social construct that originated in the 20th century, thus should also be deconstructed as they did with ``subject.''

The respect for ``experience'' by post-structuralism and its branch, queer theory, is highly selective, conditional, and full of unstated political considerations. It is not a consistent principle. They are actually executing a hidden rule: ``We respect those subjective experiences consistent with our philosophy and political agenda.''

The reason is simple. For questions about the shape of the earth, whether spiciness is a taste or pain, and the formation of self-identity, the scientific conclusions are unequivocal. Doubting these things would immediately associate them with anti-science conspiracy theories like ``flat-earth theory'' and would make post-structuralists lose all credibility. In contrast, gender identity is still a complex and controversial area in neuroscience and cognitive science, far from reaching a definitive conclusion. This is a kind of ``critique in the gaps,'' similar to ``God in the gaps.'' They claim to be ``postmodern,'' but their behaviour in this regard is very pre-modern.

This is, in fact, a Cartesian mind-body dualism. They claim to oppose Descartes, but they have secretly resurrected Descartes's ghost, attempting to preserve a domain in human consciousness that is exempt from the tests of natural science, arguing that the general scientific methodology based on empirical evidence is invalid in this category. This is a pre-modern fantasy doomed to fail.