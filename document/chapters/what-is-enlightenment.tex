\textcite{Kant1996Answer} proposed the famous definition of Enlightenment, ``\textit{Enlightenment} is the human being’s emergence from his self-incurred minority. Minority is inability to make use of one’s own understanding without direction from another. This minority \textit{is self-incurred} when its cause lies not in lack of understanding but in lack of resolution and courage to use it without direction from another.'' The emergence is accomplished not in other ways, but only in \textit{Sapere aude} (dare to know). This motto is not only epistemological, but more importantly, ethical and political.

\textcite{Kant1996Answer} gave three examples of ``minority,'' ``\ldots a book that understands for me, a spiritual advisor who has a conscience for me, a doctor who decides upon a regimen for me \ldots'' This is a precise description of the communities I encountered. They loudly proclaim their opposition to authority, they haven't actually escaped what Kant called ``minority.'' They've simply changed their ``guardian.'' The previous guardian might have been the church, tradition, or a monarch. The current guardian is their theory.

When they see somebody with a different opinion, they don't need to use their curiosity to ask, ``What did they say? Why did they reach this conclusion?'' Instead, they open their handbook. The handbook reads: ``Using biology to explain gender = gender essentialism = oppression; transgender + gender essentialism = internalised transphobia.'' Thus, the comprehension ends. They let a theory framework understand others for them, hiding behind their guardian (the theory), abandoning the ethical responsibility.

Admittedly, understanding does not necessarily lead to ideal communication and consensus, they might still refuse my theory even if they read it, just like I refused theirs. Nevertheless, ``Knowing'' others is the first step of communication. Even if we still refuse each other's theory, at least we can reach a \textit{meta-consensus}: I thought \ldots, while you thought \ldots, we cannot agree with each other because \ldots~ However, unfortunately, they did not participate in a mutual dialogue. The asymmetry is manifested in the fact that I invested huge intellectual effort to understand them, while they did not do the same thing to me.

%Let us illustrate what this \textit{meta-consensus} might look like in our case:

%After delving into post-structuralism, queer theory, and transgender activism, I can fully empathise with how mainstream transgender people would view my gender abolitionism, because post-structuralist philosophers have done the exact same thing to me. I negated ``gender identity,'' a concept many see as a survival need; Foucault negated ``reason,'' a concept I see as my only refuge. I see gender identity as a prison that enchained our soul; you see reason as an oppressive tool. This is precisely what makes us unable to have meaningful dialogue. Because this is not a matter of philosophical theory. We both see the ``liberation'' in the other's eyes as the most significant threat, and the oppression in the other's eyes as the most important survival need. As Saussure pointed out, a word has no intrinsic meaning, so there is no transcendent metaphysical standard to judge whose use of ``reason/rationality'' and ``gender'' is correct. There is no Platonic ``Form of Reason'' or Aristotelian ``Essence of Gender,'' no theory or empirical evidence that can bridge this gap.

%This may not be a perfect meta-consensus, because,

I believe that my research in biology and my attempt to comprehend their philosophy, though methodologically different, share a common epistemic root: a ``desire for knowledge'' characterised by pure and irrepressible wonder and curiosity.

``Dare to know'' is a kind of modesty, which means ``I NEED to know,'' acknowledging our own epistemological deficiency. This is the drive to bridge the gap between the self and the external, and the yearning to approach truth. Love of Truth and Love of Neighbour are fundamentally connected. Scientists acknowledge that nature is the ``other,'' possessing its own logic independent of human will. Studying nature means humbly listening to its voice and understanding its laws, independent of our own. Similarly, true understanding of others means acknowledging them as independent rational beings with their own theories I have not yet comprehended. Reading their theories, as I have done, is ``daring to know'' their world, even if it would make me uncomfortable. They abandoned this curiosity along with abandoning science, allowing a guardian to understand others on their behalf.
%``Knowing'' is essentially respecting and asking favours from others.

We need to acknowledge not only our epistemological deficiency of nature and of others, but also our deficiency of ourselves. My analysis of myself at the beginning is precisely ``dare to know.'' I did not ``disrespect'' my subjective experience. Otherwise, I would have simply said, ``How could I be a girl? It seems I'm crazy. Period.'' But I did not do that. On the contrary, I very seriously analysed my subjective experiences of gender dysphoria and gender euphoria, placing them on the same epistemological level as all other knowledge. When I proposed that ``natural selection cannot encode abstract concepts,'' I was not just discussing evolutionary biology and neuroscience. I also recalled seeing my younger siblings and cousins randomly sucking things as kids, and elders teaching them, ``Don't suck that, it's so dirty.'' Our naive knowledge and scientific knowledge are interconnected and mutually shape each other, forming a rhizome -- or rather, mycelium.

This reminds me of Baruch Spinoza's words: ``\textit{Non ridere, non lugere, neque detestari, sed intelligere}'' (not to ridicule, not to lament, not to detest, but to understand). ``Dare to know'' is not merely an epistemological pursuit; it represents the zenith of human affect -- a manifestation of love and beatitude.

The Enlightenment spirit promotes the mutual understanding and recognition of human diversity not only through communicative rationality, but also through scientific rationality. For example, Johann Friedrich Blumenbach, through comparative anatomical studies of skulls, reached scientific conclusions quite close to modern population genetics: ``the highest diversity is found within the black race,'' ``the five races are not clearly divided, with transitional states between them,'' ``there is no scientific evidence for superiority or inferiority among different races,'' and ``the intelligence of black people is not inferior to any other race''~\parencite{Rupke2019Johann}.

I know post-structuralists will immediately say: his views and theories were later appropriated by ``scientific'' racist theories, which is a manifestation of Foucault's theory of the ``episteme,'' it exceeded the ``conditions of possibility'' of knowledge at his era. Nevertheless, this also shows that Foucault's ``power'' is not an omnipresent, omnipotent, insurmountable thing that exists beyond the scope of human cognition. (Otherwise, how did Foucault himself know it?) \textcite{Foucault1978History} himself also claimed that ``power'' is just a nominalist, functional concept for analysis: ``Power is not an institution, and not a structure; neither is it a certain strength we are endowed with; it is the name that one attributes to a complex strategical situation in a particular society.''

Scientific research loyal to evidence can provide us with a relatively reliable picture of the world and help us understand complexity and diversity. Science is proved to be a reliable way to understand objective reality. Otherwise, it would be incredible for Blumenbach to have reached conclusions largely consistent with population genetics under a completely different ``episteme.'' \footnote{This is the classic no-miracles argument, but in Blumenbach's case, it has an ethical and political meaning. }

We have previously mentioned Page's genomic study in 2023 revealed that the ``inactive'' X chromosome regulates the active X chromosome in humans \parencite{San2023Human}. A similar instance is that, in the 1920s, Richard Goldschmidt proposed a theory of sex (phenotypic sex) determination based on his hybridisation experiments of two \textit{Lymantria} species. In his ``balance theory of sex,'' the sex is determined by the balance and the quantitative relationship between the male and female factors. The sex chromosomes carry some of them and autosomes carry the others. Two sexes are not clearly distinct opposites. An individual is on a position in the female-to-male continuum \parencite{Dietrich2016Experimenting}. This is quite consistent with the modern genomic research: the development of fetal gonads depends on the antagonism between two sets of gene networks, which are widely distributed on sex chromosomes and autosomes in the genome \parencite{Graves2010Homologies}.

Goldschmidt, similar to Einstein, was a German Jewish scientist who fled Nazi persecution. It is well demonstrated that science itself is not oppressive, but other, a tragic hero. Historically, a true Kallipolis ruled by pure reason has never existed; reason has almost always been the dissenting voice, weak, excluded, and suppressed by power. Oppression stems from politics-influenced academic misconduct, cheery-picking of evidence, circular reasoning, whether it is scientific racism or binary gender theory. Rigorous science, which is based on evidence and uses mathematics to understand complexity and diversity, has always been the glorious tradition of Enlightenment.

However, the value of the Enlightenment extends far beyond its epistemological contribution to science. Its most profound legacy lies in the construction of an independent, autonomous subject. In the ideal world of Enlightenment, reason allows us to understand diversity; in the darkest corners of human history, reason empowers the individual to preserve their soul when their body is being destroyed. The rational subject constructed by the Enlightenment, with its inherent reason and autonomous judgment, is an important and powerful weapon against dehumanisation.

That sentence in the Mauthausen concentration camp is deafening: If there is a God, He will have to beg my forgiveness. \footnote{Wenn es einen Gott gibt muß er mich um Verzeihung bitten.} \parencite{Lassley2015Defective} When this nameless prisoner wrote this sentence on the wall of their cell, they were choosing human autonomy and agency, using clear and unshakable reason to confront the evil and irrationality. What they did was to use the independent, strong, agentic, and liberating subject of self to resist the oppression of Nazism. In the face of ultimate evil, where human dignity was systematically stripped away, this individual refused to seek comfort in the ultimate ``spiritual guardian'' -- God. Instead, they chose to rely solely on their own reason and moral judgment to confront the absurdity of their fate. We are ignorant of the ethnicity of this nameless prisoner -- both German political prisoners, Polish, Soviet and Yugoslav prisoners of war, and Dutch and Hungarian Jews were held in the Mauthausen concentration camp. At that moment, their resilience transcended all identities. Behind this sentence is a shared destiny, one that still intimately connects the \textit{Aufklärung} and Haskalah, as well as the Enlightenment, \textit{Lumières}, Nahda, and the May Fourth Movement. When the external world strips an individual of all social dignity, ethnicity, gender, philosophical and political views, and even their physical freedom, the ``rational subject'' constructed by the Enlightenment becomes the indestructible core of resistance.

Critics from the post-structuralist camp often argue that ``Universal Reason'' is a grand narrative constructed by the privileged (white, male, colonial) to erase the ``Other.'' This is a fundamental misunderstanding of the concept. A rule is universal only when it applies to everyone without exception. A privilege, by definition, is an exception granted to a specific group; it cannot apply to everyone and is therefore not universalist. Therefore, a firm commitment to universalism and rationalism is not a tool for privileged groups to defend their status -- logically, it cannot be. On the contrary, it is the last refuge of the weak.

I am not privileged. I was born in a small town in a third-world country. I am not European, not white, not male (well, that depends on how we define ``male''), not cisgender, not heterosexual, and not a coloniser. My mother was a textile worker, and my father was a machine worker. I was constantly excluded and bullied in elementary and middle school because of my personality. After growing up, as a transgender biologist who supports gender abolition, I have become an outsider to all political camps. I have no group, no tribe, no community, no recognised identity to shelter me. The only thing I have is these principles that should apply to all people equally and without exception. The bullies at school could pull down my pants and throw away my shoes, the discourse police in online communities can distort my words and ban my account, but no one can stop me from, as Kant said, having the courage to use my own \textit{reason} to point out how arbitrary and baseless their rules are.

\textit{A man can be destroyed but not defeated.} Everything I have done is precisely like the anonymous prisoner in the Mauthausen concentration camp, using the powerful, rational subject of self from the Enlightenment to resist the irrational violence of the world. Universalism and rationalism are not a claim to power, but a relentless pursuit of a pure justice, a justice that can even empower an ultimate outsider like me. The goal of Enlightenment is precisely to use the light of reason to enlighten all the darkness of pre-modern superstition, privilege, dogmatism, tribalism, and obscurantism, to create a world where there is no more of this pain and harm, a world where no child will be bullied for wearing a pair of ``wrong'' shoes.