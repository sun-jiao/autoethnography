I know what the post-structuralists and the communities influenced by them will say when they read this:

\begin{enumerate}
    \item We acknowledge that you have had a personal experience different from ours, and we fully respect your relentless pursuit of universal reason, which began with the division of household labour in your childhood. However, you cannot universalise your experience to everyone.\\
    — This may be the kindest and most moderate critique, but unfortunately, it is an invalid one. Because my personal experience is precisely the pursuit of universalism, ``the universalisation of universalism'' is a tautology that provides no new information and thus is not a valid critique. If I said, ``I like to eat apples, so I will pursue the universalism of eating apples,'' that would constitute ``the universalisation of personal experience.'' In contrast, ``I support universalism, so I want all principles to be universally applied'' is not ``the universalisation of personal experience.''
    \item You are using your own experience to destroy the narrative legitimacy of lived experience. ``Autoethnography'' is usually used to present the special experiences of marginalised groups. You are using it to argue for a universalist, revolutionary grand narrative (the abolition of gender), which is a performative contradiction. \\
    — I readily admit this. That's right, just as Foucault used reason to destroy reason, you have done it too. Moreover, your so-called ``respect for lived experience'' is itself selective. As I have already mentioned, you yourselves are systematically denying the subjective experience of an ``independent subject.'' If we set post-structuralism as the ``centre,'' I am a marginalised person. Denying this perspective is a kind of self-exceptionalism.
    \item The science you love is not neutral, but a historically constructed ``episteme'' that is inseparable from the operation of power. \parencite{Foucault1972Archaeology} It is a local knowledge originating from Europe. To treat it as a meta-discourse is a failure to reflect on your own position of power.\\
    — This is the point most worthy of a response, and I will respond to it with the highest respect I can give to an idea.
\end{enumerate}

No. Science is not something that originated from the European Enlightenment, but rather a natural extension of the naive human epistemology.

Except for solipsism, any form of philosophy, ethics, or sociology is to some extent a form of realism, relying on two propositions that cannot be ultimately justified: ``the external world is real, I am not a brain in a vat'' and ``all people have the same mind and subjective feelings as I do'' (the problem of other minds). Although they can suspend ontological problems in their theories, this is logically self-evident. It is a prerequisite for their work to have practical meaning.

Otherwise, I could perfectly well believe that everyone except me is essentially a neural signal inputted into a brain in a vat, or that I live in The Truman Show and everyone else is a biochemically created philosophical zombie. Then, discussing ethics or sociology for such fictional things would be as absurd as ``you can't kill people in a video game.'' Why can't I go out and kill people randomly? As long as we don't want the conclusion ``randomly killing people on the street is ethical,'' we must accept these two points, a minimalised realism.

Once we accept these two premises, we also accept:

\begin{enumerate}
    \item Our senses and neural signals are trustworthy. We cannot bypass our own nervous system to directly access external reality without mediation. Whether it's seeing others' joy, anger, sorrow, and happiness, talking to and hugging others, seeing others' videos, or reading others' stories, all of this is perceived by us through a series of neural signals. But how do we confirm it? We cannot. We don't need any biological knowledge.\footnote{I will not cite biological knowledge here, as that would be using science to argue for the scientific epistemology, which would be a circular argument. } We only need a series of phenomenological visual illusion experiments to know that ``our senses do not always output consistent and trustworthy signals; at least some of them are not real.'' If we can determine that ``some are not real,'' then we have no metaphysical reason to believe that its other signals are ``real'' or that it is ``overall real.'' But we still make this choice. We believe that our neural signals are a reliable proxy for objective reality.
    \item An epistemological strategy based on Abductive Reasoning and Inference to the Best Explanation (IBE). ``Our senses and neural signals are trustworthy'' can only tell us ``there are some people here, doing these things,'' it cannot tell us they are not philosophical zombies. The reason we believe that ``all people have the same mind and subjective feelings as I do'' is because we observe that all people have similar faces, eyes, expressions, and they cry and laugh like me. The most convincing explanation is that they have the same inner feelings as I do. Otherwise, we would need to argue how different foundations produce the same external manifestations.
    \item Falsifiability based on empirical evidence. We choose to constantly revise our understanding based on reality, rather than believing in a closed story that cannot be changed.\footnote{Similarly, I will not cite the ``Bayesian brain theory'' here, as that would also be a circular argument.} When we wake up from a dream in the morning, get dressed, get out of bed\ldots we build a model of our life: ``I woke up from a dream, got dressed, got out of bed.'' After a while, we wake up again, and we are surprised to find that we are still in bed, and everything we just did was a dream. We falsify the original model and revise it to ``I fell asleep again after waking up the first time.'' Any understanding we have about the external world is fundamentally falsifiable and constantly being revised. This is the prerequisite for us to live a meaningful life.
\end{enumerate}

Things that are unfalsifiable, like God or witchcraft, are \textit{ad hoc} and violate basic human cognition. If we use a religious epistemology in our daily lives, for example, a girl who constantly experiences gender discrimination, abuse, preferring her brother, has all the inheritance left to her brother, and selling her to an old bachelor for money, and still constantly adds \textit{ad hoc} justifications like, ``they must have their reasons, Mom and Dad still love me,'' most people would think this is an unhealthy thought, a form of escapism, weakness, not daring to face reality, being brainwashed and gaslighted. So why is the same epistemology, e.g., ``these sufferings are a test from God,'' reasonable?

Since we have already accepted that one type of signal is a reliable proxy for objective reality, we have no reason to reject the reliability of another type of signal. A photon hitting our retina and being converted into a neural signal is isomorphic to the cosmic microwave background radiation hitting the sensor of a space telescope and being converted into a digital signal. Neither is the entity itself, but a signal from a mediator or proxy.

If someone believes that ``quarks are just an explanation for deep inelastic scattering experiments that cannot be ultimately confirmed,'' then ``here is a brick'' should also be seen as an explanation for ``photons reflected by a reddish-brown rectangular solid with a porous microstructure hit my retina'' that cannot be ultimately confirmed, rather than a credible ontological claim.

From the perspective of the extended mind theory, we can view instruments as an extension of our senses, just as a notebook is an extension of Otto's memory. For a person with poor vision, is what they see with the help of glasses a ``direct experience''? For a person with poor hearing, with the help of a hearing aid? This line seems to be a product of accidental human biological basis, rather than a solid philosophical foundation.

What we see with a telescope or microscope is also, in our intuition, ``directly'' perceived by us, like with glasses. Using a telescope for bird watching is considered a way to get close to nature, an immersive connection with the natural world. If telescopes and microscopes are reliable, then is the optical viewfinder of a DSLR camera with a telephoto/macro lens reliable? Optically, it is essentially a telescope or microscope. If what we see through the optical viewfinder is trustworthy, then is the electronic viewfinder of a mirrorless camera reliable? From this step on, the light received by our retina is no longer directly from the object itself. It has been digitised and re-simulated because the electronic viewfinder of a mirrorless camera is essentially a screen. However, for advanced models with high resolution and high refresh rates, there is no phenomenologically perceptible difference in user experience from an optical viewfinder. Users trust the photos taken with a mirrorless camera to be genuine, just as they trust those from a DSLR. The key to trustworthiness does not lie in the physical mechanism of mediation, nor the method it interacts with physiological senses, but the consistency and verifiability of all evidence. I see a bird with my naked eye, but I can't see the details. I pick up a mirrorless camera with a super-telephoto lens, see the bird in the electronic viewfinder, press the shutter to take a photo, and review it later. The bird and its background, as seen through this series of actions and the photo I took, are consistent, so I believe this photo truly reflects the real image of the bird. I identify the species of the bird from this photo and add it to my birding lifer (life list), which means I believe I really saw it.

On this basis, we will continue to ask, are high-speed photography, infrared/ultraviolet photography, and computational photography reliable? Are medical imaging techniques like MRI and CT scans reliable? Are spectrometers reliable? We will eventually reach radio telescopes and interferometry, space probes, and the Large Hadron Collider\ldots This process of change seems to be continuous. It is difficult to draw a line of demarcation between these instruments as reliable and unreliable.

This ordering is also a product deeply shaped by the biological characteristics of humans, especially the sensotypical humans, rather than based on an objective standard. A case in point is that for a tetrachromat, they might think that it is a fundamental break from optical to digital, because the camera's CMOS digitises colour based on trichromatic vision, and some colour information they can see is permanently lost. A tetrachromat might feel that a digital photo is just a crude imitation, completely different from what they see with their naked eye or a microscope/telescope. Conversely, this tetrachromat can use a spectrometer to let trichromatic family or friends distinguish metamerism and understand their ability. The spectrometer is a more reliable extension of their senses, more ``concrete,'' more ``real,'' closer to their ``phenomenological experience'' and ``lifeworld.'' For a blind person, the entire ordering above is nonsensical and meaningless, just as ultrasound and infrared light. If this blind person were to study physics, then the Large Hadron Collider and a computer that can announce the experimental results in voice would be more real than any object, like a rainbow that is visible but cannot be heard or touched.

Therefore, this seemingly natural ordering is actually a form of sensotypical-centrism. Any theory that maintains a distinction between ``sensory experience'' and ``abstract theory,'' or ``observable'' and ``unobservable,'' please state your ``centrist'' assumption. Whose ``senses''? Whose ``observability''? When you set this standard with your own qualia, which cannot be directly accessed and cross-validated by others, you place your own sensory modality at the centre and define all other sensory modalities (colourblind, blind, deaf, tetrachromats, synaesthetes\ldots) as marginal, Other. Their ``senses'' are defined as non-standard, abstract, and needing to be ``translated'' by scientific instruments. There is no universal distinction between ``observable'' and ``unobservable'' that applies to all humans. Science, on the contrary, provides us with a more inclusive way to accommodate sensory diversity, allowing different groups to establish a shared intersubjective reality in a way that is, although very imperfect, abstract, and indirect,\footnote{A sensotypical person can never directly and perfectly experience the qualia of tetrachromacy, and the same applies to the blind and deaf towards sensotypical people. } still relatively reliable. This is completely consistent with our initially accepting the existence of ``external world'' and ``other minds'' to construct a shared intersubjective world.

Moreover, the narrative of ``discover unobservable theories from directly observable phenomena'' does not always proceed in the order we expect. If we train a deep visual neural network model with photos of two traditionally indistinguishable cryptic species, and find that the network can distinguish them with an accuracy of, for example, over 85\%, which is far higher than human biologists. Then we use Grad-CAM (Gradient-weighted Class Activation Mapping) to generate an activation heatmap of the last convolutional layer. We found that the neural network was distinguishing them through a feature we had previously ignored. We then find this feature to be very effective, and after learning it, human biologists can also distinguish them with the naked eye. In this process, we first defined them as two distinct species through the analysis of unobservable DNA sequences and phylogenetic analysis. Then we let the neural network train on these two classes\footnote{The term ``class'' here refers to the computer scientific term, not the taxonomic level. } which we artificially divided. Compared to directly ``believing'' the phylogenetic results, we additionally believed that the deep neural network is not a random number generator or a supernatural magic program. In this event, the theoretical load of naked-eye observation (phylogeny + artificial intelligence) is heavier than the phylogenetic analysis itself. The seemingly natural process of ``first observe, then discover'' is completely reversed.

Similarly, the ``social construct'' that post-structuralism is obsessed with is also a form of mediation, albeit an abstract one. Our perception of an ``apple'' is not just the physical and biological process of light and retina. It is also wrapped in numerous layers of social construction, including the cultural meaning of the word ``apple,'' the story of Snow White, and the saying ``an apple a day keeps the doctor away.'' \textcite{Foucault1978History} proposed that the family is an important site for biopower. Parents teaching a young child, ``this is an apple, apples are delicious,'' is precisely the way the biopower operates through the family node. We mentioned earlier that as a child, I believed human traffickers did not exist. I saw it as very interesting and told my mom when writing this article, and she lectured me again and warned me about ``beware of human traffickers.'' This is precisely how complicated philosophy exceeded the ``condition of possibility'' under a micro-episteme in a family. Most people will not doubt the reality of apples, as they might doubt scientific realism. If scientific instruments have theoretical presuppositions, and scientific theories have cultural and political presuppositions, then glasses (geometric optics), the naked eye (trichromatic vision), and even our most basic ability to think and know also have them. This is our universal epistemological dilemma, not a refutation of scientific realism.

Did my parents see real human traffickers? They did not, I am sure of this. They learned about the human traffickers from the TV news. Why should we believe that the television is a reliable source of news, that it will technically and faithfully reproduce what the television station wants to tell us, rather than a magic box that randomly outputs fictional content? Why should we believe that news is reliable, that ``human traffickers'' are a real threat to children, but not biopower means to control population migration? Our most basic interaction, as a child with parents, is already immersed in the social network of ``knowledge/power.'' The idea of distinguishing a ``direct lifeworld'' from an ``abstract theoretical world'' is naive. It is an obsession with unmediated, direct ``presence,'' precisely a form of logocentrism.

If we believe our ``pre-philosophical knowledge'' is trustworthy, that ``human traffickers'' are a real threat to our safety, then there is no reason to think that quarks are less real than ``human traffickers.'' Most people do not wait until themselves or their children's actual abduction to believe in the existence of human traffickers. Children believe their parents, parents believe the news, the news believes the reporters, and the reporters believe the victims\ldots This chain of trust leading to ``human traffickers'' is epistemologically isomorphic to the chain of trust leading to ``quarks'': we believe textbooks, textbooks believe scientists, scientists believe peer review, and peer review believes experimental data\ldots They all involve trust in mediation, others, and inference.

Science, by examining all these ``mediated signals'' under an equal epistemological standard, attempts to cut through the fog and establish a more objective understanding of reality itself. For example, Johann Friedrich Blumenbach, through comparative anatomical studies of skulls, reached scientific conclusions quite close to modern population genetics: ``the highest diversity is found within the black race,'' ``the five races are not clearly divided, with transitional states between them,'' ``there is no scientific evidence for superiority or inferiority among different races,'' and ``the intelligence of black people is not inferior to any other race.''~\parencite{Rupke2019Johann}

Of course, I know post-structuralists will say: his views and theories were later appropriated by ``scientific'' racist theories like phrenology, which is a manifestation of Foucault's theory of the ``episteme.'' Blumenbach's scientific views exceeded the ``conditions of possibility'' of knowledge at the time; they were ``unthinkable'' and ``unbelievable'' within the episteme, discourse rules, and cognitive framework of his era, and were thus quickly appropriated in the discursive field into a form more acceptable in the episteme.

Nevertheless, this also shows that Foucault's ``power'' is not an omnipresent, omnipotent, insurmountable thing that exists beyond the scope of human cognition. Otherwise, how did Foucault himself know it? \textcite{Foucault1978History} said that ``power'' is just a nominalist, functional concept for analysis: ``Power is not an institution, and not a structure; neither is it a certain strength we are endowed with; it is the name that one attributes to a complex strategical situation in a particular society.''

Although Blumenbach did not escape more abstract historical limitations like ``seeking ideal prototypes,'' the ``equality'' and ``transitional nature'' in his rigorously derived conclusions clashed fiercely with the hierarchical (superiority/inferiority) and essentialist (distinct categories with clear essences) tendencies in the deep structure of the episteme at his era. Scientific research loyal to evidence can provide us with a relatively reliable picture of the world. No matter how scientific data is mediated by senses, instruments, culture, and political presuppositions, science remains a reliable way to understand objective reality. Otherwise, it would be incredible for Blumenbach to have reached conclusions largely consistent with contemporary population genetics under a completely different ``episteme.''\footnote{This is the classic no-miracles argument, but in Blumenbach's case, it has an ethical and political meaning. } From the perspective of structural realism\footnote{A branch of scientific realism that holds that the ontological assumptions of scientific theories may not correspond to reality, but the mathematical structures they reveal are truly possessed by reality. }, although the ``races'' assumed in Blumenbach's theory were later denied ontological status by subsequent theories, we have reason to believe that the mathematical relationships—``this group has higher diversity,'' ``there are no clear boundaries between these groups''—captured some real properties of objective reality.

Like Foucault's ``power,'' social construction is not a deterministic thing that precedes or even transcends natural laws. Music is also a social construct; a specific melody does not necessarily have a definite ``emotion'' or ``meaning,'' and our understanding of music is culturally loaded. However, it is obviously impossible for us to compose music using ultrasound or infrasound, which would violate human biological characteristics. As we have discussed in previous cases, our vision is also limited by a series of innate factors. Social construction cannot escape the constraints of natural laws. Since social construction itself has been shaped by natural laws, it is theoretically possible for us to understand this reality through these epistemological mediations that ``internally carry the \textit{traces} of natural laws.''

If Foucault believed that the participants of discourse, the mentally ill in asylums are real,\footnote{I mean the several entities as buildings and a group of human individuals, not the abstract concept of ``mental illness,'' which I know Foucault did not believe in. } if Bruno Latour believed that the human and non-human actors in the actor-network are real, if Butler believed that the human body is real, and not their own dreams or illusions, then why can't we believe that quarks, chemical bonds, genes, and dinosaurs are (or were) real? Any form of social constructionism must admit that society and its members are real. They are the prerequisite for ``construction'' to occur.

Therefore, I am not saying, ``I can irrefutably argue from a metaphysical perspective that quarks are an absolute objective reality independent of my mind.'' I cannot argue this, just as no one can, under the same skeptical scrutiny, argue that they are not a brain in a vat or that others are not philosophical zombies. I am not interested in metaphysical questions, nor do I want to discuss ``whether quarks are objectively reality.'' Instead, I am saying: a philosophical view can claim that human knowledge of things is reality itself, or that it is merely a phenomenon and not the ``thing-in-itself,'' or that it is all ``within the text.'' While they must apply to both bricks and quarks. Whether it's metaphysical solipsism or post-structuralism, there is no valid critique that can selectively act on quarks without affecting bricks.\footnote{On this point, I think Derrida is more self-consistent than Husserl, although I know Derrida might say that my judging his theory by ``self-consistency'' is a form of logocentrism.}

Our naive beliefs, and scientific concepts such as ``quarks'' and ``genes,'' share the same unstable yet only available epistemological foundation. Scientific realism is not a naive, unreflective, vulgar position. On the contrary, it is a consistent and coherent philosophical worldview to exist in a shared, meaningful intersubjective world in a non-nihilistic manner, after seriously facing the profound challenges of skepticism and critical theory. From the moment we reject solipsism and acknowledge the existence of other minds to live a meaningful life in this world, where we are ``thrown into,'' we have already accepted an epistemological strategy based on abductive reasoning and inference to the best explanation. We have accepted that ``signals are reliable proxies for reality.'' We openly and continuously update our understanding of things based on new empirical evidence. Science is the inevitable natural extension and systematised version of naive epistemology.

My analysis of myself at the beginning is a practical application of this ``coherent epistemology.'' I did not ``disrespect'' my subjective experience. Otherwise, I would have simply said, ``How could I be a girl? It seems I'm crazy. End.'' But I did not do that. On the contrary, I very seriously analysed my subjective experiences of gender dysphoria and gender euphoria, which are themselves cognitions mediated by the nervous system and sociocultural constructs. Placing them on the same epistemological level as all the knowledge I could obtain from other sources. All this knowledge is equal, intermediated, intertwined, and connected in a rhizomatic—or rather, mycelial—way. When I proposed that ``natural selection cannot encode abstract concepts,'' I was not just discussing evolutionary biology and neuroscience. I also recalled seeing my younger siblings and cousins randomly sucking things as kids, and elders teaching them, ``Don't suck that.'' Our naive knowledge and scientific knowledge are interconnected and mutually shape each other. The belief in a pure, embodied experience independent of theory is a naive, pre-critical metaphysical illusion of ``presence.''

Respecting experience is not about unconditionally accepting it as a metaphysical truth, but about seriously understanding and analysing it, integrating it into one's complete knowledge network, and establishing cross-validating connections with all other knowledge. Far from ``disrespecting subjective experience,'' I have given my subjective experience the highest respect I can give to any knowledge.

I fully admit that neither scientific epistemology nor naive epistemology can solve ultimate metaphysical problems. We are standing on quicksand without an ultimate metaphysical foundation, making an existential ``leap of faith'' that cannot ultimately be justified. So why can't we be brave and try to jump a few more times? We will eventually find that our legs seem to be very, very strong, and we seem to be able to jump quite, quite far.

This is the most extraordinary vision of the Enlightenment. Reason is not coldness; reason does not mean the dismissal and suppression of emotion. The goal of the Enlightenment is not to shape everyone into emotionless artificial intelligence. It does not seek to eliminate our emotional reactions or our connection to history, but to separate these emotions from irrational beliefs.

In the ideal world of the Enlightenment, a person who falls will shout ``Ouch!'' but they know it's their C-fibres firing, not because they are unlucky or cursed. A German, a Jew, and a Pole visiting Auschwitz together will all feel horror and say, ``How terrible!'' While they know in their hearts that Nazism was an analysable historical and political disaster, unrelated to the ``essence'' or collective guilt of any nation or people. See, I said ``in their hearts,'' but we all know it's in their brains. When we visit historical sites, we appreciate the ancient artistic and architectural achievements, but we know that other peoples have similar things. Even if they don't, it is primarily due to geopolitical, environmental, and historical constraints, and does not mean that these less developed peoples are inherently inferior.

We are already doing this, aren't we? We are aware of the Earth's rotation and Rayleigh scattering, yet we still admire a beautiful sunset. We know that capsaicin activates pain receptors, yet we still enjoy spicy food. We know that love is associated with dopamine, oxytocin, and neural activity, yet we still shed tears over the story of \textit{Romeo and Juliet}. Scientific explanations enhance our understanding without destroying our beautiful subjective experiences. In fact, they can add new layers to these experiences, enriching the language we use to express our feelings. When we see the Milky Way, we feel not only the beauty of the night sky and many ancient stories, but also the greatness and boundlessness of the universe. Biology has given us new terms to express love. We can say not only ``butterflies in my stomach,'' but also ``my dopamine is exploding'' to our partners.

We just need to extend this state from physics, chemistry, and biology to history, politics, society, and culture, to complete the unfinished work of the Enlightenment, to persist in the brave step we have already taken. Why should we stop here? Why do we apply this mode of thinking to physics and biology, but retreat to pre-scientific essentialist myths when discussing history, politics, and identity?

Such a profound pursuit and understanding of truth is supremely liberating, capable of dismantling any prejudice, stereotype and discrimination. Recent scholarship has shown that the \textit{Song of Songs} is a love poem, devoid of theological significance and not the work of King Solomon. \parencite{Exum2022Conceptualizing} It provides a valuable secular perspective on Hebrew culture. The verse, ``Many waters cannot quench love, neither can floods drown it. If one offered for love all the wealth of one's house, it would be utterly scorned'' (Song of Songs 8:7, NRSV), is so moving that it nullifies any stereotype of the ``grasping Jewish merchant.'' The Hebrews, as ancestors of Jews, possessed a profound humanistic \textit{Ethos} that placed the precious love between individuals above all material wealth. At the very wellspring of the Jewish national spirit lies the supreme exaltation of non-material values.

However, if we interpret the text according to the traditional understanding—as an allegory for the relationship between humanity and God—the force of this argument is diminished. This is because the proposition that ``God is more important than wealth'' is an axiom in any religion and thus offers no new information. An anti-Semite could still contend that, ``apart from God, the Jewish people prioritize wealth above all else, and therefore, they remain grasping merchants.''

That sentence in the Mauthausen concentration camp is deafening: If there is a God, He will have to beg my forgiveness.\footnote{Wenn es einen Gott gibt muß er mich um Verzeihung bitten.} \parencite{Lassley2015Defective}

When this unknown prisoner wrote this sentence on the wall of their cell, they were rejecting the unfalsifiable comfort of another irrational system, choosing instead human autonomy and agency, using clear and unshakeable reason to confront the evil and irrationality of Nazism. What they did was to use the independent, strong, agentic, and liberating subject of self to resist the oppression of Nazism. In the face of violence aimed at completely erasing personal dignity, the ``self'' constructed by the Enlightenment, with its inherent reason and autonomous judgment, is an important and powerful weapon against dehumanisation.

Post-structuralism will certainly oppose this, as they are keen on promoting a myth about the life experiences of marginalised people being erased by universalist discourse.

I am not privileged.

I was born in a small town in a third-world country. I am not European, not white, not male (well, that depends on how we define ``male''), not cisgender, not heterosexual, and not a coloniser. My mother was a textile worker, and my father was a machine worker. I was constantly excluded and bullied in elementary and middle school because of my personality. After growing up, as a transgender biologist who supports gender abolition, I have become an outsider to all political camps.

If my personal experience from childhood to adulthood contradicts a particular philosophical school's critique of universalism and rationalism, why should I accept it? Insisting on logical self-consistency as the standard for understanding the world has been my most profound ``lived experience'' since my childhood questioning of housework division and kinship terms. For me, post-structuralism (including queer theory, and so on) is local knowledge from Europe. The ``harms of rationalism and universalism'' they criticise have never appeared in my life and are completely disconnected from my personal experience. I have no reason to accept it.

Moreover, the concept of ``local knowledge'' is itself a kind of local knowledge. Since it is local, there must be places where it is invalid, which means that there must be some non-local and universal knowledge.

The idea that ``rationalism originating from the Western Enlightenment oppresses the different knowledge systems of other civilisations'' is still a continuation of the ``Orientalist'' thought that views the East as an exotic and irrational Other. It simply praises this thought that emphasises ``subjective experience, emotion, intuition'' and uses it as an argument in an internal philosophical conflict within Europe, without genuinely trying to understand non-Western intellectual history. This is a form of ``reverse Orientalism.''

It is still a form of ``nation-state thinking,'' a product of the Westphalian system. As I have argued before, the Westphalian system and international geopolitics are themselves forms of identity politics that oppose universal reason, sharing the same mode of thinking as post-structuralism, albeit with different standards to define ``identity.'' Universalism offers the only means to transcend identity politics entirely.

The intra-civilisation philosophical differences are greater than the inter-civilisation ones. The School of Names' ``white horse dialogue,''~\parencite{Graham1990Studies} the Mohists' ``universal love,'' and their exploration of geometry and physics are more similar in \textit{Ethos} to Plato, Aristotle, and Euclid than to Confucianism and Taoism. There were also Sophists in Ancient Greece. The \textit{Mohist Canon}'s ``a circle is that which has the same length from a single centre''\footnote{圆,一中同长也。} and ``a square is that which has four angles and four corners equal''\footnote{方,柱隅四权也。} are almost identical to Euclid's definitions, although they did not develop into a systematised axiomatic system. \parencite{Graham1978Later} There are paradigmatic differences between analytical, logical, abstract, systematic, and universalist approaches, and contextual, emotional, concrete, empirical, and particularist ones (and their intermediate states) within different civilisations.

The similarities among analytical, logical, abstract, systematic, and universalist systems across civilisations (for example, the similarities between Mohism and Euclidean geometry) are greater than the similarities among contextual, emotional, concrete, empirical, and particularist systems across civilisations (for example, Confucianism, the Sophists, and post-structuralism). Therefore, we can say that rationality and science are the most universal knowledge.

In middle school, I was drawn to the Mohists and found Confucianism to be less appealing. Confucianism is not my ``local knowledge,'' but a paradigmatic enemy. More than two thousand years before I was born, they established their dominant position in China by systematically excluding reason and universalism as their ``Other'' and eliminating competitors whose basic cognitive paradigm was more similar to mine. My love for science and Enlightenment rationality is because it is based on the same cognitive foundation and has developed more completely and systematically than the Mohists and the School of Names. In my view, this is a case of ``seeking lost rites among the common people,''\footnote{礼失而求诸野, meaning that the authentic traditions are often better preserved among the common people than by the ruling class. Ironically, this old saying is considered to be from Confucius. } choosing the most developed one among all those that align with my cognitive mode.

For me, it is an act of colonising personal experience with theory, using a cold, academic philosophical theory to strip personal experience of its validity to accept the post-structuralist critique of reason. If I were to sincerely agree with the post-structuralist critique, then, by post-structuralist standards, I should refuse to accept post-structuralism. Therefore, it is not that I choose to reject post-structuralism out of ideological opposition originating from the ``scientistic grand narrative,'' but rather that post-structuralism cannot handle my personal experience. It would immediately throw a fatal exception and crash. Accepting it is impossible for me, a logical paradox.

A rule is universal only when it applies to everyone without exception. A privilege, by definition, is an exception granted to a specific group; it cannot apply to everyone and is therefore not universalist. Similarly, there had never been a real Kallipolis that was ruled by true reason. It has always been weak, excluded and suppressed throughout history. A firm commitment to universalism, rationalism, and the Enlightenment is not a tool for privileged groups to defend their status and it logically cannot be.

On the contrary, it is the last refuge of the weak. I have no group, no tribe, no recognised identity. The only thing I have are these principles that should apply to all people equally and without exception. The bullies at school could pull down my pants and throw away my shoes, the discourse police in online communities can distort my words and ban my account, but no one can stop me from, as Kant said, having the courage to use my own \textit{reason} to point out how arbitrary and baseless their rules are.

\textit{A man can be destroyed but not defeated.} Everything I have done is precisely like the anonymous prisoner in the Mauthausen concentration camp, using the powerful, rational subject of self from the Enlightenment to resist the irrational violence of the world.

Universalism and rationalism are not a claim to power, but a relentless pursuit of a pure justice, a justice that can even empower an ultimate outsider like me. The goal of the Enlightenment is precisely to use the light of reason to enlighten all the darkness of pre-modern superstition, privilege, dogmatism, tribalism, and obscurantism, to create a world where there is no more of this pain and harm, a world where no child will be bullied for wearing a pair of ``wrong'' shoes.

{\Large \textbf{Long live the Enlightenment.}}

{\Large \textbf{Long live the Enlightenment!}}