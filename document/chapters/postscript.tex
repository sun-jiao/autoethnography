Every sentence in the main text is completely sincere. I have indeed experienced those events and that nightmare about psychological manipulation. After that, I began to think about gender identity from the first principles of biology, asked questions in multiple communities and was refuted, attacked, or banned. I then explored my own history. Finally, to understand ``why they attacked me,'' I learned about post-structuralism and queer theory.

I did not fabricate the imitation in \cref{app:b} into a second ``Sokal affair'' (and as you can see, I also criticised Sokal's views on gender in the main text). Because I really want sincere discussion and communication. What I want to prove is ``I have spent great effort to understand your theory. I am not an ignorant outsider.''

Moreover, if we set aside the fact that ``I do not accept post-structuralism at all,'' this act of ``writing two versions of a personal story'' is very post-structuralist. It forces us to think: to what extent does the way we tell our stories determine who we are? My answer is that I believe in an independent, agentic, rational subject of self that can choose between two different discourse systems. My choice of science and reason over post-structuralism is the ultimate embodiment of this. A post-structuralist would surely interpret it as a deconstruction of the ``true self,'' a questioning of the author's ``sovereignty,'' and ``there is nothing outside the text.''

After delving into post-structuralism, queer theory, and current transgender activism, I can fully empathise with how mainstream transgender people would view my gender abolitionism, because post-structuralist philosophers have done the exact same thing to me. I negated ``gender identity,'' a concept many see as a survival need; Foucault negated ``reason,'' a concept I see as my only refuge. I see gender identity as an oppressive prison; you see reason as an oppressive prison.

In my view, the word ``reason'' refers to the cognitive mode shared by all humanity, the reason of Socrates and Kant, of Euclid and Einstein, the reason that made that young me in the kitchen ask my mom, ``Why doesn't Dad cook?'' the reason that makes me now engage in evolutionary biology research. Reason is not just a tool; in this world full of chaos and pain, it is the only clean, pure, and trustworthy thing. It is my shield, my weapon, my sanctuary. It is the sole, unified force for resisting tyranny and discovering scientific truth.

I know full well that what Foucault criticised is not this reason, but that's where the problem lies: I do not recognise ``instrumental rationality'' as a form of reason at all. I believe that the thing that creates weapons, wages wars, and harms the bodies of transgender and intersex people is merely a tyrant claiming a name that does not belong to it. I don't care what ``raison'' means in French, nor do I want to know what happened in Western history, but it cannot be translated into the word ``理性'' (lǐ xìng, reason/rationality) in my language (Chinese).

Similarly, both queer theory and I long for a space with infinite human creativity. This space may need a name, or perhaps it doesn't need a name. But it cannot be called ``gender,'' because that is the name of the prison we are trying to escape from. In my view, ``gender identity,'' as in \textcite{Marx1977Critique}'s critique of Luther's Reformation, frees the body and enchains the heart.

This is precisely what makes us unable to have meaningful dialogue. Because this is not a matter of philosophical theory, but of ethics, aesthetics, and even existentialism. We both see the ``liberation'' in the other's eyes as the most significant threat, and the oppression in the other's eyes as the most important survival need. As Saussure pointed out, a word has no intrinsic meaning, so there is no transcendent metaphysical standard to judge whose use of words like ``reason/rationality'' and ``gender'' is correct. There is no Platonic ``Form of Reason,'' no theory or empirical evidence that can bridge this gap.

Imagine this: I am a guerrilla fighter. To escape a tyranny called ``gender,'' I have, after hard struggle, created a base area called ``reason,'' the only place where I feel safe. One day, another guerrilla unit passes by. I hear they are from France and the United States, and their captain (Foucault) and vice-captain (Butler) are a gay and a non-binary person, being our comrades. I happily welcome them to visit my base, but they tell me that my base is actually the prison that oppresses us, and that the tyranny called ``gender'' is our true path to liberation.

What can I do? What can I tell them?

I can still witness. I can still read and understand.

I paid huge intellectual effort to understand them, while they did not understand me.

When that true, unforeseeable, irreducible, monstrous ``Other'' appeared at the master's door, they tragically failed to recognise the visage at all.

\appendix

\crefalias{section}{appendix}