The only solution -- gender abolition.

What we should do is to strictly limit ``sex'' and ``gender,'' detaching them from individuals or organisms.

The biological sex should be strictly confined to the gametes \parencite{Lehtonen2014Gamete, Goymann2023Biological, Hurst1996There}. It is related to, and only to, gametes. Nothing else is ``sex,'' and biological individuals do not have such a sex. The question ``What is your sex?'' is a category error, just as we cannot call the colour of hair ``the colour of a person.'' Therefore, \textcite{FaustoSterling2000Sexing}'s so-called ``5 sexes'' is also a meaningless classification. Sexual dimorphism is a dynamic spectrum that constantly changes throughout evolutionary history. The so-called ``phenotypic sex'' should be dismantled into a series of discrete, decentralised phenotypic traits such as chromosomes, hormones, as well as morphology and anatomical traits. They are distributed on the spectrum of sexual dimorphism, along with sexually dimorphic phenotypic traits not traditionally classified as ``phenotypic sex'', like height, weight, body hair, and body fat percentage. The imprecise concept of intersex should be replaced by more detailed terms like intergonadal, intergenital, and sexual intermorphism. Gender should be strictly treated as a sociocultural phenomenon. Individuals do not have an innate gender; they merely ``possess'' gametes and ``live in'' a society and culture.

By biologically confining sex strictly to gametes, it cannot be ``assigned'' or ``inferred.'' In one case, a person's phenotype was completely ``phenotypic male,'' who had fathered a daughter, but one of his two ``testicles'' was actually a pure ovary (not an ovotestis), and dissection showed it had previously ovulated \parencite{Parvin1982Ovulation}. Therefore, most people, even those who have had children, cannot know for sure if they are chimaeras capable of producing two types of gametes. This makes it lose any possibility of being used for social classification, discipline, and oppression. It becomes a technical term of reproductive biology, completely ``exiled'' from everyday language and the operation of power.

``Gender identity'' should be strictly regarded as a ``narrative identity'' produced by the individual's interaction with and internalisation of social gender constructs. We do not study what a person's ``gender'' is, but rather the correlation of many sexually dimorphic traits (including phenotypic, genetic, and neurological) with gametes, how they are shaped by sexual selection in evolutionary history, how they are mutually regulated at the molecular level, how they are interpreted by and interact with society and culture, and how this interaction shapes the individual's self-identity.

Of course, the binary of sex (gametes) is still a very useful model in evolutionary biology, ethology, and reproductive ecology. It is obviously technically and ethically impossible to capture and dissect all individual animals to examine their gonads or the gametes they produce. In these cases, researchers should state in their \textit{Materials and Methods} section what method they used to estimate the gamete production capacity and the reliability of this proxy, such as the situation in birds revealed by \textcite{Hall2025Prevalence}. Sex is a microscopic reproductive biological attribute, not an externally observable phenotypic trait.

\textcite{Polderman2018Biological} claimed that the heritability of gender identity is 30--60\% and consistent with other behavioural and personality traits, which is a category error. As we have said before, ``gender identity'' cannot be encoded in the genome (this would require extraordinary evidence). We can only have a series of genes related to personality traits, cognitive styles, and body schema. Studying the ``heritability of gender identity'' is a huge logical leap. The correct scientific questions should not be: ``What is the biological basis of gender identity?'' but rather: ``Which heritable biological traits are assigned gender meaning by society and culture?'' (sociology). ``How does the meaning interact with the individual to construct a specific gender identity?'' (psychology). ``What is the biological basis of these traits?'' (genetics and neuroscience). Autism spectrum disorder is also highly heritable \parencite{Sandin2017Heritability}, and autistic traits are higher in those working in STEM fields \parencite{Ruzich2015Sex}. Following \textcite{Polderman2018Biological}'s methods, we might also be able to calculate a mathematically reasonable number of ``heritability of STEM ability'', which everyone would consider an absurd study. Autistic neurological traits are heritable, while participating in STEM fields is an extremely complex result of innate neurological traits, personal experience, academic training and so on. Biological studies of ``gender identity'' are precisely the 21st-century replication of ``scientific'' racism.

At the social level, we should abolish all gender concepts and any social constructs based on gender. The biological characteristics previously considered ``phenotypic sex'' should have no more social significance than height or weight. We should abolish gendered pronouns and titles, treat sexual orientation as an aesthetic preference, treat gender incongruence as a form of body anxiety, and treat sex reassignment surgery as a form of cosmetic surgery. A person's desire to change their body, whether from innate body schema incongruence or any acquired factor, is a matter of pure personal freedom. Its legitimacy does not need an unprovable ``gender identity.'' Gender markers should be removed from all nonmedical records, and sex markers in medical records should be changed to multi-level records (chromosomes, reproductive organs, fertility, hormone levels, etc.) rather than just male/female. We should eliminate gender-segregated facilities; the assumption that people are willing to expose their private parts in front of the ``same sex'' or share a room with the ``same sex'' is also a bias. Clothing stores should no longer be divided into ``men's/women's'' sections, but organised by clothing type (tops, pants, outerwear) and size/fit. Toys should no longer be divided into ``boys'/girls','' but by function and type (e.g., building blocks, dolls, science experiments, art creation) and appropriate age.

We should recognise that the concept of ``gender identity'' is not liberatory but oppressive. It leads individuals to consider the result of being brainwashed as their ``true self.'' Xenogenders should not be counted as gender identities. This is not because I, like conservatives or transgender medicalists (truscum), think these things are ``unqualified'' to be considered gender identities, but because I think gender identity is ``unqualified'' to include them. Xenogenders are a space of infinite human creativity. What right does the oppressive concept of ``gender'' have to include them?

This transforms the issue from a special identity under identity politics, a special psychological state (identity), and a specific phenomenon, to the dismantling of an oppressive system. Everyone, whether ``cisgender'' or ``transgender,'' is a victim of this system. It is a political issue that requires the participation of the entire society: why do bathrooms need to be segregated by gender, and why do documents need to be marked with gender?

Some people might object to my previous statement that ``neither feminism nor LGBTQ has opposed gender-centrism,'' claiming that ``queer theory is against gender-centrism.'' \footnote{I have once been told by someone that ``queer theory advocates for the abolition of gender,'' which left me extremely stunned.(\cref{fig:comments}) When did queer theory advocate for the ``abolition of gender''? Don't they, as post-structuralists, hate such top-down, revolutionary grand narratives?} I admit that queer theory is ontologically anti-gender-centric, as argued by \textcite{Butler1990Gender} that they do not consider ``sex'' (phenotypic sex) or gender to be natural, real or stable. Still, there is a huge difference between ontology and methodology. \textcite{Butler1990Gender} advocated for a strategy of ``repeating'':

\begin{quotation}
    The task is not whether to repeat, but how to repeat, or, indeed, to repeat and, through a radical multiplexing of gender, to displace the very gender norms that enable the repetition itself.
\end{quotation}

The issue is neither whether to repeat nor how to repeat, but what to repeat. It is a kind of ``repeat'' to create a lot of neopronouns, however, this is precisely because of the gendered pronouns in English, Butler's native language. Nobody would create a lot of neoterms about bicycles, because there are no masculine and feminine forms of it, e.g., \textit{bicycle} and \textit{bicyclette}. This strategy recognises the gender norm's power to define the area and boundary of the battlefield. Therefore, queer theory is ontologically anti-gender-centric yet methodologically extremely gender-centric.

The intended outcome of this strategy was to ``displace the very gender norms'' by showing their constructed and unstable nature. However, as the actual outcome, by constantly talking about and analysing gender, by ``repeating'' gender inside the boundary defined by itself, it has established ``gender'' as the central issue for understanding power, self, and society. \textit{Gender} was originally just an external social construct, but it has now become an internal psychological phenomenon. As we have discussed in the case of \textcite{Oberle2023Benefits}, gender identity (a psychological phenomenon) has been abbreviated to gender (a social construct), and these two concepts have been equivalent.

Queer theory created a powerful cultural incentive: to treat all kinds of unconventional personal experiences and identities as a ``gender'', because it has made ``gender'' the most attractive ``liberative'' discourse. If ``gender'' were still strictly understood as a sociocultural construct, xenogender people would likely not see their ``identity'' as a ``gender identity.'' They do so because ``gender'' is currently the most prominent and available discourse of resistance. This paradoxically expands the category of gender to encompass phenomena that might otherwise have been understood in different terms (e.g., as personality, philosophy, or simply as radical individuality). In fact, within the xenogender community, there is already reflection on this issue, and the alternative term xenoidentity has been proposed for those who have similar identities but are unwilling to classify their identity as a ``gender identity.'' Every new identity they create is a new ``true self'' that individuals can choose. This makes the most fundamental rule, ``you must have a gender identity'', more unquestionable. The queer theory theoretically opposes the identity politics, yet its strategy seems to offer infinite choices of identities.

Queer theory's strategy was to engage gender norms on the ``battlefield'' those norms had already defined. The unintended consequence of this engagement was that the battlefield itself -- the very concept of ``gender'' -- became more central, all-encompassing, and ideologically powerful than ever before, attracting everyone to it. The strategy recognised the prison's power to define its walls, and sought to destabilise them by operating inside the prison, painting them many different colours, but not destroy and escape from it. The outcome was that the ``colourful and comfortable'' prison became more fundamentally entrenched, attracted more and more people to move in and reinforcing the core idea that one must be a prisoner. The strategy reinforced the premise that a choice must be made (you must have a gender identity).

If the gender system cannot accommodate our existence, shouldn't we completely smash this system? Why must we tell them, ``Actually, we are also a kind of `gender'?'' By saying ``we are also a kind of `gender','' they have accepted that ``gender'' is a valuable category worth joining, which gave up the fundamental right to ask, ``Can we abolish the social norm centred on `gender'?'' \textcite{Butler2025Who} is a typical figure of this reformist ideology:

\begin{quotation}
    The critique of the gender binary, for instance, did not claim that ``women'' and ``men'' are over and done with. On the contrary, it asked why gender is organized that way and not in some other way. It was also a way of imagining living otherwise. The critique of the gender binary turned out to give rise to a proliferation of genders beyond the established binary versions -- and beyond the gender hierarchy that feminism rightly opposes.
\end{quotation}

As \textcite{Marx1977Critique} said in his \textit{A Contribution to the Critique of Hegel's Philosophy of Right}:

\begin{quotation}
    Luther, we grant, overcame bondage out of \textit{devotion} by replacing it by bondage out of \textit{conviction}. He shattered faith in authority because he restored the authority of faith. He turned priests into laymen because he turned laymen into priests. He freed man from outer religiosity because he made religiosity the inner man. He freed the body from chains because he enchained the heart.
\end{quotation}

The concept of ``gender identity'' leads individuals to equate a social construct with their true self, to internalise an external social construct as a core part of their self. It freed the body from ``gender'' because it enchained the heart. It successfully achieved what classical gender theory could not do for thousands of years. We should award it a one-kilogram-heavy medal.

\textcite{Cull2019Against}'s claim that ``abolishing gender would endanger the lives of transgender people'' is a pure category fallacy. Their entire article argued that ``A genderless society is harmful to transgender people by refusing to recognise their identity.'' In an interview with \textcite{Williams2014Gender}, Butler claims that ``some people really love the gender that they have claimed for themselves. If gender is eradicated, so too is an important domain of pleasure for many people. And others have a strong sense of self bound up with their genders, so to get rid of gender would be to shatter their self-hood'' is the same fallacy. They presumed the existence of ``transgender people'' as a fixed category in a society where the concept of gender no longer operates, which is essentialist. As we argued earlier, any innate trait, including the innate body schema inconsistency, is not naturally related to ``gender.'' In contradiction, it is the society that assigned gender meaning to them, we internalised the assigned gender meaning and classified ourselves under this framework. It's the same as saying, ``We need to use pathogens to make vaccines, so eliminating pathogens would harm the patient's life.''

The root cause of the suffering (such as gender dysphoria) and oppression is the social construct of gender and ``phenotypic sex.'' ``Transgender'' is not an individual condition stemming from an ``inner self'' that requires medical care, \footnote{I completely acknowledge that body schema inconsistency is an innate individual condition that needs medical care, while as we previously argued, considering it ``gender'' is a category error.} but a serious political issue concerning human reason and existence. Insisting on ``gender identity'' is to treat only the symptoms while allowing the pathogen (the gender construct) to persist.

True liberation is not achieved by merely managing the symptoms while allowing the pathogen to thrive. It requires eliminating the pathogen altogether. The central flaw of Cull's view is its failure to recognise that the category ``transgender'' is itself a product of the gender system. In a world without gender, the experience of having an atypical body schema might persist, but it would no longer be defined as a ``gender'' issue. The ultimate solution is not to fix this ``mismatch,'' nor to fight for rights through piecemeal social engineering, but to completely destroy the entire system. It is not to defend the identity of ``transgender'' but to achieve a world where such categories are no longer necessary to describe human experience.

True freedom is not infinite ``genders,'' but an infinite space filled with human creativity.

I believe that not only is xenogender a misclassification and not a part of gender identity, but even the term xenoidentity is inappropriate.

The current classification of ``gender identity'' relies on an unprovable assumption of importance. Why are terms derived from ``sex'' the most core identities, while queer, agender, and gender fluid are secondary, and other words are grouped into a general category of xenogender, making xenogender the ``other of the other of the other'' (gender $\supseteq$ non-binary gender $\supseteq$ xenogender $\supseteq$ a specific xenogender, like catgender)? The diversity and internal heterogeneity of the xenogender are far greater than the typical gender identity, and the sources of the words also cover a much wider range (catgender, doggender, parrotgender, and lizardgender already cover the entire Amniota, and men and women, as part of \textit{Homo sapiens}, are necessarily phylogenetically nested within). This classification is extremely unnatural, a non-monophyletic group, and anthropocentric.

Xenogender is neither ``xeno-'' (strange) nor ``-gender'' (gender). On the contrary, this is the broadest, freest ``self-identity,'' even ``meta-identity.'' It is the basic human ability of self-cognition, the purest, freest, most unrestricted ``identity.'' It is the act of ``an individual choosing a word to locate and describe themselves.'' It is what made us human beings, the organism with reason. It phylogenetically includes everything, linguistically (in theory) covers all words. Outside of it, we have no way to describe and define ourselves. This is the prerequisite for the birth of ``gender identity'' and even sociocultural gender constructs. Gender identity is essentially just a small part of this ``meta-identity''; it is a narrow identity that specifically uses a few specific words. The greatest guilt of ``gender identity'' is its attempt to elevate itself to a supreme position, even above the very thing that makes its existence possible.

The Earth is not the centre of the universe; it is just a planet in the solar system. The solar system is just a part of the Milky Way. The stars and nebulae on that ``outermost celestial sphere,'' once considered distant and unimportant, are the real universe.