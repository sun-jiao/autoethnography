Critics might oppose my previous statement that ``neither feminism nor LGBTQ has opposed gender-centrism,'' claiming that ``queer theory is against gender-centrism.'' I admit that queer theory is ontologically anti-gender-centric, as argued by \textcite{Butler1990Gender} that they do not consider ``sex'' (phenotypic sex) or gender to be natural, real or stable. Still, there is a huge difference between ontology and methodology.

\textcite{Butler1990Gender} advocated for a strategy of ``repeating'':

\begin{quotation}
    The task is not whether to repeat, but how to repeat, or, indeed, to repeat and, through a radical multiplexing of gender, to displace the very gender norms that enable the repetition itself.
\end{quotation}

The issue is neither whether to repeat nor how to repeat, but what to repeat. For example, it is a kind of ``repeat'' to create a lot of neopronouns, however, this is precisely because of the gendered pronouns in English. Nobody would create a lot of neoterms about bicycles, because there are no masculine and feminine forms of it. This strategy recognises the gender norm's power to define the area and boundary of the battlefield. Therefore, queer theory is ontologically anti-gender-centric yet methodologically extremely gender-centric.

The intended outcome of this strategy was to ``displace the very gender norms'' by showing their constructed and unstable nature. However, as the actual outcome, by constantly talking about and analysing gender, by ``repeating'' gender inside the boundary, it has reinforced ``gender'' as the central issue for understanding ourselves and society. ``Gender'' was originally just an external sociocultural construct, but it is now frequently used as a synonym of ``gender identity,'' an internal psychological phenomenon. \textcite{Oberle2023Benefits} is a typical case, they argued that using the word \textit{gender} to describe plants ``harms'' transgender people. It is an obvious category error to claim that the borrowing of an abstract sociological concept, \textit{gender}, will ``harm'' humans who possess a psychological phenomenon, \textit{gender identity}.

%Every new identity they create is a new ``true self'' that individuals can choose. This makes the most fundamental rule, ``you must have a gender identity'', more unquestionable. The queer theory theoretically opposes the identity politics, yet its strategy seems to offer infinite choices of identities.
Queer theory's strategy created a powerful cultural incentive: to treat all kinds of unconventional personal experiences and identities as a ``gender'', because it has made ``gender'' the most attractive ``liberative'' discourse. If ``gender'' were still strictly understood as a sociocultural construct, xenogender people would likely not see their ``identity'' as a ``gender identity.'' They do so because ``gender'' is currently the most prominent and available discourse of resistance. This paradoxically expands the category of gender to encompass phenomena that might otherwise have been understood in different categories (e.g., as personality, philosophy, or simply as radical individuality). Within the xenogender community, there is already reflection on this issue, and the alternative term xenoidentity has been proposed for those who have similar identities but are unwilling to classify it as a ``gender identity.''

Queer theory's strategy was to engage gender norms on the ``battlefield'' which was defined by itself. The unintended consequence of this engagement was that the battlefield itself -- the very concept of ``gender'' -- became more central, omni-encompassing, and ideologically powerful than ever before, attracting everyone to join it. The strategy recognised the prison's power to define its walls, and sought to ``destabilise'' them by painting them many different colours, but not destroy and escape from it. The outcome was that the ``colourful and comfortable'' prison became more fundamentally entrenched, attracted more and more people to move in and reinforcing the core idea that one must be a prisoner.

%If the gender system cannot accommodate our existence, shouldn't we completely smash this system? Why must we tell them, ``Actually, we are also a kind of `gender'?'' By saying ``we are also a kind of `gender','' they have accepted that ``gender'' is a valuable category worth joining, which gave up the fundamental right to ask, ``Can we abolish the social norm centred on `gender'?'' \textcite{Butler2025Who} is a typical figure of this reformist ideology:
%
%\begin{quotation}
%    The critique of the gender binary, for instance, did not claim that ``women'' and ``men'' are over and done with. On the contrary, it asked why gender is organized that way and not in some other way. It was also a way of imagining living otherwise. The critique of the gender binary turned out to give rise to a proliferation of genders beyond the established binary versions -- and beyond the gender hierarchy that feminism rightly opposes.
%\end{quotation}

As I previously argued, this is precisely because of the ontology of post-structuralism. They do not believe that objective knowledge of the external world is possible. Therefore, the goal is not to match an external truth, but to play with, subvert, and reconfigure the discourses we live within. In contrast, an objective understand of the reality is possible and desirable in scientific realism, which provides an unshakeable foothold for scientific and philosophical criticism and political resistance. The concept of ``gender'' causes most people's perception of the world to be inconsistent with this external, objective reality. The highest goodness and truth is about aligning our internal models with external reality. Gender is a buggy model that creates a mismatch, so it must be abolished. This is a concise demonstration of the fundamental, irreconcilable gap between the scientific worldview of Enlightenment and the post-structuralist one.

%As \textcite{Marx1977Critique} said in his \textit{A Contribution to the Critique of Hegel's Philosophy of Right}:
%
%\begin{quotation}
%    Luther, we grant, overcame bondage out of \textit{devotion} by replacing it by bondage out of \textit{conviction}. He shattered faith in authority because he restored the authority of faith. He turned priests into laymen because he turned laymen into priests. He freed man from outer religiosity because he made religiosity the inner man. He freed the body from chains because he enchained the heart.
%\end{quotation}
%
%The concept of ``gender identity'' leads individuals to equate a social construct with their true self, to internalise an external social construct as a core part of their self. It freed the body from ``gender'' because it enchained the heart. It successfully achieved what classical gender theory could not do for thousands of years.

In an interview with \textcite{Williams2014Gender}, Butler claims that ``some people really love the gender that they have claimed for themselves. If gender is eradicated, so too is an important domain of pleasure for many people. And others have a strong sense of self bound up with their genders, so to get rid of gender would be to shatter their self-hood.'' This is a reasonable concern with humanistic care. It is also found in the analytical tradition: \textcite{Cull2019Against} argued that ``A genderless society is harmful to transgender people by refusing to recognise their identity.'' However, the fallacy is that Butler and Cull presumed the existence of ``transgender people'' as a fixed category in a society where the concept of gender no longer operates, which is essentialist. It is the same as saying, ``We need to use pathogens to make vaccines, so eliminating pathogens would harm the patient's life.''

I do not shy away from this point: according to my model, in a world where gender is abolished, transgender people will not exist. Cisgender people will also not exist. Because ``gender identity'' will not exist, at least not in its current form. However, this is not an elimination of contemporary transgender people, nor does it mean that gender-affirming care for contemporary transgender people is not needed. Quite the contrary, this is based on a profound acknowledgement of the suffering of contemporary transgender people (including myself). The root cause of the suffering (such as gender dysphoria) and oppression is the social construct about gender. Transgenderness is not an individual condition stemming from an ``inner self'' that requires medical care, but a serious political threat to our very existence. Insisting on ``gender identity'' is to treat only the symptoms while allowing the pathogen (the gender construct) to persist. True liberation is not achieved by merely managing the symptoms while allowing the pathogen to thrive. It requires eliminating the pathogen altogether. It is not to defend the identity of ``transgender'' but to achieve a world where such categories are no longer necessary to describe human experience. Gender abolitionism is the only structuralist solution, as it correctly identifies the problem in the ``training data'' that shapes all individuals.

Another fallacy of Butler's this claim is that we can abolish stereotypes while keeping gender/gender identity. However, gender identity is the product of socialisation within a gender norm based on stereotypes. If this gender norm no longer exists, gender identity -- at least for most people -- will not exist. It might become a niche, historical subculture, like believers in the Athenian pantheon today. It will be considered a form of freedom of speech and freedom of religious belief.

We have no need at all to shy away from ``gender identity is the internalisation of social gender norms.'' The gender identity of cisgender people is also the internalisation of gender norms. However, what if it's an internalisation of gender norms? I even believe my gender identity is a product of childhood trauma. So, why is it a virtue when childhood trauma causes a person to become strong, but when it causes a person to develop an incongruent gender identity, it needs to be ``converted?'' This conservative logic is ``very easy to refute'' because it rests on an unstated, bigoted value judgment, not on a consistent principle.

What should be emphasised is: gender identity is an acquired psychological state, but it is one that is acquired subtly and unconsciously, like a mother tongue. For the person themselves, it feels very much like it is innate. Therefore, most people find it difficult to realise that it is acquired, and even if they do realise it, they cannot get rid of it through ``realisation.''
