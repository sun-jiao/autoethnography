Since the real problem lies in an omni-encompassing ideology that attempts to assign ``gender'' to everything, the most fundamental solution is to abolish gender. What we should do is to strictly limit ``sex'' and ``gender,'' detaching them from individuals.

The biological sex should be strictly confined to the gametes \parencite{Lehtonen2014Gamete, Goymann2023Biological, Hurst1996There, Griffiths2025Biology}. It is related to, and only to, gametes. Individuals do not have such a sex. Sexual dimorphism is a dynamic spectrum that constantly changes throughout evolutionary history. The so-called ``phenotypic sex'' should be dismantled into a series of discrete, decentralised phenotypic traits such as chromosomes, hormones, as well as all morphology and anatomical traits. They are distributed on the sexual dimorphism spectrum, along with sexually dimorphic traits not traditionally classified as ``phenotypic sex'', like height, weight, body hair, and body fat percentage \parencite{Wells2007Sexual}. Individuals do not have an innate gender; they merely ``possess'' gametes and ``live in'' a society and culture.
%The imprecise concept of intersex should be replaced by more detailed terms like intergonadal, intergenital, and sexual intermorphism. Gender should be strictly treated as a sociocultural phenomenon.

By biologically confining sex strictly to an unobservable trait, it cannot be ``assigned'' or ``inferred.'' \textcite{Parvin1982Ovulation} reported a person who was completely ``phenotypic male'' and had fathered a daughter, but one of their two ``testicles'' was actually a pure ovary (not an ovotestis), and dissection showed it had previously ovulated. Therefore, most people, even those who have had children, cannot know for sure if they are capable of producing two types of gametes. This makes it lose any possibility of being used for social classification, discipline, and oppression. It becomes a technical term of reproductive biology, completely ``exiled'' from everyday language and the operation of power.

``Gender identity'' should be strictly regarded as a ``narrative identity'' shaped by one's interaction with and internalisation of social gender constructs. We do not study what a person's ``gender'' is, but rather the correlation of many sexually dimorphic traits (including phenotypic, genetic, and neurologic) with gametes, how they are shaped by sexual selection in evolutionary history, how they are mutually regulated at the molecular level, how they are interpreted by and interact with society and culture, and how this interaction shapes the individual's self-identity.

The binary of sex is still a very useful model in evolutionary biology, ethology, and reproductive ecology. It is obviously technically and ethically impossible to capture and dissect all individual animals to examine the gametes they produce. In these cases, researchers should state in their \textit{Materials and Methods} section what proxy they used to estimate the gamete production capacity and the reliability, such as the situation in birds revealed by \textcite{Hall2025Prevalence}. Sex itself is a microscopic reproductive biological trait, not an externally observable morphological character.

%\textcite{Polderman2018Biological} claimed that the heritability of gender identity is 30--60\% and consistent with other behavioural and personality traits, which is a category error. As we have said before, ``gender identity'' cannot be encoded in the genome (this would require extraordinary evidence). We can only have a series of genes related to personality traits, cognitive styles, and body schema. Studying the ``heritability of gender identity'' is a huge logical leap. The correct scientific questions should not be: ``What is the biological basis of gender identity?'' but rather: ``Which heritable biological traits are assigned gender meaning by society and culture?'' (sociology). ``How does the meaning interact with the individual to construct a specific gender identity?'' (psychology). ``What is the biological basis of these traits?'' (genetics and neuroscience). Autism spectrum disorder is also highly heritable \parencite{Sandin2017Heritability}, and autistic traits are higher in those working in STEM fields \parencite{Ruzich2015Sex}. Following \textcite{Polderman2018Biological}'s methods, we might also be able to calculate a mathematically reasonable number of ``heritability of STEM ability'', which everyone would consider an absurd study. Autistic neurological traits are heritable, while participating in STEM fields is an extremely complex result of innate neurological traits, personal experience, academic training and so on. Biological studies of ``gender identity'' are precisely the 21st-century replication of ``scientific'' racism.

At the social level, all gender concepts and any social constructs should be abolished. The biological characteristics previously considered ``phenotypic sex'' should have no more social significance than height or weight. Gendered pronouns and titles should be abolished, sexual orientation should be considered an aesthetic preference, gender incongruence should be considered a form of body anxiety, and gender-affirming surgery should be considered a form of cosmetic surgery. A person's desire to change their body, whether because of innate body schema incongruence or any acquired factor, is a matter of pure personal freedom. These are merely mechanical modifications to our body, its legitimacy stems from bodily autonomy. It does not need an unprovable ``gender identity'' to justify it. Gender markers should be removed from all nonmedical records, and sex markers in medical records should be multi-level (chromosomes, reproductive organs, fertility, hormone levels, etc.) rather than just male/female. All gender-segregated facilities should be eliminated. Clothing stores should no longer be divided into ``men's/women's'' sections, but organised by clothing type (tops, pants, outerwear) and size/fit. Toys should no longer be divided into ``boys'/girls','' but by function and type (e.g., building blocks, dolls, science experiments, art creation) and appropriate age.

Most importantly, ``gender identity'' should be considered not liberatory but oppressive. It leads individuals to consider a internalised social construct as their ``true self.'' Non-binary and xenogenders should not be considered gender identities. This is not because I, like conservatives or transgender medicalists (truscum), consider they ``unqualified'' to be gender identities, but because I consider gender identity ``unqualified'' to include them. Xenogenders are a space of infinite human creativity. Why should it be included in an oppressive concept, in which it is considered ``the other of the other of the other'' (gender $\supseteq$ non-binary gender $\supseteq$ xenogender $\supseteq$ a specific xenogender, like catgender)? The diversity and internal heterogeneity of the xenogender are far greater than the typical gender identity, and the sources of the words also cover a much wider range (catgender, doggender, parrotgender, and lizardgender already cover the entire Amniota, and men and women, as part of \textit{Homo sapiens}, are necessarily phylogenetically nested within). This classification is extremely unnatural and anthropocentric.
%Not only is xenogender a misclassification and not a part of gender identity, but even the term xenoidentity is inappropriate.

%The current classification of ``gender identity'' relies on an unprovable assumption of importance. Why are terms derived from ``sex'' the most core identities, while queer, agender, and gender fluid are secondary, and other words are grouped into a general category of xenogender, making xenogender

Xenogender is neither ``xeno-'' (strange) nor ``-gender'' (gender). On the contrary, this is the broadest, freest ``meta-identity.'' It is the basic, purest, freest, and most unrestricted ``identity.'' It is the act of ``an individual choosing a word to locate and describe themselves.'' It is what made us human beings, the organism with reason. It linguistically (theoretically) covers all words. Outside of it, we have no way to describe and define ourselves. This is the prerequisite for the birth of ``gender identity'' and even sociocultural gender constructs. Gender identity is essentially just a small part of the ``meta-identity''; it is a narrow identity that specifically uses a few specific words. The greatest guilt of ``gender identity'' is its attempt to elevate itself to a supreme position, even above the very thing that makes its existence possible.

The Earth is not the centre of the universe; it is just a planet in the solar system. The solar system is just a part of the Milky Way. The stars and nebulae on that ``outermost celestial sphere,'' once considered distant and unimportant, are the real universe.

%This transforms the issue from a special identity under identity politics, a special psychological state (identity), and a specific phenomenon, to the dismantling of an oppressive system. Everyone, whether ``cisgender'' or ``transgender,'' is a victim of this system. It is a political issue that requires the participation of the entire society: why do bathrooms need to be segregated by gender, and why do documents need to be marked with gender?

\textcite{Cull2019Against} argued that ``A genderless society is harmful to transgender people by refusing to recognise their identity.'' Similarly, in an interview with \textcite{Williams2014Gender}, Butler claims that ``some people really love the gender that they have claimed for themselves. If gender is eradicated, so too is an important domain of pleasure for many people. And others have a strong sense of self bound up with their genders, so to get rid of gender would be to shatter their self-hood'' is the same fallacy. They presumed the existence of ``transgender people'' as a fixed category in a society where the concept of gender no longer operates, which is essentialist. As we argued earlier, any innate trait, including the innate body schema inconsistency, is not naturally related to ``gender.'' In contradiction, it is the society that assigned gender meaning to everything, we internalised the assigned gender meaning and classified ourselves under this framework. It is the same as saying, ``We need to use pathogens to make vaccines, so eliminating pathogens would harm the patient's life.''

The root cause of the suffering (such as gender dysphoria) and oppression is the social construct about gender and ``phenotypic sex.'' Transgenderness is not an individual condition stemming from an ``inner self'' that requires medical care, but a serious political issue concerning human reason and existence. Insisting on ``gender identity'' is to treat only the symptoms while allowing the pathogen (the gender construct) to persist. True liberation is not achieved by merely managing the symptoms while allowing the pathogen to thrive. It requires eliminating the pathogen altogether. It is not to defend the identity of ``transgender'' but to achieve a world where such categories are no longer necessary to describe human experience.
