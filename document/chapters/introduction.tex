The contemporary discourse surrounding gender identity often presents a dichotomy between essentialist narratives -- whether rooted in religious conservatism or a search for a ``gendered brain'' -- and post-structuralist frameworks that view identity as purely discursive and fluid. For an observer grounded in the natural sciences, both extremes frequently appear to bypass the material realities of biology and the rigorous demands of logical consistency. The former often relies on a metaphysical ``soul'' or unproven neurological determinism, while the latter frequently risks dissolving the subject entirely into language games, potentially alienating the very lived experiences it seeks to describe. This article attempts to bridge this epistemological schism by examining the phenomenon of gender not through the lens of ideology, but through the analytic tools of evolutionary biology and the philosophical tradition of the Enlightenment.

I am a PhD student in plant taxonomy, using programming and statistical methods based on morphology and molecular phylogenetics to resolve the classification of plants. This methodology has deeply influenced the methods I used in my own journey of gender exploration. This article aims to document how an individual with a firm commitment to Enlightenment rationalism, who has received rigorous scientific training, attempts to understand a crucial aspect of themself. Autoethnography is a unique genre in which the author's own philosophical and scientific stances are part of the data. The soul of autoethnography lies in honestly presenting how the ``self'' (auto-) experiences and understands ``culture'' (ethno-). My ``self'' is the one who has been scientifically trained, committed to Enlightenment rationality, and tries to use logic and order to understand a chaotic and painful world. The positivist and rationalist attitudes are essential parts of the ``sanctuary'' in which I find my place in the world. Therefore, it must be faithfully presented here as the core of the story.

This inquiry necessitates a departure from standard narratives of transition. Rather than seeking to validate a pre-existing category of identity, this work interrogates the category itself. It asks whether the concept of ``gender identity'' possesses ontological weight when subjected to the same scrutiny one would apply to a taxonomic classification or an evolutionary hypothesis. By engaging with personal memory -- ranging from childhood socialisation to adult interactions within academic and online communities -- this article treats the self as a case study for a broader theoretical proposition: that the path to genuine human liberation lies not in the proliferation of gender categories, but in their abolition.

Furthermore, this text addresses the isolation often felt by those who fall outside politically convenient taxonomies. It explores the tension of being a ``gender abolitionist transgender'' individual -- a position that frequently invites hostility from both trans-exclusionary radical feminists and dogmatic sectors of the queer community. By grounding this analysis in the universalist principles of the Enlightenment, the following sections argue for a reconstruction of subjectivity that transcends the ``prison'' of gender, aiming instead for a humanism defined by reason, autonomy, and the courage to know.
