The contemporary intellectual landscape surrounding gender is dominated by two primary, often contending, theoretical frameworks: biological essentialism and post-structuralist queer theory. The former frequently posits ``gender identity'' as an inherent essence \parencite{APA2015Guidelines, NHS2022Gender}. This perspective, often adopting a ``born this way'' narrative, attempts to anchor social identity in brains or metaphysical souls. Conversely, post-structuralism, exemplified by the work of \textcite{Butler1990Gender}, approaches gender as a socially constructed and performative phenomenon, denying an objective understanding of the pre-discursive biological reality. While these theories ostensibly oppose one another, this article argues that they share a critical flaw: both ultimately reinforce ``gender-centrism,'' an ideology that establishes gender as the central category for understanding human existence, thereby obscuring the objective diversity of the material world.

This article presents criticisms of both transgender essentialism and queer theory, demonstrating gender abolitionism based on a synthesis of evolutionary biology, neurology, and the philosophy of science. Central to this political solution is the definition of sex \textit{sensu stricto} as strictly limited to gametes, distinguishing it from the dynamic spectrum of phenotypic traits and dismantling the social category of gender entirely.

