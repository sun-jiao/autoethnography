The contemporary intellectual landscape surrounding gender is dominated by two primary, often contending, theoretical frameworks: biological essentialism and post-structuralist queer theory. The former frequently posits ``gender identity'' as an inherent essence \parencite{APA2015Guidelines}. This perspective, often adopting a ``born this way'' narrative, attempts to anchor social identity in brains or metaphysical souls. Conversely, post-structuralism, exemplified by the work of \textcite{Butler1990Gender}, approaches gender as a socially constructed and performative phenomenon, denying a pre-discursive biological reality. While these theories ostensibly oppose one another, this article argues that they share a critical flaw: both ultimately reinforce ``gender-centrism,'' an ideology that establishes gender as the central category for understanding human existence, thereby obscuring the objective diversity of the material world.

This article presents a critique of ``gender identity'' and a proposal for gender abolitionism based on a synthesis of evolutionary biology, neurology, and the philosophy of science. Central to this argument is the definition of sex \textit{sensu stricto} as strictly limited to gametes, distinguishing it from the dynamic spectrum of phenotypic traits and dismantling the social category of gender entirely.

To elucidate the mechanisms by which a ``gender identity'' is formed without resorting to essentialism, this narrative incorporates a detailed autobiographical case study. It is necessary to explicitly acknowledge the epistemological limitations of this approach. As a subjective account of a single individual (N=1), this autobiography does not claim to generate universal statistical data or prove a biological rule on its own. Instead, it is offered as supplementary evidence to the broader theoretical and scientific argument.

In this article, I use ``phenotypic sex'' as an alternative to the conventional term ``biological sex,'' because the academic biological definition of sex (sex \textit{sensu stricto}) is about gametes.
