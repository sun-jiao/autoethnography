In contemporary socio-political discourse, gender identity is frequently articulated as an inherent essence \parencite{APA2015Guidelines}. Proponents of this view argue that humans possess a gendered essence (whether located in a ``gendered brain'' or a metaphysical soul) that is distinct from their phenotypic sex. However, treating a complex sociological category as a biological a priori demands rigorous scrutiny, as it risks regressing into a form of metaphysical essentialism akin to Plato’s theory of Forms. It is necessary to challenge this social axiom against the principles of evolutionary biology and neurology.

I will use ``phenotypic sex'' as an alternative to the conventional term ``biological sex,'' because the academic biological definition of sex (sex \textit{sensu stricto}) is about gametes.
