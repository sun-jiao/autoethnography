I became aware of my ``gender identity'' late in my undergraduate studies. The first thing I noticed was that I subconsciously thought from a girl's perspective. For instance, once, while discussing physical fitness tests with classmates, I said, ``My 800-metre run.'' A classmate pointed out (without malice, but with confusion), ``It's 1000 meters.'' I didn't respond directly, joking, ``It's actually 800 plus.'' They jokingly asked if I was a girl. I didn't deny it directly, replying with a Chinese internet slang, ``u1s1, qs'' (to be honest, yes). \footnote{In Chinese universities, the long-distance running test is 1000 metres for male students and 800 metres for female students.}

During this period, I developed a strong desire to adopt a more feminine name: 娇 (\textit{Jiāo}). This character means ``cute'' or ``adorable'' and features the ``woman'' radical (will be addressed as ``the feminine \textit{Jiāo}''). The feminine \textit{Jiāo} is a homophone and graphically like my legal name (骄, \textit{Jiāo}, means ``pride'', relatively neutral), with only the left-side radical different. I often used the Romanised version instead of Chinese characters because they have the same Pinyin. When signing, I would use a cursive script (行书, \textit{xíng shū}) to make it indistinguishable. Furthermore, because the feminine name is much more popular than my legal name, it is the first choice in many Chinese pinyin input methods. \footnote{When typing in Chinese, we use a software called ``input method.'' It gives all possible Chinese characters based on the pinyin the user inputs, and the user chooses one from them. So, it's a common thing to choose a wrong character.} Many people would type my name in the feminine version, and every time this happened, I was delighted and afraid that someone would ``kindly'' point it out. If this really happened, I would be very ``tolerant'' and say, ``It's okay, as long as I know you're addressing me, a wrong character doesn't matter.'' In an artificial intelligence class, one homework was to determine the gender based on provided names. I changed all the people with the same name as me in the dataset to ``female.'' Sometimes I even used that feminine name myself, and if discovered, I would blame the input method.

Because I never corrected it, many personal friends who knew me in informal places (like in student clubs) thought it was my legal name. Occasionally, someone would ask, ``Is your name really the feminine \textit{Jiāo}?'' and friends would argue with them, ``Why can't a boy be named the feminine \textit{Jiāo}? It's such a cute name! That's a rude question.'' I would pretend not to see the group messages, secretly enjoying the protection from friends. Some friends thought it was a nickname, which I also didn't correct.

I was also delighted when friends described me as ``quiet'' and ``virtuous'' (文静 \textit{wén jìng} and 贤惠 \textit{xián huì}, both are traditionally ``feminine'' adjectives in Chinese) because I didn't talk much and cooked during home parties. Once, we went camping and played an ice-breaking game called ``King and Angel.'' \footnote{This is a game in which everyone will be an ``angel'' of their ``King''  and need to do everyone's best to take care of their ``King'' In the final reveal, everyone needs to try to name their ``Angel'' according to the care they received. And the word ``国王'' \textit{guó wáng}, which means ``King,'' is theoretically gender neutral in Chinese.} I gave my ``King'' a handmade hog plum (\textit{Choerospondias axillaris}) bracelet, placing it into their clothes with a note. During the final reveal, the ``King'' said that upon seeing such a fantastic bracelet and delicate handwriting, they thought it would be from a girl. I secretly felt extremely pleased about that.

However, because the proportion of female students at my undergraduate university was very high (70\%), I didn't take this matter very seriously at the time. I initially rationalised this feeling as ``I was assimilated by girls after spending a long time with them.'' \footnote{I'm not saying that girls naturally possess so-called ``feminine traits'' and ``feminine behaviours.''}

Later, I attended graduate school, where the proportion of male students was significantly higher, and this feeling became increasingly difficult to ignore.

At an academic conference, someone wrote the feminised version of my name on a poster. I saw it in advance and said nothing. A junior male student discovered it and contacted the responsible officer to correct it. I felt a strong sense of disappointment at that moment.

For a period after that, I used deep learning-based image generation models to create some feminised photos of myself. I also scanned my ID card and graduation certificate, changed the sex marker to female, replaced the photo with an AI-generated female version, and saved them on my computer. Since I remained in my undergraduate student club's group chat after graduation, one day I saw some new students joining the group. I told one of them I was a girl and sent an AI-generated photo. They replied, ``Wow, a pretty sis.'' I was incredibly delighted.

I bought many ``gender-neutral looking'' women's shoes and clothes, including my hiking boots, which were specifically designed for women. Not only that, but I bought women's hiking boots from TOREAD and Decathlon Quechua. I even wore them on fieldwork in Xizang (Tibet) and western Sichuan, where I climbed many mountains, saw many birds, and collected many plant specimens. (\cref{fig:1})

\begin{figure}[htbp]
    \centering
    \includegraphics[width=0.7\textwidth]{F1}
    \caption{Photos of the author wearing Decathlon Quechua MH100 women's boots during a birding trip in western Sichuan Province: a) The author (\textit{Homo sapiens}), b) Golden Bush Robin (\textit{Tarsiger chrysaeus}), c) Blood Pheasant (\textit{Ithaginis cruentus}), d) Przevalski's Suthora (\textit{Suthora przewalskii}).\label{fig:1}} % title and label
\end{figure}