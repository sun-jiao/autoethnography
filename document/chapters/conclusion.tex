This article has presented a multifaceted critique of the prevailing concepts of gender identity and the theoretical underpinnings of queer theory. By synthesising evolutionary biology, neurology, philosophy of science, and an autobiographical case study, I have argued that gender identity is not an innate biological essence but a product of neuroplasticity shaped by interaction with social norms. Furthermore, the narrative demonstrates how the queer theory paradoxically reinforces the very gender-centrism they seek to dismantle, and caused exclusion and violence towards transgender individuals who do not comply with their philosophical stance. Finally, I proposed a radical shift toward gender abolitionism: a framework that strictly restricts biological sex to gametes while dismantling phenotypic sex into a spectrum of independent traits, thereby removing gender as a valid category for social, legal, and self-identity classification.

Even for readers who may not fully embrace the political strategy of gender abolitionism, the scientific realist framework offers an alternative perspective for understanding gender identity, moving beyond essentialism and radical constructivism. For specific transgender individuals who, like myself, from scientific community or analytic tradition, and find their experiences rooted in material reality and personal history, this framework offers a path to self-understanding that reconciles biological reality with psychological experience. Simultaneously, it invites onlookers to perceive gender dysphoria not as an inexplicable metaphysical conflict, but as an understandable physical status, fostering a more grounded and rational basis for empathy and dialogue. The author's autobiographical case study also demonstrates the urgent need of an exploratory mode of healthcare for transgender individuals that empower themselves with intellectual respect. The goal is to give the client the intellectual tools to build their own understanding of themselves.
