This autoethnography serves not merely as a retrospective analysis of a single individual’s ``gender identity,'' but as a manifesto for a fundamental paradigm shift. It has traversed the intimate landscapes of personal memory -- from the stigma of childhood footwear to the alienation within online communities -- and the rigorous terrains of evolutionary biology and political philosophy. The synthesis of these disparate elements points toward a future that is not defined by the proliferation of identities, but by their dissolution. The ultimate prospect offered here is not the expansion of the gender spectrum, but the complete abolition of the ``gender'' category as a structural element of human society. The abolition of gender paves the way for the emergence of a true ``meta-identity.'' Just as the xenogender movement inadvertently hints at the infinite creativity of self-definition, a post-gender world allows individuals to define themselves through the boundless vocabulary of humanity rather than the narrow dialect of gender. The future prospects of this abolitionist project must address the reconstruction of scientific inquiry, social governance, and human subjectivity.

Ultimately, this project advocates for a return to, and a radical expansion of, the Enlightenment ideal. We should move beyond the identity politics that fragment humanity into competing tribes. The future should be an intersubjective world where independent, rational subjects can cooperate and communicate ideally. The abolition of gender is, therefore, a necessary step toward a universal humanism, where individuals are defined not by the categories imposed upon them, but by their capacity for reason, their moral agency, and their shared destiny.
