This article has presented a multifaceted critique of the prevailing concepts of gender identity and the theoretical underpinnings of queer theory. By synthesising evolutionary biology, neurology, philosophy of science, I have argued that gender identity is not an innate biological essence but a product of neuroplasticity shaped by interaction with social norms. Furthermore, the narrative demonstrates how the queer theory paradoxically reinforces gender-centrism they seek to dismantle by rejecting the premise of objective knowledge. Therefore, queer theory is harmful and politically dangerous to transgender rights. Finally, I proposed a blueprint for gender abolitionism: a framework that strictly restricts biological sex to gametes while dismantling phenotypic sex into a spectrum of independent traits, thereby removing gender as a valid category for social, legal, and self-identity classification.

Even for readers who may not fully embrace the political strategy of gender abolitionism, the scientific realist framework offers an alternative perspective for understanding gender identity, moving beyond essentialism and radical constructivism. For specific transgender individuals who, like myself, from scientific community or analytic tradition, and find their experiences rooted in material reality and personal history, this framework offers a path to self-understanding that reconciles biological reality with experience. Simultaneously, it invites onlookers to perceive gender dysphoria not as an inexplicable metaphysical conflict, but as an understandable physical status, fostering a more grounded and rational basis for empathy and dialogue, which will benefit transgender rights.
