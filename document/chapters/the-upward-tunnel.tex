The first thing I needed to consider was whether to view ``gender identity'' as internal or external. I almost immediately chose the latter, as the former contradicts some fundamental principles of evolutionary biology and neuroscience.

We cannot find a plausible selection pressure that could shape ``gender identity,'' regardless of what this ``gender'' refers to. If it refers to ``phenotypic sex,'' an organism does not need to know its ``sex'' to reproduce; in fact, most organisms do not. A male peacock's display is an instinctive behaviour driven by hormones. It does not require a self-concept of ``I am a male peacock.'' We can infer that human ancestors did not need such a concept before the emergence of advanced cognitive functions. Common childhood misconceptions about sex also show that humans do not ``naturally'' possess concepts of sex and gender, nor do they innately know which parts of their bodies are related to sexual activity or a specific sex. Some children believe that kissing leads to pregnancy, and some boys, upon seeing that girls do not have a penis, think that girls lost this organ because of injury.

If it refers to ``gender,'' then asserting that humans have an innate gender identity is too anti-Darwinist and illogical. To think that humans can have innate knowledge of an abstract, socially constructed concept like ``gender'' is almost a regression to Plato's theory of Forms. If it is neither ``sex'' nor ``gender,'' then, as we mentioned earlier, please specify what it is.

What can be innately encoded in the nervous system are only more basic biological things, such as the sucking reflex. However, a baby performing the sucking reflex does not mean they know what a ``breast'' is. In fact, they do not, as babies will suck on almost anything. Another example is the simple and complex cells that recognise specific visual patterns, but this is not equivalent to innately knowing what a ``square'' is. The brain needs to consciously integrate and coordinate these simple, basic neural signals and understand abstract concepts like ``equal,'' ``angle,'' and ``side'' to recognise a ``square.''

Of course, I do not deny that from a purely physicalist and reductionist perspective, the human brain could theoretically encode an innate ``gender identity.'' The so-called ``gender identity'' is just a physical state of the brain, involving the connection patterns of some neurons. With enough genes, the human brain could be initialised in this state. However, this claim is too \textit{ad hoc}. It is puzzling why gender identity would be encoded rather than any other abstract concept. For humans, innately encoding a ``gender identity'' or ``gender'' in the genome is far less practical than encoding a ``breast.'' The latter would allow babies to recognise breasts innately, distinguish what should and should not suck, and avoid sucking on harmful things. The evolutionary advantage is much more useful than encoding a ``gender identity.'' Since we don't even have an innate concept of a ``breast,'' it is even less likely that we would have an innate concept of ``gender identity.''

Moreover, the purpose of wasting so many genes to encode such a metaphysical, abstract concept is unclear, and the selection pressure is absent. Directly encoding such a complex abstract concept into the genome is biologically extremely inefficient and impractical. In contrast, evolving a brain with high neuroplasticity to learn abstract concepts is far more reasonable than pre-encoding them.

Almost all neuroscience research claiming an ``innate gender identity'' involves serious circular reasoning. Where does the brain get a sex? Does the bed nucleus of the stria terminalis (BSTc) \textit{have} a sex? \textit{Is} the BSTc a sex? Can the BSTc encode a complex concept like ``gender identity''? Rats also have BSTcs. Do rats have gender identities?

When some scientists claim that ``the brain characteristics of transgender people (e.g., the BSTc) are closer to their identified gender,'' they are dragging an objective and neutral neurological feature into the realm of human society and culture. They created a link between ``brain sex'' and ``gender identity''. They then named the brains of ``phenotypic male''/``phenotypic female'' individuals who conform to traditional gender patterns (so-called ``cisgender'') as ``male brains''/``female brains.'' The act of naming is a cultural act, not a scientific one. Without human society and culture, this would be a sexually dimorphic neurological feature statistically related to gamete type, but not ``sex'' or ``gender.'' If it could be considered ``sex'' or ``gender,'' then height and body fat percentage could also be. We could then refer to tall/short individuals as ``male height'' and ``female height,'' which is a severe contradiction of scientific methodology.

Of course, discomfort with or preference for a specific body morphology may indeed stem from innate differences in body schema. Some neuroscience studies show that some transgender men have an innate phantom penis sensation \parencite{Ramachandran2008Phantom}. However, according to the sex/gender division, neurological and physiological differences should be a part of \textit{sex}, not \textit{gender}. A preference for or aversion to a specific body morphology should also be considered an identity with \textit{sex} (``sex identity''), not with \textit{gender} (gender identity). For these transgender individuals, their ``gender identity'' might be a socially constructed conceptual chimaera -- it forcibly connects an innate, pre-linguistic feeling about the body with a purely socially constructed category of identity. Then it claims this connection is ``natural.''

As for the innate body schema differences, the reason they are interpreted within the framework of ``gender'' is still a product of society and culture, because we have artificially selected specific physical characteristics and deemed them to be ``gendered.'' ``I feel my body should have a penis'' or ``I feel an alienation from my breasts'' are innate, physical, neurological experiences. ``I am a man/woman/non-binary person'' is social, cultural, and political, carrying a great deal of cultural baggage, social expectations, and power relations. The mainstream gender-affirming discourse connected them, using the former to legitimise the latter, claiming that the former directly and necessarily leads to the latter.

Also, ``conversion therapy cannot change it'' is not a valid argument for ``gender identity is innate.'' Let's conduct a thought experiment: try using ``conversion therapy'' to change a person's native language. If we apply this logic honestly, consistently, and without compromise, we will inevitably conclude that ``native language is also innate.'' \footnote{I am not seriously suggesting a ``forced native language conversion''; this is clearly a violent, anti-human act.}

Without the intervention of society and culture, an individual would not choose words like ``male'' and ``female'' to describe themselves. Our social constructs have encoded these two words with meanings beyond their initial reproductive biological sense. Neurological features are not gender; they are simply personality traits, behavioural patterns, and body schemas. In social and cultural interactions, they may lead an individual to develop a so-called sense of ``identity'' with specific gender social constructs and gender roles. Still, they are not ``gender.'' This process occurs within the research domains of sociology and psychology. Using biological methods to study them is a serious category error, as ineffective as calculating the relativistic velocity of each car to study traffic flow on a road.