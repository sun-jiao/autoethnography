\documentclass[
  journal=,
  manuscript=article,  %% article (default), rescience, data, software, proceedings, poster
  layout=preprint,  %% preprint (for submission) or publish (for publisher only)
  year=2025,
%  volume=x,
]{extra/joas}
\usepackage[fontset=none, punct=plain]{ctex}
\usepackage[backend=biber, style=authoryear]{biblatex}
\usepackage{cleveref}
\usepackage{hyperref}
\setCJKmainfont{Noto Serif CJK SC}
\renewcommand{\abstractname}{Abstract}
\renewcommand{\figurename}{Figure}
\renewcommand{\tablename}{Table}
\renewcommand{\refname}{References}
% \renewcommand{\bibname}{References}
\crefname{appendix}{Appendix}{Appendices}
\Crefname{appendix}{Appendix}{Appendices}
\interfootnotelinepenalty=10000
\AtAppendix{\crefalias{section}{appendix}}
\addbibresource{references.bib}
\DeclareNameAlias{sortname}{last-first}
\DeclareNameAlias{default}{last-first}

\nolinenumbers

\title{An Autoethnography of a Gender Abolitionist Transgender Evolutionary Biologist}

\author{Jiao Sun\orcidlink{0000-0002-5028-8132}}
\affiliation{Division of Ecology and Evolutionary Biology, School of Biological Science, University of Reading, Whiteknights, Reading, RG6 6EX, United Kingdom}
\email{j.sun@pgr.reading.ac.uk}
\date{10/02/2025}

% maximum five keywords
\keywords{autoethnography, gender abolitionism, gender studies, introspective empiricism, philosophy of biology, reflective empiricism, scientific realism, sex/gender distinction, transgender philosophy, transgender studies}

% Important: don't overuse abbreviations. Only use abbreviations if the term is used more than ten times throughout the paper. Otherwise, write them in full.
\abbreviations{
The author uses ``phenotypic sex'' as an alternative to the conventional term ``biological sex,'' given the fact that actual biological sex (sex \textit{sensu stricto}) is about gametes.
}

\begin{document}

%\maketitle

\begin{abstract}
This autoethnography presents a comprehensive personal journey of a transgender evolutionary biologist examining the origins of their gender identity, finally culminating in an argument for gender abolitionism. Employing a methodology of introspective and reflective empiricism, cross-validated with personal records and memories, the author applies the principles of evolutionary biology and neuroscience to analyse their own gendered experiences. The analysis traces the author's so-called ``gender identity'' not to an innate, ontological essence, but to a complex synthesis of internalised social norms, childhood trauma, aesthetic preferences, and reactions to a ``pervasively gendered'' society that repeatedly assigned gendered meaning to neutral behaviours, objects, and personality traits.

The narrative critically engages with and deconstructs mainstream gender theories, revealing logical inconsistencies, circular reasoning, and category errors within concepts such as the sex/gender distinction, ``innate gender identity'', ``assigned sex at birth'' (ASAB), and the cisgender/transgender binary. These concepts are \textit{ad hoc} additions designed to preserve a flawed gender-centric model of the world, akin to the Ptolemaic system, rather than instigating a necessary Copernican Revolution. The author proposes this revolution in the form of gender abolitionism: a framework that strictly limits sex \textit{sensu stricto} to gametes, dismantles phenotypic ``sex'' into a spectrum of sexual dimorphic traits, and advocates for the complete removal of gender as a social, legal, cultural, and self-identity category.

Furthermore, the article extends its critique to post-structuralist and queer theories, arguing that while their deconstructive intentions are noted, their real-world effect paradoxically reinforces gender-centrism and engages in performative contradictions, ultimately failing to provide a path to genuine liberation. Ultimately, the work is a defence of universalism and scientific realism as essential tools for dismantling all forms of oppression, arguing that a consistent application of reason offers the most profound and inclusive path toward human freedom, transcending the particularist trappings of contemporary identity politics.
\end{abstract}

\addcontentsline{toc}{section}{Abstract}

\section{Introduction}\label{sec:introduction}
The contemporary discourse surrounding gender identity often presents a dichotomy between essentialist narratives -- whether rooted in religious conservatism or a search for a ``gendered brain'' -- and post-structuralist frameworks that view identity as purely discursive and fluid. For an observer grounded in the natural sciences, both extremes frequently appear to bypass the material realities of biology and the rigorous demands of logical consistency. The former often relies on a metaphysical ``soul'' or unproven neurological determinism, while the latter frequently risks dissolving the subject entirely into language games, potentially alienating the very lived experiences it seeks to describe. This article attempts to bridge this epistemological schism by examining the phenomenon of gender not through the lens of ideology, but through the analytic tools of evolutionary biology and the philosophical tradition of the Enlightenment.

I am a PhD student in plant taxonomy, using programming and statistical methods based on morphology and molecular phylogenetics to resolve the classification of plants. This methodology has deeply influenced the methods I used in my own journey of gender exploration. This article aims to document how an individual with a firm commitment to Enlightenment rationalism, who has received rigorous scientific training, attempts to understand a crucial aspect of themself. Autoethnography is a unique genre in which the author's own philosophical and scientific stances are part of the data. The soul of autoethnography lies in honestly presenting how the ``self'' (auto-) experiences and understands ``culture'' (ethno-). My ``self'' is the one who has been scientifically trained, committed to Enlightenment rationality, and tries to use logic and order to understand a chaotic and painful world. The positivist and rationalist attitudes are essential parts of the ``sanctuary'' in which I find my place in the world. Therefore, it must be faithfully presented here as the core of the story.

This inquiry necessitates a departure from standard narratives of transition. Rather than seeking to validate a pre-existing category of identity, this work interrogates the category itself. It asks whether the concept of ``gender identity'' possesses ontological weight when subjected to the same scrutiny one would apply to a taxonomic classification or an evolutionary hypothesis. By engaging with personal memory -- ranging from childhood socialisation to adult interactions within academic and online communities -- this article treats the self as a case study for a broader theoretical proposition: that the path to genuine human liberation lies not in the proliferation of gender categories, but in their abolition.

Furthermore, this text addresses the isolation often felt by those who fall outside politically convenient taxonomies. It explores the tension of being a ``gender abolitionist transgender'' individual -- a position that frequently invites hostility from both trans-exclusionary radical feminists and dogmatic sectors of the queer community. By grounding this analysis in the universalist principles of the Enlightenment, the following sections argue for a reconstruction of subjectivity that transcends the ``prison'' of gender, aiming instead for a humanism defined by reason, autonomy, and the courage to know.


\section{Shadows in the Cave}\label{sec:shadows-in-the-cave}
I became aware of my ``gender identity'' late in my undergraduate studies. The first thing I noticed was that I subconsciously thought from a girl's perspective. For instance, once, while discussing physical fitness tests with classmates, I said, ``My 800-metre run.'' A classmate pointed out (without malice, but with confusion), ``It's 1000 meters.'' I didn't respond directly, joking, ``It's actually 800 plus.'' They jokingly asked if I was a girl. I didn't deny it directly, replying with a Chinese internet slang, ``u1s1, qs'' (to be honest, yes). \footnote{In Chinese universities, the long-distance running test is 1000 metres for male students and 800 metres for female students.}

During this period, I developed a strong desire to adopt a more feminine name: 娇 (\textit{Jiāo}). This character means ``cute'' or ``adorable'' and features the ``woman'' radical (will be addressed as ``the feminine \textit{Jiāo}''). The feminine \textit{Jiāo} is a homophone and graphically like my legal name (骄, \textit{Jiāo}, means ``pride'', relatively neutral), with only the left-side radical different. I often used the Romanised version instead of Chinese characters because they have the same Pinyin. When signing, I would use a cursive script (行书, \textit{xíng shū}) to make it indistinguishable. Furthermore, because the feminine name is much more popular than my legal name, it is the first choice in many Chinese pinyin input methods. \footnote{When typing in Chinese, we use a software called ``input method.'' It gives all possible Chinese characters based on the pinyin the user inputs, and the user chooses one from them. So, it's a common thing to choose a wrong character.} Many people would type my name in the feminine version, and every time this happened, I was delighted and afraid that someone would ``kindly'' point it out. If this really happened, I would be very ``tolerant'' and say, ``It's okay, as long as I know you're addressing me, a wrong character doesn't matter.'' In an artificial intelligence class, one homework was to determine the gender based on provided names. I changed all the people with the same name as me in the dataset to ``female.'' Sometimes I even used that feminine name myself, and if discovered, I would blame the input method.

Because I never corrected it, many personal friends who knew me in informal places (like in student clubs) thought it was my legal name. Occasionally, someone would ask, ``Is your name really the feminine \textit{Jiāo}?'' and friends would argue with them, ``Why can't a boy be named the feminine \textit{Jiāo}? It's such a cute name! That's a rude question.'' I would pretend not to see the group messages, secretly enjoying the protection from friends. Some friends thought it was a nickname, which I also didn't correct.

I was also delighted when friends described me as ``quiet'' and ``virtuous'' (文静 \textit{wén jìng} and 贤惠 \textit{xián huì}, both are traditionally ``feminine'' adjectives in Chinese) because I didn't talk much and cooked during home parties. Once, we went camping and played an ice-breaking game called ``King and Angel.'' \footnote{This is a game in which everyone will be an ``angel'' of their ``King''  and need to do everyone's best to take care of their ``King'' In the final reveal, everyone needs to try to name their ``Angel'' according to the care they received. And the word ``国王'' \textit{guó wáng}, which means ``King,'' is theoretically gender neutral in Chinese.} I gave my ``King'' a handmade hog plum (\textit{Choerospondias axillaris}) bracelet, placing it into their clothes with a note. During the final reveal, the ``King'' said that upon seeing such a fantastic bracelet and delicate handwriting, they thought it would be from a girl. I secretly felt extremely pleased about that.

However, because the proportion of female students at my undergraduate university was very high (70\%), I didn't take this matter very seriously at the time. I initially rationalised this feeling as ``I was assimilated by girls after spending a long time with them.'' \footnote{I'm not saying that girls naturally possess so-called ``feminine traits'' and ``feminine behaviours.''}

Later, I attended graduate school, where the proportion of male students was significantly higher, and this feeling became increasingly difficult to ignore.

At an academic conference, someone wrote the feminised version of my name on a poster. I saw it in advance and said nothing. A junior male student discovered it and contacted the responsible officer to correct it. I felt a strong sense of disappointment at that moment.

For a period after that, I used deep learning-based image generation models to create some feminised photos of myself. I also scanned my ID card and graduation certificate, changed the sex marker to female, replaced the photo with an AI-generated female version, and saved them on my computer. Since I remained in my undergraduate student club's group chat after graduation, one day I saw some new students joining the group. I told one of them I was a girl and sent an AI-generated photo. They replied, ``Wow, a pretty sis.'' I was incredibly delighted.

I bought many ``gender-neutral looking'' women's shoes and clothes, including my hiking boots, which were specifically designed for women. Not only that, but I bought women's hiking boots from TOREAD and Decathlon Quechua. I even wore them on fieldwork in Xizang (Tibet) and western Sichuan, where I climbed many mountains, saw many birds, and collected many plant specimens. (\cref{fig:1})

\begin{figure}[htbp]
    \centering
    \includegraphics[width=0.7\textwidth]{F1}
    \caption{Photos of the author wearing Decathlon Quechua MH100 women's boots during a birding trip in western Sichuan Province: a) The author (\textit{Homo sapiens}), b) Golden Bush Robin (\textit{Tarsiger chrysaeus}), c) Blood Pheasant (\textit{Ithaginis cruentus}), d) Przevalski's Suthora (\textit{Suthora przewalskii}).\label{fig:1}} % title and label
\end{figure}

\section{Broken Chains}\label{sec:broken-chains}
\input{chapters/broken-chains}

\section{The Upward Tunnel}\label{sec:the-upward-tunnel}
The first thing I needed to consider was whether to view ``gender identity'' as internal or external. I almost immediately chose the latter, as the former contradicts some fundamental principles of evolutionary biology and neuroscience.

We cannot find a plausible selection pressure that could shape ``gender identity,'' regardless of what this ``gender'' refers to. If it refers to ``phenotypic sex,'' an organism does not need to know its ``sex'' to reproduce; in fact, most organisms do not. A male peacock's display is an instinctive behaviour driven by hormones. It does not require a self-concept of ``I am a male peacock.'' We can infer that human ancestors did not need such a concept before the emergence of advanced cognitive functions. Common childhood misconceptions about sex also show that humans do not ``naturally'' possess concepts of sex and gender, nor do they innately know which parts of their bodies are related to sexual activity or a specific sex. Some children believe that kissing leads to pregnancy, and some boys, upon seeing that girls do not have a penis, think that girls lost this organ because of injury.

If it refers to ``gender,'' then asserting that humans have an innate gender identity is too anti-Darwinist and illogical. To think that humans can have innate knowledge of an abstract, socially constructed concept like ``gender'' is almost a regression to Plato's theory of Forms. If it is neither ``sex'' nor ``gender,'' then, as we mentioned earlier, please specify what it is.

What can be innately encoded in the nervous system are only more basic biological things, such as the sucking reflex. However, a baby performing the sucking reflex does not mean they know what a ``breast'' is. In fact, they do not, as babies will suck on almost anything. Another example is the simple and complex cells that recognise specific visual patterns, but this is not equivalent to innately knowing what a ``square'' is. The brain needs to consciously integrate and coordinate these simple, basic neural signals and understand abstract concepts like ``equal,'' ``angle,'' and ``side'' to recognise a ``square.''

Of course, I do not deny that from a purely physicalist and reductionist perspective, the human brain could theoretically encode an innate ``gender identity.'' The so-called ``gender identity'' is just a physical state of the brain, involving the connection patterns of some neurons. With enough genes, the human brain could be initialised in this state. However, this claim is too \textit{ad hoc}. It is puzzling why gender identity would be encoded rather than any other abstract concept. For humans, innately encoding a ``gender identity'' or ``gender'' in the genome is far less practical than encoding a ``breast.'' The latter would allow babies to recognise breasts innately, distinguish what should and should not suck, and avoid sucking on harmful things. The evolutionary advantage is much more useful than encoding a ``gender identity.'' Since we don't even have an innate concept of a ``breast,'' it is even less likely that we would have an innate concept of ``gender identity.''

Moreover, the purpose of wasting so many genes to encode such a metaphysical, abstract concept is unclear, and the selection pressure is absent. Directly encoding such a complex abstract concept into the genome is biologically extremely inefficient and impractical. In contrast, evolving a brain with high neuroplasticity to learn abstract concepts is far more reasonable than pre-encoding them.

Almost all neuroscience research claiming an ``innate gender identity'' involves serious circular reasoning. Where does the brain get a sex? Does the bed nucleus of the stria terminalis (BSTc) \textit{have} a sex? \textit{Is} the BSTc a sex? Can the BSTc encode a complex concept like ``gender identity''? Rats also have BSTcs. Do rats have gender identities?

When some scientists claim that ``the brain characteristics of transgender people (e.g., the BSTc) are closer to their identified gender,'' they are dragging an objective and neutral neurological feature into the realm of human society and culture. They created a link between ``brain sex'' and ``gender identity''. They then named the brains of ``phenotypic male''/``phenotypic female'' individuals who conform to traditional gender patterns (so-called ``cisgender'') as ``male brains''/``female brains.'' The act of naming is a cultural act, not a scientific one. Without human society and culture, this would be a sexually dimorphic neurological feature statistically related to gamete type, but not ``sex'' or ``gender.'' If it could be considered ``sex'' or ``gender,'' then height and body fat percentage could also be. We could then refer to tall/short individuals as ``male height'' and ``female height,'' which is a severe contradiction of scientific methodology.

Of course, discomfort with or preference for a specific body morphology may indeed stem from innate differences in body schema. Some neuroscience studies show that some transgender men have an innate phantom penis sensation \parencite{Ramachandran2008Phantom}. However, according to the sex/gender division, neurological and physiological differences should be a part of \textit{sex}, not \textit{gender}. A preference for or aversion to a specific body morphology should also be considered an identity with \textit{sex} (``sex identity''), not with \textit{gender} (gender identity). For these transgender individuals, their ``gender identity'' might be a socially constructed conceptual chimaera -- it forcibly connects an innate, pre-linguistic feeling about the body with a purely socially constructed category of identity. Then it claims this connection is ``natural.''

As for the innate body schema differences, the reason they are interpreted within the framework of ``gender'' is still a product of society and culture, because we have artificially selected specific physical characteristics and deemed them to be ``gendered.'' ``I feel my body should have a penis'' or ``I feel an alienation from my breasts'' are innate, physical, neurological experiences. ``I am a man/woman/non-binary person'' is social, cultural, and political, carrying a great deal of cultural baggage, social expectations, and power relations. The mainstream gender-affirming discourse connected them, using the former to legitimise the latter, claiming that the former directly and necessarily leads to the latter.

Also, ``conversion therapy cannot change it'' is not a valid argument for ``gender identity is innate.'' Let's conduct a thought experiment: try using ``conversion therapy'' to change a person's native language. If we apply this logic honestly, consistently, and without compromise, we will inevitably conclude that ``native language is also innate.'' \footnote{I am not seriously suggesting a ``forced native language conversion''; this is clearly a violent, anti-human act.}

Without the intervention of society and culture, an individual would not choose words like ``male'' and ``female'' to describe themselves. Our social constructs have encoded these two words with meanings beyond their initial reproductive biological sense. Neurological features are not gender; they are simply personality traits, behavioural patterns, and body schemas. In social and cultural interactions, they may lead an individual to develop a so-called sense of ``identity'' with specific gender social constructs and gender roles. Still, they are not ``gender.'' This process occurs within the research domains of sociology and psychology. Using biological methods to study them is a serious category error, as ineffective as calculating the relativistic velocity of each car to study traffic flow on a road.

\section{Leaving the Cave}\label{sec:leaving-the-cave}
Having established this premise, what I should do next was clear: to introspect on everything related to gender in my grown-up and analyse which neurological features they might stem from, how they interacted with society and culture, and how these interactions might have shaped my so-called ``gender identity.''

I began with my current experiences, as described earlier, to identify the experiences that might have shaped them.

My desire for a feminine name might stem from my name being frequently miswritten or mistyped as the feminine \textit{Jiāo} by others in my childhood due to the input method issue mentioned earlier. (\cref{fig:2}) I was furious at the time. I suspect my preference for the feminine \textit{Jiāo} was shaped by repeated misuse.

\begin{figure}[htbp]
    \centering
    \includegraphics[width=0.7\textwidth]{F2}
    \caption{The author's name was spelt as the feminine \textit{Jiāo} by others. Left: The author's middle school name tags, with the middle one written as the feminine \textit{Jiāo}, and the top and bottom ones as the author's legal name; right: The author's border pass in Nyalam County, Tibet, with the name spelt as the feminine \textit{Jiāo} by officers.\label{fig:2}}
\end{figure}

The second thing that came to mind quite clearly was that in primary school, the shoes I wore were a type of white cloth shoe with red trim, locally known as ``gymnastics shoes'' (体操鞋 \textit{tǐ cāo xié}). My classmates considered them ``girls' shoes,'' and I was mocked and bullied for it.

It wasn't just that the colour and style were ``feminine,'' but were part of the girls' school uniform (the boys' version was the same style but blue). For occasions such as ceremonies and performances, girls were required to wear the red version, and boys the blue one. Therefore, in the micro-society of the school, they were absolutely and unquestionably considered girls' shoes. A few bullies snatched my shoes and threw them away, pulled down my pants, and pushed me onto the broken branches of a bush in the school garden, injuring me.

During that time, I often dreamt of losing my feet in an accident and would feel a strong sense of comfort and relief upon waking. I don't think this is body integrity identity disorder. I hypothesise that the dream might have been because in my dream, I believed that wearing girls' shoes meant having girls' feet, and I wanted to separate them from myself. This might indicate that I did not identify as a girl at the time, was very resistant, and wanted to separate that from myself.\footnote{I am not saying that my feeling of ``male'' or ``boy'' was internal and innate; it was necessarily learned as well. It may occur too early to be remembered by me. } Then, this idea of seeing a specific part of my body as a girl's might have gradually extended to my entire body being a girl.

After I grew up, I developed a unique attachment to these shoes. I even took a photo of myself wearing them, uploaded it to Wikimedia Commons,\footnote{\url{https://commons.wikimedia.org/wiki/File:Red_Uwabaki.jpg}} and added it to its Wikipedia article, making a specific private childhood experience a part of humanity's public knowledge repository.

I read some literature and researched their history, discovering that these shoes likely originated from Japanese indoor shoes (上履き, \textit{uwabaki}), mainly used in schools and kindergartens in Japan. Chinese clothing factories might got some Japanese orders to produce these shoes, which were later sold in China. Due to their simple style and low price, they became widely used as uniform shoes for elementary school students. In some Japanese schools, the colour of indoor shoes also depends on gender, but some schools assign colours by grade level. \parencite{Kanzaki2019Shogakko} Their study revealed that the shoe itself has no fixed gender meaning; its gender meaning is artificially assigned in a specific context. (\cref{fig:3})

\begin{figure}[htbp]
    \centering
    \includegraphics[width=0.5\textwidth]{F3}
    \caption{Application of gymnastics shoes/indoor shoes in different contexts: a-b) Photos of the author wearing red gymnastics shoes in childhood; c) An performance at a Chinese kindergarten, where boys wear the blue version and girls wear the red version, from Qilu.com; d) Screenshot from the social platform Xiaohongshu, where a Japanese blogger shares their school life, with both boys and girls wearing the red version of indoor shoes. Chinese users in the comments discuss this phenomenon with the Japanese blogger.\label{fig:3}}
\end{figure}

Following the thread of being bullied in elementary school, I recalled that the boys in my class often fought. I disliked playing with them. The girls were harmonious and friendly, and they were kind to me, so I enjoyed playing with them.\footnote{I am not saying that this gender-specific behaviour pattern is innate. } One thing I remember vividly is when we visited a museum with an interactive exhibit. Our class was split into two groups, boys and girls. When the boys' group played, if someone failed, they were harshly mocked and heckled by the majority, telling them to get down quickly and not waste others' time. When the same thing happened in the girls' group, it was filled with encouragement. I really envied the girls' group at that moment.\footnote{Again: I am not saying that this gender-specific behaviour pattern is innate.}

Then, a few other childhood memories flooded my mind: adults used to say my ``personality was like a little girl's,'' probably because I liked playing with stuffed animal toys and disliked sports and fighting. Another thing was that I was punished for imitating a little girl on a TV series, covering her mouth to laugh, being told, ``Boys can't laugh like girls.'' Children have no gender bias and imitate and learn all behaviours within their capabilities, and the gender meaning is externally imposed. I might have some innate personality traits and temperaments from my nervous system that are like the ``feminine temperament'' in gender stereotypes, making me more willing to imitate specific behaviours. However, these innate traits are fundamentally neutral. It has nothing to do with gender before being interpreted by adults as ``this is girls' behaviour.''

I also recalled a few things related to the girls' bathroom. In elementary school, I was playing tag with a few girls, and they repeatedly ran into the girls' bathroom to hide. I stood guard at the door, waiting to catch them when they came out, but a teacher saw me and punished me. Another incident was when a maths teacher punished some mischievous boys by making them clean the girls' bathroom. Another time, I got sick and vomited during class, and the teacher took me to the girls' bathroom to clean up. This was probably because the teacher was female, our classroom was close to the girls' bathroom, and it was during class time, so no one was in there.

These incidents are difficult to interpret with normal logic. I suspect that, in some way, they shaped the girls' bathroom in my young mind into a symbolic place filled with indescribable, chaotic, and contradictory implies and metaphors: it was a safe zone for girls, where they could hide during a game, while my attempt to use a logical strategy to win the game was inexplicably punished; it was a place of ``degradation'' for boys, where misbehaving boys were forced to enter and clean as a form of humiliation; it was a place of care, where when I was unwell, the usually forbidden rules were broken, and the teacher helped me clean my body and clothes. These events may have somehow shaped my fascination with female spaces, femininity, and female symbols.

Another significant difference between me and ``typical'' transgender people is that my dislike and anxiety about my body do not seem to be ontological. One piece of evidence is that I discovered the mechanism of masturbation at a very young age (around kindergarten age) and excitedly shared it with others. This indicates that my body schema was consistent with my physiological body.\footnote{I know this behaviour might be considered ``shameless'' or ``perverted'' by adults, but for my kindergarten self, it was just an objective exploration of the body, which I believe is no different in essence from sucking one's thumb or playing with one's hair.}

As we've discussed before, gender incongruence about physical characteristics and gender incongruence about social roles do not have the same neurological mechanisms. Otherwise, it would imply that humans can innately feel the traditionally defined ``gender'' (the sex-gender complex), which suggests a ``return to gender essentialism.'' Innate body incongruence may stem from the body schema. In contrast, nurture body incongruence may stem from the meaning that society and culture assign to the body and its impact on body image. Mine seems to be the latter. My dysphoria is primarily about identity and social roles\footnote{It has been explained by childhood experiences like the shoes and the girls' bathroom. }, and my longing for a female body is much milder than what other transgender people describe.

One thing that left a deep impression on me was that I had a crush on a girl in my childhood, but she didn't love me. I happened to read Stefan Zweig's \textit{Letter from an Unknown Woman}, in which the female protagonist has a one-night stand with the male protagonist and raises their child on her own. I thought at the time, ``Wow, I also want to have XX's child and raise them secretly.'' It's so great that girls can have babies; it's so enviable. Why can't a boy's body have babies? What a pity. This was an envy of a specific function, stemming from a longing for a romantic relationship.

Moreover, I find a kind of beauty in the female body that is hard to describe. It's a pre-linguistic aesthetic feeling, which could perhaps be described as ``a sense of elegance.'' I find the female body very pleasing to look at and feel envious. It's like how one might envy a bird for being able to fly, but it's unlikely to cause a body integrity disorder regarding one's own arms.\footnote{Our arms are homology with birds' wings. } I interpreted it as my longing for a female body is aesthetic and functional, not metaphysical.

I suspect that both my so-called ``sexual orientation'' (towards women) and the aesthetic part of my ``gender identity'' are products of this aesthetic experience, combined with different ``other factors'' (to use the term loosely). Combined with intimate emotions and sexual instincts, it becomes sexual orientation; combined with body image and external stimuli, it becomes the aesthetic part of ``gender identity.''

Another piece of evidence is that a DeepNude version of my feminised self has sexually aroused me.\footnote{I have to say, it's quite ironic that an AI application originally intended for sexual abuse was used by me to alleviate my gender dysphoria. } It is revealed that in my case, ``sexual orientation'' and ``gender identity'' share some more fundamental factors. This is phenotypically like \textcite{Blanchard1991Clinical}'s theory of autogynephilia. However, he explained that sexual orientation is the root cause, from which gender identity stems. This implies an innate, ontological sexual orientation, which I disagree with. I believe humans have genetically determined, neurological innate aesthetic and mate preferences, but they are not innately ``gendered.'' They are interpreted by society and culture. Individuals who have internalised social norms then integrated them into a ``gender.''

On the other hand, I have neurodermatitis (a.k.a. lichen simplex chronicus) in my genital area. There are multiple studies supporting that chronic itching and pain can affect one's body image. \parencite{Simsek2020Body, Vamos1993Body} This could be another reason why I dislike my reproductive organs. However, it should be noted that I developed neurodermatitis much later than the childhood events mentioned above, and neurodermatitis itself is a psychosomatic disease heavily influenced by psychological and mental states. \parencite{Lotti2008Prurigo, Tey2013Psychosomatic} Therefore, it could be a result of gender dysphoria rather than a cause.

My conclusion is: my current so-called ``gender identity'' is a synthesis of this aesthetic longing for the female body, envy of the reproductive function, and an internalised reaction to childhood experiences and trauma. Our life experiences and psychological responses to external stimuli are interwoven like a dense net, or rather, like a chain reaction, where one event triggers multiple preceding events, which continue to trigger subsequent events. Then we pick out a few phenotypically similar phenomena, give them a name: ``gender identity.'' This is pure tautology and has no other meaning.

The ``similarity'' between them is also guided and shaped by society and culture; otherwise, they would be independent and unrelated phenomena. Does a name have a gender?\footnote{Chinese is a language without grammatical gender. } Does a specific colour and style of shoe have a gender? Does encouragement and care from friends have a gender? Does playing with stuffed animals, disliking sports, roughhousing, or covering one's mouth to laugh have a gender? Does a specific geographical space, if not labelled ``Girls' Bathroom,'' have a gender? Does a specific body morphology and aesthetic preference have a gender? Does wanting to establish a relationship with a romantic partner through childbirth have a gender?\footnote{Some might argue that the last two do have a ``gender,'' which involves innate body schema differences, a point we have already discussed. Moreover, even from a biological perspective, a person can have a ``phenotypic female'' body or a uterus and simultaneously produce sperm.}

From beginning to end, gender is not inherent in the individual (me), but in the external society. It is the pervasively gendered society and culture that assigns a gender to all things. The individual, in their interaction with these ``gendered'' things, internalises social norms and develops a so-called ``gender identity.'' Therefore, ``gender identity'' is not an internal attribute but a product of humans' internalisation of social norms through interaction with the external world.

\section{Vertigo}\label{sec:vertigo}
After organising this self-analysis, I looked it over with satisfaction a few times. It was not long before I began to doubt this narrative, because I realised that some things seemed to be just "invented" rather than recalled by me, such as the story of being bullied because of my shoes. I have a memory of wearing red gymnastics shoes in elementary school, and I have a memory of being bullied by bad kids; both vivid and rich. In contrast, the causal relationship is entirely abstract and empty, without detail. I cannot find a memory of "10-year-old me crying and clearly realising that they were bullying me because of my shoes." I cannot even recall realising it at age 12 or 14. Not only that, but I had never seriously thought about why they bullied me before.

On the other hand, I cannot and will not ask them, and they have likely forgotten it entirely -- as we all know, this is typical for bullies. Thus, the truth is, I do not know why they bullied me at all. Perhaps it was simply because I looked easy to bully, or because I was introverted and didn't talk much. I "invented" this causal relationship after setting the goal of "finding the origin of my gender identity" and then recalling and analysing. This is a classic Texas Sharpshooter Fallacy.

In contrast, the memory of being mocked by classmates for wearing red gymnastics shoes is very vivid, specific, and full of detail. A case in point is that once in Chinese class, the teacher asked us to bring some childhood photos and tell stories based on them. In the photo I brought, I was wearing such shoes, and a classmate mocked me, "So you've been wearing girls' shoes since you were a kid" (or something like it), and I angrily snatched the photo back. The causal relationship between them is supported by details. Additionally, I found these childhood photos of me wearing red gymnastics shoes at home. (\cref{fig:shoes})

The story of visiting the museum is similar. The memory of being mocked and heckled by some male classmates, urging me to get down so they could play, and me crying in a corner while enviously watching the girls' group encouraging each other, is very real. The memory of my name being mistyped in childhood is also like this. I remember my mum angrily saying, "Can't they see it is a boy in the photo? Why would they use this character?" and demanding that I go to school the next day and have the teacher change it. This still happens today; many friends who clearly know my legal name, and even government officials, have typed my name as the feminised version. I even found physical evidence, such as name tags or documents. (\cref{fig:2})

Similarly, although I have a vivid memory of lying in bed and recalling the dream. I remember trying to fantasise about the dream's content to find peace and calm to help me sleep during a night of insomnia, the so-called causal chain from "my feet are girls' feet" gradually spreading to "I am a girl" is a "recent invention."

I even began to question those detailed and rich memories, as they could theoretically be retrospectively constructed. I tried to find physical evidence, such as the photos of me wearing red gymnastics shoes and the middle school's name tag with the wrong name. I also found the book \textit{Letter from an Unknown Woman}, which I read as a child, but most of my memories lack such corroborating evidence.

I tried to rationalise the reliability. Although our memories is not 100\% factual, a study by \textcite{Diamond2020Truth} shows that 93--95\% of verifiable details in human memory are accurate. With such a rate, the broad framework of the narrative and the conclusion we reached -- "gender is a product of the individual's interaction with and internalisation of social norms" -- are relatively reliable.

It no longer relies on a single, highly hypothetical causal chain like "being bullied because of the shoes, therefore seeing my feet as 'girls' feet,' and then gradually extending that to 'I am a girl'." Instead, it is based on a social pattern that repeatedly appeared throughout my development, summarised from a large amount of memory data: the individual's neutral personality traits, behaviours, and used items were repeatedly labelled with "gender" by the external social environment, accompanied by strong emotional feedback such as ridicule, discipline, exclusion, and even trauma. This series of interactions shaped the individual's perception of "gender" like a network or a "chain reaction."

After resolving the issue of memory reliability, I turned to another question: Is my "gender identity" \footnote{We will continue to use this term for now, although my definition differs from the mainstream.} really "female"?

This still seems to be a Texas Sharpshooter Fallacy. Before I began my analysis, I had presupposed that "I have a strong sense of identity with female identity and related social symbols." However, sometimes I very naturally and automatically think of myself as "male," especially when arguing with trans-exclusionary radical feminists (TERFs) online.

When they criticise or insult all "phenotypic males" in some gender-essentialist way (e.g., "all men are oppressors"), I feel very angry and use myself as a counterexample to refute them. Of course, this is partly because I know very well that their so-called "men" refers to phenotypic sex, not gender identity. It at least shows that although I usually feel uncomfortable when being called or classified as "male," this discomfort is not greater than my hatred for irrationality. If classifying myself as "male" can provide a valid counterexample to refute their argument. I am happy to substitute myself into $ \text{Male}(x) $ and logically falsify their universal proposition of $ \forall x (\text{Male}(x) \rightarrow P(x)) $.

I am not sure if this counts as a kind of "gender identity." When I argue with TERFs, the structure of the anger is the same as the anger I feel when arguing with extreme nationalists who proclaim that "all Japanese people are guilty, there are no innocent souls under the atomic bomb." \footnote{Referring to Hiroshima and Nagasaki.} I believe my anger is from the irrationality and collective responsibility. These two scenarios are almost logically isomorphic: an oppressive power (patriarchy/Japanese militarism) commits evil in the name of a specific group, which does not grant the oppressed group the right to indiscriminately attack the former group in return, because this power clearly did not receive authorization from the group it claims to represent.

While for me personally, there is a significant difference between these two situations: I am obviously not Japanese, so when I argue with extreme nationalists, I am very clear that this is a purely rational anger against irrationality. In the other situation, because I had not deeply thought about gender identity at the time, and their definition of "male," along with that of the broader society, did include me, the target and direction of this anger were often confused. I sometimes genuinely felt that I was a (specifically defined) "male" and that I was being insulted.

From a particular perspective, this is also "internalising a gender identity through interaction with a pervasively gendered society." TERFs use a crude, essentialist method to impose the label "male" on me. For the sake of debate, I strategically accept this label and develop complex emotional reactions around it. This "contextual male identity" and the "female identity," I feel, in many other scenarios, are formed by the exact same mechanism.

Thus, this phenomenon not only occurs in childhood but also appears continuously throughout a person's life. It is just that for most people, whether cisgender or transgender, after their gender identity is formed, they will consciously resist the invasion from another gender. I happen to not care much about "gender." I don't think my gender identity is very important to me, not my core identity, so I didn't resist it. Ironically, my ignorance of gender has allowed "gender" to be able to freely "invade" my "self." Some other "gender fluid" people may also stem from a similar cognitive mechanism. \footnote{I did not rule out other mechanisms of gender fluid. }

\section[“See-the-light”]{``See the Light''}\label{sec:“See-the-light”}
\input{chapters/see-the-light}

\section{Epicycles}\label{sec:epicycles}
\input{chapters/epicycles}

\section{The Copernican Revolution}\label{sec:the-copernican-revolution}
The only solution -- gender abolition.

What we should do is to strictly limit ``sex'' and ``gender,'' detaching them from individuals or organisms.

The biological sex should be strictly confined to the gametes \parencite{Lehtonen2014Gamete, Goymann2023Biological, Hurst1996There}. It is related to, and only to, gametes. Nothing else is ``sex,'' and biological individuals do not have such a sex. The question ``What is your sex?'' is a category error, just as we cannot call the colour of hair ``the colour of a person.'' Therefore, \textcite{FaustoSterling2000Sexing}'s so-called ``5 sexes'' is also a meaningless classification. Sexual dimorphism is a dynamic spectrum that constantly changes throughout evolutionary history. The so-called ``phenotypic sex'' should be dismantled into a series of discrete, decentralised phenotypic traits such as chromosomes, hormones, as well as morphology and anatomical traits. They are distributed on the spectrum of sexual dimorphism, along with sexually dimorphic phenotypic traits not traditionally classified as ``phenotypic sex'', like height, weight, body hair, and body fat percentage. The imprecise concept of intersex should be replaced by more detailed terms like intergonadal, intergenital, and sexual intermorphism. Gender should be strictly treated as a sociocultural phenomenon. Individuals do not have an innate gender; they merely ``possess'' gametes and ``live in'' a society and culture.

By biologically confining sex strictly to gametes, it cannot be ``assigned'' or ``inferred.'' In one case, a person's phenotype was completely ``phenotypic male,'' who had fathered a daughter, but one of his two ``testicles'' was actually a pure ovary (not an ovotestis), and dissection showed it had previously ovulated \parencite{Parvin1982Ovulation}. Therefore, most people, even those who have had children, cannot know for sure if they are chimaeras capable of producing two types of gametes. This makes it lose any possibility of being used for social classification, discipline, and oppression. It becomes a technical term of reproductive biology, completely ``exiled'' from everyday language and the operation of power.

``Gender identity'' should be strictly regarded as a ``narrative identity'' produced by the individual's interaction with and internalisation of social gender constructs. We do not study what a person's ``gender'' is, but rather the correlation of many sexually dimorphic traits (including phenotypic, genetic, and neurological) with gametes, how they are shaped by sexual selection in evolutionary history, how they are mutually regulated at the molecular level, how they are interpreted by and interact with society and culture, and how this interaction shapes the individual's self-identity.

Of course, the binary of sex (gametes) is still a very useful model in evolutionary biology, ethology, and reproductive ecology. It is obviously technically and ethically impossible to capture and dissect all individual animals to examine their gonads or the gametes they produce. In these cases, researchers should state in their \textit{Materials and Methods} section what method they used to estimate the gamete production capacity and the reliability of this proxy, such as the situation in birds revealed by \textcite{Hall2025Prevalence}. Sex is a microscopic reproductive biological attribute, not an externally observable phenotypic trait.

\textcite{Polderman2018Biological} claimed that the heritability of gender identity is 30--60\% and consistent with other behavioural and personality traits, which is a category error. As we have said before, ``gender identity'' cannot be encoded in the genome (this would require extraordinary evidence). We can only have a series of genes related to personality traits, cognitive styles, and body schema. Studying the ``heritability of gender identity'' is a huge logical leap. The correct scientific questions should not be: ``What is the biological basis of gender identity?'' but rather: ``Which heritable biological traits are assigned gender meaning by society and culture?'' (sociology). ``How does the meaning interact with the individual to construct a specific gender identity?'' (psychology). ``What is the biological basis of these traits?'' (genetics and neuroscience). Autism spectrum disorder is also highly heritable \parencite{Sandin2017Heritability}, and autistic traits are higher in those working in STEM fields \parencite{Ruzich2015Sex}. Following \textcite{Polderman2018Biological}'s methods, we might also be able to calculate a mathematically reasonable number of ``heritability of STEM ability'', which everyone would consider an absurd study. Autistic neurological traits are heritable, while participating in STEM fields is an extremely complex result of innate neurological traits, personal experience, academic training and so on. Biological studies of ``gender identity'' are precisely the 21st-century replication of ``scientific'' racism.

At the social level, we should abolish all gender concepts and any social constructs based on gender. The biological characteristics previously considered ``phenotypic sex'' should have no more social significance than height or weight. We should abolish gendered pronouns and titles, treat sexual orientation as an aesthetic preference, treat gender incongruence as a form of body anxiety, and treat sex reassignment surgery as a form of cosmetic surgery. A person's desire to change their body, whether from innate body schema incongruence or any acquired factor, is a matter of pure personal freedom. Its legitimacy does not need an unprovable ``gender identity.'' Gender markers should be removed from all nonmedical records, and sex markers in medical records should be changed to multi-level records (chromosomes, reproductive organs, fertility, hormone levels, etc.) rather than just male/female. We should eliminate gender-segregated facilities; the assumption that people are willing to expose their private parts in front of the ``same sex'' or share a room with the ``same sex'' is also a bias. Clothing stores should no longer be divided into ``men's/women's'' sections, but organised by clothing type (tops, pants, outerwear) and size/fit. Toys should no longer be divided into ``boys'/girls','' but by function and type (e.g., building blocks, dolls, science experiments, art creation) and appropriate age.

We should recognise that the concept of ``gender identity'' is not liberatory but oppressive. It leads individuals to consider the result of being brainwashed as their ``true self.'' Xenogenders should not be counted as gender identities. This is not because I, like conservatives or transgender medicalists (truscum), think these things are ``unqualified'' to be considered gender identities, but because I think gender identity is ``unqualified'' to include them. Xenogenders are a space of infinite human creativity. What right does the oppressive concept of ``gender'' have to include them?

This transforms the issue from a special identity under identity politics, a special psychological state (identity), and a specific phenomenon, to the dismantling of an oppressive system. Everyone, whether ``cisgender'' or ``transgender,'' is a victim of this system. It is a political issue that requires the participation of the entire society: why do bathrooms need to be segregated by gender, and why do documents need to be marked with gender?

Some people might object to my previous statement that ``neither feminism nor LGBTQ has opposed gender-centrism,'' claiming that ``queer theory is against gender-centrism.'' \footnote{I have once been told by someone that ``queer theory advocates for the abolition of gender,'' which left me extremely stunned.(\cref{fig:comments}) When did queer theory advocate for the ``abolition of gender''? Don't they, as post-structuralists, hate such top-down, revolutionary grand narratives?} I admit that queer theory is ontologically anti-gender-centric, as argued by \textcite{Butler1990Gender} that they do not consider ``sex'' (phenotypic sex) or gender to be natural, real or stable. Still, there is a huge difference between ontology and methodology. \textcite{Butler1990Gender} advocated for a strategy of ``repeating'':

\begin{quotation}
    The task is not whether to repeat, but how to repeat, or, indeed, to repeat and, through a radical multiplexing of gender, to displace the very gender norms that enable the repetition itself.
\end{quotation}

The issue is neither whether to repeat nor how to repeat, but what to repeat. It is a kind of ``repeat'' to create a lot of neopronouns, however, this is precisely because of the gendered pronouns in English, Butler's native language. Nobody would create a lot of neoterms about bicycles, because there are no masculine and feminine forms of it, e.g., \textit{bicycle} and \textit{bicyclette}. This strategy recognises the gender norm's power to define the area and boundary of the battlefield. Therefore, queer theory is ontologically anti-gender-centric yet methodologically extremely gender-centric.

The intended outcome of this strategy was to ``displace the very gender norms'' by showing their constructed and unstable nature. However, as the actual outcome, by constantly talking about and analysing gender, by ``repeating'' gender inside the boundary defined by itself, it has established ``gender'' as the central issue for understanding power, self, and society. \textit{Gender} was originally just an external social construct, but it has now become an internal psychological phenomenon. As we have discussed in the case of \textcite{Oberle2023Benefits}, gender identity (a psychological phenomenon) has been abbreviated to gender (a social construct), and these two concepts have been equivalent.

Queer theory created a powerful cultural incentive: to treat all kinds of unconventional personal experiences and identities as a ``gender'', because it has made ``gender'' the most attractive ``liberative'' discourse. If ``gender'' were still strictly understood as a sociocultural construct, xenogender people would likely not see their ``identity'' as a ``gender identity.'' They do so because ``gender'' is currently the most prominent and available discourse of resistance. This paradoxically expands the category of gender to encompass phenomena that might otherwise have been understood in different terms (e.g., as personality, philosophy, or simply as radical individuality). In fact, within the xenogender community, there is already reflection on this issue, and the alternative term xenoidentity has been proposed for those who have similar identities but are unwilling to classify their identity as a ``gender identity.'' Every new identity they create is a new ``true self'' that individuals can choose. This makes the most fundamental rule, ``you must have a gender identity'', more unquestionable. The queer theory theoretically opposes the identity politics, yet its strategy seems to offer infinite choices of identities.

Queer theory's strategy was to engage gender norms on the ``battlefield'' those norms had already defined. The unintended consequence of this engagement was that the battlefield itself -- the very concept of ``gender'' -- became more central, all-encompassing, and ideologically powerful than ever before, attracting everyone to it. The strategy recognised the prison's power to define its walls, and sought to destabilise them by operating inside the prison, painting them many different colours, but not destroy and escape from it. The outcome was that the ``colourful and comfortable'' prison became more fundamentally entrenched, attracted more and more people to move in and reinforcing the core idea that one must be a prisoner. The strategy reinforced the premise that a choice must be made (you must have a gender identity).

If the gender system cannot accommodate our existence, shouldn't we completely smash this system? Why must we tell them, ``Actually, we are also a kind of `gender'?'' By saying ``we are also a kind of `gender','' they have accepted that ``gender'' is a valuable category worth joining, which gave up the fundamental right to ask, ``Can we abolish the social norm centred on `gender'?'' \textcite{Butler2025Who} is a typical figure of this reformist ideology:

\begin{quotation}
    The critique of the gender binary, for instance, did not claim that ``women'' and ``men'' are over and done with. On the contrary, it asked why gender is organized that way and not in some other way. It was also a way of imagining living otherwise. The critique of the gender binary turned out to give rise to a proliferation of genders beyond the established binary versions -- and beyond the gender hierarchy that feminism rightly opposes.
\end{quotation}

As \textcite{Marx1977Critique} said in his \textit{A Contribution to the Critique of Hegel's Philosophy of Right}:

\begin{quotation}
    Luther, we grant, overcame bondage out of \textit{devotion} by replacing it by bondage out of \textit{conviction}. He shattered faith in authority because he restored the authority of faith. He turned priests into laymen because he turned laymen into priests. He freed man from outer religiosity because he made religiosity the inner man. He freed the body from chains because he enchained the heart.
\end{quotation}

The concept of ``gender identity'' leads individuals to equate a social construct with their true self, to internalise an external social construct as a core part of their self. It freed the body from ``gender'' because it enchained the heart. It successfully achieved what classical gender theory could not do for thousands of years. We should award it a one-kilogram-heavy medal.

\textcite{Cull2019Against}'s claim that ``abolishing gender would endanger the lives of transgender people'' is a pure category fallacy. Their entire article argued that ``A genderless society is harmful to transgender people by refusing to recognise their identity.'' In an interview with \textcite{Williams2014Gender}, Butler claims that ``some people really love the gender that they have claimed for themselves. If gender is eradicated, so too is an important domain of pleasure for many people. And others have a strong sense of self bound up with their genders, so to get rid of gender would be to shatter their self-hood'' is the same fallacy. They presumed the existence of ``transgender people'' as a fixed category in a society where the concept of gender no longer operates, which is essentialist. As we argued earlier, any innate trait, including the innate body schema inconsistency, is not naturally related to ``gender.'' In contradiction, it is the society that assigned gender meaning to them, we internalised the assigned gender meaning and classified ourselves under this framework. It's the same as saying, ``We need to use pathogens to make vaccines, so eliminating pathogens would harm the patient's life.''

The root cause of the suffering (such as gender dysphoria) and oppression is the social construct of gender and ``phenotypic sex.'' ``Transgender'' is not an individual condition stemming from an ``inner self'' that requires medical care, \footnote{I completely acknowledge that body schema inconsistency is an innate individual condition that needs medical care, while as we previously argued, considering it ``gender'' is a category error.} but a serious political issue concerning human reason and existence. Insisting on ``gender identity'' is to treat only the symptoms while allowing the pathogen (the gender construct) to persist.

True liberation is not achieved by merely managing the symptoms while allowing the pathogen to thrive. It requires eliminating the pathogen altogether. The central flaw of Cull's view is its failure to recognise that the category ``transgender'' is itself a product of the gender system. In a world without gender, the experience of having an atypical body schema might persist, but it would no longer be defined as a ``gender'' issue. The ultimate solution is not to fix this ``mismatch,'' nor to fight for rights through piecemeal social engineering, but to completely destroy the entire system. It is not to defend the identity of ``transgender'' but to achieve a world where such categories are no longer necessary to describe human experience.

True freedom is not infinite ``genders,'' but an infinite space filled with human creativity.

I believe that not only is xenogender a misclassification and not a part of gender identity, but even the term xenoidentity is inappropriate.

The current classification of ``gender identity'' relies on an unprovable assumption of importance. Why are terms derived from ``sex'' the most core identities, while queer, agender, and gender fluid are secondary, and other words are grouped into a general category of xenogender, making xenogender the ``other of the other of the other'' (gender $\supseteq$ non-binary gender $\supseteq$ xenogender $\supseteq$ a specific xenogender, like catgender)? The diversity and internal heterogeneity of the xenogender are far greater than the typical gender identity, and the sources of the words also cover a much wider range (catgender, doggender, parrotgender, and lizardgender already cover the entire Amniota, and men and women, as part of \textit{Homo sapiens}, are necessarily phylogenetically nested within). This classification is extremely unnatural, a non-monophyletic group, and anthropocentric.

Xenogender is neither ``xeno-'' (strange) nor ``-gender'' (gender). On the contrary, this is the broadest, freest ``self-identity,'' even ``meta-identity.'' It is the basic human ability of self-cognition, the purest, freest, most unrestricted ``identity.'' It is the act of ``an individual choosing a word to locate and describe themselves.'' It is what made us human beings, the organism with reason. It phylogenetically includes everything, linguistically (in theory) covers all words. Outside of it, we have no way to describe and define ourselves. This is the prerequisite for the birth of ``gender identity'' and even sociocultural gender constructs. Gender identity is essentially just a small part of this ``meta-identity''; it is a narrow identity that specifically uses a few specific words. The greatest guilt of ``gender identity'' is its attempt to elevate itself to a supreme position, even above the very thing that makes its existence possible.

The Earth is not the centre of the universe; it is just a planet in the solar system. The solar system is just a part of the Milky Way. The stars and nebulae on that ``outermost celestial sphere,'' once considered distant and unimportant, are the real universe.

\section{Dialogue with the Cave}\label{sec:dialogue-with-the-cave}
I posted my initial viewpoint on three different platforms. On two of them, I received insults or was banned. On the third, I received a more subtle form of violence. They asked me, ``Isn't this just queer?'' or ``You've reinvented queer theory.'' I told them, ``This is not queer. The difference between queer theory and my viewpoint is like the difference between Giordano Bruno's pantheistic philosophy and the Hubble model of the universe. They are within two completely different disciplines and frameworks. This is also not a reinvention. You can't say that Hubble reinvented Bruno's pantheism.'' \footnote{This is a fact. I started thinking entirely based on the first principles of evolutionary biology. I read \textit{Gender Trouble} after I completed my introspection.} When they interpreted my points as ``reinventing queer theory,'' they did what Derrida spent his whole life opposing: they eliminated the uniqueness of the Other. Some others said that my viewpoint was ``solipsism'' or ``Landian Accelerationism'' and I completely fail to understand where these labels came from. \footnote{``Accelerationism'' is the most absurd label that I have ever encountered. Both left-wing accelerationism and Land's right-wing accelerationism are theories about ``pushing the inherent logic of capitalism to its limits, leading to its self-destruction and creating a new world.'' The difference lies in the ``new world'' they envision (socialism/anti-humanism). Therefore, queer theory is more like ``gender accelerationism'' than my view. They're all about resisting a system by repeating/accelerating its own mechanism. } Moreover, they responded to my analysis of ``Indo-European-centric cultural invasion in gendered pronouns \footnote{I am not a linguistic purist. I completely accept reasonable loanwords. However, gendered pronouns is not a reasonable case. Chinese originally lacked gendered third-person pronouns. Before the New Culture Movement, 他 (\textit{tā}, will be mentioned as ``neutral \textit{tā}'' in the following text) referred to anyone, regardless of gender. During the Westernization wave in the 20th century, Liu Bannong created 她 (\textit{tā}, she, will be mentioned as ``female \textit{tā}'' in the following text) in 1917 to mirror the gendered pronouns in European languages, especially English. This sparked controversy; for instance, the magazine \textit{Women's Resonance} (妇女共鸣) argued that replacing the ``human'' radical (亻) with the female radical (女) from the neutral \textit{tā} to create the female \textit{tā} dehumanized women. Nonetheless, the utility of the distinction, especially in translation and new literature, led to its widespread adoption. Ironically, some ``modern Liu Bannongs'' brainwashed by the West propose creating a ``gender-neutral pronoun'' to mirror English's \textit{they}. It is not a natural and equal language contact, but a linguistic pattern that continuously replicates in history: imitating English pronouns, making English almost the ``upstream'' (in the software engineering sense) of Chinese. Additionally, to be honest, I completely cannot understand why do you use gendered pronouns in English. }, gender study and LGBTQ terms'' with the personal attack, ``Your mum is also a form of cultural invasion.'' (\cref{fig:6})

\begin{figure}[htbp]
    \centering
    \includegraphics[width=0.95\textwidth]{F6}
    \caption{The author's experience of being insulted in dogmatic communities on Zhihu, Xiaohongshu, and Reddit. Chinese comments translated using AI.\label{fig:6}}
\end{figure}

Later, with the goal of so-called ``community solidarity,'' I wanted to participate in a more moderate way, no longer mentioning that ``gender is an evil concept,'' ``we should completely abolish gender.'' I still couldn't understand why, after sharing an article about the lack of inclusivity of ASAB for intersex people (a brief version was mentioned earlier; the full version is in \cref{app:a}) in one community, I was immediately banned. Another community accused me of ``gender essentialism'' for thinking that ``phenotypic sex is more accurate and inclusive than ASAB.'' Moreover, I don't believe ``phenotypic sex'' is a real thing at all (as we discussed earlier). It's just a model created by humans, a tool, like ASAB, to represent the gender socialisation patterns. If ``phenotypic sex'' can more accurately represent gender socialisation patterns than ASAB, then of course we should use it. A more precise representation means higher inclusivity, recognising a richer biological and sociological diversity. As a man-made scientific model, what does it have to do with ``essentialism''? I don't believe there is a metaphysical essence or a Platonic Form of ``male'' and ``female'' behind this model. I never discussed the metaphysical status of this concept at all.

I was extremely puzzled. Why did it seem that no one was willing to read others' viewpoints with basic respect and seriousness? This viewpoint is moderate and morally necessary. I am curious if they really believe that the person in the thought experiment is a ``cisgender female.'' If they had seriously considered my point, they could not have reached this conclusion. It is extremely absurd and completely violates their own values.

It took me a long time to understand that their standard of ``inclusivity'' seems to be whether it sounds comfortable to them, and this standard of comfort seems to be arbitrary and without any fixed logic. The inclusivity I understand is using more accurate scientific models to represent the human biological and sociological diversity accurately. Suppose a model can describe and distinguish more diversity (e.g., different types of intersex people). In that case, it is more precise, and therefore more inclusive, than a model that flattens or ignores these situations (like ``assigned sex at birth'').

The prerequisite for any dialogue is that both parties are equal, mutually respectful, willing to listen, engage with the opponent's actual arguments, and willing to modify their own positions based on logic and empirical evidence. A dogmatic community completely refuses to engage in these activities, refusing to understand what others are saying, refusing to explain the specific content of their own views. They don't care what ``gender essentialism'' or ``solipsism'' really are. It's a trumped-up charge like ``blasphemy'' or ``witchcraft,'' the ultimate political tool an ideological system uses when facing a challenge it cannot respond to with logic.

Extending from this, what I find even more incomprehensible is why many people not only lack respect for others' views, but also lack respect for their own. Like the cases we have already discussed:

Conservatives also believe that sex is only related to gametes. Then, assigned sex at birth is technically impossible to implement, and almost all gender-segregated facilities should be completely abolished, because using bathrooms based on gametes is meaningless, and most people will never know if they are chimaeras capable of producing two types of gametes. While they dare not, as I do (I also believe sex should be strictly limited to gametes), push this view to this point.

This even includes some conservative scientists, like \textcite{Sokal2024Sex}, who claim that sex is an ``objective biological reality'', ``determined at conception and observed at birth.'' I agree that sex \textit{sensu stricto} is a biological reality. While according to \textcite{Jones2006Gamete}, ``Conception occurs when a sperm and an ovum fuse to become a zygote,'' ``The process of fertilisation, or conception, involves fusion of the nucleus of a male gamete (sperm) and a female gamete (ovum) to form a new individual.'' At ``conception'' (fertilisation), it is just a zygote. A zygote is just one cell. It cannot produce gametes and does not have a body. It has neither sex \textit{sensu stricto} nor phenotypic sex. Even if its genome will guide it to develop into a typical phenotypic male or female under normal conditions, it will not necessarily be a typical phenotypic male or female. To give a simple example, if a normal 46, XY zygote loses its Y chromosome during the first mitotic division after fertilisation, it will develop into a 45, X0/46, XY chimaera (mixed gonadal dysgenesis) or a 45, X0 individual with Turner syndrome \parencite{Gravholt2017Clinical, Jacobs1997Turner, Lopes2014Mosaicism}. Richard Dawkins, as an evolutionary biologist, cannot be unaware of this.

Liberals and some gender studies scholars believe that gender identity is an internal self-awareness, and imposing an identity is wrong. Then, the gender identity of most people should be undefined. Their imposing the label ``cisgender'' is not only self-contradictory, but also implies that transgender people are ``cisgender'' at birth.

Similar to the distinction between ``emotional inclusivity'' and ``epistemological inclusivity'' we mentioned earlier, many concepts within the EDI movement remain incomprehensible to me.

I've never understood why the English-speaking world chose to fix \textit{man} as masculine, rather than reviving the Old English masculine word \textit{wǣpnedmann} and modernising it to \textit{weaponedman} (or shortening it to something like \textit{poman}). In this way, we only need to revise the masculine \textit{man}, preserving all words with the suffix \textit{-man}. This solution has minimal impact and perfectly solves the problem of ``why \textit{woman} includes \textit{man}'': \textit{woman} is a kind of \textit{man}, so of course it includes \textit{man}. In contrast, fixing \textit{man} as masculine required modifying almost all words with the suffix \textit{-man}, which is far more troublesome and could not resolve the issue of \textit{woman}.

From a linguistic descriptivist perspective, both meanings of \textit{man} are lived modern English phenomena, neither superior nor inferior. Both solutions are artificial normative approaches. This isn't a competition between descriptivism and normative, but rather between two different normative approaches. From a linguistic normative perspective, my proposal is both ethically and cost-effectively the best one. They declared that this linguistic normative approaches is for ``equality'' and ``inclusivity,'' but they were unwilling to use the most equal and most inclusive solution.

The similar thing occurred in the paper by \textcite{Oberle2023Benefits}, which we discussed earlier. The conclusion they reach at the end of the paper is to recommend that botanists not address the reproductive function of plants \textit{gender}. Ironically, the second author, Emily Fairchild, is engaged in so-called ``gender studies'' and ``queer studies.'' Yet, in this paper, they try to fix a stable meaning for a word. They try to terminate the ``dissemination'' of the word \textit{gender}, turning it from a fluid signifier in endless différance into a well-defined, strictly controllable scientific concept, and want to give this word a meaning that is unambiguous, unmediated, and fully ``present.'' This is a typical logocentric obsession with presence, and is extremely anti-Derridean. In this scenario, \textit{gender} has already escaped its original domain (sociology) and established a rhizomatic new connection in another discipline (botany), yet they try to crudely pull it back, re-establishing rigid, hierarchical disciplinary boundaries. This reterritorialisation of thought is extremely anti-Deleuzian. Their attempt to establish a ``regime of truth'' and implement a cross-disciplinary linguistic discipline is extremely anti-Foucauldian.

They completely ignored the fact that they are also using ``biological sex'' \footnote{I acknowledge that the term ``sex'' is polysemous, while using the prefix ``biological'' is ridiculous. }, ``dimorphism'', ``science'' and ``rhizome'' in ways inconsistent with the scientific community, and that the word ``gender'' itself was borrowed from linguistics. The term ``rhizome'' in botany is also hierarchical; although it sometimes looks like a network, its branches are only mechanically connected, not physiologically. The mycelium of fungi is a better metaphor, because the hyphae of the branches can reconnect, which is called anastomosis. This is more consistent with \textcite{Deleuze2004Thousand}'s conception: any point of a rhizome can be connected to anything other, and must be. This is very different from the tree or root\ldots

\textcite{Butler2025Who} said that ``when they [TERF] argue that the problem is not trans, but `sex,' they mean \textit{biological sex}, \ldot (we will consider this question of \textit{biological sex} in the following chapter)'', but what TERF called ``sex'' is not the ``biological sex,'' which is only about gametes and, as we previously argued, is not observable or assignable \parencite{Lehtonen2014Gamete, Goymann2023Biological, Hurst1996There}. Many people do not know what kinds of gametes they can produce in their whole life. Sex \textit{sensu stricto} (gametes) is (at least in Vertebrata) a stable and strictly binary biological fact. If Butler is interested in the ``instability'' of ``\textit{biological} sex,'' they should look for the green algae (Chlorophyta) and Fungi. Of course, they discussed it in another chapter: ``However, even the drawing of this distinction proves to be a convention wrongly applied to the human species, \textit{given} [my emphasis] that all the members of some species of algae, fungi, and protozoans produce the same size gametes. In these cases, the species is divided into genetic groups known as `mating types,' but sex falls out of the picture.'' Do they really know what they were talking about? There is no causal relationship between ``algae, fungi, and protozoans'' and it (gametes binary) is ``wrongly applied to the human species.'' It is almost same to say that ``However, even the flying ability proves to be a convention wrongly applied to the passerine species, \textit{given} that all the members of some species of Spheniscidae cannot fly.'' What they should cite is \textcite{Parvin1982Ovulation} if they want to prove that the binary sex is ``wrongly applied to the human species [individuals].'' If they want to use ``algae, fungi, and protozoans'' to prove that the binary sex is not a universal natural law created by Deities, they should state this point clearly.

In the same book, they said that ``For some Christians, natural law and divine will are the same: God made the sexes in a binary way \ldots Regardless, this older \textit{science} holds to the proposition that sex differences are established in natural law \ldots'' This is almost the most ridiculous text that I have ever read. What does Christianity have to do with ``science''? Additionally, they also created the term ``gender dimorphism'' without giving a clear definition, but what on earth is it? We only have ``sexual dimorphism.'' In another chapter, they arbitrarily switched to ``sexual dimorphism.'' What is the difference between the two terms? If they are interchangeable, why do Butler use both of them in the same book? They stated that it ``is neither a simple fact nor an innocent hypothesis. It functions as a norm, if not a demand, that orders the way we see \ldots In such cases, the hypothesis is not revised by the evidence that is found; it forecloses that evidence, revealing itself as an obligatory epistemic norm, a compulsory phantasm, rather than good science.'' It seems that they confused ``sex'' (phenotypic sex) with sexual dimorphism. The phenotypic sex is a social norm, because it has direct impacts on our bodies, especially for intersex (intergenital) people. I take it that the traits included in phenotypic sex are a subset of sexual dimorphic traits. Nonetheless, sexual dimorphism itself is not a norm, given that it is also manifested in height, fat mass, muscle mass, bones, body shape and so on \parencite{Wells2007Sexual}. The difference is clear: a person with sexually atypical height, fat, muscle, bones or body shape is not considered ``intersex'' and forced to undergo surgery like what intersex people experienced.

Biologists usually don't write a commentary to a philosophy journal ``condemning'' Deleuze's ``misuse'' of the word ``rhizome'' or Butler's ``misuse'' of ``biological sex'' and ``dimorphism.''

Similarly, queer theory and the communities influenced by it claim to protect the ``Other'' and respect ``lived experience.'' \footnote{\textcite{Williams2014Gender}: I think I needed to pay more attention to what people feel, how the primary experience of the body is registered, ... \par \textcite{Butler2025Who}: ``Gender identity'' is a deeply felt sense of how one fits in the gendered scheme of things, the lived reality of one's own body in the world.} However, what they seem to do is respect only those experiences that are consistent with their own theoretical framework. A rationalist, materialist, transgender biologist trying to understand their own experiences and feelings in their own way does not seem to qualify as a valid, respectable human experience in their eyes. Or perhaps my experience is a ``dead experience.'' Furthermore, I believe that my practice of the categorical imperative as a child before reading Kant should absolutely be considered a valid personal experience of universalism and rationalism.

They claim to respect human experience, but they arbitrarily set boundaries to maintain the absolute authority of a specific kind of experience, and systematically disrespect, censor, and suppress all other conflicting experiences. What they seem to do is ``finding an Other, assimilating the Other with their theoretical framework, making the Other understand and speak of their experience with the post-structuralist terms, and ultimately expelling those who refuse the post-structuralist paradigm.'' Is it empowerment or a form of intellectual colonisation?

Although the framework I use is evolutionary biology, neuroscience, psychology, and cognitive science, my personal experience can be described perfectly well using Butler's gender performativity, Althusser's ``interpellation,'' and Foucault's diffuse power and discipline. My views and post-structuralism, although one from an internal individual perspective, one from a social perspective, both point to the same conclusion: identity is constructed in interaction with the internal and external environment, not an innate, fixed, \textit{a priori} essence. I am actually fully capable of rewriting this article using their frameworks and terms; I just don't want to. (I did write it, see \cref{app:b}.) As I said in the introduction, a firm commitment to science and Enlightenment rationality is an inseparable part of my ``self.'' At the same time, my analysis of ``using gendered pronouns to disrupt gender is Indo-European-centric'' is also fully consistent with post-colonialism. The more I understand their views, the more I cannot understand why they reject me.

My views were rejected, most likely because they broke a discourse monopoly. In their eyes, the form is more important than the content. What kind of gender worldview I actually described is irrelevant; what matters most is what language I used to say it. The tools of critique are no longer used for difficult and painful analysis; they have become obscure postmodern jargon, a sophistry to dismiss any criticism as ``you don't understand,'' and a way to show allegiance to the tribe. In that specific ideological framework, only discourse originating from specific thinkers is considered ``orthodox'' and ``safe.'' Scientific discourse, even if it reaches similar conclusions, is seen as a ``heretical,'' unwelcome external competitor. It threatens the purity and authority of that closed theoretical system.

Another reason is that they view so-called ``logocentrism,'' ``scientism,'' ``rationality,'' and ``grand narratives'' as a source of oppression even more terrifying than patriarchy and sexism. So much so that when they encounter someone speaking scientific language, who could originally be a companion of them, they immediately turn their guns on them, completely ignoring the fact that we have the same enemy. This is precisely a manifestation of ideological struggle and grand narratives. \textcite{Lyotard1994Postmodern} argued that ``postmodernism is the incredulity of all metanarratives.'' However, ``incredulity of all metanarratives'' itself has become an unquestionable metanarrative.

This is what I find most sad and desperate: almost no one -- whether conservative, liberal, or post-structuralist -- is interested in pure, consistent, self-coherent thought. No one is willing to bear the full logical consequences of their own ideology. They are only interested in winning. They are keen to defend their tribe, traditions, emotions, and political goals, and will adopt any temporarily useful premise. Once these premises become inconvenient or lead to unpleasant conclusions, they will quietly abandon them. They proclaim a series of premises but refuse to live in a world built on those premises as axioms. This is a destructive act full of intellectual bad faith.

This reminds me of \textcite{Habermas1987Philosophical}'s critique of Michel Foucault. He argued that Foucault's reduction of everything to ``power'' made rational communication impossible and destroyed the foundation for building a truly free society based on communicative rationality. \footnote{I do not agree with Habermas's all views. However, I do not intend to elaborate on this point in this article, as it would be off-topic. } This is perfectly shown in the communities deeply influenced by Foucault. Foucault has taught some social activists that every communication is just a move in a power game. When you tell people that there is no truth, only ``regimes of truth''; no reason, only ``power/knowledge''; no living author, only the ``author function''; no common, universal human destiny, only various discourses fighting each other -- what do you expect them to do? Under this influence, if someone presents a challenging argument, you have no obligation to listen to its reasoning; you only need to analyse its ``discursive effect'' \footnote{In my case, it was not even a real discursive effect, but their own prediction of a possible discursive effect. } and see it as an ``operation of power.''

According to this logic, Foucault himself should be the first to be judged, because ``morally judging and punishing an author based on discursive effects'' is precisely the real discursive effect produced in the discourse network by Foucault as an author function. Regardless of whether this was his own intention, \footnote{I know that it was not his intention. Foucauldian theory is an analysing framework.} its discursive effects have caused exclusion and harm in the real world. Foucault is the only author who cannot, like others (such as myself), claim ``that's not what I meant,'' because he himself (as the author function) has forbidden himself (as a natural person) from doing so.

This is not just a ``vulgarisation'' that has occurred in a few specific communities influenced by post-structuralism. As Habermas criticised Foucault, Foucault used reason to destroy reason, which is a performative contradiction, already containing the seeds of self-destruction.

The greatest strength of Habermas's critique is its immediacy and structural nature. Habermas's critique was fully articulated before Foucault's theory was ``vulgarised'' by the followers we see today. It did not need to wait for any subsequent events for verification, because it targeted an inherent structural flaw -- the ``performative contradiction'' of ``using reason to destroy reason.'' If Foucault was still using academic language, making arguments, writing, and expecting to be understood, he was inevitably caught in this contradiction. It is somewhat ironic that Foucault~\parencite{Boesers1977Folter} appealed to intent to defend himself; he claimed, ``I was talking about the French \textit{raison}, not the German \textit{Vernunft}; \textit{raison} is instrumental rationality, \textit{Vernunft} also includes value rationality, they are not the same.'' In this event, the natural person Foucault ``resurrected'' the ``dead'' author Foucault to practice Habermas's communicative rationality, trying to use rational argument to make others understand, ``I am not destroying reason.''

Out of the principle of communicative rationality, I will accept Foucault's own defence. We must also acknowledge that this phenomenon likely existed before Foucault; it could not have been created by him, at least the intellectual dishonesty of conservatives cannot be attributed to him. However, analysing it with their own theory: Foucault's description of the world as an anonymous, impersonal discursive field, claiming that the author's own intention is not essential, is not just descriptive but also performative. His works, through discursive effects, ``reproduced'' this impersonal discourse operation, providing people with a powerful new discourse and new arguments as intellectual weapons, allowing them to ignore the author's intent comfortably, refuse interpretation in good faith, and dismiss demands for intellectual coherence and logical self-consistency as ``power'' or ``discipline,'' equating it with ``instrumental rationality'' or ``governmentality,'' thereby further exacerbating this pre-existing problem. Foucault himself should have been the one to foresee this consequence most accurately.

Present-day post-structuralist scholars also live in such a performative contradiction, because academic activities presuppose the validity of rationality. Writing papers, organising arguments, citing literature, and responding to criticism all require the use of the ``reason'' they claim to critique, and require the expectation that others will sincerely understand them. Peer review presupposes the existence of public standards. The very act of submitting a paper to peer review implicitly acknowledges that there exists a public standard that transcends positions and can be commonly understood and judged. Without this public standard, peer review becomes purely a matter of ``taking sides'' -- ``if you are one of us (whether this `us' is divided by economic interests, academic factions, or `lived experience'), I'll pass you; if you are not, I'll reject you.'' Although this situation does exist in reality, at least in theory, this is not the stated purpose of peer review.

According to Foucault's definition of power, a queer theorist who teaches at a top university, can influence the thinking of a generation of students, can decide the academic future and fate of students, defines what are ``valuable'' questions and ``legitimate'' research methods in a specific field, publishes articles in famous journals, and is honoured by the media as a representative of ``progressive thought,'' should also be considered an essential node of power. Their self-proclaimed role of ``speaking for the marginalised,'' claiming to be on the margins of ``power'' and the Other, is sociologically absurd. From a Foucauldian perspective, this self-claiming is not an objective description of power, but a power strategy that attempts to produce a kind of authority for itself through discursive effects. At the same time, the mainstream liberal transgender community produces the ``innate gender identity'' through discursive effect and, through a series of social movements, implements its own political claims into written law, setting the threshold for community recognition, medical resources, and legal identity. This is also a form of power. They are not powerless victims, but an actively operating ``regime of truth.''

However, post-structuralists rarely include themselves in the list of ``powers'' when analysing the operation of power. They always use external power to define the ``centre'' and the ``Other.'' When essentialist transgender people (e.g., the mechanical materialist ``gender brain'' or the idealist ``gender soul'') declare ``do not deconstruct our gender identity,'' queer theorists tend to become very gentle and retract to an essentialist position (as in~\cite{Williams2014Gender}), even if they theoretically disagree with these people's essentialist claims (as in~\cite{Butler1990Gender}).

This is not an accident, but an inherent structural problem of their narratives. Because the size of a politically effective group has the lowest limit, although this is not a mathematically precise boundary. If post-structuralists want to maintain the political effectiveness of their actions, they must stop when they touch this lowest limit. Therefore, those who ultimately receive the most attention and protection are definitely not the most special, the most vulnerable, and the most excluded people, but a self-declared ``minority'' group that is large and major enough to make the loudest sound and form effective political actions.

If they consistently applied their own philosophy, they would have to stand with the ``Other'' of the dogmatic transgender community, and then, after these Others have formed a power, continue to stand with the Other's Other\ldots until one day, they must stand with the individual. In this way, they would return to the position of individualism and universalism.

A transgender, gender-abolitionist, anti-essentialist evolutionary biologist is clearly a multiple-dimensional ``Other,'' whether to conservatives (obviously), liberals (they like the ``born this way'' narrative), the mainstream academic community (I oppose both Richard Dawkins's ``scientific'' gender essentialism and the ``scientific'' practice of searching for a ``neurological proxy'' for ``gender identity''), the mainstream essentialist transgender community (they believe in an internal, essential gender identity), or the mainstream anti-essentialist transgender community (they are often post-structuralist, like to talk about discourse and power, and are wary of people who talk about science)\ldots I have never seen post-structuralists stand with this kind of politically valueless ``Other.''

If they truly respect ``lived experience,'' most people subjectively feel that the earth is flat and still, that spicy is a sense of taste, that we have a complete, unified, innate self-awareness, but almost no one rigorously respects these subjective experiences. Including post-structuralism itself, from Foucault onwards, it is built on the systematic negation of the subjective experience of a ``complete self-subject.'' If they defend themselves by saying ``the subject is a social construct that originated in the 18th century''~\parencite{Foucault1994Order}, then gender identity is exactly the same, a social construct that originated in the 20th century, thus should also be deconstructed as they did with ``subject.''

The respect for ``experience'' by post-structuralism and its branch, queer theory, is highly selective, conditional, and full of unstated political considerations. It is not a consistent principle. They are actually executing a hidden rule: ``We respect those subjective experiences consistent with our philosophy and political agenda.''

The reason is simple. For questions about the shape of the earth, whether spiciness is a taste or pain, and the formation of self-identity, the scientific conclusions are unequivocal. Doubting these things would immediately associate them with anti-science conspiracy theories like ``flat-earth theory'' and would make post-structuralists lose all credibility. In contrast, gender identity is still a complex and controversial area in neuroscience and cognitive science, far from reaching a definitive conclusion. This is a kind of ``critique in the gaps,'' similar to ``God in the gaps.'' They claim to be ``postmodern,'' but their behaviour in this regard is very pre-modern.

This is, in fact, a Cartesian mind-body dualism. They claim to oppose Descartes, but they have secretly resurrected Descartes's ghost, attempting to preserve a domain in human consciousness that is exempt from the tests of natural science, arguing that the general scientific methodology based on empirical evidence is invalid in this category. This is a pre-modern fantasy doomed to fail.

\subsection*{Concluding Scientific Postscript}

\textcite{Butler1990Gender} criticised \textcite{Page1987Sex}'s study on sex-determining gene \footnote{This gene is not the famous SRY gene but the Zinc finger Y-chromosomal protein gene, which was once an important candidate for the testis-determining factor.} in the subchapter \textit{Concluding Unscientific Postscript} of their famous book \textit{Gender Trouble}. Butler proposed two main points:

\begin{enumerate}
    \item Why do we want to find a \textit{master gene} [sic]?
    \begin{quotation}
        The framework suggests a refusal from the outset to consider that these individuals implicitly challenge the descriptive force of the available categories of sex; the question he pursues is that of how the “binary switch” gets started, not whether the description of bodies in terms of binary sex is adequate to the task at hand.
    \end{quotation}
    \item Why do we want to find a \textit{mater gene} [sic] determining males?
    \begin{quotation}
        Ovary-determination is never considered in the literature on sex-determination and that femaleness is always conceptualized in terms of the absence of the male-determining factor or of the passive presence of that factor. As absent or passive, it is definitionally disqualified as an object of study.
        \par The concentration on the ``master gene'' suggests that femaleness ought to be understood as the presence or absence of maleness or, at best, the presence of a passivity that, in men, would invariably be active.
        \par Unfortunately for Page, there was one persistent problem that haunted the claims made on behalf of the discovery of the DNA sequence. Exactly the same stretch of DNA said to determine maleness was, in fact, found to be present on the X chromosomes of females. \footnote{The Zinc finger X-chromosomal protein gene.} Page first responded to this curious discovery by claiming that perhaps it was not the presence of the gene sequence in males versus its absence in females that was determining, but that it was active in males and passive in females (Aristotle lives!).
    \end{quotation}
\end{enumerate}

I agree with the first point that Page's hypothesis about a ``binary switch'' \footnote{\textcite{Page1987Sex}:
\par \begin{quotation}
    The mammalian Y chromosome, by its presence or absence, constitutes a binary switch upon which hinge all sexually dimorphic characteristics. \ldots There must exist on the Y chromosome one or more genes whose products, directly or indirectly, determine all aspects of sexual dimorphism.
\end{quotation}} is over-simplified and untenable, even in 1987.

As early as the 1920s, Richard Goldschmidt proposed a theory of sex (phenotypic sex) determining based on his hybridisation experiments of two \textit{Lymantria} species. In his ``balance theory of sex,'' the sex is determined by the balance and the quantitative relationship between the male and female factors. The sex chromosomes carry some of them and autosomes carry the others. Two sexes are not clearly distinct opposites. An individual is on a position in the female-to-male continuum \parencite{Dietrich2016Experimenting}. This is quite consistent with the modern genomic research: the development of fetal gonads depends on the antagonism between two sets of gene networks, which are widely distributed on sex chromosomes and autosomes in the genome \parencite{Graves2010Homologies}. If sex in simple organisms like \textit{Lymantria} is determined by the antagonism between multiple factors, then it is logically unreasonable for more complex vertebrates to have a single ``binary switch.''

Nevertheless, Butler's second critique was based on a misinterpretation of Page. He, actually, proposed four hypotheses:

\begin{enumerate}
    \item The X-encoded protein does not function in gonadal sex determination.
    \item The X and Y loci determine sex antagonistically.
    \item The X and Y loci determine sex in concert, while they are not interchangeable.
    \item The X and Y loci are interchangeable. However, in females, one of the two X chromosomes is inactivated. Therefore, a single dose (on the active X) determines female and two doses (on X and Y, respectively) determine male.
\end{enumerate}

Page himself preferred the fourth model, he said, ``Models 1, 2, and 3 all fit well with the prevailing notion of a dominantly acting sex-determining factor unique to the Y chromosome. A fourth model does not fit with this prevailing notion, but its simplicity is
attractive.'' More than half of this section was used to discuss the fourth model. In other words, Page considered the most anti-Aristotelian model simple and attractive. This is a qualitative hypotheses (one dose vs two doses), this gene is not ``inactive'' or ``passive'' in females, on the contrary, the single dose was actively determining the gonadal sex. Moreover, he did not use Aristotelian terms ``active/passive'' in the original paper. Yet Butler imposed these terms on their study and satirised him by saying ``Aristotle lives.'' I agree that analysing the cultural and philosophical load behind a scientific study is reasonable, which is helpful for better scientific practice. However, it is irresponsible to attribute it to scientists' own ``claiming.''

Another issue lies in Butler's criticise is that even Page really proposed a ``master gene'' for male determining, what is its relationship with ``Aristotle lives?'' The same structure can, on the contrary, be interpreted as ``female is natural and default (the first sex) and male is derived (the second sex).'' It was, in fact, the mainstream interpretation of the Sry gene before the discovery of the WNT4 gene \parencite{Ainsworth2015Sex}. This interpretation was, and is still, used by feminists and transgender advocators to rebut the androcentrism and ``sex'' ideology \footnote{I coined this term to imitate Trump's ``gender ideology.'' } of patriarchy, including \textcite{Trump2025Defending}'s Executive Order (as in~\cite{Garcia2025McBride, Yeo2025Trump}).

If the same scientific result can be interpreted as both ``male is active, female is passive'' and ``female is default, male is derived,'' it is untenable to choose a specific interpretation and use it to criticise the original study. This indicates that Butler has fallen into a confirmation bias. They presupposed a template of ``metaphysical binary opposition'' rooted in the ``Western philosophical tradition,'' then misinterpret the text to comfort their preset framework, ultimately using this to ``prove'' their conclusion: I have found the unstated power behind the metaphysical preset of scientists. This is both the straw man fallacy (misinterpret others' viewpoints) and Texas sharpshooter fallacy (choose the interpretation that is easiest to criticise from two or many of them).

It devastatingly demonstrates what happens when a post-structuralist, even the most elite of them, places their theoretical framework above the rigorous close reading of text. A better, more responsible post-structuralist analysis should: a) accurately presents all four hypotheses, b) acknowledge Page's personal preference for the least binary model, c) analyse why, despite the complex thinking within the scientific community, like Page's, the simple binary ``master gene'' narrative is still dominant in our society.

Ironically, Page himself published a genomic study in 2023 \parencite{San2023Human}, which revealed that the ``inactive'' X chromosome regulates the active X chromosome in humans.


\section{Enabling Act}\label{sec:enabling-act}
Moreover, such situations also exist in areas other than gender. Considering this is an ``autoethnography,'' let's choose the historical event discussed in my middle school language arts course as a case study. Just as I thought in middle school, Germany's post-war ``Vergangenheitsbewältigung'' (overcoming the past) historiography also has huge internal tensions:

Why should those former anti-Nazi heroes, Nazi victims, their families, and descendants ``reflect'' and pay taxes for Nazi war reparations? Why should new immigrants, even those from countries invaded by the Nazis, also ``reflect'' and pay taxes for Nazi war reparations? Constitutional patriotism holds that the government does not represent a specific ethnic group, culture, or history. Then it is simply a legal entity responsible for administrative affairs, employed by the people to serve them. If the people did not directly support its crimes, why should they reflect on its actions?

What is special about the sovereign state? Why is it the sovereign state (Germany) and not a higher-level political entity (the EU) or a lower-level political entity (the state of Prussia) that needs to reflect on war history? The designation of this level lacks inherent, logical reason. This seems to imply that ``Germany is more likely to start another war in the future, so it must be prevented in advance,'' which cannot be proven. By limiting the scope of ``reflecting on Nazism'' to Germany, it allows other countries -- especially the United States and Israel -- to comfortably see themselves as victors or victims, thus naively believing they are ``naturally immune'' to fascism, providing spaces for fascism to return under a different name. Furthermore, Germany comfortably supports it under the absurd and irrational so-called ``special responsibility to Israel.''

Why can a person openly display the Iron Cross symbol but not the swastika? Although the use of the swastika in Germany for Buddhism and Hinduism (as well as for historical education and research) is still permitted, there is no similar restrictions for the Iron Cross, like ``for military uses only.'' Is this fair to Germany's East and South Asian immigrants? Both are cultural symbols with long histories that predate the Nazis. The swastika's history is even longer than the Iron Cross's, and we can even find right-facing, rotated 45° Buddhist swastikas on some ancient buildings in China. (\cref{fig:sayagata}) Its pre-Nazi usage was also more peaceful than the Iron Cross, which was associated with the Prussian army even before the Nazis. Is the only difference that the Iron Cross is ``German,'' while the swastika is not? The former is ``our,'' ``Deutschness'' national culture, to be cherished, saved, and purified; the latter is ``external,'' ``Other,'' ``non-Deutschness,'' can be arbitrarily defined by the Nazis, and cannot be reclaimed after being tainted by them. Does this mean that Germany is still essentially a nation-state of the German people, just claiming not to be one in words?

\begin{figure}[htbp]
    \centering
    \includegraphics[width=0.7\textwidth]{Sayagata_motives_on_wall}
    \caption{Both left- and right-facing, rotated 45° Buddhist swastikas on a Chinese ancient building. Author: \href{https://commons.wikimedia.org/wiki/User:Yongxinge}{Yongxinge}, CC BY-SA 3.0 \label{fig:sayagata}}
\end{figure}

Why is the Iron Cross not an unconstitutional symbol? Because the Wehrmacht is not an unconstitutional organisation. Why is the Wehrmacht not an unconstitutional organisation? Because ``we'' stipulated that it is not. However, historical research shows that the Wehrmacht was far from innocent \parencite{Wette2006Wehrmacht}. This seems to be a self-fulfilling prophecy. The legislators of the Federal Republic of Germany prophesied: ``The Iron Cross can be saved and purified, while the swastika cannot.'' Then, they passed legislation allowing the unrestricted use of the Iron Cross while completely banning the swastika, ensuring that this prophecy would forever be true.

Germany's so-called ``overcoming the past'' is filled with a series of lies, self-contradictions, and doublethink.

I believe the only logically consistent solution is to execute all high-ranking leaders with leadership responsibility in the Nazi Party, concentration camps, Wehrmacht, SS, and SA, as well as low-ranking members who directly killed people, as murderers. Low-ranking members who did not directly kill people should be sentenced or acquitted based on their specific actions (e.g., helping the victims). The released members and the rest of the German people would live freely and without original sin in a clean Germany.

Some will say my solution is too naive. Nevertheless, this is not naivety, but the logically consistent extension of the Nuremberg trials.

Suppose we believe that international humanitarian law can be applied retroactively. In that case, we should apply it universally, extending the principle established by the Nuremberg trials to all levels of members of the Nazi party and its related organisations. Similarly, any form of international humanitarian law should also be applied retroactively. The bombing of Dresden, Tokyo, and the atomic bombings of Hiroshima and Nagasaki violated the principle of proportionality in the 1977 Additional Protocol to the Geneva Conventions. Roosevelt, Truman, and all US military personnel involved in these actions are guilty. If the international law of 1945 can be used to try crimes of 1942, why can't the international law of 1977 be used to try crimes of 1945? Moreover, why is the ``principle of proportionality'' considered a reasonable argument for war crimes? If that ``any country's municipal law does not allow killing innocent people'' is a reasonable argument for the retroactive application of Nuremberg laws, any country's municipal law does not accept the ``principle of proportionality'' as a reasonable argument for killing. I don't see any difference between them.

By the same token, if it is said that after 70 years of aggression and occupation, Hamas killing hundreds of Israeli civilians is called ``terrorism,'' let's try to extract a universal rule from this event: the invaded and occupied country cannot kill civilians of the aggressor or occupier country during its resistance, otherwise it is ``terrorism.'' \footnote{As an abstract principle, I do not oppose this. Actually, I completely agree with it. ``The crimes of the rulers are not the fault of the ruled; governments may sometimes be robbers, but the people never are.''}

Thus, let's now apply it universally: after the United States experienced the Japanese invasion of Pearl Harbor and four years of the Pacific War, the US military killing hundreds of thousands of Japanese civilians with incendiary bombs and atomic bombs is also terrorism, and should be a more heinous form of terrorism -- state terrorism, because the order to massacre civilians came directly from the US president. In contrast, Hamas is not the legitimate government of Palestine, so Hamas's actions are just ordinary terrorism, not ``state terrorism.'' \footnote{I am not saying that Hamas killing Israeli civilians is correct. There are, of course, innocent civilians who do not support Zionism, just as there were innocent civilians in Hiroshima and Nagasaki who did not support Japanese militarism. Actually, the Japanese Communist Party and Socialist Party clearly documented some Japanese political prisoners who died in the atomic bombings.}

We, as I have analysed, either use retroactivity universally or do not use retroactivity at all.

Germany's military aid to Israel's genocide crime in Gaza~\parencite{Soussi2023War}, defining the Boycott, Divestment, and Sanctions (BDS) movement and accusing Israel of genocide as so-called ``anti-Semitism''~\parencite{Kuras2023Strange, Whittle2024Germany} is also this kind of exceptionalism. The concept of ``special responsibility'' is not a universal principle; it is an exceptionalist political tool used to justify specific foreign policies. It allows Germany to hypocritically present its geopolitical actions as an inevitable ``moral'' responsibility stemming from its unique history, rather than a matter of its national geopolitical interest. If Germany has a ``special responsibility'' towards Israel because of Nazis, there is a frightening, hidden logic: ``Israel,'' a sovereign state, has an \textit{a priori} inherent relationship with Jews, an ethnic group. This is almost isomorphic with the Nazis' racial theory that claimed an \textit{a priori} relationship between Germany and ``Aryans.''

 Many early Zionists were secular Jews who received European education, who are a product of the Haskalah. However, they believed the Haskalah ideal (Jewish integration into European civilisation) was impossible and instead employed another European tool: nationalism, to attempt to establish a Jewish nation-state. The Haskalah's universalist ideal, in its own failure, transformed into its opposite: a particularist, nationalist practice. As a modern nationalist project, it employed the instrumental rationality and colonial logic. It required calculations of land, population, resources, and security, viewing non-Jewish populations (particularly Palestinians) as resources to be calculated, managed, and controlled. To achieve its instrumental rationality, Israel's national security apparatus both exploited Palestinian labour and excluded and even expelled them for the so-called ``security.'' The instrumental rationality is almost isomorphic to \textcite{Adorno1997Dialectic}'s critique in the \textit{Dialectic of Enlightenment}. Israel's occupation and blockade of Palestine can be viewed as a form of ``the Dialectic of Haskalah.'' \footnote{This is an imitation of \textcite{Adorno1997Dialectic} rather than a serious analysis. In my view, the Enlightenment, as well as the Haskalah, is pure and flawless. Nazis and Zionism are betrayals of them. } However, Adorno himself supported Israel. \footnote{ \textcite{Braunstein2018Wahrheit}:

\begin{quotation}
    Adorno schrieb am 5. Juni 1967, während des Sechstagekrieges, an seine Wiener Freundin Lotte Tobisch: »Wir machen uns schreckliche Sorgen wegen Israel. \ldots In einem Eck meines Bewußtseins habe ich mir immer vorgestellt, daß das auf Dauer nicht gutgehen wird, aber daß sich das so rasch aktualisiert, hat mich doch völlig überrascht. Man kann nur hoffen, daß die Israelis einst -- weilen immer noch militärisch den Arabern soweit überlegen sind, daß sie die Situation halten können.«
    \par Adorno wrote to his Viennese friend Lotte Tobisch on June 5, 1967, during the Six-Day War: ``We are terribly worried about Israel. \ldots In a corner of my consciousness, I have always imagined that this would not go well in the long run, but that it would actualize so quickly has still completely surprised me. One can only hope that the Israelis are, for the time being, still so militarily superior to the Arabs that they can hold the situation.''
\end{quotation}}

Let us take a look at what we have now:

\begin{enumerate}
    \item Conservatives who do not believe that sex is only about gametes.
    \item Richard Dawkins, an evolutionary biologist, who does not believe in postzygotic mutations.
    \item Liberals and gender theorists who do not believe that externally assigning identities is immoral, assigning ``cisgender'' to billions of people.
    \item Emily Fairchild, a queer theorist, who does not believe in différance and dissemination.
    \item Dogmatic transgender communities which do not respect lived experiences.
    \item Queer theorists who do not (or rarely) consider themselves and dogmatic transgender communities as nodes of power.
    \item Micheal Foucault, who did not believe in the discursive effects and author function, using his personal intent to debate with Habermas.
    \item The German government, which does not believe that sovereign states are not inherently related to ethnic groups, both for Germany (the Iron Cross case) and Israel.
    \item An international humanity law system which does not believe that humanity laws can retroactively apply.
    \item Theodor Adorno, who does not believe in the dialectic of rationality.
\end{enumerate}

The whole world is profoundly hypocritical and self-exceptionalist. The rule of the Nazis itself was built on such self-exceptionalism, a complete betrayal of the Enlightenment spirit, reason, science, and ``daring to know.'' Questioning Nazi political propaganda with true Enlightenment reason and scientific \textit{Ethos} would absolutely not be tolerated: why are Jews (and Roma, Slavs) inferior to Germans? How can morality be quantified? Which area of the brain is related to morality? What are the anatomical differences between morally inferior and morally superior people? How to prove that Jews generally have this brain structure? What selection pressures made morally inferior Jews have higher fitness and leave more offspring than morally superior Jews? When did the morality of the Jews begin to decline? During the period of the Kingdom of Israel or during the Diaspora? If it were during the Kingdom of Israel, why didn't the same selection pressures act on other Levantine peoples living in the same environment? If it was during the Jewish Diaspora, why didn't the same selection pressures act on European peoples? \footnote{Honestly, the first time I came up with the idea that ``a world that completely follows science and reason would be so free and equal'' was about a question very similar to the Nazi's anti-scientific propaganda, and that question was about gender: some people say that female students are not suited for learning science. What is the evidence? Which area of the brain and what brain structure is suitable for learning science? What is the relationship between these brain structures and sex? Do sex hormones promote the differentiation of brain structure in different directions in the early fetus? Was this process shaped by sexual selection? Where did the selection pressure come from? How was this hypothesis verified?}

The situation with Germany and Israel is precisely the same as the ``scientific'' gender essentialism of conservatives and the dogmatic, exclusionary transgender communities I encountered. The problem is not the specific content of this exceptionalism (whether it's ``the living space of the Aryans,'' ``the gender order created by God,'' ``Germany's special responsibility to Israel,'' or ``lived experience''), but that it creates a zone where reason, science, empirical evidence, logical consistency, international humanity laws, universal human rights, and justice are legitimately declared invalid, and power will ultimately do what it wants to do.

This is nothing else but a historical self-repeating of the Enabling Act of 1933. The real Enabling Act, by creating a legal ``state of exception,'' authorised the Hitler government to bypass the constitution, thereby destroying the Weimar Republic. Every accepted principle of ``exception,'' no matter how big or small, no matter what content is in it, is a micro ``enabling act.'' It authorises power to override rules, thereby destroying the foundations of justice.

The so-called ``Germany's special responsibility to Israel'' is such an enabling act. When state actions touch upon the issue of so-called ``Israel's right to exist,'' the enabling act is activated, and regular, universalist considerations -- whether it's compliance with international law or the universal human rights of all peoples, including Palestinians -- can be legitimately suspended or downgraded.

Richard Dawkins and conservatives' so-called ``science'' is such an enabling act. They do not care about the biological reality, but use it to justify their oppressive ideology. They do not care about the origin and biological function of sex and anisogamy, but use it to justify the social and political norm based on phenotypic sexes.

The so-called ``lived experience'' is also such an enabling act. Its purpose is not to protect people's freedom to express their experiences, but to use this mystifying and obscurantist language to conceal its true purpose: systematically promoting a specific type of experience and devaluing other experiences.

Moreover, the sovereign state, the nation-state, and the Westphalian system are also enabling acts. Its purpose is to permanently prevent the universal implementation of fundamental human rights. Every sovereign state's constitution is also an enabling act that delineates boundaries, establishes states of exception, and promotes exceptionalism. It aims to declare that its principles are not universal. They apply only to this group of people (citizens) and not to another (foreigners).

The entire global nation-state system is equivalent to saying that the full enjoyment of human rights depends on the accidental facts of their geographical location of birth and their nationality. This cannot pass the Kantian test of universalisability. Moreover, it is not like ``it is permissible to kill another person when it is convenient for achieving one's own goals.'' $\to$ ``a rational person must will their own continued existence; to will such a world could lead to one's own destruction. This creates a conflict in the will,'' which relies on reasoning for a \textit{reductio ad absurdum}. It directly denies universalism in its very wording. It makes an entirely accidental, irrational factor (place of birth) the core of human rights, which is fundamentally contradictory to the requirement that ``a law must be universal.'' The only special aspect is that it has endured for so long that people consider it a normal state of affairs, and jurists and philosophers even regard it as a self-evident starting point for analysis.

The only way to guarantee justice and prevent tyranny is to abolish all zones of exception mercilessly. Universal laws, including fundamental human rights, must apply to everyone, everywhere, at all times.

If Germany had adopted a rationalist and universalist narrative, such as ``the disaster created by Nazi was an analysable historical event rooted in a specific historical and political-economical context; its racial theory is pseudoscience, the word Aryan refers to Iranians and has nothing to do with Germans; there is no evidence that race or ethnicity has any biological essence, Germans are neither noble nor guilty; `never again' means never again for anyone, anywhere,'' the current tragedy would not have happened.

The Nazis had destroyed Germany's rationalist tradition to such an extent that even the reflection of Nazism still remains within a mindset contaminated by it. They used a new exceptionalism (Germany's guilt and special responsibility towards Israel) to replace the old exceptionalism (Aryan superiority). It is a symptom of the profound damage that Nazism has caused to Germany's intellectual tradition. As the homeland of Kant, Germany has never truly returned to the glorious Enlightenment tradition. It continues to use a tribal, mythical, and irrational narrative to resolve an issue that can only be truly understood and overcome through the universal reason. The special responsibility is not the antidote to Nazism. It is precisely a painful, lingering aftereffect left by the Nazis' ideological virus, a ``long Nazism'' similar to the long COVID.

The doubt and abandonment of universal reason is not a profound reflection on Nazism. On the contrary, it is itself a manifestation of Nazism's ideological victory. The Nazis not only destroyed life but, on a spiritual level, also destroyed the trust in the only intellectual tradition that could truly falsify and resist it -- the universalism of the Enlightenment.

\section{What is Enlightenment?}\label{sec:what-is-enlightenment?}
I know what the post-structuralists and the communities influenced by them will say when they read this:

\begin{enumerate}
    \item We acknowledge that you have had a personal experience different from ours, and we fully respect your relentless pursuit of universal reason, which began with the division of household labour in your childhood. However, you cannot universalise your experience to everyone.\\
    — This may be the kindest and most moderate critique, but unfortunately, it is an invalid one. Because my personal experience is precisely the pursuit of universalism, ``the universalisation of universalism'' is a tautology that provides no new information and thus is not a valid critique. If I said, ``I like to eat apples, so I will pursue the universalism of eating apples,'' that would constitute ``the universalisation of personal experience.'' In contrast, ``I support universalism, so I want all principles to be universally applied'' is not ``the universalisation of personal experience.''
    \item You are using your own experience to destroy the narrative legitimacy of lived experience. ``Autoethnography'' is usually used to present the special experiences of marginalised groups. You are using it to argue for a universalist, revolutionary grand narrative (the abolition of gender), which is a performative contradiction. \\
    — I readily admit this. That's right, just as Foucault used reason to destroy reason, you have done it too. Moreover, your so-called ``respect for lived experience'' is itself selective. As I have already mentioned, you yourselves are systematically denying the subjective experience of an ``independent subject.'' If we set post-structuralism as the ``centre,'' I am a marginalised person. Denying this perspective is a kind of self-exceptionalism.
    \item The science you love is not neutral, but a historically constructed ``episteme'' that is inseparable from the operation of power. \parencite{Foucault1972Archaeology} It is a local knowledge originating from Europe. To treat it as a meta-discourse is a failure to reflect on your own position of power.\\
    — This is the point most worthy of a response, and I will respond to it with the highest respect I can give to an idea.
\end{enumerate}

No. Science is not something that originated from the European Enlightenment, but rather a natural extension of the naive human epistemology.

Except for solipsism, any form of philosophy, ethics, or sociology is to some extent a form of realism, relying on two propositions that cannot be ultimately justified: ``the external world is real, I am not a brain in a vat'' and ``all people have the same mind and subjective feelings as I do'' (the problem of other minds). Although they can suspend ontological problems in their theories, this is logically self-evident. It is a prerequisite for their work to have practical meaning.

Otherwise, I could perfectly well believe that everyone except me is essentially a neural signal inputted into a brain in a vat, or that I live in The Truman Show and everyone else is a biochemically created philosophical zombie. Then, discussing ethics or sociology for such fictional things would be as absurd as ``you can't kill people in a video game.'' Why can't I go out and kill people randomly? As long as we don't want the conclusion ``randomly killing people on the street is ethical,'' we must accept these two points, a minimalised realism.

Once we accept these two premises, we also accept:

\begin{enumerate}
    \item Our senses and neural signals are trustworthy. We cannot bypass our own nervous system to directly access external reality without mediation. Whether it's seeing others' joy, anger, sorrow, and happiness, talking to and hugging others, seeing others' videos, or reading others' stories, all of this is perceived by us through a series of neural signals. But how do we confirm it? We cannot. We don't need any biological knowledge.\footnote{I will not cite biological knowledge here, as that would be using science to argue for the scientific epistemology, which would be a circular argument. } We only need a series of phenomenological visual illusion experiments to know that ``our senses do not always output consistent and trustworthy signals; at least some of them are not real.'' If we can determine that ``some are not real,'' then we have no metaphysical reason to believe that its other signals are ``real'' or that it is ``overall real.'' But we still make this choice. We believe that our neural signals are a reliable proxy for objective reality.
    \item An epistemological strategy based on Abductive Reasoning and Inference to the Best Explanation (IBE). ``Our senses and neural signals are trustworthy'' can only tell us ``there are some people here, doing these things,'' it cannot tell us they are not philosophical zombies. The reason we believe that ``all people have the same mind and subjective feelings as I do'' is because we observe that all people have similar faces, eyes, expressions, and they cry and laugh like me. The most convincing explanation is that they have the same inner feelings as I do. Otherwise, we would need to argue how different foundations produce the same external manifestations.
    \item Falsifiability based on empirical evidence. We choose to constantly revise our understanding based on reality, rather than believing in a closed story that cannot be changed.\footnote{Similarly, I will not cite the ``Bayesian brain theory'' here, as that would also be a circular argument.} When we wake up from a dream in the morning, get dressed, get out of bed\ldots we build a model of our life: ``I woke up from a dream, got dressed, got out of bed.'' After a while, we wake up again, and we are surprised to find that we are still in bed, and everything we just did was a dream. We falsify the original model and revise it to ``I fell asleep again after waking up the first time.'' Any understanding we have about the external world is fundamentally falsifiable and constantly being revised. This is the prerequisite for us to live a meaningful life.
\end{enumerate}

Things that are unfalsifiable, like God or witchcraft, are \textit{ad hoc} and violate basic human cognition. If we use a religious epistemology in our daily lives, for example, a girl who constantly experiences gender discrimination, abuse, preferring her brother, has all the inheritance left to her brother, and selling her to an old bachelor for money, and still constantly adds \textit{ad hoc} justifications like, ``they must have their reasons, Mom and Dad still love me,'' most people would think this is an unhealthy thought, a form of escapism, weakness, not daring to face reality, being brainwashed and gaslighted. So why is the same epistemology, e.g., ``these sufferings are a test from God,'' reasonable?

Since we have already accepted that one type of signal is a reliable proxy for objective reality, we have no reason to reject the reliability of another type of signal. A photon hitting our retina and being converted into a neural signal is isomorphic to the cosmic microwave background radiation hitting the sensor of a space telescope and being converted into a digital signal. Neither is the entity itself, but a signal from a mediator or proxy.

If someone believes that ``quarks are just an explanation for deep inelastic scattering experiments that cannot be ultimately confirmed,'' then ``here is a brick'' should also be seen as an explanation for ``photons reflected by a reddish-brown rectangular solid with a porous microstructure hit my retina'' that cannot be ultimately confirmed, rather than a credible ontological claim.

From the perspective of the extended mind theory, we can view instruments as an extension of our senses, just as a notebook is an extension of Otto's memory. For a person with poor vision, is what they see with the help of glasses a ``direct experience''? For a person with poor hearing, with the help of a hearing aid? This line seems to be a product of accidental human biological basis, rather than a solid philosophical foundation.

What we see with a telescope or microscope is also, in our intuition, ``directly'' perceived by us, like with glasses. Using a telescope for bird watching is considered a way to get close to nature, an immersive connection with the natural world. If telescopes and microscopes are reliable, then is the optical viewfinder of a DSLR camera with a telephoto/macro lens reliable? Optically, it is essentially a telescope or microscope. If what we see through the optical viewfinder is trustworthy, then is the electronic viewfinder of a mirrorless camera reliable? From this step on, the light received by our retina is no longer directly from the object itself. It has been digitised and re-simulated because the electronic viewfinder of a mirrorless camera is essentially a screen. However, for advanced models with high resolution and high refresh rates, there is no phenomenologically perceptible difference in user experience from an optical viewfinder. Users trust the photos taken with a mirrorless camera to be genuine, just as they trust those from a DSLR. The key to trustworthiness does not lie in the physical mechanism of mediation, nor the method it interacts with physiological senses, but the consistency and verifiability of all evidence. I see a bird with my naked eye, but I can't see the details. I pick up a mirrorless camera with a super-telephoto lens, see the bird in the electronic viewfinder, press the shutter to take a photo, and review it later. The bird and its background, as seen through this series of actions and the photo I took, are consistent, so I believe this photo truly reflects the real image of the bird. I identify the species of the bird from this photo and add it to my birding lifer (life list), which means I believe I really saw it.

On this basis, we will continue to ask, are high-speed photography, infrared/ultraviolet photography, and computational photography reliable? Are medical imaging techniques like MRI and CT scans reliable? Are spectrometers reliable? We will eventually reach radio telescopes and interferometry, space probes, and the Large Hadron Collider\ldots This process of change seems to be continuous. It is difficult to draw a line of demarcation between these instruments as reliable and unreliable.

This ordering is also a product deeply shaped by the biological characteristics of humans, especially the sensotypical humans, rather than based on an objective standard. A case in point is that for a tetrachromat, they might think that it is a fundamental break from optical to digital, because the camera's CMOS digitises colour based on trichromatic vision, and some colour information they can see is permanently lost. A tetrachromat might feel that a digital photo is just a crude imitation, completely different from what they see with their naked eye or a microscope/telescope. Conversely, this tetrachromat can use a spectrometer to let trichromatic family or friends distinguish metamerism and understand their ability. The spectrometer is a more reliable extension of their senses, more ``concrete,'' more ``real,'' closer to their ``phenomenological experience'' and ``lifeworld.'' For a blind person, the entire ordering above is nonsensical and meaningless, just as ultrasound and infrared light. If this blind person were to study physics, then the Large Hadron Collider and a computer that can announce the experimental results in voice would be more real than any object, like a rainbow that is visible but cannot be heard or touched.

Therefore, this seemingly natural ordering is actually a form of sensotypical-centrism. Any theory that maintains a distinction between ``sensory experience'' and ``abstract theory,'' or ``observable'' and ``unobservable,'' please state your ``centrist'' assumption. Whose ``senses''? Whose ``observability''? When you set this standard with your own qualia, which cannot be directly accessed and cross-validated by others, you place your own sensory modality at the centre and define all other sensory modalities (colourblind, blind, deaf, tetrachromats, synaesthetes\ldots) as marginal, Other. Their ``senses'' are defined as non-standard, abstract, and needing to be ``translated'' by scientific instruments. There is no universal distinction between ``observable'' and ``unobservable'' that applies to all humans. Science, on the contrary, provides us with a more inclusive way to accommodate sensory diversity, allowing different groups to establish a shared intersubjective reality in a way that is, although very imperfect, abstract, and indirect,\footnote{A sensotypical person can never directly and perfectly experience the qualia of tetrachromacy, and the same applies to the blind and deaf towards sensotypical people. } still relatively reliable. This is completely consistent with our initially accepting the existence of ``external world'' and ``other minds'' to construct a shared intersubjective world.

Moreover, the narrative of ``discover unobservable theories from directly observable phenomena'' does not always proceed in the order we expect. If we train a deep visual neural network model with photos of two traditionally indistinguishable cryptic species, and find that the network can distinguish them with an accuracy of, for example, over 85\%, which is far higher than human biologists. Then we use Grad-CAM (Gradient-weighted Class Activation Mapping) to generate an activation heatmap of the last convolutional layer. We found that the neural network was distinguishing them through a feature we had previously ignored. We then find this feature to be very effective, and after learning it, human biologists can also distinguish them with the naked eye. In this process, we first defined them as two distinct species through the analysis of unobservable DNA sequences and phylogenetic analysis. Then we let the neural network train on these two classes\footnote{The term ``class'' here refers to the computer scientific term, not the taxonomic level. } which we artificially divided. Compared to directly ``believing'' the phylogenetic results, we additionally believed that the deep neural network is not a random number generator or a supernatural magic program. In this event, the theoretical load of naked-eye observation (phylogeny + artificial intelligence) is heavier than the phylogenetic analysis itself. The seemingly natural process of ``first observe, then discover'' is completely reversed.

Similarly, the ``social construct'' that post-structuralism is obsessed with is also a form of mediation, albeit an abstract one. Our perception of an ``apple'' is not just the physical and biological process of light and retina. It is also wrapped in numerous layers of social construction, including the cultural meaning of the word ``apple,'' the story of Snow White, and the saying ``an apple a day keeps the doctor away.'' \textcite{Foucault1978History} proposed that the family is an important site for biopower. Parents teaching a young child, ``this is an apple, apples are delicious,'' is precisely the way the biopower operates through the family node. We mentioned earlier that as a child, I believed human traffickers did not exist. I saw it as very interesting and told my mom when writing this article, and she lectured me again and warned me about ``beware of human traffickers.'' This is precisely how complicated philosophy exceeded the ``condition of possibility'' under a micro-episteme in a family. Most people will not doubt the reality of apples, as they might doubt scientific realism. If scientific instruments have theoretical presuppositions, and scientific theories have cultural and political presuppositions, then glasses (geometric optics), the naked eye (trichromatic vision), and even our most basic ability to think and know also have them. This is our universal epistemological dilemma, not a refutation of scientific realism.

Did my parents see real human traffickers? They did not, I am sure of this. They learned about the human traffickers from the TV news. Why should we believe that the television is a reliable source of news, that it will technically and faithfully reproduce what the television station wants to tell us, rather than a magic box that randomly outputs fictional content? Why should we believe that news is reliable, that ``human traffickers'' are a real threat to children, but not biopower means to control population migration? Our most basic interaction, as a child with parents, is already immersed in the social network of ``knowledge/power.'' The idea of distinguishing a ``direct lifeworld'' from an ``abstract theoretical world'' is naive. It is an obsession with unmediated, direct ``presence,'' precisely a form of logocentrism.

If we believe our ``pre-philosophical knowledge'' is trustworthy, that ``human traffickers'' are a real threat to our safety, then there is no reason to think that quarks are less real than ``human traffickers.'' Most people do not wait until themselves or their children's actual abduction to believe in the existence of human traffickers. Children believe their parents, parents believe the news, the news believes the reporters, and the reporters believe the victims\ldots This chain of trust leading to ``human traffickers'' is epistemologically isomorphic to the chain of trust leading to ``quarks'': we believe textbooks, textbooks believe scientists, scientists believe peer review, and peer review believes experimental data\ldots They all involve trust in mediation, others, and inference.

Science, by examining all these ``mediated signals'' under an equal epistemological standard, attempts to cut through the fog and establish a more objective understanding of reality itself. For example, Johann Friedrich Blumenbach, through comparative anatomical studies of skulls, reached scientific conclusions quite close to modern population genetics: ``the highest diversity is found within the black race,'' ``the five races are not clearly divided, with transitional states between them,'' ``there is no scientific evidence for superiority or inferiority among different races,'' and ``the intelligence of black people is not inferior to any other race.''~\parencite{Rupke2019Johann}

Of course, I know post-structuralists will say: his views and theories were later appropriated by ``scientific'' racist theories like phrenology, which is a manifestation of Foucault's theory of the ``episteme.'' Blumenbach's scientific views exceeded the ``conditions of possibility'' of knowledge at the time; they were ``unthinkable'' and ``unbelievable'' within the episteme, discourse rules, and cognitive framework of his era, and were thus quickly appropriated in the discursive field into a form more acceptable in the episteme.

Nevertheless, this also shows that Foucault's ``power'' is not an omnipresent, omnipotent, insurmountable thing that exists beyond the scope of human cognition. Otherwise, how did Foucault himself know it? \textcite{Foucault1978History} said that ``power'' is just a nominalist, functional concept for analysis: ``Power is not an institution, and not a structure; neither is it a certain strength we are endowed with; it is the name that one attributes to a complex strategical situation in a particular society.''

Although Blumenbach did not escape more abstract historical limitations like ``seeking ideal prototypes,'' the ``equality'' and ``transitional nature'' in his rigorously derived conclusions clashed fiercely with the hierarchical (superiority/inferiority) and essentialist (distinct categories with clear essences) tendencies in the deep structure of the episteme at his era. Scientific research loyal to evidence can provide us with a relatively reliable picture of the world. No matter how scientific data is mediated by senses, instruments, culture, and political presuppositions, science remains a reliable way to understand objective reality. Otherwise, it would be incredible for Blumenbach to have reached conclusions largely consistent with contemporary population genetics under a completely different ``episteme.''\footnote{This is the classic no-miracles argument, but in Blumenbach's case, it has an ethical and political meaning. } From the perspective of structural realism\footnote{A branch of scientific realism that holds that the ontological assumptions of scientific theories may not correspond to reality, but the mathematical structures they reveal are truly possessed by reality. }, although the ``races'' assumed in Blumenbach's theory were later denied ontological status by subsequent theories, we have reason to believe that the mathematical relationships—``this group has higher diversity,'' ``there are no clear boundaries between these groups''—captured some real properties of objective reality.

Like Foucault's ``power,'' social construction is not a deterministic thing that precedes or even transcends natural laws. Music is also a social construct; a specific melody does not necessarily have a definite ``emotion'' or ``meaning,'' and our understanding of music is culturally loaded. However, it is obviously impossible for us to compose music using ultrasound or infrasound, which would violate human biological characteristics. As we have discussed in previous cases, our vision is also limited by a series of innate factors. Social construction cannot escape the constraints of natural laws. Since social construction itself has been shaped by natural laws, it is theoretically possible for us to understand this reality through these epistemological mediations that ``internally carry the \textit{traces} of natural laws.''

If Foucault believed that the participants of discourse, the mentally ill in asylums are real,\footnote{I mean the several entities as buildings and a group of human individuals, not the abstract concept of ``mental illness,'' which I know Foucault did not believe in. } if Bruno Latour believed that the human and non-human actors in the actor-network are real, if Butler believed that the human body is real, and not their own dreams or illusions, then why can't we believe that quarks, chemical bonds, genes, and dinosaurs are (or were) real? Any form of social constructionism must admit that society and its members are real. They are the prerequisite for ``construction'' to occur.

Therefore, I am not saying, ``I can irrefutably argue from a metaphysical perspective that quarks are an absolute objective reality independent of my mind.'' I cannot argue this, just as no one can, under the same skeptical scrutiny, argue that they are not a brain in a vat or that others are not philosophical zombies. I am not interested in metaphysical questions, nor do I want to discuss ``whether quarks are objectively reality.'' Instead, I am saying: a philosophical view can claim that human knowledge of things is reality itself, or that it is merely a phenomenon and not the ``thing-in-itself,'' or that it is all ``within the text.'' While they must apply to both bricks and quarks. Whether it's metaphysical solipsism or post-structuralism, there is no valid critique that can selectively act on quarks without affecting bricks.\footnote{On this point, I think Derrida is more self-consistent than Husserl, although I know Derrida might say that my judging his theory by ``self-consistency'' is a form of logocentrism.}

Our naive beliefs, and scientific concepts such as ``quarks'' and ``genes,'' share the same unstable yet only available epistemological foundation. Scientific realism is not a naive, unreflective, vulgar position. On the contrary, it is a consistent and coherent philosophical worldview to exist in a shared, meaningful intersubjective world in a non-nihilistic manner, after seriously facing the profound challenges of skepticism and critical theory. From the moment we reject solipsism and acknowledge the existence of other minds to live a meaningful life in this world, where we are ``thrown into,'' we have already accepted an epistemological strategy based on abductive reasoning and inference to the best explanation. We have accepted that ``signals are reliable proxies for reality.'' We openly and continuously update our understanding of things based on new empirical evidence. Science is the inevitable natural extension and systematised version of naive epistemology.

My analysis of myself at the beginning is a practical application of this ``coherent epistemology.'' I did not ``disrespect'' my subjective experience. Otherwise, I would have simply said, ``How could I be a girl? It seems I'm crazy. End.'' But I did not do that. On the contrary, I very seriously analysed my subjective experiences of gender dysphoria and gender euphoria, which are themselves cognitions mediated by the nervous system and sociocultural constructs. Placing them on the same epistemological level as all the knowledge I could obtain from other sources. All this knowledge is equal, intermediated, intertwined, and connected in a rhizomatic—or rather, mycelial—way. When I proposed that ``natural selection cannot encode abstract concepts,'' I was not just discussing evolutionary biology and neuroscience. I also recalled seeing my younger siblings and cousins randomly sucking things as kids, and elders teaching them, ``Don't suck that.'' Our naive knowledge and scientific knowledge are interconnected and mutually shape each other. The belief in a pure, embodied experience independent of theory is a naive, pre-critical metaphysical illusion of ``presence.''

Respecting experience is not about unconditionally accepting it as a metaphysical truth, but about seriously understanding and analysing it, integrating it into one's complete knowledge network, and establishing cross-validating connections with all other knowledge. Far from ``disrespecting subjective experience,'' I have given my subjective experience the highest respect I can give to any knowledge.

I fully admit that neither scientific epistemology nor naive epistemology can solve ultimate metaphysical problems. We are standing on quicksand without an ultimate metaphysical foundation, making an existential ``leap of faith'' that cannot ultimately be justified. So why can't we be brave and try to jump a few more times? We will eventually find that our legs seem to be very, very strong, and we seem to be able to jump quite, quite far.

This is the most extraordinary vision of the Enlightenment. Reason is not coldness; reason does not mean the dismissal and suppression of emotion. The goal of the Enlightenment is not to shape everyone into emotionless artificial intelligence. It does not seek to eliminate our emotional reactions or our connection to history, but to separate these emotions from irrational beliefs.

In the ideal world of the Enlightenment, a person who falls will shout ``Ouch!'' but they know it's their C-fibres firing, not because they are unlucky or cursed. A German, a Jew, and a Pole visiting Auschwitz together will all feel horror and say, ``How terrible!'' While they know in their hearts that Nazism was an analysable historical and political disaster, unrelated to the ``essence'' or collective guilt of any nation or people. See, I said ``in their hearts,'' but we all know it's in their brains. When we visit historical sites, we appreciate the ancient artistic and architectural achievements, but we know that other peoples have similar things. Even if they don't, it is primarily due to geopolitical, environmental, and historical constraints, and does not mean that these less developed peoples are inherently inferior.

We are already doing this, aren't we? We are aware of the Earth's rotation and Rayleigh scattering, yet we still admire a beautiful sunset. We know that capsaicin activates pain receptors, yet we still enjoy spicy food. We know that love is associated with dopamine, oxytocin, and neural activity, yet we still shed tears over the story of \textit{Romeo and Juliet}. Scientific explanations enhance our understanding without destroying our beautiful subjective experiences. In fact, they can add new layers to these experiences, enriching the language we use to express our feelings. When we see the Milky Way, we feel not only the beauty of the night sky and many ancient stories, but also the greatness and boundlessness of the universe. Biology has given us new terms to express love. We can say not only ``butterflies in my stomach,'' but also ``my dopamine is exploding'' to our partners.

We just need to extend this state from physics, chemistry, and biology to history, politics, society, and culture, to complete the unfinished work of the Enlightenment, to persist in the brave step we have already taken. Why should we stop here? Why do we apply this mode of thinking to physics and biology, but retreat to pre-scientific essentialist myths when discussing history, politics, and identity?

Such a profound pursuit and understanding of truth is supremely liberating, capable of dismantling any prejudice, stereotype and discrimination. Recent scholarship has shown that the \textit{Song of Songs} is a love poem, devoid of theological significance and not the work of King Solomon. \parencite{Exum2022Conceptualizing} It provides a valuable secular perspective on Hebrew culture. The verse, ``Many waters cannot quench love, neither can floods drown it. If one offered for love all the wealth of one's house, it would be utterly scorned'' (Song of Songs 8:7, NRSV), is so moving that it nullifies any stereotype of the ``grasping Jewish merchant.'' The Hebrews, as ancestors of Jews, possessed a profound humanistic \textit{Ethos} that placed the precious love between individuals above all material wealth. At the very wellspring of the Jewish national spirit lies the supreme exaltation of non-material values.

However, if we interpret the text according to the traditional understanding—as an allegory for the relationship between humanity and God—the force of this argument is diminished. This is because the proposition that ``God is more important than wealth'' is an axiom in any religion and thus offers no new information. An anti-Semite could still contend that, ``apart from God, the Jewish people prioritize wealth above all else, and therefore, they remain grasping merchants.''

That sentence in the Mauthausen concentration camp is deafening: If there is a God, He will have to beg my forgiveness.\footnote{Wenn es einen Gott gibt muß er mich um Verzeihung bitten.} \parencite{Lassley2015Defective}

When this unknown prisoner wrote this sentence on the wall of their cell, they were rejecting the unfalsifiable comfort of another irrational system, choosing instead human autonomy and agency, using clear and unshakeable reason to confront the evil and irrationality of Nazism. What they did was to use the independent, strong, agentic, and liberating subject of self to resist the oppression of Nazism. In the face of violence aimed at completely erasing personal dignity, the ``self'' constructed by the Enlightenment, with its inherent reason and autonomous judgment, is an important and powerful weapon against dehumanisation.

Post-structuralism will certainly oppose this, as they are keen on promoting a myth about the life experiences of marginalised people being erased by universalist discourse.

I am not privileged.

I was born in a small town in a third-world country. I am not European, not white, not male (well, that depends on how we define ``male''), not cisgender, not heterosexual, and not a coloniser. My mother was a textile worker, and my father was a machine worker. I was constantly excluded and bullied in elementary and middle school because of my personality. After growing up, as a transgender biologist who supports gender abolition, I have become an outsider to all political camps.

If my personal experience from childhood to adulthood contradicts a particular philosophical school's critique of universalism and rationalism, why should I accept it? Insisting on logical self-consistency as the standard for understanding the world has been my most profound ``lived experience'' since my childhood questioning of housework division and kinship terms. For me, post-structuralism (including queer theory, and so on) is local knowledge from Europe. The ``harms of rationalism and universalism'' they criticise have never appeared in my life and are completely disconnected from my personal experience. I have no reason to accept it.

Moreover, the concept of ``local knowledge'' is itself a kind of local knowledge. Since it is local, there must be places where it is invalid, which means that there must be some non-local and universal knowledge.

The idea that ``rationalism originating from the Western Enlightenment oppresses the different knowledge systems of other civilisations'' is still a continuation of the ``Orientalist'' thought that views the East as an exotic and irrational Other. It simply praises this thought that emphasises ``subjective experience, emotion, intuition'' and uses it as an argument in an internal philosophical conflict within Europe, without genuinely trying to understand non-Western intellectual history. This is a form of ``reverse Orientalism.''

It is still a form of ``nation-state thinking,'' a product of the Westphalian system. As I have argued before, the Westphalian system and international geopolitics are themselves forms of identity politics that oppose universal reason, sharing the same mode of thinking as post-structuralism, albeit with different standards to define ``identity.'' Universalism offers the only means to transcend identity politics entirely.

The intra-civilisation philosophical differences are greater than the inter-civilisation ones. The School of Names' ``white horse dialogue,''~\parencite{Graham1990Studies} the Mohists' ``universal love,'' and their exploration of geometry and physics are more similar in \textit{Ethos} to Plato, Aristotle, and Euclid than to Confucianism and Taoism. There were also Sophists in Ancient Greece. The \textit{Mohist Canon}'s ``a circle is that which has the same length from a single centre''\footnote{圆,一中同长也。} and ``a square is that which has four angles and four corners equal''\footnote{方,柱隅四权也。} are almost identical to Euclid's definitions, although they did not develop into a systematised axiomatic system. \parencite{Graham1978Later} There are paradigmatic differences between analytical, logical, abstract, systematic, and universalist approaches, and contextual, emotional, concrete, empirical, and particularist ones (and their intermediate states) within different civilisations.

The similarities among analytical, logical, abstract, systematic, and universalist systems across civilisations (for example, the similarities between Mohism and Euclidean geometry) are greater than the similarities among contextual, emotional, concrete, empirical, and particularist systems across civilisations (for example, Confucianism, the Sophists, and post-structuralism). Therefore, we can say that rationality and science are the most universal knowledge.

In middle school, I was drawn to the Mohists and found Confucianism to be less appealing. Confucianism is not my ``local knowledge,'' but a paradigmatic enemy. More than two thousand years before I was born, they established their dominant position in China by systematically excluding reason and universalism as their ``Other'' and eliminating competitors whose basic cognitive paradigm was more similar to mine. My love for science and Enlightenment rationality is because it is based on the same cognitive foundation and has developed more completely and systematically than the Mohists and the School of Names. In my view, this is a case of ``seeking lost rites among the common people,''\footnote{礼失而求诸野, meaning that the authentic traditions are often better preserved among the common people than by the ruling class. Ironically, this old saying is considered to be from Confucius. } choosing the most developed one among all those that align with my cognitive mode.

For me, it is an act of colonising personal experience with theory, using a cold, academic philosophical theory to strip personal experience of its validity to accept the post-structuralist critique of reason. If I were to sincerely agree with the post-structuralist critique, then, by post-structuralist standards, I should refuse to accept post-structuralism. Therefore, it is not that I choose to reject post-structuralism out of ideological opposition originating from the ``scientistic grand narrative,'' but rather that post-structuralism cannot handle my personal experience. It would immediately throw a fatal exception and crash. Accepting it is impossible for me, a logical paradox.

A rule is universal only when it applies to everyone without exception. A privilege, by definition, is an exception granted to a specific group; it cannot apply to everyone and is therefore not universalist. Similarly, there had never been a real Kallipolis that was ruled by true reason. It has always been weak, excluded and suppressed throughout history. A firm commitment to universalism, rationalism, and the Enlightenment is not a tool for privileged groups to defend their status and it logically cannot be.

On the contrary, it is the last refuge of the weak. I have no group, no tribe, no recognised identity. The only thing I have are these principles that should apply to all people equally and without exception. The bullies at school could pull down my pants and throw away my shoes, the discourse police in online communities can distort my words and ban my account, but no one can stop me from, as Kant said, having the courage to use my own \textit{reason} to point out how arbitrary and baseless their rules are.

\textit{A man can be destroyed but not defeated.} Everything I have done is precisely like the anonymous prisoner in the Mauthausen concentration camp, using the powerful, rational subject of self from the Enlightenment to resist the irrational violence of the world.

Universalism and rationalism are not a claim to power, but a relentless pursuit of a pure justice, a justice that can even empower an ultimate outsider like me. The goal of the Enlightenment is precisely to use the light of reason to enlighten all the darkness of pre-modern superstition, privilege, dogmatism, tribalism, and obscurantism, to create a world where there is no more of this pain and harm, a world where no child will be bullied for wearing a pair of ``wrong'' shoes.

{\Large \textbf{Long live the Enlightenment.}}

{\Large \textbf{Long live the Enlightenment!}}

\section*{Postscript: Story behind the Story}
%Every sentence in the main text is completely sincere. I have indeed experienced those events and that nightmare about psychological manipulation. After that, I began to think about gender identity from the first principles of biology, asked questions in multiple communities and was refuted, attacked, or banned. I then explored my own history. Finally, to understand ``why they attacked me,'' I learned about post-structuralism and queer theory.

%The \cref{app:b} is not a bad faith imitation like the ``Sokal affair'' (and as you can see, I also criticised Sokal's views on gender in the main text). I really want some sincere discussion and communication. I wrote it to prove that ``I have spent great effort to understand your theory. I am not an ignorant outsider.''

%Moreover, if we set aside the fact that ``I do not accept post-structuralism at all,'' this act of ``writing two versions of a personal story'' is very post-structuralist. It forces us to think: to what extent does the way we tell our stories determine who we are? My answer is that I believe in an independent, agentic, rational subject of self that can choose between two different discourse systems. My choice of science and reason over post-structuralism is the ultimate embodiment of this. A post-structuralist would surely interpret it as a deconstruction of the ``true self,'' a questioning of the author's ``sovereignty,'' and ``there is nothing outside the text.''

%After delving into post-structuralism, queer theory, and current transgender activism, I can fully empathise with how mainstream transgender people would view my gender abolitionism, because post-structuralist philosophers have done the exact same thing to me. I negated ``gender identity,'' a concept many see as a survival need; Foucault negated ``reason,'' a concept I see as my only refuge. I see gender identity as an oppressive prison; you see reason as an oppressive prison.

In my view, the word ``reason'' refers to the cognitive mode shared by all humanity, the reason of Socrates and Kant, of Euclid and Einstein, the reason that made that young me in the kitchen ask my mother, ``Why doesn't Dad cook?'' the reason that makes me now engage in evolutionary biology research. Reason is not just a tool; in this world full of chaos and pain, it is the only clean, pure, and trustworthy thing. It is my shield, my weapon, my sanctuary. It is the sole, unified force for resisting tyranny and discovering scientific truth.

I know full well that what Foucault criticised is not this reason, but that's where the problem lies: I do not recognise ``instrumental rationality'' as a form of reason at all. I believe that the thing that creates weapons, wages wars, and harms the bodies of transgender and intersex people is merely a tyrant claiming a name that does not belong to it. I don't care what ``raison'' means in French, nor do I want to know what happened in Western history, but it cannot be translated into the word ``理性'' (lǐ xìng, reason/rationality) in my language (Chinese).

Similarly, both queer theory and I long for a space with infinite human creativity and freedom. \footnote{``The task is to help produce a world in which we can move and breathe and love without fear of violence, with the radical and unrealistic hope in a world no longer driven by moral sadism cloaked as morality.''\parencite{Butler2025Who}} This space may need a name, or perhaps it doesn't need a name. But it cannot be called ``gender,'' because that is the name of the prison we are trying to escape from. Calling this space ``gender'' would be like naming the liberated territory ``The Prison'' or naming the peace ``The War.'' In my view, ``gender identity,'' as in \posscite{Marx1977Critique} critique of Luther's Reformation, frees the body and enchains the heart. The same pattern replicated on ``sex.'' We all consider phenotypic sex unstable, while I do not agree with referring to it as ``sex.''

%This is precisely what makes us unable to have meaningful dialogue. Because this is not a matter of philosophical theory, but of ethics, aesthetics, and even existentialism. We both see the ``liberation'' in the other's eyes as the most significant threat, and the oppression in the other's eyes as the most important survival need. As Saussure pointed out, a word has no intrinsic meaning, so there is no transcendent metaphysical standard to judge whose use of words like ``reason/rationality'' and ``gender'' is correct. There is no Platonic ``Form of Reason'' or Aristotelian ``Essence of Gender,'' no theory or empirical evidence that can bridge this gap. I cannot accept their philosophy, just like they cannot accept mine.
%
%Imagine this: I am a guerrilla fighter. To escape a tyranny called ``gender,'' I have, after a hard struggle, created a base area called ``reason,'' the only place where I feel safe. One day, another guerrilla unit passes by. I hear they are from France and the United States, and their captain (Foucault) and vice-captain (Butler) are a gay and a non-binary person, being our comrades. I happily welcome them to visit my base, but they tell me that my base is actually the prison that oppresses us, and that the tyranny called ``gender'' is our true path to liberation.

%What can I do? What can I tell them?
%
%At least, I can still witness. I can still read and understand.
%
%I paid huge intellectual effort to understand them, while they did not understand me.


\section*{Source Code Availability}

Code for generating this document and visualising the author's experience (\cref{fig:5}): \url{https://github.com/sun-jiao/autoethnography}

\printbibliography

\section[“Assigned Sex at Birth” Lacks Inclusivity for Intersex People, and a Philosophical Critique of its “Mechanical Idealism”]{``Assigned Sex at Birth'' Lacks Inclusivity for Intersex People, and a Philosophical Critique of its ``Mechanical Idealism''} \label{app:a}
\subsection[“Assigned Sex/Gender at Birth” Lacks Inclusivity for Intersex People]{``Assigned Sex/Gender at Birth'' Lacks Inclusivity for Intersex People}\label{subsec:assigned-sex-gender-at-birth-lacks-inclusivity-for-intersex-people}

Let's start with a basic thought experiment. There is an intersex child whose external genitalia more closely resemble a male. At birth, in a hospital with a high level of medical expertise, doctors noticed ambiguities in the external genitalia. After performing karyotype analysis and PCR, they ``assigned'' the child as female and advised the parents to consider genital reconstruction surgery when the child was older. The parents, being highly irresponsible and finding it troublesome, abandoned the child. The child was then found by a couple with a lower level of education and was raised as a boy. Theoretically, this child's ASAB/AGAB is female, but this assignment has no bearing on the child's entire life. If this child grows up and identifies as female, according to current mainstream definitions, this child would be a cisgender woman. This is clearly absurd, as their experience is vastly different from that of a cisgender woman.

This illustrates two points:

\begin{enumerate}
    \item The method of sex assignment is artificial. Although visual inspection of external genitalia is the common method, theoretically, karyotype or PCR sequencing could be used.
    \item Later ``sex assignment'' can potentially override the assignment at birth, becoming the main gender used to raise the child.
\end{enumerate}

Although this thought experiment is extreme, it involves highly skilled doctors, irresponsible biological parents, and less educated adoptive parents. There are less extreme but more common cases in reality. In the following examples, ASAB/AGAB had some impact on the child's early development, but subsequent experiences were equally or more important.

Many intersex individuals undergo medical ``reassignment'' at some point during childhood or adolescence.2 Some parents with strong stereotypes may immediately make a 180-degree turn in their raising methods. For example, I read a story by a patient with Persistent Müllerian Duct Syndrome (PMDS). After being reassigned as female, their parents immediately bought them dresses and bras, told them to pay attention to their appearance, deportment, and posture to avoid indecent exposure, and forbade them from playing with ``boys' toys'' (though the good news is that the author self-identifies as female in adulthood). This would not happen in the lives of typical ``phenotypic males'' or ``phenotypic females'' (even if they identify as transgender). For them, their phenotypic sex and assigned sex are completely consistent (it's hard for them not to be, barring extreme medical errors) and will not be ``reassigned'' during their lives.

For intersex people, labels such as ``AGAB to Identify'' (e.g., MtF) cannot fully encapsulate their experiences of interacting with and resisting externally imposed gender roles. This also differs from the experiences of transgender individuals with ``typical phenotypic sex'' who share the same label (e.g., a typical MtF).

Therefore, ASAB/AGAB erases the unique experiences of intersex people. It attempts to categorise the patterns of a child's social gendering into two typical phenotypic sexes (most medical systems only offer male and female options for sex assignment), ignoring the experiences of children whose gender roles and expectations undergo drastic shifts during their development. Consequently, the so-called ASAB/AGAB replicates the power framework it intended to oppose using a new discursive system; it is, in fact, gender binary and ``typical sex hegemonism'' (a term I coined).

Furthermore, when we interact with others daily, no one asks them to show identification to see their legal ASAB/AGAB. We merely judge their sex or gender based on their physical and social appearance. In other words, we are constantly performing such ``reassignment'' on others. Moreover, to some extent, this ``daily assignment'' we perform also influences the gender experiences of others (especially children). Both medical ``reassignment'' and our ``daily assignment'' can lead to unique experiences for individuals that cannot be encapsulated by ASAB/AGAB. Indeed, this ongoing ``daily assignment,'' based on perceived social cues, can be seen as an integral part of the broader phenomenon often described as gender performativity, where gender is constituted through repeated social interactions and interpretations.

\subsection{Philosophical Critique of ``Mechanistic Idealism''}\label{subsec:philosophical-critique-of-``mechanistic-idealism''}

Philosophically, ASAB/AGAB has a rather bizarre and, for me, incomprehensible philosophical basis.

I believe it is primarily idealistic because it denies the role of phenotypic sex, or the phenotypic basis (material level), in the child's social gendering process, reducing it instead to an ``assignment'' -- in other words, a specific social activity (mental level). (Although phenotypic sex is also a socially constructed concept, it is relatively more ``material'' compared to ASAB/AGAB, encompassing many material phenotypic characteristics, even if these characteristics are organised into the concept of ``phenotypic sex'' through human mental activities.)

This view might have some reasonable aspects, as the ways of social gendering are diverse and cannot be fully represented by phenotypic sex (though it remains a very influential factor, as guardians cannot invent a gender role out of thin air and apply it to a child). I can understand what it intends to convey: transgender people are not denying their phenotypic sex but rather the externally imposed gender roles. The concern behind this viewpoint may be valid.

However, it further attempts to use an initial mental act to determine everything, bestowing upon it an absolute, ontological priority, and on this basis, it further negates all subsequent mental activities. It treats the initial mental act (ASAB/AGAB) as a static, unchanging, and ultimately decisive factor, disregarding the impact of all subsequent ``mental activities'' (such as medical and daily ``reassignment'') on individual experiences. It places all subsequent mental activities under the absolute dominion of this ``initial mental act.'' It reduces a complex, dynamic, lifelong process of social gendering and the individual's reactions and interactions with this process to a static, single-point, initial ``label.''

This philosophical view is not comprehensible yet profoundly striking. I name it ``mechanistic idealism.'' To my knowledge, many idealistic philosophies (such as Hegel's absolute idealism, Berkeley's subjective idealism, etc.) incorporate change, development, dialectical movement, or agency. Few idealistic philosophies advocate for mechanism and dogmatism. Therefore, in my view, this is a rather atypical and anomalous philosophy, possessing a certain high degree of ``intellectual originality.''

\subsection{New Definitions: Balancing Accuracy and Operability}\label{subsec:new-definitions:-balancing-accuracy-and-operability}

Under my definition, transgender is essentially about the ego feeling discomfort with and resisting externally imposed ``social gendering'' patterns. The resistance can manifest in myriad ways, from overt political activism to deeply personal, internal acts of self-definition and emotional negotiation.

ASAB/AGAB itself is just a part of the ``social gendering pattern.'' The source of oppression is an externally imposed, artificially constructed social norm, not the word written on a birth certificate. This process is dynamic and diverse, rather than fixed and monolithic.

Under this definition, being transgender is not a passive, objectively existing state, but an individual's non-conformity with and resistance to externally imposed rigid social norms, an active act of self-empowerment. It begins with the internal refusal to accept imposed norms, regardless of whether this leads to external action. This definition mentions the specificity of gender issues under the current social construction and can attract broader support from those resisting oppression beyond gender issues by relying on a narrative of resistance.

However, this definition is too abstract and lacks operability in daily life. Nobody's social gendering pattern can be completely the same with others. What constitutes ``discomfort''? What constitutes ``resistance''? It is fundamentally undefinable and inapplicable.

\subsubsection{``Phenotypic Sex'' with a More Inclusive Definition}

From the perspective of ``All models are wrong, but some are useful,'' I believe ``phenotypic sex'' is a criterion that can better balance operability and accuracy. Phenotypic sex is the primary factor determining how we are socially gendered; the way others perceive us depends on this material existence (at least under current social norms), and based on this, they further actively impose socialization patterns on us, including how elders raise us and how others interact with us, for both ``typical phenotypic sex'' individuals and intersex people.

Phenotypic sex includes not only male and female (typical phenotypic sexes) but also intersex people. Intersex people can even be further subdivided into different groups, all of whom may have experiences different from typical males and females. We should acknowledge XtM, XtF, XtX cases (where the first X represents atypical phenotypic sex), and can even extend this to, for example, KtM (Klinefelter syndrome identifying as male), TtF (Turner syndrome identifying as female), 5tQ (5-alpha-reductase deficiency identifying as queer), etc. (Of course, specific symbols can be negotiated and adjusted). At the same time, phenotypic sex also avoids the problem of ``social gendering patterns'' being too abstract and inoperable, thus being a good standard that combines accuracy, inclusivity, and operability.

\subsubsection{All Formally Assigned Genders (AFAG)}

Another useful ``model'' is an extension of ASAB/AGAB, incorporating all formally assigned genders:

Transgender individuals are those whose gender identity does not align with at least one of their formally assigned genders (AFAG). ``Formal'' refers to designations made in legal or medical documents and the gender role \textit{de facto} used by guardians.

The value of introducing the concept of ``\textit{de facto} use by guardians'' lies in its pointing to an important social reality -- the individual's early social gendering experiences in the absence of formal documentation or when documentation conflicts with upbringing practices, like the one in our initial thought experiment, where the child was not formally assigned male but was raised as male. It can also include stateless persons without documents. Even if it is challenging to quantify perfectly, acknowledging its existence is itself progress.

``At least one'' emphasises that if any formal assignment in an individual's life does not align with their gender identity (meaning a social gendering pattern was once imposed in this way), they can be classified as transgender.

Although this definition omits ``daily assignment,'' which we previously suggested could also create unique social gendering experiences, omitting a factor with relatively less impact is reasonable for the sake of operability.

Under this definition, an individual's gender identity needs only to be inconsistent with any one of their AFAG during their lifetime to meet the definition of transgender.

\subsubsection{Comparison of Definitions}

The expanded definition of phenotypic sex emphasises the material reference point society uses when socially gendering an individual. Its advantage is the high inclusivity, accommodating any phenotypic sex, such as the many different types of intersex people we mentioned. Moreover, it points out that the social gendering process is mainly based on phenotypic characteristics, reminding us to reflect on this practice. The disadvantage is that it might be misinterpreted by some as a degree of biological determinism if they deliberately ignore that ``this is just a model.''

AFAB, All Formally Assigned Genders, emphasises that social gendering patterns are externally imposed and also highlights the continuity of this process throughout an individual's life. However, its disadvantage is that it might be too simplistic for different types of intersex people. If an individual is required to fill out their AFAB on a document, should they list all of them? What if there is not enough space? Furthermore, ``\textit{de facto} used by guardians'' lacks empirical evidence and strict quantification or criteria. For example, does it count if parents dress their child in different gendered clothing on a whim one afternoon? In the future, we might develop more detailed sociological or psychological methods to assess the impact of such ``\textit{de facto} use.''

Which one to choose? Frankly, I do not know. Because both models are simplifications and generalisations of objective reality, not reality itself, they are only ``useful'' or not. The final adoption depends on consultation among all relevant parties, including the transgender community, medical professionals, psychologists, sociologists, policymakers, etc. I am merely conducting purely theoretical deductions and philosophical critiques, aiming to provide new perspectives and possibilities for discussions in related fields. Specific applications and choices still await interdisciplinary cooperation and practical testing, and I will not make any irresponsible predictions about this.

\section{Becoming-Woman: An Autoethnography of Interpellation and Performativity} \label{app:b}

[Omitting the pre-reflective experience in the section ``The Shadows in the Cave.'']
\subsection{From Interpellation to Performativity}\label{subsec:from-interpellation-to-performativity}

During this time, I was in a state of confused self-exploration, mostly treating it as a meaningless fantasy. However, it increasingly affected my life. The most serious issue was that I would cast myself in the role of the female protagonist when reading some adult stories, and sometimes I couldn't distinguish between fantasy and reality. A few months ago, I was reading an adult story about psychological manipulation. I identified with the manipulated character and, for a long time, did not realise it was a story about psychological manipulation, believing the manipulator was genuinely trying to protect ``me.'' However, when I realised he was just manipulating and using ``me,'' I had a severe nightmare. I dreamt that ``I'' was crying and begging him to continue deceiving ``me.'' I woke up from the dream in the middle of the night and sat in my room for a long time, unable to fall back asleep.

After waking up the next day, I realised I had to take this matter seriously. Being suggested by friends, I began to read the works of Foucault, Althusser, and Butler to understand my experiences.

Butler points out that gender is not what we intrinsically ``are,'' but what we constantly ``do.'' It is not a stable essence, but an effect produced by the ``stylised repetition of acts in time.'' There is no pre-existing ``doer'' behind the deed. On the contrary, it is the deed itself, in its constant repetition, that fictitiously constitutes and consolidates the illusion of a stable ``doer.'' I found that many of my actions perfectly illustrate this theory.

In informal settings, I used Latin letters or cursive script for my signature to make my name ambiguous. In my artificial intelligence class, I changed the dataset. These were secret performances about the name. I even used that feminine name myself, and if discovered, I would blame the ``mistake'' on the input method. This series of performances—using an ambiguous signature, altering the dataset, and blaming the input method—is precisely what Judith Butler describes as the ``stylised repetition of acts.'' They are not expressions of an inner essence but produce the effect of identity through constant reiteration in a specific social context.

My feeling happy when praised by friends as ``quiet'' and ``virtuous'' was also a performative citation of the script of traditional femininity. Making a bracelet and enclosing a note with delicate handwriting, the ``King'' in the game said they thought their ``angel'' would be a girl, and I ``secretly felt extremely delighted.'' These are not passive reactions, but a series of active, ``stylised acts.'' By repeating these acts that conform to social imaginings of ``femininity,'' I produced the effect of a female gender identity.

A strong sense of pleasure always accompanied these moments of successful performance. The source of this pleasure is key to understanding Butler's theory. When friends defended my feminine name, when the ``King'' in the game mistook my gender, and especially when the junior female student replied, ``Wow, a pretty sis,'' I felt delighted. This gender euphoria is not the joy of an inner, ``true'' self being seen—that view still falls into the trap of essentialism. Instead, this pleasure is the profound satisfaction and relief that comes after a successful performative utterance. At that moment, the social world reflects and confirms the identity that my actions sought to create. It is the joy of making a fiction feel real, the satisfaction of successfully bringing a social reality into existence.

This discourse also plunged me into a brief confusion: if gender is a performance, then what is my gender identity, and where did it come from? Is it also the result of my performances at a younger age? I began to try to reflect on all gender-related matters in my upbringing, trying to understand how my gender identity was constructed.

My desire for a feminine name may not have been fabricated out of thin air, but like a ghost, had long been lurking in my life, stemming from the experience of my name being frequently misspelt or mistyped by others in my childhood.

A few other childhood memories also flooded my mind: adults used to say my ``personality was like a little girl's,'' probably because I liked playing with stuffed animals and disliked sports and fighting. I was also punished for imitating a little girl on TV covering her mouth to laugh, being told, ``Boys can't laugh like girls.'' Children have no gender bias and spontaneously explore the world, imitating and performing all behaviours within their capabilities as a naive performative.

Whether it's an input method error or the correction of behaviour, it is the continuous interpellation of the ideological state apparatus. Louis Althusser describes interpellation as the process by which ideology ``hails'' individuals and constitutes them as subjects, like a policeman shouting on the street: ``Hey, you there!'' Every misuse of a name, every disciplining of behaviour, is an interpellation from the social machine: ``Hey, the little girl there!'' Initially, my reaction to this call was anger and rejection. I was eager to correct it every time, trying to maintain the integrity of my identity as a ``boy.'' This anger was an instinctive resistance to a ``wrong'' ideological positioning.

This long negotiation around my name reveals the contingent origins of gender identity. The initial seed of my feminised identity did not come from some \textit{a priori} inner essence, but from a systematic ``error''—a product of an input method algorithm, a technological noise. This origin negates the Cartesian, Western Enlightenment narrative of ``finding the true self.'' The construction of identity is sometimes not a heroic journey of self-exploration, but more like an opportunistic, creative use of system loopholes and contingencies. The generation of queer identity may often occur in the cracks, glitches, and noises of the normative system. It is a form of bricolage, not an architect's creation. The trajectory of emotion, from initial anger at the ``wrong'' interpellation to the eventual joyful acceptance of an alternative subject position, reveals the core of ideology's operation. Ideology is not a cold set of ideas, but a power that orchestrates, manages, and even produces emotion. When mainstream ideology fails to provide emotional comfort, an alternative, marginal call can bring liberating pleasure, because it happens to bridge the pre-existing, ineffable gap between emotion and social positioning.

We can see a dynamic feedback loop between Althusser's interpellation and Butler's performativity. Initially, an interpellation from the external world (like the misspelt name) provides an ideological ``script.'' Then, the subject actively ``adopts'' and ``performs'' this script through a series of stylised, repetitive acts (like deliberately blurring a signature, cooking, making a bracelet). These successful performances, in turn, elicit new, more precise interpellations from others (like the address ``pretty sis''), and these new interpellations further strengthen the subject's desire to continue performing. Gender construction is thus a never-ending circular process: we are interpellated into a specific subject position (Althusser), then we consolidate and inhabit these positions through performativity (Butler), and our actions in turn invite more interpellations. Identity is not a destination to be reached, but a continuous, dynamic process of negotiation between action and response.

\subsection{A Pair of Shoes}\label{subsec:a-pair-of-shoes}

If the interpellation of a name was a relatively gentle negotiation at the symbolic level, the ``gymnastics shoes'' incident was a cruel disciplinary practice inscribed upon the body. This memory is an excellent model for understanding Foucauldian power: power does not always oppress in a grand, visible, monarchical way; it more often exists in a micro, diffuse form of discipline aimed at producing ``docile bodies.''

In elementary school, I wore a type of white cloth shoe with red trim, colloquially known as ``gymnastics shoes.'' These shoes were far more than just a colour and style; they were a strictly encoded gender symbol in the school. They were part of the girls' school uniform. In all occasions requiring a uniform, such as ceremonies and performances, girls were required to wear the red version and boys the blue one. Therefore, in this micro-society, ``red shoes are for girls'' was not a debatable opinion, but an unquestionable ``truth.'' By wearing red shoes, I blatantly violated the established norm. My body, because of these shoes, became an ``abnormal,'' transgressive body, exposed to the gaze of disciplinary power.

The power that executed the punishment did not come from the ``monarchical'' authority of the principal or teachers, but was diffused among the peer group. A few boys snatched my shoes and threw them away, pulled down my pants, and pushed me onto the broken branches of a bush in the garden. These acts of bullying were a cruel, decentralised enforcement of gender norms. The classmates at that moment acted as guards in a prison; they were the most terminal and effective nodes of the power network. The punishment was directly applied to my body: the shoes were stripped away, the body was exposed, and the skin was pierced. This precisely confirms Foucault's assertion: the target of disciplinary power is the body. It operates on the body, shaping it into a compliant ``docile body'' through training, marking, and punishment.

This micro-power operates not only by punishing transgressors but also by constantly reproducing gender through the ``truth mechanisms'' within the group. I recall that the boys in my class often established hierarchies through fighting, and I preferred to play with the girls because their interactions seemed more friendly. This preference did not stem from any innate gender essence but was a reaction to two different regimes of discipline. During a museum visit, this difference was dramatically displayed: the boys' group responded to failure with ridicule and expulsion, a punitive practice aimed at eliminating the weak and reinforcing competitive norms; the girls' group, on the other hand, offered encouragement to those who failed, a technology of care aimed at maintaining the collective and emphasising cooperation. My ``envy'' at that moment was a subject's longing for a more attractive subject position within two different micro-power networks. This reveals that the gender divide is not only marked by symbols like clothing, but is also continuously reproduced through distinctly different behavioural norms, emotional patterns, and power technologies.

Faced with this violence inscribed on the body, psychological resistance unfolded peculiarly. I often dreamt of losing my feet in an accident and would feel a strong sense of comfort and relief upon waking. It is a profound psychological metaphor: since the feet wearing ``girls' shoes'' were the source of social punishment and pain, removing these ``girls' feet'' was a fantasy of liberation from the grasp of power/knowledge. It was a complete refusal to become the ``docile body'' that the system demanded. This experience provides a powerful, non-essentialist genealogical explanation for the emergence of body dysphoria. It shows that body dysphoria does not necessarily stem from a pre-social, innate sense of mismatch. Instead, it can be the direct consequence of disciplinary power violently inscribing social meaning (``girlish'') onto the body. Dysphoria arises when this inscription is so painful that the subject internalises it as an alienation from their own flesh. I initially had no aversion to my body. Still, when a part of it (my feet) became the target of social punishment because of its association with a ``transgressive'' gender symbol (red shoes), I developed a psychological desire to get rid of that body part, seeing it as a foreign object. This localised sense of alienation may be the origin of a more widespread body dysphoria. This ``dysphoria'' is the scar left on the psyche by normalised violence.

I read some literature and researched their history, discovering that these shoes likely originated from Japanese indoor shoes (uwabaki), mainly used in schools and kindergartens in Japan. Chinese clothing factories likely took on some Japanese orders to produce these shoes, which were later sold in China. Due to their simple style and low price, they became widely used as uniform shoes for elementary school students. In some Japanese schools, the colour of indoor shoes also depends on gender, but some schools assign colours by grade level. \parencite{Kanzaki2019Shogakko} Such a genealogical analysis reveals the contingency and arbitrariness of the gendered signifier. The ``truth'' that ``red shoes are for girls'' is filled with historical contingency and arbitrariness, a result of the operation of a power/knowledge mechanism, but not a metaphysical truth.

As an adult, I developed a unique, complex attachment to these shoes. I took a photo of myself wearing them, uploaded it to Wikimedia Commons, and added it to the relevant Wikipedia article. This act is a highly self-aware reversal of power. It repositions the disciplined body as the controller, transforming a symbol laden with personal trauma and shame into public knowledge. This is a modern form of ``reverse discourse.'' It utilises one of the most powerful tools of knowledge production and dissemination of our time to rewrite the public meaning of a symbol once used to discipline me. It is an attempt to transform from an object of power/knowledge into a subject that produces power/knowledge. In the digital age, resistance sometimes manifests as an information war, a battle over the public meaning of symbols.

\subsection{The Heterotopia of Childhood}\label{subsec:the-heterotopia-of-childhood}

The operation of power is not only inscribed on the body but also divides and encodes space. My childhood memories related to the girls' bathroom reveal how this space became what Foucault called a ``heterotopia''—a real place that is nevertheless outside of all other places, an ``other space.'' The girls' bathroom was not a neutral, functional place, but a symbolic domain full of contradictory meanings, where the conventional logic of power was suspended, inverted, and reconfigured.

In elementary school, I was playing tag with a few girls, and they repeatedly ran into the girls' bathroom to hide. I stood guard at the door, waiting to catch them when they came out, but a teacher saw me and scolded me. Here, power was not about upholding the rules of the game, but about maintaining the gendered segregation of space. The punishment I received was not for disrupting the game, but because my male body ``lingering'' outside a space coded as female constituted a potential threat to the spatial order.

Another incident was when a maths teacher used ``cleaning the girls' bathroom'' as a punishment for mischievous boys. This act was highly symbolic. It transformed a private space usually associated with femininity and ``impurity'' into a public theatre of punishment. The boys were forced to enter this forbidden zone, and the punishment lay not only in the hard labour but also in the ``humiliation'' of being forced to cross a gender boundary. Here, space was used as a disciplinary technology, reinforcing the norms of masculinity through forced ``effeminacy''—that is, ``real boys'' do not belong here.

Finally, at certain times, it offered care that violated the rules. Once, I got sick and vomited during class. Perhaps because the teacher was female, our classroom was close to the girls' bathroom, and it was during class time, when no one was there, the teacher took me to the girls' bathroom to clean the vomit off my clothes and body. In my most vulnerable moment, all the taboos of this space were temporarily lifted. In a state of bodily crisis, the strict rules of gender segregation gave way to a more pressing ethic of care, creating a ``state of exception'' that instantly transformed this forbidden space into a caring, healing place.

The superposition of these three memories constructs the girls' bathroom as a paradoxical heterotopia. It was simultaneously a safe zone for girls, a place of degradation for boys, and a place of care that could be transgressed in a crisis. It was both a symbol of strict gender order and proof that this order could be broken. This chaotic, contradictory meaning, in a pre-linguistic way, shaped the female space, female symbols, and even femininity itself in my mind into a complex amalgam of fascination, taboo, safety, and care, profoundly influencing my early perception of gender.

\subsection{Becoming-woman}\label{subsec:becoming-woman}

I am significantly different from ``typical'' transgender people in the mainstream discourse, as my dislike and anxiety about my biological body do not seem to be an ontological rejection. One piece of evidence is that I discovered the mechanism of masturbation at a very young age (around kindergarten age) and excitedly shared it with others. I know this behaviour might be considered ``shameless'' or ``perverted'' by adults, but for my kindergarten self, it was just an objective exploration of the body. I believe it is no different in essence from sucking one's thumb or playing with one's hair. A child sees the body as a field of possibilities, not a cage pre-coded by gender norms.

One thing that left a deep impression on me was that I had a crush on a girl in my childhood, but she didn't like me back. I happened to read Stefan Zweig's Letter from an Unknown Woman, in which the female protagonist has a one-night stand with the male protagonist and raises their child on her own. I thought at the time, ``Wow, I also want to have XX's child and raise it secretly.'' It's so great that girls can have babies; it's so enviable. Why can't a boy's body have babies? What a pity. This was an envy of a specific function, stemming from a longing for a romantic relationship.

I find a kind of beauty in the female body that is hard to describe. It's like an aesthetic feeling, which could perhaps be loosely called a sense of elegance. I find the female body very pleasing to look at and feel envious. A DeepNude version of my feminised self has sexually aroused me. This experience has an unsettling phenomenological similarity to Blanchard's  controversial theory of autogynephilia, which is often used as a weapon to pathologise and deny the identity of transgender women. This phenomenological similarity is my undeniable ``lived experience,'' but we do not have to accept his explanation. He explains this phenomenon by positing that sexual orientation is the root cause, from which gender identity is derived, which implies the existence of an innate, ontological sexual orientation.

This experience radically deconstructs the binary opposition between identity and desire. Traditional models of identity, including many mainstream LGBTQ+ popular discourses, usually treat gender identity (who you are) and sexual orientation (who you are attracted to) as two independent axes. However, my experience of being sexually aroused by the gender image I identify with completely dismantles this clear division. Is this an act of identity affirmation or an expression of sexual desire?

This demonstrates that my ``sexual orientation'' (towards women) and my ``gender identity'' may stem from some shared, more fundamental aesthetic preference. The preference itself is neutral. It is only when combined with different sociocultural scripts that it is differentiated and interpreted as different phenomena: when combined with intimate emotions and sexual instincts, it is named ``sexual orientation''; when it interacts with body image and external discipline (like the ``red shoes'' incident), it is constructed as part of ``gender identity.'' Identity, desire and other affairs are not independent axes in a Cartesian coordinate system, but fluid, entangled, paradoxical, and rhizomatic connected. My body is therefore not an error waiting to be ``corrected,'' but a dynamic site where various desires, memories, and aesthetic judgments constantly flow, recombine, and generate meaning.

\subsection{Cyber-Pharmakon}\label{subsec:cyber-pharmakon}

Among all the practices of constructing gender, the experience of using FaceApp to generate a feminised version of myself and then using DeepNude to create a nude version is undoubtedly the most ethically and psychologically complex and contradictory. It embodies Jacques Derrida's concept of the ``pharmakon.'' Pharmakon is a core concept Derrida proposed in his reading of Plato's Phaedrus, pointing to the inherent ``undecidability'' of a word in ancient Greek: pharmakon means both ``remedy'' and ``poison,'' and these two meanings are inseparable; it is always both at the same time. My use of DeepNude is a vivid enactment of the pharmakon concept in contemporary digital technology.

The ``poisonous'' nature of this application is obvious. We must admit that its algorithm, design purpose, and even its very existence are steeped in the violent logic of misogyny and the objectification of the female body. It was initially a tool for sexual exploitation. When I was sexually aroused by the feminised image of myself generated by DeepNude, this experience, though closely linked to my exploration of gender identity, became extremely complex due to its toxic origin. Using a tool designed to objectify others for self-exploration inherently carries the great risk of self-objectification. This is its ``poison'' side.

However, at the same time, this application acted as a powerful ``remedy.'' For the immense mental suffering caused by the mismatch between social roles and inner feelings, it provided a visual, concrete comfort. Those generated female nude images, along with the scanned ID documents I modified with Photoshop, provided a tangible, visible confirmation for my then-vague and unrecognised identity. This is its ``remedy'' side.

The core is that these two aspects are inseparable; we cannot cleanly separate the ``remedy'' from the ``poison.'' The power of this tool to heal my gender anxiety is derived from the same visual, sexualizing logic as its power as a tool of exploitation. It is the ability to generate realistic female nude images that allows it to generate a feminised body image that I find gender-affirming. My female identity was generated under the interpellation and performativity of the same misogyny and male gaze that are at the root of the AI software that creates objectified women. The affirmation of identity and sexual arousal are intertwined here, making it impossible to distinguish which is the cause and which is the effect. They are two sides of the same coin.

This act is a prime example of queer reappropriation, and queer reappropriation itself is a practice with the nature of a pharmakon. The word ``queer'' itself is the most famous example of the queer community reappropriating an oppressive tool and stigma (poison) into a symbol of community pride and identity (remedy). My action was to turn a technological tool designed for misogyny inward, using it for self-construction and self-healing. The political potential of queer practice often lies not in creating ``pure,'' utopian tools free from the contamination of the oppressive system, but in seizing the contaminated, ``toxic'' tools within that system and using them to resist the system's own logic. The power of reappropriation comes precisely from its impurity.

\subsection{Strategic Habitation}\label{subsec:strategic-habitation}

The archaeological excavation of this autoethnography finally arrives at the present. I turn to another question: Is my gender identity really ``female''? This seems natural, but in fact, I sometimes genuinely believe I am ``male,'' especially when arguing with trans-exclusionary radical feminists (TERFs) online.

When they criticise or insult all ``phenotypic males'' in some gender-essentialist way (e.g., ``all men are oppressors''), I feel very angry and use myself as a counterexample to refute them. Of course, this is partly because I know very well that their so-called ``men'' refers to phenotypic sex, not gender identity. It at least shows that although I usually feel uncomfortable when being called or classified as ``male,'' this discomfort is not greater than my hatred for irrationality. I feel comfortable with it if classifying myself as ``male'' can provide a valid counterexample to refute their argument.

I am not sure if this counts as a kind of ``gender identity.'' When I argue with TERFs, the structure of the anger is the same as the anger I feel when arguing with extreme nationalists who proclaim that ``all Japanese people are guilty, there are no innocent souls under the atomic bomb.'' I believe my anger stems from the ``collective responsibility.''

While for me personally, there is a significant difference between these two situations: I am obviously not Japanese, so when I argue with extreme nationalists, I am very clear that this is a purely moral anger against collective punishment. In the other situation, because I had not deeply thought about gender identity at the time, and their definition of ``male,'' along with that of the broader society, did include me, the target and direction of this anger were often confused. I sometimes genuinely felt that I was a (specifically defined) ``male'' and that I was being insulted.

From a particular perspective, this is also ``internalising a gender identity through interaction with a pervasively gendered society''. TERFs use a crude, essentialist method to impose the label ``male'' on me. For the sake of debate, I strategically accept this label and develop complex emotional reactions around it. This ``contextual male identity'' and the ``female identity,'' I feel, in many other scenarios, are formed by the exact same mechanism.

It demonstrated that this phenomenon not only occurs in childhood but also appears continuously throughout a person's life. It's just that for most people, whether cisgender or transgender, after their gender identity is formed, they will consciously resist the invasion from another gender. I happen to not care much about ``gender.'' I don't think my gender identity is essential to me, not my core identity, so I didn't resist it. Ironically, my ignorance of gender has allowed ``gender'' to be able to freely ``invade'' my ``self.'' Some other ``gender fluid'' people may also stem from a similar cognitive mechanism.

This is an unconscious practice of ``strategic essentialism'': I strategically occupy the category of ``male'' imposed on me by my opponents to deconstruct their arguments from within. This demonstrates an understanding of identity as a tool to be used, a position to be occupied, rather than a fixed, inner truth. It ultimately embodies the core idea of post-structuralism: ``the subject does not pre-exist discourse, but is constituted within discourse,'' and can therefore be strategically reconfigured and reused according to context and need.

\subsection{Towards a Practice of Queer Life}\label{subsec:towards-a-practice-of-queer-life}

Through the theoretical analysis of a series of my personal experiences, we have painted a picture of the generation of gender identity: it is interpellated by ideology, disciplined by power, constituted through performativity, and navigated with the help of contradictory pharmakon-like tools. The conclusion is that gender identity is not a stable essence, but a contingent, socially mediated, and ongoing process. The fluidity of gender is not ``indecisiveness,'' but a politically conscious practice of refusing to be fixed by any single category. It ultimately embodies the idea that ``the subject does not pre-exist discourse, but is constituted within discourse,'' and can therefore be strategically reconfigured as needed.

Thus, what this autoethnography describes is not a journey of ``discovering'' a true self, but a journey of learning how to deconstruct and use identity. From the body disciplined by ``red shoes'' in childhood to the strategic deployment of a ``male'' identity in online debates as an adult, this is a process of transforming from an object shaped by power to a subject capable of using discourse and identity for resistance. The conclusion is that gender identity is not a stable essence, but a contingent, socially mediated, and ongoing process of becoming.

Queer theory is therefore not just an academic discipline, but a crucial set of life tools. The most liberating response to the gendered power diffused throughout society may be continuously self-reflecting on how our own identities are shaped by power; refusing the temptation to find a single ``true'' gender identity, and instead embracing the complexity, contradiction, and fluidity of life experience; and learning to performatively and strategically use identity categories to challenge power, rather than letting them define us. The ultimate goal is not, as we usually think, to ``discover'' who we are, but to actively refuse the person that power tries to shape us into. This is a never-ending, generative self-creation.

\end{document}
