First of all, it must be clarified that ``transgender'' in the title is a pragmatic term.
I indeed have some transgender-like experience.
However, I find it challenging to have myself truly categorised.
If we strictly apply the ``gender identity'' classifications,
my gender identity might be gender fluid,
but this is also a pragmatic term to help others understand me.

I am a PhD student in plant taxonomy, using programming and statistical methods based on
morphology and molecular phylogenetics to resolve the classification of plants.
This methodology has deeply influenced the methods I used in my own journey of gender exploration.

I understand that writing an autoethnography based on Enlightenment rationalism to advocate for
a grand narrative may be seen by some scholars as contrary to the fundamental paradigms of this field.
However, I am not trying to convince my readers that I am completely correct, but
documenting how an individual with a firm commitment to rationalism,
who has received rigorous scientific training, attempts to understand a crucial aspect of themself.
This is, in fact, a story of how a person resolves a significant existential crisis.

Additionally, the basic methods I used are introspection and cross-validation with relatives and friends
to propose explanations of myself. I then extend my own experiences to the scope of ``ethnography''
and analyse the related social and philosophical issues.
I did not scan my brain with an fMRI machine and record the data while
experiencing gender dysphoria or gender euphoria.

%Moreover, if we examine this work from a post-structuralist perspective, it holds that reason and science are essentially just one specific discourse among many. Under their epistemology, ``a transgender biology PhD student exploring their gender identity'' is not essentially different from ``a transgender Buddhist exploring their gender identity.'' The conflict between rationalism and an ``innate, profound gender identity'' (an unverifiable ontological assumption) is no different from the conflict between Buddhism and the idea of a ``female soul born in a male body'' (Buddhism does not consider the soul to have a ``gender;'' a person could even be an animal in their next life, making a discussion of the soul's gender meaningless). Both are stories of an individual's experience in coping with two conflicting discourses, both of which relate to important aspects of their existence.

On the other hand, autoethnography is a unique genre in which the author's own philosophical and scientific stances are part of the data. The soul of autoethnography lies in honestly presenting how the ``self'' (auto-) experiences and understands ``culture'' (ethno-). My ``self'' is the one who has been scientifically trained, committed to Enlightenment rationality, and tries to use logic and order to understand a chaotic and painful world. Not to do so would be dishonest. My insistence on ``evidence,'' ``credibility of introspection,'' and my desire to construct a ``hypothesis'' with ``explanatory power'' are not merely tools for writing and research; they are an inseparable part of my very ``self.'' The positivist and rationalist attitudes are essential parts of the ``sanctuary'' in which I find my place in the world. Therefore, it must be faithfully presented here as the core of the story.

%I do not intend to formally publish it, as I am not in the philosophy department and do not need a philosophy paper to get a degree, so it may not be written in a particularly formal style.

