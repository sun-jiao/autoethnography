I became aware of my ``gender identity'' when I subconsciously thought from a girl's perspective. For instance, once, while discussing physical fitness tests, I said ``my 800-metre run.'' A friend said, ``It's 1000 meters.'' I didn't respond directly, joking, ``It's actually 800 plus.'' They jokingly asked if I was a girl.\footnote{In Chinese universities, the long-distance running test is 1000 metres for male students and 800 metres for female students.} I didn't deny it directly, replying with a Chinese internet slang, ``u1s1, qs'' (有一说一,确实 you yi shuo yi, que shi, means to be honest, yes).

During this period, I developed a strong desire to adopt a more feminine name: 娇 (\textit{Jiāo}, will be addressed as ``the feminine \textit{Jiāo}''). This character means ``cute'' or ``adorable'' and features the ``woman'' radical. It is a homophone and graphically like my legal name (骄, \textit{Jiāo}, means ``pride'', relatively neutral). I often used cursive script (行书, \textit{xíng shū}) or the  Romanisation (Pinyin) to make them indistinguishable. Furthermore, because the feminine name is much more popular than my legal name, it is the first choice in many Chinese pinyin input methods. \footnote{When typing in Chinese, we use a software called ``input method.'' It gives all possible Chinese characters based on the pinyin, and the user chooses the correct character from them. It is a common thing to choose a wrong character.} Every time somebody types my name as the feminine \textit{Jiāo}, I was extremely delighted and afraid that someone would ``kindly'' point it out. If this really happened, I would be very ``tolerant'' and say, ``It's okay, as long as I know you're addressing me, a wrong character doesn't matter.'' At an academic conference, the curators wrote my name as the feminine \textit{Jiāo} on a poster. A junior male student discovered it and contacted the responsible officer to correct it. I felt a strong sense of disappointment at that moment. Sometimes I even used the feminine \textit{Jiāo} myself, and if discovered, I would blame the input method.

Because I never corrected it, many personal friends who knew me in informal places (like in student clubs) thought it was my legal name. Occasionally, someone would ask, ``Is your name really the feminine \textit{Jiāo}?'' and friends would argue with them, ``Why can't a boy be named the feminine \textit{Jiāo}? It's such a cute name! That's a rude question.'' I would pretend not to see the group messages, secretly enjoying the protection from friends. Some friends thought it was a nickname, which I also didn't correct.

I was also delighted when friends described me as ``quiet'' and ``virtuous'' (文静 \textit{wén jìng} and 贤惠 \textit{xián huì}, both are traditionally ``feminine'' adjectives in Chinese) because I didn't talk much and cooked during home parties, as well as when my handmade hog plum (\textit{Choerospondias axillaris}) bracelet was considered to be from a girl at a gift-exchange event. I secretly used deep learning-based image generation models to create some feminised photos of myself. I also scanned my ID card and graduation certificate, changed the sex marker, replaced the photo. Additionally, I bought some women's shoes and clothes, including my hiking boots, which I wore on fieldwork in Xizang (Tibet) and western Sichuan.% (\cref{fig:quechua})

%Once, we went camping and played an ice-breaking game called ``King and Angel.'' \footnote{This is a game in which everyone will be an ``angel'' of their ``King''  and need to do everyone's best to take care of their ``King'' In the final reveal, everyone needs to try to name their ``Angel'' according to the care they received. And the word ``国王'' \textit{guó wáng}, which literally means ``monarch of a country,'' is theoretically gender-neutral in Chinese, though has been used to translate "King."} I gave my ``King'' a handmade hog plum (\textit{Choerospondias axillaris}) bracelet, placing it into their clothes with a note. During the final reveal, the ``King'' said that upon seeing such a fantastic bracelet and delicate handwriting, they thought it would be from a girl. I secretly felt extremely pleased about that.

%However, because the proportion of female students at my undergraduate university was very high (70\%), I didn't take this matter very seriously at the time. I initially rationalised this feeling as ``I was assimilated by girls after spending a long time with them.'' \footnote{I'm not saying that girls naturally possess so-called ``feminine traits'' and ``feminine behaviours.''}

%Since I remained in my undergraduate student club's group chat after graduation, one day I saw some new students joining the group. I told one of them I was a girl and sent an AI-generated photo. They replied, ``Wow, a pretty sis.'' I was incredibly delighted.

%\begin{figure}[htbp]
%    \centering
%    \includegraphics[width=0.7\textwidth]{figures/Quechua_and_birds}
%    \caption{Photos of the author wearing Decathlon Quechua MH100 women's boots during a birding trip in western Sichuan Province: a) The author (\textit{Homo sapiens}), b) Golden Bush Robin (\textit{Tarsiger chrysaeus}), c) Blood Pheasant (\textit{Ithaginis cruentus}), d) Przevalski's Suthora (\textit{Suthora przewalskii}).\label{fig:quechua}} % title and label
%\end{figure}

During this time, I had wondered if I was ``transgender'' and tried to understand the mainstream transgender narratives. However, I kept encountering intellectual barriers, such as: What is gender identity? Where does it come from? What is its relationship to ``gender''? Why is it ``gendered''? The term ``gender'' refers to a sociocultural category, then it is essentially an identity with a social construct. This contradicts the so-called ``innate, profound feeling,'' because social norms are nurtured. It is also politically problematic, as it seems to advocate that people should identify with the oppressive gender roles. If it points to phenotypic sex, this contradicts the sex/gender distinction, and how does an identity with gender lead to a desire for bodily modification (sex)? If it refers to ``gender identity'' itself, this constitutes a ridiculous tautology, i.e., ``gender identity is an identity with gender identity.''

I posted my questions on some platforms, including Reddit, RedNote (Xiaohongshu) and Zhihu (a Chinese Q\&A website), hoping for some advice or help. However, almost no one responded to me seriously. I was insulted or degraded on all three platforms, and my account was banned on Reddit. I was both angry and disappointed. However, beyond that, the more crucial matter was to solve my own problem.

%Therefore, I concluded that this theory was intellectually unsatisfying. I quickly abandoned it and returned to my original explanation: this was just a meaningless fantasy.

%The most serious issue was that I would cast myself in the role of the female protagonist when reading some adult stories, and sometimes I couldn't distinguish between fantasy and reality. A few months ago, I was reading an adult story about psychological manipulation. I identified with the manipulated character and, for a long time, did not realise it was a story about psychological manipulation, believing the manipulator was genuinely trying to protect ``me.'' However, when I realised he was manipulating and using ``me,'' I had a severe nightmare. I dreamt that ``I'' was crying and begging him to continue deceiving ``me.'' I woke up from the dream in the middle of the night and sat in my room for a long time, unable to fall back asleep.

%For a long time afterwards, I continued this ``meaningless fantasy,'' but it increasingly affected my life. I realised that I had to take this matter seriously. I initially intended to continue learning about mainstream theories, so I tried to organise all my questions clearly. This included the ambiguity of the sex/gender distinction, the circular reasoning behind ``gender identity'' and ``brain sex,'' and how the current mainstream narrative overemphasises European-American culture and Indo-European languages (e.g., the impact on languages without gendered pronouns), thereby forming a cultural invasion of other cultures.

%I posted my questions on some platforms, including Reddit, Xiaohongshu (RedNote) and Zhihu (a Chinese Q\&A website), hoping for some advice, help, or response. What I couldn't understand was that almost no one responded to me seriously. I even doubt they read my post carefully. ``Nobody is interested in reading your AI-generated bullshit.'' ``You just need to post the AI prompt you used.'' ``He is probably a 14-year-old incel who just told ChatGPT to write him an essay promoting gender essentialism.'' I argued with them for a long time, repeatedly telling them, ``This is not AI-generated; I wrote it very seriously,'' and ``If you ask ChatGPT to write an essay promoting gender essentialism and get this result, it only proves that ChatGPT is lying to you.''
%
%I tried to explain to them, ``We might have some misunderstandings,'' but they seemed to refuse to accept it. This ultimately led to my account being banned. I was both angry and disappointed. However, beyond that, the more crucial matter was to solve my own problem.