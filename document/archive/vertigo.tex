After organising this self-analysis, I looked it over with satisfaction a few times. It was not long before I began to doubt this narrative, because I realised that some things seemed to be just "invented" rather than recalled by me, such as the story of being bullied because of my shoes. I have a memory of wearing red gymnastics shoes in elementary school, and I have a memory of being bullied by bad kids; both vivid and rich. In contrast, the causal relationship is entirely abstract and empty, without detail. I cannot find a memory of "10-year-old me crying and clearly realising that they were bullying me because of my shoes." I cannot even recall realising it at age 12 or 14. Not only that, but I had never seriously thought about why they bullied me before.

On the other hand, I cannot and will not ask them, and they have likely forgotten it entirely -- as we all know, this is typical for bullies. Thus, the truth is, I do not know why they bullied me at all. Perhaps it was simply because I looked easy to bully, or because I was introverted and didn't talk much. I "invented" this causal relationship after setting the goal of "finding the origin of my gender identity" and then recalling and analysing. This is a classic Texas Sharpshooter Fallacy.

In contrast, the memory of being mocked by classmates for wearing red gymnastics shoes is very vivid, specific, and full of detail. A case in point is that once in Chinese class, the teacher asked us to bring some childhood photos and tell stories based on them. In the photo I brought, I was wearing such shoes, and a classmate mocked me, "So you've been wearing girls' shoes since you were a kid" (or something like it), and I angrily snatched the photo back. The causal relationship between them is supported by details. Additionally, I found these childhood photos of me wearing red gymnastics shoes at home. (\cref{fig:shoes})

The story of visiting the museum is similar. The memory of being mocked and heckled by some male classmates, urging me to get down so they could play, and me crying in a corner while enviously watching the girls' group encouraging each other, is very real. The memory of my name being mistyped in childhood is also like this. I remember my mum angrily saying, "Can't they see it is a boy in the photo? Why would they use this character?" and demanding that I go to school the next day and have the teacher change it. This still happens today; many friends who clearly know my legal name, and even government officials, have typed my name as the feminised version. I even found physical evidence, such as name tags or documents. (\cref{fig:2})

Similarly, although I have a vivid memory of lying in bed and recalling the dream. I remember trying to fantasise about the dream's content to find peace and calm to help me sleep during a night of insomnia, the so-called causal chain from "my feet are girls' feet" gradually spreading to "I am a girl" is a "recent invention."

I even began to question those detailed and rich memories, as they could theoretically be retrospectively constructed. I tried to find physical evidence, such as the photos of me wearing red gymnastics shoes and the middle school's name tag with the wrong name. I also found the book \textit{Letter from an Unknown Woman}, which I read as a child, but most of my memories lack such corroborating evidence.

I tried to rationalise the reliability. Although our memories is not 100\% factual, a study by \textcite{Diamond2020Truth} shows that 93--95\% of verifiable details in human memory are accurate. With such a rate, the broad framework of the narrative and the conclusion we reached -- "gender is a product of the individual's interaction with and internalisation of social norms" -- are relatively reliable.

It no longer relies on a single, highly hypothetical causal chain like "being bullied because of the shoes, therefore seeing my feet as 'girls' feet,' and then gradually extending that to 'I am a girl'." Instead, it is based on a social pattern that repeatedly appeared throughout my development, summarised from a large amount of memory data: the individual's neutral personality traits, behaviours, and used items were repeatedly labelled with "gender" by the external social environment, accompanied by strong emotional feedback such as ridicule, discipline, exclusion, and even trauma. This series of interactions shaped the individual's perception of "gender" like a network or a "chain reaction."

After resolving the issue of memory reliability, I turned to another question: Is my "gender identity" \footnote{We will continue to use this term for now, although my definition differs from the mainstream.} really "female"?

This still seems to be a Texas Sharpshooter Fallacy. Before I began my analysis, I had presupposed that "I have a strong sense of identity with female identity and related social symbols." However, sometimes I very naturally and automatically think of myself as "male," especially when arguing with trans-exclusionary radical feminists (TERFs) online.

When they criticise or insult all "phenotypic males" in some gender-essentialist way (e.g., "all men are oppressors"), I feel very angry and use myself as a counterexample to refute them. Of course, this is partly because I know very well that their so-called "men" refers to phenotypic sex, not gender identity. It at least shows that although I usually feel uncomfortable when being called or classified as "male," this discomfort is not greater than my hatred for irrationality. If classifying myself as "male" can provide a valid counterexample to refute their argument. I am happy to substitute myself into $ \text{Male}(x) $ and logically falsify their universal proposition of $ \forall x (\text{Male}(x) \rightarrow P(x)) $.

I am not sure if this counts as a kind of "gender identity." When I argue with TERFs, the structure of the anger is the same as the anger I feel when arguing with extreme nationalists who proclaim that "all Japanese people are guilty, there are no innocent souls under the atomic bomb." \footnote{Referring to Hiroshima and Nagasaki.} I believe my anger is from the irrationality and collective responsibility. These two scenarios are almost logically isomorphic: an oppressive power (patriarchy/Japanese militarism) commits evil in the name of a specific group, which does not grant the oppressed group the right to indiscriminately attack the former group in return, because this power clearly did not receive authorization from the group it claims to represent.

While for me personally, there is a significant difference between these two situations: I am obviously not Japanese, so when I argue with extreme nationalists, I am very clear that this is a purely rational anger against irrationality. In the other situation, because I had not deeply thought about gender identity at the time, and their definition of "male," along with that of the broader society, did include me, the target and direction of this anger were often confused. I sometimes genuinely felt that I was a (specifically defined) "male" and that I was being insulted.

From a particular perspective, this is also "internalising a gender identity through interaction with a pervasively gendered society." TERFs use a crude, essentialist method to impose the label "male" on me. For the sake of debate, I strategically accept this label and develop complex emotional reactions around it. This "contextual male identity" and the "female identity," I feel, in many other scenarios, are formed by the exact same mechanism.

Thus, this phenomenon not only occurs in childhood but also appears continuously throughout a person's life. It is just that for most people, whether cisgender or transgender, after their gender identity is formed, they will consciously resist the invasion from another gender. I happen to not care much about "gender." I don't think my gender identity is very important to me, not my core identity, so I didn't resist it. Ironically, my ignorance of gender has allowed "gender" to be able to freely "invade" my "self." Some other "gender fluid" people may also stem from a similar cognitive mechanism. \footnote{I did not rule out other mechanisms of gender fluid. }