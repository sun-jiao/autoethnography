\documentclass[
  journal=PREPRINT,
  manuscript=article,  %% article (default), rescience, data, software, proceedings, poster
  layout=preprint,  %% preprint (for submission) or publish (for publisher only)
  year=,
  volume=,
]{extra/joas}
%\doi{10.13140/RG.2.2.27723.66085}
%\received {1 Sep 2025}
%\revised  {1 Dec 2025}
%\accepted {10 Dec 2025}
%\published{20 Dec 2025}
%\editor{Joanna Thorne}
%\reviewers{Anonymous reviewers}

\usepackage{biblatex}
\usepackage{cleveref}
\usepackage{hyperref}
\usepackage{csquotes}
\MakeOuterQuote{"}
\renewcommand{\abstractname}{Abstract}
\renewcommand{\figurename}{Figure}
\renewcommand{\tablename}{Table}
\renewcommand{\refname}{References}
\addbibresource{references.bib}
\DeclareNameAlias{sortname}{family-given}
\DeclareNameAlias{default}{family-given}

%\providecommand{\posscite}[1]{\citeauthor{#1}'s (\citeyear{#1})}
\usepackage{lineno}
\nolinenumbers

\title{For Gender Abolition -- A Scientific Realistic Argument}

\author{Jiao Sun\orcidlink{0000-0002-5028-8132}}
\affiliation{Division of Ecology and Evolutionary Biology, School of Biological Science, University of Reading, Whiteknights, Reading, RG6 6EX, United Kingdom}
\email{j.sun@pgr.reading.ac.uk}
\date{12/12/2025}

\keywords{Gender abolitionism, gender identity, gender studies, philosophy of biology, criticism of queer theory, philosophy of mind}

\begin{document}

\maketitle

\begin{abstract}
This article presents a critique of the concept ``gender identity'' based on a synthesis of evolutionary biology, neurology and philosophy of science, finally culminating in an argument for gender abolitionism. The narrative critically engages with mainstream gender theories, revealing logical fallacies within concepts such as the sex/gender distinction, ``innate gender identity'', ``assigned sex at birth'' (ASAB), and the cisgender/transgender binary. The author proposes a revolution in the form of gender abolitionism: a framework that strictly limits sex \textit{sensu stricto} to gametes, dismantles phenotypic sex into a spectrum of sexual dimorphic traits, and advocates for the complete removal of gender as a social, legal, cultural, and self-identity category. Furthermore, the article extends its critique to queer theory. While their deconstructive intentions are noted, their real-world effect paradoxically reinforces gender-centrism and gender essentialism, ultimately failing to provide a path to liberation. Even for readers who do not agree with this strategy, it also provides a physicalist framework to help specific transgender people and onlookers to understand themselves or others.
\end{abstract}

\addcontentsline{toc}{section}{Abstract}

\section{Introduction}\label{sec:introduction}
The contemporary discourse surrounding gender identity often presents a dichotomy between essentialist narratives -- whether rooted in religious conservatism or a search for a ``gendered brain'' -- and post-structuralist frameworks that view identity as purely discursive and fluid. For an observer grounded in the natural sciences, both extremes frequently appear to bypass the material realities of biology and the rigorous demands of logical consistency. The former often relies on a metaphysical ``soul'' or unproven neurological determinism, while the latter frequently risks dissolving the subject entirely into language games, potentially alienating the very lived experiences it seeks to describe. This article attempts to bridge this epistemological schism by examining the phenomenon of gender not through the lens of ideology, but through the analytic tools of evolutionary biology and the philosophical tradition of the Enlightenment.

I am a PhD student in plant taxonomy, using programming and statistical methods based on morphology and molecular phylogenetics to resolve the classification of plants. This methodology has deeply influenced the methods I used in my own journey of gender exploration. This article aims to document how an individual with a firm commitment to Enlightenment rationalism, who has received rigorous scientific training, attempts to understand a crucial aspect of themself. Autoethnography is a unique genre in which the author's own philosophical and scientific stances are part of the data. The soul of autoethnography lies in honestly presenting how the ``self'' (auto-) experiences and understands ``culture'' (ethno-). My ``self'' is the one who has been scientifically trained, committed to Enlightenment rationality, and tries to use logic and order to understand a chaotic and painful world. The positivist and rationalist attitudes are essential parts of the ``sanctuary'' in which I find my place in the world. Therefore, it must be faithfully presented here as the core of the story.

This inquiry necessitates a departure from standard narratives of transition. Rather than seeking to validate a pre-existing category of identity, this work interrogates the category itself. It asks whether the concept of ``gender identity'' possesses ontological weight when subjected to the same scrutiny one would apply to a taxonomic classification or an evolutionary hypothesis. By engaging with personal memory -- ranging from childhood socialisation to adult interactions within academic and online communities -- this article treats the self as a case study for a broader theoretical proposition: that the path to genuine human liberation lies not in the proliferation of gender categories, but in their abolition.

Furthermore, this text addresses the isolation often felt by those who fall outside politically convenient taxonomies. It explores the tension of being a ``gender abolitionist transgender'' individual -- a position that frequently invites hostility from both trans-exclusionary radical feminists and dogmatic sectors of the queer community. By grounding this analysis in the universalist principles of the Enlightenment, the following sections argue for a reconstruction of subjectivity that transcends the ``prison'' of gender, aiming instead for a humanism defined by reason, autonomy, and the courage to know.


\section{Biological Critique of "Sex"}\label{sec:dismantling-sex}
In this section, I will illustrate that the definition of the so-called "biological sex" involves the fallacy of begging the question and contradicts Occam's razor as a scientific redundancy. This argument is necessary for further discussion because many scientific models I will discuss repeat this fallacy when explaining gender identity. In this article, I use ``phenotypic sex'' as an alternative to the conventional term ``biological sex,'' because the academic biological definition of sex is strictly confined to the types of gametes \parencite{Lehtonen2014Gamete, Goymann2023Biological, Hurst1996There, Griffiths2025Biology}, and its determination mechanisms are homologous among the entire vertebrate lineage \parencite{Bellott2017Avian, Graves2010Homologies, Smith2007Bird}. Different clades have developed different sex chromosomes, genitals, and sexual dimorphic characteristics in the evolutionary history. These characteristics are dynamically shaped by sexual selection. Genitals can change extremely fast in some groups, e.g., some ducks \parencite{Brennan2007Coevolution, Orbach2018Evolution}, which is not different from other sexual dimorphic traits. The phenotypic sex (so-called ``biological sex'') is merely an arbitrarily organised subset of sexual dimorphic traits. Its scope is unjustified. \textcite{Zieminska2022Toward} proposed the so-called "five layers of sex," namely sex chromosomes, gonads, internal sex organs, external genitals, and secondary sex characteristics. From an evolutionary perspective, the sexual dimorphism of height possesses no essential difference from that of sex chromosomes, genitals, or secondary sex characteristics. The difference lies solely in the degree of overlap; height exhibits significantly greater overlap than genitalia. This is a quantitative difference rather than a fundamental one. They are all adaptive traits that promote the rate of reproduction success. Then why are tall "women" or short "men" not considered "intersex?" In what way are the five layers different from other sexual dimorphic traits? This is a typical instance of begging the question. As an instrumental proxy of gametic sex, its functions can be replaced by sexual dimorphism or some specific sexual dimorphic traits, therefore framing it as a scientific redundancy. Moreover, using the same term "sex" for gametes and an arbitrary subset of sexual dimorphic traits wrongly implies that they are more closely related than other sexual dimorphic traits. It should be named "strongly sexual dimorphic traits" rather than "sex."

Secondary sexual characteristics are sometimes considered part of phenotypic sex (e.g., \cite{Zieminska2022Toward}). However, intersex conditions are practically diagnosed only by their gonads and genitals, not the whole range of "sex." Individuals with gynaecomastia are not considered intersex. For individuals with gynaecomastia who self-identify as male, medical treatment for their breasts is also to ``make their body consistent with their gender identity,'' while it is not considered a "gender-affirming surgery." What does the component "sex" mean in this term? \textcite{AMA2021Advancing} define "intersex" as "individuals whose reproductive organs and anatomy (e.g., primary sex characteristics, hormones, chromosomes, etc.) do not align with medically defined and socially expected notions of male and female." Why aren't individuals with polycystic ovaries syndrome (PCOS) "intersex," since their hormones do not align with typical male or female phenotypes. The definition of phenotypic sex and related terms are not only unjustified, but also blurred, nebulous, and unstable, even sometimes self-contradictory. 


Some biologists who support instrumental or pragmatic methodologies might argue that even if phenotypic sex doesn't truly exist, it's useful in many areas. Indeed, like Linnaeus' taxonomic ranks, which I do not deny. However, I insist that it's not sex itself, but merely a phenotypic proxy.

Others might argue that chromosomes, gonads, genitals, and hormones aren't simply correlated, but rather form a tightly coupled, cascaded regulatory system. While genitals aren't gametes themselves, they are crucial for gamete transmission.

Following this logic, the sex of birds (other than ducks) cannot be observed externally and genetic sequencing or dissection is necessary because their genitals are isomorphic. However, most ornithologists wouldn't support this idea. They use other sexual dimorphic traits like plumage colour to identify sex during ecological field surveys, and plumage colour is ontologically closer to height than genitals.

In this situation, the opposing side has two possible responses:

\begin{enumerate}
    \item Insist on the "core system" argument: This would mean that all ecological studies that identify bird sex based on plumage colour are methodologically flawed. They confuse "sex" with "sexual dimorphic traits." Unless they perform genetic testing or dissection on every bird, their data is invalid. -- This is clearly absurd and would negate decades of avian ecology and behavioural research. This is an unacceptable price.
    \item Acknowledge the legitimacy of "proxies": "Okay, in some cases, non-core traits like plumage colour can also be legitimately considered as components of sex." -- Then why not height? What is the ontological difference between identifying bird sex by plumage colour and identifying human sex by genitalia?
\end{enumerate}


\section{Biological Critique of Gender Identity}\label{sec:biological-critique}
In this section, I will demonstrate that the idea of an "innate," or "inherent" gender identity is groundless and directly contradicts many biological laws.

\textcite{APA2015Guidelines} defines "gender identity" as "a person's deeply felt, inherent sense of being a girl, woman, or female; a boy, a man, or male; a blend of male or female; or an alternative gender." \textcite{NHS2022Gender} defines it as "a way to describe a person's innate sense of their own gender, whether male, female, or non-binary."

It is untenable that we have an innate feeling about "gender" because it is a social construct. From a physicalist and reductionist perspective, one's gender identity is the product of a specific physical state of their brain, involving the connection patterns of neurons. Theoretically, the human brains could be initialised in this state as long as there are enough genes about it. However, this claim is too \textit{ad hoc}. As an analogy, infants have the sucking reflex without knowing what a ``breast'' is. Innately encoding a ``gender identity'' is more useless than encoding a ``breast.'' The latter would allow infants to recognise breasts innately, distinguish what should and should not suck, and avoid sucking on harmful things. Therefore, an innate ``gender identity'' is less probable than an innate concept of breasts. Moreover, using so many genes to encode such a metaphysical, abstract concept is biologically inefficient and impractical, without reasonable selection pressure. In contrast, evolving a brain with high neuroplasticity to learn abstract concepts is far more reasonable.

%By definition, it clearly falls under the concept of "narrative identity" or "narrative self", specifically, "self-concept," which is defined as the "conscious beliefs about the self that are descriptive or evaluative" \parencite{Fanti2024Dual}. In contemporary psychology, cognitive science and philosophy of mind, there is a broad consensus that narrative identity is essentially a posteriori. It is formed through the integration and interpretation of personal life experiences and is deeply influenced by sociocultural frameworks \parencite{Fanti2024Dual}. Therefore, claiming that gender identity is a priori or innate creates a profound philosophical contradiction.

Many neuroscience research claiming an ``innate gender identity'' involves circular reasoning and explanatory gaps. They used the brains of individuals who conform to assigned gender roles (so-called ``cisgender'') as ``male brains''/``female brains.'' However, without human society and culture, this would just be a sexual dimorphic neurological trait, but not ``male'' or ``female.'' If it could be considered ``male'' or ``female,'' tall/short individuals would also be considered ``male height'' and ``female height,'' so would fat/thin individuals, because they are also sexual dimorphic characteristics \parencite{Wells2007Sexual}. Additionally, research shows that almost no one's brain is entirely "male" or entirely "female"; the vast majority of people have a mixture of "typical male" and "typical female" characteristics \parencite{Baxendale2025Brain, Joel2015Sex}.

Primatologist Frans de Waal even claimed that ``primates are born with a gender identity'' because their behaviour does not conform to the statistical typic for their phenotypic sex, such as the phenotypically female chimpanzee Donna, who occupies territory and fights with others~\parencite{DeWaal2022Different, Morin2022Frans}. However, nobody has ever asked ``Hi, Donna, what is your gender identity?'' De Waal's claim is effectively saying that gender identity is a behavioural characteristic of animals. Animals exhibit sexual dimorphic behaviour, while equating "aggression" with "male identity" is not only anthropomorphism, but also a logical leap. From an evolutionary perspective, "aggression" offers a survival advantage, but an abstract concept like "I am a male chimpanzee" would have been entirely unprofitable and highly energy-intensive for animals without language.

Discomfort with a specific body morphology starts from an early age, reported by some transgender individuals, may indeed stem from internal body representation, which was revealed by some neuroscience studies \parencite{Case2017Altered, Lin2014Neural, Ramachandran2008Phantom}. However, according to the sex/gender division, a preference for a specific body morphology should not be considered "gender identity" itself. This is a category error.

The reason they are interpreted within the framework of ``gender'' is a product of society and culture. The sex \textit{sensu stricto} is strictly confined to the gametes in biology \parencite{Lehtonen2014Gamete, Goymann2023Biological, Hurst1996There, Griffiths2025Biology}, and its determination mechanisms are homologous among the entire vertebrate lineage \parencite{Bellott2017Avian, Graves2010Homologies, Smith2007Bird}. Different clades have developed different sex chromosomes, sex organs, and sexual dimorphism in the evolutionary history. These characteristics are dynamically shaped by sexual selection. The phenotypic sex (so-called ``biological sex'') is merely a socially organised set of sexual dimorphic traits from all of them. The boundaries are very blurry, and the standards are inconsistent and unstable. For example, breasts are sometimes considered part of phenotypic sex (e.g., in gender-affirming surgery), but not when determining intersex conditions. Intersex individuals are judged only by their genitals; individuals with gynaecomastia are not considered intersex. For individuals with gynaecomastia who self-identify as male, medical treatment for their breasts is also to ``make their body consistent with their gender identity,'' which therefore should also be considered a gender-affirming surgery. It is shaped by the sociocultural construct that why we understand body representation incongruence about some parts as ``gender dysphoria,'' while other parts as ``body dysmorphic disorder'' or ``body integrity identity disorder.''

A nature-nurture co-operation hypothesis is strongly supported by the aforementioned neurological studies. \textcite{Case2017Altered} suggest an innate multimodal body representation but acknowledge that culture and experience can shape it, making the ``innate'' vs. ``acquired'' line blurry. \textcite{Lin2014Neural} revealed that transgender individual's brains showed high connectivity between the body representation areas and visual/auditory processing areas. They suggest this means transgender individuals are ``integrating massive visual and auditory cues to shape their body image,'' and transgender people's ``distinct neural network of body representation can be coterminal to genetic constitution, developmental factors and learned experience in their life.'' This means that our brains use learned socio-cultural knowledge to interpret a vague, underlying discomfort into a specific, socially coherent narrative. Evolution may have encoded certain raw feelings about body representation, personality, or other traits. Nonetheless, these studies, as well as similar neurological studies, mainly focus on ``typical,'' early-onset transgender individuals. In contrast, studies focus on ``atypical'' transgender or non-binary individuals are relatively rare. \textcite{Bonazzi2025Gender} demonstrate the significant relevance between autism and transgender identity. The neurological heterogeneity between body representation incongruence and autism implies that ``transgender'' people do not share a biological essence.

According to the predictive coding framework, the human brain functions as a continuous prediction engine, leveraging past knowledge and internal models to efficiently anticipate sensory data and thereby minimise unexpected outcomes \parencite{Clark2013Whatever}. \textcite{Tacikowski2020Fluidity} demonstrated that the perceptual illusion of owning a phenotypic opposite-sex body causes a dynamic, robust, and automatic shift in gender identity, characterized by a more balanced subjective identification with both genders, updated implicit self-associations, and reduced gender-stereotypical beliefs regarding one's own personality. If an individual has a sensory processing variation (e.g., the breast feels ``alien''), this creates a prediction error. The brain can resolve this error by updating its ``self-model.'' If the social environment offers a category (``Transgender Man'') that explains this feeling, the brain adopts this identity to minimise the error. Similarly, as demonstrated by \textcite{Clausen2021Action}, footstep sounds can modulate gender identity and the sense of self-group relation in cisgender participants. Therefore, ``gender identity'' is shaped through updating our self model to resolve the continuous prediction failure in multiple aspects because one or many of our raw feelings are incongruence with gender categories. Body representation incongruent people's raw feeling might be ``this body part isn't mine.'' Autistic people's raw feeling might be ``the arbitrary gender norm is hard to conform and makes me uncomfortable.'' Some people with gender non-conforming personalities and behaviours might feel ``I dislike the personality/temperament externally forced on me.'' Other transgender individuals might develop their gender identity through personal history. The gender construct ties disparate causes together and make us interpret them as ``This means I am a girl/boy/man/woman/non-binary person.'' ``Gender identity'' is therefore not an internal essence, but a post-hoc rationalisation, a ``conceptual chimaera.'' It connects a pre-linguistic raw feeling with a purely socially constructed category of identity. Then it claims this connection is ``natural,'' using the raw feeling to legitimise a social construct.

Therefore, we must distinguish between the biological causes and the psychological explanation. The former are continuous, mosaic, and low-level biological traits. The latter is a discrete, high-level semantic label recruited by the brain's predictive machinery to make sense of the former within a specific cultural framework. Confusing them is the fundamental error of gender essentialism.

To elucidate the mechanisms by which a "gender identity" is formed without resorting to essentialism, this narrative incorporates a detailed autobiographical case study (\hyperref{app:case-study}). It is necessary to explicitly acknowledge the epistemological limitations of this approach. As a subjective account of a single individual (N=1), this autobiography does not claim to generate universal statistical data or prove a biological rule on its own. Instead, it is offered as supplementary evidence to the broader theoretical and scientific argument.

%If we were to judge intersex conditions in humans as we do in birds, based on external genitalia alone, then most birds (except for a few groups like ducks, whose males have a penis) have the same external genitalia (a cloaca), so they should not have the concept of ``intersex'' at all. Some ornithologists classify birds as intersex based on mixed plumage colour \parencite{Choudhary2024Intersex}, but feathers' relationship to reproduction is less direct than that of mammalian breasts (used for lactation). Their function is more like our body hair, Adam's apple, beard, and facial features, which are purely for attracting mates and not physiologically related to reproduction.

%If birds' mixed plumage can be used to classify intersex, then individuals with gynaecomastia should also be intersex. The neutral neurological traits related to an individual's personality, behaviour patterns, and body representation, which make them more likely to form a specific gender identity when interacting with current social gender norms, are also a form of sexual dimorphism and have a significant statistical correlation with gametes (sex \textit{sensu stricto}). Thus, theoretically, they should also be part of ``phenotypic sex.''

%If we apply the standard for birds to ourselves, an individual whose certain sexually dimorphic traits differ from the statistical norm for their gametic sex could be considered intersex. Then, transgender people whose identity stems from innate body representation incongruence are all intersexes, and in fact, almost everyone is intersex, such as a ``female'' with a lot of body hair or a short ``male.'' The ``standard'' and ``typical'' ``male'' and ``female'' (endosex) would be rare exceptions.

%Also, ``conversion therapy cannot change it'' is not a valid argument for ``gender identity is innate.'' Let's conduct a thought experiment: try using ``conversion therapy'' to change a person's native language. If we apply this logic honestly, consistently, and without compromise, we will inevitably conclude that ``native language is also innate.'' \footnote{I am not seriously suggesting a ``forced native language conversion''; this is clearly a violent, anti-human act.}

%Without the intervention of society and culture, an individual would not choose words like ``male'' and ``female'' to describe themselves. Our social constructs have encoded these two words with meanings beyond their reproductive biological sense. Neurological features are not gender; they are simply personality traits, behavioural patterns, and body schemas. In social and cultural interactions, they may lead an individual to develop a sense of ``identity'' with specific gender constructs and gender roles. Still, they themselves are not ``gender.''
%This process occurs within the research domains of sociology and psychology. Using biological methods to study them is a serious category error, as ineffective as calculating the relativistic velocity of each car to study traffic flow on a road.

\section{Philosophical Critique of Gender Identity}\label{sec:philosophical-critique}
%As we said above, if we apply the standards of Aves to Mammalia, we will conclude that almost all humans are intersex. However, in the following argument, let's not jump to such a radical stage.

%Let's first achieve a kind of logical consistency within humanity: individuals with gynaecomastia are intersex. This is the necessary result of applying the standard of breasts as part of ``phenotypic sex'' from other domains to the domain of ``intersex.''

%Then, according to the old diagnostic criteria, like in ICD-10~\parencite{WHO1992ICD10} or DSM-IV-TR~\parencite{APA2000DSM4}, an individual with gynaecomastia who self-identifies as male should be considered a specific kind of transsexual (XtM), because they have ``a sense of discomfort with their anatomic sex,'' and want to make it ``as congruent as possible with one's preferred sex.'' The medical treatment should also be considered ``gender affirming surgery.'' Ironically, both diagnostic criteria explicitly exclude intersex individuals, which is extremely hard to understand.

%However, rather than achieving greater inclusivity by including intersex people, the editors continue to maintain a \textit{de facto} exclusion of intersex people by introducing the concept of  ``assigned sex at birth''~\parencite{APA2013DSM5, WHO2019ICD11}. Not only individuals with gynaecomastia were excluded from transgender, but many traditionally defined

The preceding biological analysis established that ``gender identity'' is not an innate, immutable essence housed within the brain or soul, but rather a product of neuroplasticity formed as the brain ``overfits'' to a noisy and arbitrary social norms. Consequently, because the biological hardware is merely processing the software of culture, I shift the focus from the biological mechanism to the philosophical structure of the social constructs themselves. I will demonstrate that the contemporary theoretical framework surrounding gender -- specifically the dichotomy of ``cisgender'' and ``transgender'' and the concept of ``assigned sex at birth'' (ASAB) -- is full of internal contradictions to maintain a precarious ideological order.

Let us have a thought experiment: an intersex child is assigned female at birth, but gets lost at a very young age, is reassigned male by adoptive parents, and is raised as a boy. After adulthood, their body is closer to a ``phenotypic male,'' but their gender identity is female. According to the definition of \textcite{APA2013DSM5, WHO2019ICD11}, this individual is a ``cisgender female.''

By introducing the concept of  ``assigned sex at birth,'' some intersex people are excluded from transgender. Ironically, if we use ``phenotypic sex'' to define transgender people, we can capture this individual's unique gender socialisation experience (XtF) very well. Since the so-called ``assigned sex'' is almost always binary in actual medical practice, it simplifies the patterns of children's gender socialisation using two typical phenotypic sexes, replicating the power framework it seeks to oppose, and in fact, brings back the ghost of gender binary.

When we examine ASAB more furtherly, we will inevitably ask: Is ``assigned sex/gender at birth'' \footnote{In English, both ``assigned sex'' and ``assigned gender'' are used. } a kind of ``sex'' or ``gender?'' What is the relationship between ASAB and gender identity? If gender is unrelated to phenotypic sex and gender identity is not identifying with sex, then logically, being phenotypic or assigned male/female and identifying as male/female are two or three unrelated things. They just share the words ``male/female,'' like the ecological and sociological ``community.''

Therefore, everyone's phenotypic sex (a series of biological characteristics, although artificially selected) or ASAB (a label on a legal document) is ``inconsistent'' with their gender identity (a psychological state). Or more accurately, it is impossible to discuss whether they are ``consistent'' or ``inconsistent.''. Because they describe three essentially different kinds of things, just as we would not say that the social ``community'' a person belongs to is ``consistent'' with the ecological ``community'' they inhabitants.

Conversely, if those whose ASAB use the same word as their gender identity are called ``cisgender,'' and their phenotypic sex or ASAB is ``consistent'' with their gender identity, this means that phenotypic sex and gender are related in some way. Their view on the relationship between sex and gender is not a rigorous philosophical nominalism, but a vulgarised, selectively applied nominalist fallacy. When they need to separate the body from identity, they use nominalism; when they need to establish the cisgender/transgender binary opposition, they secretly retreat to an unreflective realist position.

The conflict between ``nominalism'' and ``realism'' occurs not only between ASAB and gender identity, but also within ``gender identity'' itself: is the ``woman'' of feminism and the ``woman'' of patriarchy the same concept? The ``woman'' who supports feminism and the traditional ``woman'' both self-identify as ``woman.'' Are their gender identities the same? They claim that all people who identify as ``woman'' share some real, existing ``essence'' called ``female identity'', no matter how different their political stances, lifestyles, and values are. And it uses realist labels like ``cisgender/transgender'' to classify these completely unrelated self-identities as the same kind, simply because their name is or is not consistent with their birth certificate.

Moreover, the cisgender/transgender binary opposition fundamentally violates the ``principle of self-identification,'' because it is equivalent to assigning a gender identity that is ``consistent'' with the ASAB to all people who have not explicitly identified as a different one. Most ``cisgender'' people have never made such a statement, nor have they ever been asked ``What is your gender identity?'' This act is almost completely isomorphic to the ASAB; both are top-down, external assignments without consent. A movement, whose important ethical principle is ``opposing non-consensual identity assignment,'' \footnote{\textcite{Butler2025Who}: \ldots Can we decide what being or having a sex means outside of a framework that establishes and reestablishes sex, that is, a framework that has to be imposed with regularity through time, one where the power to self-assign is exercised by those who have already \textit{been} assigned? Some trans people turn against all assignment, claiming that it invariably works in the service of hierarchy.} relies on the non-consensual identity assignment of most people. It repeats the oppressive structure it criticises, but only packaging itself in a ``critical'' language.

It implicitly introduces an \textit{a priori} assumption that ``everyone has a gender identity,'' because the definition of ``cisgender/transgender'' depends on the existence of it and its ``consistency'' with ASAB. By establishing a classification based on ``gender identity,'' it made ``having a gender identity'' a necessary condition for being a complete-socialised person. If you are not ``transgender,'' then you must be ``cisgender'' -- humans are deprived of the options ``I don't have this thing at all,'' ``null,'' or ``undefined.'' Under this framework, the position of ``I have no gender identity'' becomes invisible or politically incorrect. Individuals are deprived of the right to ``quit the game''; they are either ``cisgender'' or ``transgender.''
%\footnote{This topic reminds me that I once intended to declare my gender identity as null. I was using it in a very sincere, technical sense (value unassigned). Nevertheless, I immediately realised that I cannot express it like how a computer handles null. When someone asks about my gender identity, I cannot directly throw a ``NullPointerException: field \`gender\` not found in object \`user\`'' in their brains. In real life, it is still a textual label. The ``null'' I am writing in this article is four Latin letters, not the technical null.}

Most fatally, when they set ``cisgender'' as a synonym for ``non-transgender'' or ``having not stated their gender identity,'' they are saying that all transgender people are cisgender at birth. This is equivalent to saying that ``transgender people were cisgender and changed for specific reasons,'' which becomes the conservative viewpoint. This flaw is logically and politically devastating.

%``Transgender'' is also this kind of imposed identity if we view it from a specific perspective. Some activists and scholars like \textcite{Arraiza2024After} argue that neurological markers are ``biological determinism'' and strip trans people of their rights. However, if we completely abandon neurological markers, it is effectively saying a person must participate in the social gender construct and have a ``gender identity'' to get medical treatment. People are striped the right of refusing to participate in gender constructs and identity politics, and just going to a doctor and say, ``I have a persistent rejection of this part, it affects my life, and I want to remove it to resolve my suffering.'' The only reason this isn't the primary path is that the ``gender identity'' narrative currently has more followers.


\section{Abolition}\label{sec:abolition}
Critics might accuse me, ``your theory is just a wrap of queer theory with analytic and scientific languages.'' I admit that my model is similar with \posscite{Butler1990Gender} theory in some senses, including denying a Cartesian self and supporting a constructed/shaped self-consciousness by the external environment through discursive effect/neuroplasticity. The most significant difference lies in our ontological premise: I believe in an objective reality that exists independently of the human mind \footnote{I admit that it may sound like begging the question, but obviously it's off-topic to provide a sophisticated argument for scientific realism here. }; and for queer theory, our understanding of reality is constructed through discourse and we cannot perceive it without mediation. In most contexts, ontology might be nothing more than a boring metaphysical intellectual game. While in this case, ontology is everything, it decides two completely different strategy.

From a scientific realistic aspect, gender shapes our understanding about the biological, psychological and sociological diversity of human beings, which is inconsistent with its real mathematical structure. This misunderstanding, also known as ``stereotype,'' affects our self-identity. Being analogous to deep learning (connectionism), gender construct is a corrupted training data that contains two flawed classes: Class 0 (female) is consisted of feminine faces, vaginas, clitorises, developed breasts, pink, purple, brooms, mops, kitchens, dolls, stuffed toys, dresses, Mary Jane shoes, jewellery, make-up \ldots~On the contrary, class 1 (male) is made up of masculine faces, penises, testicles, undeveloped breasts, Adam's apples, beards, blue, computers, documents, architecture, toy cars, pants, suits, ties \ldots~An individual's gender identity is the product of it. ``Gender identity'' is our brains desperately trying to find a pattern in a chaotic, nonsensical, and wrongly labelled set of social data. It latches onto these arbitrary connections and creates a rigid internal model that doesn't actually reflect a fundamental reality.

Therefore, the most fundamental ``-centrism'' in our society is not androcentrism, heterocentrism, or cisgender-centrism, but gender-centrism -- an ideology that establishes ``gender'' as the core category for understanding the world and ourselves. By assigning a gender to everything, ``gender'' attempts to bring the entire world under its dominion, to establish itself as the centre of the way humans perceive the world. It prevents humans from accurately perceiving the real mathematical structure of the world. The gender-centrism is so ancient, so deep, that even the most so-called ``liberatory'' theories or social movements -- feminism or LGBTQ -- have not escaped it or do not dare to escape it. They merely try to change cells within the prison. They resist the ``oppression of gender,'' but they are afraid, unwilling, unable, or refuse to resist the existence of ``gender'' itself.

For other socially constructed identities, such as nation, race, class, profession, etc., their supporters clearly know that these identities are socially constructed. Only gender, a completely externally imposed system, is regarded by the entire political spectrum as something \textit{a priori} and essential. The only disagreement is where this essence lies -- in our genitals (``scientific'' gender essentialism), brains (transgender mechanistic materialism), or souls (transgender idealism and religious conservative). It is even regarded by some groups as the ``true self,'' a symbol of ``freedom and empowerment.''

``Gender essence'' or ``gender identity'' is proclaimed by all political camps to be innate, internal, profound, and constitutive of essence, yet it is clearly acquired, social, superficial, and shaped by society and culture. It defines our essence, governs our entire lives, is assigned before the formation of the self identity, and claims to be the self itself. It simultaneously occupies the domains of science, humanities, politics, and religions. \footnote{The ``religion'' refers to the conservative narrative of ``the two sexes are the natural order created by God'' and the liberal narrative of ``a soul born in the wrong body.''} It has a unique, unquestionable, cross-political-spectrum ontological privilege.

% To declare a specific one as our ``core'' self is a questionable assumption. It suppresses the expression of other types of self-identity. The dominant status of ``gender'' over other personality traits and self-identities also reduces the culturally and socially available combinations, leading to a decrease in the freedom of self-expression.

Since the real problem lies in an omni-encompassing ideology that attempts to assign ``gender'' to everything, the most fundamental solution is to abolish the whole gender construct. I hereby propose a solution aiming at abolishing gender by strictly limiting ``sex'' and ``gender'' to atomic entity, detaching them from individuals.

% (\cref{tab:1})
%
%\begin{table}[bt]
%    \caption{Comparison of the Claims and Realities of Gender.}
%    \label{tab:1}
%    % Use "S" column identifier to align on decimal point
%    \begin{tabularx}{\textwidth}{l X X}
%    \toprule
%    Characteristic & Claim          & Reality     \\
%    \midrule
%    Time	& Innate, before self-awareness	& Acquired after learning \\
%    Source	& Internal, in our brains, genitals or souls	& Internalised from society and culture \\
%    Hierarchy	& Profound, equivalent to self-perception	& Superficial, merely a descriptive tool \\
%    Function	& Constitutive of essence, defining existence	& Alienating, psychological colonisation \\
%    Ontological Status	& Cross-political-spectrum ontological privilege	& Arbitrary symbols of language and culture \\
%    Domain	& Science + Humanities + Politics + Religions	& An accidental product of history and culture \\
%    \bottomrule
%    \end{tabularx}
%\end{table}

%Concepts like ``sexual orientation,'' ``gender identity,'' and ``intersex'' were not created to ``liberate humanity from gender binary,'' but to ``protect gender binary from humanity.'' They are just a series of ``epicycles.'' Like the pre-Copernican astronomers, our modern psychologists and sociologists are creating these epicycles for those who do not conform to the norm (homosexuals, bisexuals, transgender people, intersex people). In this way, the system can safely isolate them as ``exceptions,'' thus maintaining the Ptolemaic model (the cisgender heterosexual endosex binary norm) in most situations. The true purpose of historical ``epicycles'' was not to ``help planets find their true orbits,'' but to avoid a radical paradigm shift.
%Otherwise, as we discussed earlier, why does a change intended to increase ``inclusivity'' -- using ``assigned sex'' instead of ``phenotypic sex'' in the definition of transgender -- logically reduce inclusivity and the scope of people who can be considered ``transgender''?
%
%Some zoologists call non-human animal individuals ``gender non-conforming'' (GNC) because their behaviour does not conform to the statistical typic for their own ``phenotypic sex'', such as the female chimpanzee Donna, who occupied territory and fought. Primatologist Frans de Waal even claimed that ``primates are born with a gender identity''~\parencite{DeWaal2022Different, Morin2022Frans}. This is effectively saying that gender is a behavioural characteristic of animals, thereby reducing sociology and psychology to a branch of ethology that specifically studies the species \textit{Homo sapiens}, thus making gender a part of sex.
%
%Thus, if we strictly follow this framework, considering the inaccessibility of subjective experience, ``gender identity'' should be regarded as ``using the vocal cords to produce encoded mechanical waves, or using visual symbols created by the species itself to express oneself as male/female.'' It is a behavioural sexual dimorphism of \textit{Homo sapiens}, which is obviously absurd.
%
%Similarly, under this definition, the sex of sexually monomorphic organisms cannot be determined without dissection. The so-called ``male/female'' that is generally judged based on behavioural characteristics should logically be gender, not sex. A case in point is that many lovebird owners judge their pet birds' ``sex'' based on their behaviour, such as incubation, nest building, and position during mating.
%
%If animal behaviour is gender, then botanists describing the reproductive function of a plant as \textit{gender} is a natural extension of the former. The claim by some transgender activists or the biologists who support them that using the word \textit{gender} to describe plants ``harms'' transgender people~\parencite{Oberle2023Benefits} is baseless. The same word, when used for humans, is a political and human rights issue; when used for great apes, it ``proves the naturalness of transgender''; but when used for plants, it becomes a form of ``harm.'' What is the reason for this distinction? This is essentially anthropocentrism, taxonomic chauvinism, and an act of granting our closest relatives the title of ``honorary humans.''
%
%Moreover, even if we analyse it under the mainstream theory, gender is a sociocultural construct, while gender identity is an individual's internal psychological state. These are two different concepts. ``Society'' and ``culture'' cannot be the object of harm. Claiming that the borrowing of an abstract sociological concept, \textit{gender}, will ``harm'' humans who possess an internal psychological state, \textit{gender identity}, is a category error.
%
%Considering that:
%
%\begin{enumerate}
%    \item Sex is not a natural kind, but a social construct, therefore a part of gender.
%    \item Gender is a behavioural characteristic of animals, therefore a biological attribute.
%\end{enumerate}
%
%Therefore: $ S \subseteq G \wedge G \subseteq S \Rightarrow S = G $
%
%QED\@.
%
%It is clearly demonstrated that no one really knows what the difference between sex and gender is. No one really knows what they are talking about. Different activists and scholars, according to their political needs, arbitrarily borrow concepts from other disciplines and redefine them, completely ignoring whether these acts will cause the entire gender discourse system to collapse logically.
%

The biological sex should be strictly confined to the gametes \parencite{Lehtonen2014Gamete, Goymann2023Biological, Hurst1996There, Griffiths2025Biology}. It is related to, and only to, gametes. Individuals do not have such a sex. Sexual dimorphism is a dynamic spectrum that constantly changes throughout evolutionary history. The so-called ``phenotypic sex'' should be dismantled into a series of discrete, decentralised phenotypic traits such as chromosomes, hormones, as well as all morphology and anatomical traits. They are distributed on the sexual dimorphism spectrum, along with sexually dimorphic traits not traditionally classified as ``phenotypic sex'', like height, weight, body hair, and body fat percentage \parencite{Wells2007Sexual}. Individuals do not have an innate gender; they merely ``possess'' gametes, ``live in''  a society and ``internalise'' the culture.
%The imprecise concept of intersex should be replaced by more detailed terms like intergonadal, intergenital, and sexual intermorphism. Gender should be strictly treated as a sociocultural phenomenon.

By biologically confining sex strictly to an unobservable trait, it cannot be ``assigned'' or ``inferred.'' \textcite{Parvin1982Ovulation} reported a person who was completely ``phenotypic male'' and had fathered a daughter, but one of their two ``testicles'' was actually a pure ovary (not an ovotestis), and dissection showed it had previously ovulated. Therefore, most people, even those who have had children, cannot know for sure if they are capable of producing two types of gametes. This makes it lose any possibility of being used for social classification, discipline, and oppression. It becomes a technical term of reproductive biology, completely ``exiled'' from everyday language and the operation of power.

My proposal is completely different from conservatives, who claim to adopt gametic definition, but still rely on the genital in practice, which is self-contradictory. This even includes some conservative scientists, like \textcite{Sokal2024Sex}, who claim that sex is an ``objective biological reality,'' ``determined at conception and observed at birth.'' I agree that sex \textit{sensu stricto} is a objective biological reality, and is strictly binary in Vertebrata. While according to \textcite{Jones2006Gamete}, ``Conception occurs when a sperm and an ovum fuse to become a zygote.'' ``The process of fertilisation, or conception, involves fusion of the nucleus of a male gamete (sperm) and a female gamete (ovum) to form a new individual.'' At ``conception'' (fertilisation), it is just a zygote, a single cell. It has neither sex \textit{sensu stricto} because it cannot produce its own gametes, nor does it have phenotypic sex by lacking a developed body. Even if its genome will guide it to develop into a typical phenotypic male or female under normal conditions, it will not necessarily be so. For instance, if a 46, XY zygote loses its Y chromosome during the first mitotic division after fertilisation, it will develop into a 45, X0/46, XY chimaera (mixed gonadal dysgenesis) or a 45, X0 individual with Turner syndrome \parencite{Gravholt2017Clinical, Jacobs1997Turner, Lopes2014Mosaicism}.

The binaric sex is still a very useful model in evolutionary biology, ethology, and reproductive ecology. It is obviously technically and ethically impossible to capture and dissect all individual animals to examine the gametes they produce. In these cases, researchers should state in their \textit{Materials and Methods} section what proxy they used to estimate the gamete production capacity and the reliability, such as the situation in birds revealed by \textcite{Hall2025Prevalence}. Sex \textit{sensu stricto} itself is a microscopic reproductive biological trait, not an externally observable morphological character.

%\textcite{Polderman2018Biological} claimed that the heritability of gender identity is 30--60\% and consistent with other behavioural and personality traits, which is a category error. As we have said before, ``gender identity'' cannot be encoded in the genome (this would require extraordinary evidence). We can only have a series of genes related to personality traits, cognitive styles, and body representation. Studying the ``heritability of gender identity'' is a huge logical leap. The correct scientific questions should not be: ``What is the biological basis of gender identity?'' but rather: ``Which heritable biological traits are assigned gender meaning by society and culture?'' (sociology). ``How does the meaning interact with the individual to construct a specific gender identity?'' (psychology). ``What is the biological basis of these traits?'' (genetics and neuroscience). Autism spectrum disorder is also highly heritable \parencite{Sandin2017Heritability}, and autistic traits are higher in those working in STEM fields \parencite{Ruzich2015Sex}. Following \posscite{Polderman2018Biological} methods, we might also be able to calculate a mathematically reasonable number of ``heritability of STEM ability'', which everyone would consider an absurd study. Autistic neurological traits are heritable, while participating in STEM fields is an extremely complex result of innate neurological traits, personal experience, academic training and so on. Biological studies of ``gender identity'' are precisely the 21st-century replication of ``scientific'' racism.

At the social level, all gender concepts and any social constructs should be abolished. The biological characteristics previously considered ``phenotypic sex'' should have no more social significance than height or weight. All mention of gender in laws and regulations should be removed. In cases not involving a relationship, replace "man" or "woman" with "person." In cases involving a relationship (such as marriage), replace them with "one person" and "the other person." Gendered pronouns and titles should be abolished, sexual orientation should be considered mating preferences or aesthetic preferences, gender incongruence should be considered a form of body representation incongruent, and gender-affirming surgery should be considered a form of cosmetic surgery. A person's desire to change their body, whether because of innate body representation incongruence or any acquired factor, is a matter of pure personal freedom. These are merely mechanical modifications to our body, its legitimacy stems from bodily autonomy. It does not need an unprovable ``gender identity'' to justify it. Gender markers should be removed from all nonmedical records, and sex markers in medical records should be multi-level (chromosomes, reproductive organs, fertility, hormone levels, etc.) rather than just male/female. All gender-segregated facilities should be eliminated. Clothing stores should no longer be divided into ``men's/women's'' sections, but organised by clothing type (tops, pants, outerwear) and size/fit. Toys should no longer be divided into ``boys'/girls','' but by function and type (e.g., building blocks, dolls, science experiments, art creation) and appropriate age.

Most importantly, ``gender identity'' should be understood literally, ``identity with a gender,'' which is shaped in one's interaction with and internalisation of social gender constructs. The psychotherapist should play the role of a philosophy and science teacher, promoting self-understanding in clients who are exploring their gender identity through Socratic dialogue. It's not about providing answers; it's about providing tools for thinking and empowering the individual to find their own answers. They are rational beings capable of and deserving of a deep, philosophical engagement with their own identity.

We have no need at all to shy away from ``gender identity is the internalisation of social gender norms and gender stereotypes.'' The gender identity of cisgender people is also the internalisation of gender norms and stereotypes. However, it is politically not important. Why is it a virtue when childhood trauma causes a person to have a strong heart, but when it causes a person to develop an incongruent gender identity, it needs to be ``converted?'' This conservative logic is very easy to refute. What should be emphasised is: gender identity is an acquired psychological state, but it is acquired subtly and unconsciously, like the first language. For the person themselves, it feels very "innate." This is how our brains work, it does not mean that we should blame people for "internalising gender stereotypes."

%This transforms the issue from a special identity under identity politics, a special psychological state (identity), and a specific phenomenon, to the dismantling of an oppressive system. Everyone, whether ``cisgender'' or ``transgender,'' is a victim of this system. It is a political issue that requires the participation of the entire society: why do bathrooms need to be segregated by gender, and why do documents need to be marked with gender?



\section{Dialogue with Gender Pluralism}\label{sec:dialogue-with-pluralism}
\input{chapters/dialogue-with-pluralism}

\section{Dialogue with Queer Theory}\label{sec:dialogue-with-queer}
Critics might oppose my previous statement that ``neither feminism nor LGBTQ has opposed gender-centrism,'' claiming that ``queer theory is against gender-centrism.'' I admit that queer theory is ontologically anti-gender-centric, as argued by \textcite{Butler1990Gender} that they do not consider ``sex'' (phenotypic sex) or gender to be natural, real or stable. Still, there is a huge difference between ontology and methodology.

\textcite{Butler1990Gender} advocated for a strategy of ``repeating'':

\begin{quotation}
    The task is not whether to repeat, but how to repeat, or, indeed, to repeat and, through a radical multiplexing of gender, to displace the very gender norms that enable the repetition itself.
\end{quotation}

The issue is neither whether to repeat nor how to repeat, but what to repeat. For example, it is a kind of ``repeat'' to create a lot of neopronouns, however, this is precisely because of the gendered pronouns in English. Nobody would create a lot of neoterms about bicycles, because there are no masculine and feminine forms of it. This strategy recognises the gender norm's power to define the area and boundary of the battlefield. Therefore, queer theory is ontologically anti-gender-centric yet methodologically extremely gender-centric.

The intended outcome of this strategy was to ``displace the very gender norms'' by showing their constructed and unstable nature. However, as the actual outcome, by constantly talking about and analysing gender, by ``repeating'' gender inside the boundary, it has reinforced ``gender'' as the central issue for understanding ourselves and society. ``Gender'' was originally just an external sociocultural construct, but it is now frequently used as a synonym of ``gender identity,'' an internal psychological phenomenon. \textcite{Oberle2023Benefits} is a typical case, they argued that using the word \textit{gender} to describe plants ``harms'' transgender people. It is an obvious category error to claim that the borrowing of an abstract sociological concept, \textit{gender}, will ``harm'' humans who possess a psychological phenomenon, \textit{gender identity}.

%Every new identity they create is a new ``true self'' that individuals can choose. This makes the most fundamental rule, ``you must have a gender identity'', more unquestionable. The queer theory theoretically opposes the identity politics, yet its strategy seems to offer infinite choices of identities.
Queer theory's strategy created a powerful cultural incentive: to treat all kinds of unconventional personal experiences and identities as a ``gender'', because it has made ``gender'' the most attractive ``liberative'' discourse. If ``gender'' were still strictly understood as a sociocultural construct, xenogender people would likely not see their ``identity'' as a ``gender identity.'' They do so because ``gender'' is currently the most prominent and available discourse of resistance. This paradoxically expands the category of gender to encompass phenomena that might otherwise have been understood in different categories (e.g., as personality, philosophy, or simply as radical individuality). Within the xenogender community, there is already reflection on this issue, and the alternative term xenoidentity has been proposed for those who have similar identities but are unwilling to classify it as a ``gender identity.''

Queer theory's strategy was to engage gender norms on the ``battlefield'' which was defined by itself. The unintended consequence of this engagement was that the battlefield itself -- the very concept of ``gender'' -- became more central, omni-encompassing, and ideologically powerful than ever before, attracting everyone to join it. The strategy recognised the prison's power to define its walls, and sought to ``destabilise'' them by painting them many different colours, but not destroy and escape from it. The outcome was that the ``colourful and comfortable'' prison became more fundamentally entrenched, attracted more and more people to move in and reinforcing the core idea that one must be a prisoner.

%If the gender system cannot accommodate our existence, shouldn't we completely smash this system? Why must we tell them, ``Actually, we are also a kind of `gender'?'' By saying ``we are also a kind of `gender','' they have accepted that ``gender'' is a valuable category worth joining, which gave up the fundamental right to ask, ``Can we abolish the social norm centred on `gender'?'' \textcite{Butler2025Who} is a typical figure of this reformist ideology:
%
%\begin{quotation}
%    The critique of the gender binary, for instance, did not claim that ``women'' and ``men'' are over and done with. On the contrary, it asked why gender is organized that way and not in some other way. It was also a way of imagining living otherwise. The critique of the gender binary turned out to give rise to a proliferation of genders beyond the established binary versions -- and beyond the gender hierarchy that feminism rightly opposes.
%\end{quotation}

As I previously argued, this is precisely because of the ontology of post-structuralism. They do not believe that objective knowledge of the external world is possible. Therefore, the goal is not to match an external truth, but to play with, subvert, and reconfigure the discourses we live within. In contrast, an objective understand of the reality is possible and desirable in scientific realism, which provides an unshakeable foothold for scientific and philosophical criticism and political resistance. The concept of ``gender'' causes most people's perception of the world to be inconsistent with this external, objective reality. The highest goodness and truth is about aligning our internal models with external reality. Gender is a buggy model that creates a mismatch, so it must be abolished. This is a concise demonstration of the fundamental, irreconcilable gap between the scientific worldview of Enlightenment and the post-structuralist one.

%As \textcite{Marx1977Critique} said in his \textit{A Contribution to the Critique of Hegel's Philosophy of Right}:
%
%\begin{quotation}
%    Luther, we grant, overcame bondage out of \textit{devotion} by replacing it by bondage out of \textit{conviction}. He shattered faith in authority because he restored the authority of faith. He turned priests into laymen because he turned laymen into priests. He freed man from outer religiosity because he made religiosity the inner man. He freed the body from chains because he enchained the heart.
%\end{quotation}
%
%The concept of ``gender identity'' leads individuals to equate a social construct with their true self, to internalise an external social construct as a core part of their self. It freed the body from ``gender'' because it enchained the heart. It successfully achieved what classical gender theory could not do for thousands of years.

In an interview with \textcite{Williams2014Gender}, Butler claims that ``some people really love the gender that they have claimed for themselves. If gender is eradicated, so too is an important domain of pleasure for many people. And others have a strong sense of self bound up with their genders, so to get rid of gender would be to shatter their self-hood.'' This is a reasonable concern with humanistic care. It is also found in the analytical tradition: \textcite{Cull2019Against} argued that ``A genderless society is harmful to transgender people by refusing to recognise their identity.'' However, the fallacy is that Butler and Cull presumed the existence of ``transgender people'' as a fixed category in a society where the concept of gender no longer operates, which is essentialist. It is the same as saying, ``We need to use pathogens to make vaccines, so eliminating pathogens would harm the patient's life.''

I do not shy away from this point: according to my model, in a world where gender is abolished, transgender people will not exist. Cisgender people will also not exist. Because ``gender identity'' will not exist, at least not in its current form. However, this is not an elimination of contemporary transgender people, nor does it mean that gender-affirming care for contemporary transgender people is not needed. Quite the contrary, this is based on a profound acknowledgement of the suffering of contemporary transgender people (including myself). The root cause of the suffering (such as gender dysphoria) and oppression is the social construct about gender. Transgenderness is not an individual condition stemming from an ``inner self'' that requires medical care, but a serious political threat to our very existence. Insisting on ``gender identity'' is to treat only the symptoms while allowing the pathogen (the gender construct) to persist. True liberation is not achieved by merely managing the symptoms while allowing the pathogen to thrive. It requires eliminating the pathogen altogether. It is not to defend the identity of ``transgender'' but to achieve a world where such categories are no longer necessary to describe human experience. Gender abolitionism is the only structuralist solution, as it correctly identifies the problem in the ``training data'' that shapes all individuals.

Another fallacy of Butler's this claim is that we can abolish stereotypes while keeping gender/gender identity. However, gender identity is the product of socialisation within a gender norm based on stereotypes. If this gender norm no longer exists, gender identity -- at least for most people -- will not exist. It might become a niche, historical subculture, like believers in the Athenian pantheon today. It will be considered a form of freedom of speech and freedom of religious belief.

We have no need at all to shy away from ``gender identity is the internalisation of social gender norms.'' The gender identity of cisgender people is also the internalisation of gender norms. However, what if it's an internalisation of gender norms? I even believe my gender identity is a product of childhood trauma. So, why is it a virtue when childhood trauma causes a person to become strong, but when it causes a person to develop an incongruent gender identity, it needs to be ``converted?'' This conservative logic is ``very easy to refute'' because it rests on an unstated, bigoted value judgment, not on a consistent principle.

What should be emphasised is: gender identity is an acquired psychological state, but it is one that is acquired subtly and unconsciously, like a mother tongue. For the person themselves, it feels very much like it is innate. Therefore, most people find it difficult to realise that it is acquired, and even if they do realise it, they cannot get rid of it through ``realisation.''


\section{Conclusion}\label{sec:conclusion}
This autoethnography serves not merely as a retrospective analysis of a single individual’s ``gender identity,'' but as a manifesto for a fundamental paradigm shift. It has traversed the intimate landscapes of personal memory -- from the stigma of childhood footwear to the alienation within online communities -- and the rigorous terrains of evolutionary biology and political philosophy. The synthesis of these disparate elements points toward a future that is not defined by the proliferation of identities, but by their dissolution. The ultimate prospect offered here is not the expansion of the gender spectrum, but the complete abolition of the ``gender'' category as a structural element of human society. The abolition of gender paves the way for the emergence of a true ``meta-identity.'' Just as the xenogender movement inadvertently hints at the infinite creativity of self-definition, a post-gender world allows individuals to define themselves through the boundless vocabulary of humanity rather than the narrow dialect of gender. The future prospects of this abolitionist project must address the reconstruction of scientific inquiry, social governance, and human subjectivity.

Ultimately, this project advocates for a return to, and a radical expansion of, the Enlightenment ideal. We should move beyond the identity politics that fragment humanity into competing tribes. The future should be an intersubjective world where independent, rational subjects can cooperate and communicate ideally. The abolition of gender is, therefore, a necessary step toward a universal humanism, where individuals are defined not by the categories imposed upon them, but by their capacity for reason, their moral agency, and their shared destiny.


\printbibliography

\end{document}
